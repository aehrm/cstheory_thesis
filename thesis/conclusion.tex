%! TEX root = ./thesis.tex
\chapter{Diskussion und Fazit}\label{chap:conclusion}


%\note{Abschließend erweitern wir noch unsere Notation, die uns insbesondere bei der Orakelkonstruktion helfen wird. Diese folgt Überlegungen von \textcite{dose_np-completeness_2019}. Anstelle von Orakeln als Menge zu verstehen, können wir äquivalent Orakel auch als unendlich lange Wörter $u\in\Sigma^\omega$ formulieren, die wir als das Orakel $\{ i\mid w[i]=1 \}\subseteq \mathbb N$ interpretieren. Mit der obigen Identifikation von Wörtern und natürlichen Zahlen beschreibt nun $u$ sowohl ein Orakel über $\mathbb N$ als auch über $\Sigma^*$; wir können also z.B. von der relativen Berechnung $M^w(x)$ sprechen. Analog fassen wir endlich lange Wörter $w\in\Sigma^*$ als \emph{partielles} Orakel $\{ i\mid w[i]=1 \}$, welches die Zugehörigkeit der Wörter $x<|w|$ festlegt, aber die Zugehörigkeit aller Wörter $y\geq|w|$ noch nicht endgültig festlegt.
%Auf dieser Idee der endgültigen bzw. noch nicht endgültigen Zugehörigkeit aufbauend können wir auch von \emph{definiten} Berechnungen sprechen: Eine Rechnung $M^w(x)$ ist \emph{definit} wenn auf allen Rechenwegen von $M^w(x)$ nur Orakelfragen gestellt werden, welche eine Länge $<|w|$ haben. }
