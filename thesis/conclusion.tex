%! TEX root = ./thesis.tex
\chapter{Fazit}\label{chap:conclusion}

Zum Abschluss werden hier noch einmal die wesentlichen Ergebnisse dieser Arbeit zusammengefasst, sowie noch einmal offene Fragen gesammelt, welche sich im Verlauf der Arbeit ergeben haben.
Bezüglich der Ergebnisse können wir uns an den vier Forschungsdesiderata orientieren, die in der Einleitung genannt wurden.

\subsection*{Suchprobleme vs. Entscheidungsprobleme}

Erstens, wurde in Kapitel~\ref{chap:searchproblems} die Beziehung zwischen NP-Suchproblemen und den entsprechenden NP-Entscheidungsproblemen erarbeitet.
Hierfür orientiert sich die Arbeit an vorhandenen Definitionen von Suchproblemen, die intuitiv neben dem Berechnungsproblem „Gilt $x\in L$?“ auch nach einem „Beweis“ oder „Zertifikat“ für die Aussage „$x\in L$“ fragen. 
Auf dieser Basis wurde gezeigt, dass die übliche Argumentation, das Suchproblem auf das Entscheidungsproblem reduzieren zu können (\emph{search reduces to decision}) im Allgemeinen nicht immer zutrifft. Ergebnisse aus der Literatur zeigen, dass dies unter geeigneten Bedingungen nicht für die NP-Intermediates (also jene Probleme, die nicht NP-vollständig sind, aber auch nicht in $\P$ liegen) zutrifft.
Im Kontext der \emph{Selbstreduzierbarkeit}, also der Lösung des Suchproblems gegeben einem Orakel für „kleinere“ Instanzen, liegen dagegen nur wenige Ergebnisse vor. In der Arbeit wurde insbesondere auf Ergebnisse von \textcite{harsha_downward_2023} hingewiesen, welche Anzeichen geben, dass das Suchproblem der Faktorisierung nicht selbstreduzierbar ist.

Wie die Suchprobleme lassen sich auch die Entscheidungsprobleme mittels Reduktionen untereinander nach ihrer Schwierigkeit vergleichen. In dieser Arbeit wurde insbesondere die Levin-Reduktion als Reduktion auf den NP-Suchproblemen produktiv gemacht. Dieser Reduktionsbegriff kann als natürliche Generalisierung der Many-one- bzw. Karp-Reduktion auf den Entscheidungsproblemen verstanden werden. Hieran anschließend wurde auf die Beziehung zwischen Levin-Vollständigkeit der Suchprobleme und der Karp-Vollständigkeit der entsprechenden Entscheidungsprobleme hingewiesen. Erstens verstärkt die Levin-Vollständigkeit die Karp-Vollständigkeit. Zweitens sehen wir, empirisch gesprochen, dass für alle \emph{natürlichen} Probleme diese beiden Vollständigkeitsbegriffe zusammenfallen. 
Dennoch bleibt Folgendes unklar: \emph{Gegeben ein beliebiges NP-Suchproblem $R$, für welches das entsprechende Entscheidungsproblem $\Proj(R)$ Karp-vollständig ist, ist dann auch $R$ Levin-vollständig?}

Die Formulierung dieser Frage ist der zentrale Beitrag dieser Arbeit.
Sie stellt zwei umfassende „Aufgabentypen“ (Suchen vs. Entscheiden) und Reduktionsbegriffe der Komplexitätstheorie gegenüber, und trotz der einfachen Formulierung beschäftigt sich nur wenig Forschung mit dieser Frage.
Es gibt keine Hinweise, welche auf eine positive oder negative Beantwortung dieser Frage hinweisen; im Kontext dieser Arbeit wurde die negative Beantwortung als Hypothese $\mathsf{KvL}$ formuliert.


%Aufbauend auf der oben genannten empirischen Beobachtung, dass die natürlichen Suchprobleme alle Levin-vollständig sind, wurde in einem weiteren Survey noch auf Verstärkungen dieser Interreduzierbarkeit inerhalb der Literatur hingewiesen, welche im Wesentlichen (analog zur $\P$-Isomorphie der natürlichen NP-vollständigen Mengen) tiefergehende strukturelle Ähnlichkeiten der (Levin-)vollständigen Suchprobleme herausarbeiten.

\subsection*{NP-Suchprobleme im Pudlákschen Programm}

Zweitens, wurde die Hypothese $\hQ$ (erstmals formuliert von \cite{fortnow_separability_1993}) in den Blick genommen, welche als Aussage 
„alle totalen NP-Suchprobleme sind effizient lösbar“ unmittelbar in direkter Beziehung zu den NP-Suchproblemen steht.
Die vorliegende Arbeit zeigt nun insbesondere auch folgende Charakterisierungen von $\hQ$:
\begin{itemize}[midpenalty=0]
    \item Es gilt $\hQ$ genau dann wenn jede Karp-Reduktion zwischen Entscheidungsvarianten von zwei NP-Suchproblemen zu einer Levin-Reduktion zwischen eben jenen Suchproblemen verstärkt werden kann. Damit ist $\neg\hQ$ auch eine notwendige Bedingung für $\mathsf{KvL}$.
    \item Es gilt $\hQ$ genau dann wenn das Standardbeweissystem eines beliebigen Levin-vollständigen NP-Suchproblems $\P$-optimal ist. Das eliminiert damit ein Padding-Argument von \textcite{messner_simulation_2001} und verallgemeinert sein Ergebnis, dass $\hQ$ äqivalent zir $\P$-Optimalität des Standardbeweissystems $\mathit{sat}$ ist. Über diese Charakterisierung wird auch deutlich, dass $\neg\hQ$ auch eine notwendige Bedingung für die Pudláksche Hypothese $\hSAT$ ist.
\end{itemize}
Im Kontext des dritten Forschungsdesiderat wurde das Pudláksche Programm um weitere Hypothesen verfeinert. Insbesondere über die zweite Charakterisierung von $\hQ$ lässt sich die Hypothese $\neg\hQ$ in das Pudláksche Programm einordnen. Für die Hypothese $\mathsf{KvL}$ war dagegen eine Einordnung schwieriger, und es konnte nur gezeigt werden, dass $\mathsf{SAT^{eff}}$, eine natürliche Verstärkung von $\hSAT$, hinreichend für $\mathsf{KvL}$ ist.
%Parallel hierzu trägt diese Arbeit in einem welteren Survey die bekannten Implikationen unter den Hypothesen des (verallgemeinerten) Pudlákschen Programms zusammen, und erarbeitet einen Überblick über die relativierenden Trennungen von Hypothesen mittels existierenden Orakelkonstruktionen. 

Viertens, wurde abschließend ein Orakel konstruiert, relativ zu diesem die (relativierten) Hypothesen $\hDisjNP$, $\hUP$ und $\hQ$ gelten.
Damit wurde insbesondere $\hQ$ und $\mathsf{KvL}$ von einigen der Pudlákschen Hypothesen getrennt. Insbesondere erhalten wir damit das Resultat, dass selbst unter Annahme von $\hDisjNP\land\hUP$ es nicht mit relativierenden Beweisen möglich ist, auf $\hQ$ (und auch nicht auf $\mathsf{KvL}$) zu schließen.
Für die Hypothese $\hQ$ ergibt sich ferner unter relativierbaren Beweisen eine paarweise Unabhängigkeit zu den originalen Pudlákschen Hypothesen, außer für jene Fälle, bei der eine relativierende Implikation bekannt ist:

\begin{corollary}
    Sei $\mathsf A \in \{\hDisjNP, \hUP, \mathsf{TAUT^{N}}, \hTAUT, \hDisjCoNP, \hTFNP, \hSAT\}$.
    Für jede Wahl von $\mathsf A$ existiert je ein Orakel, relativ zu diesem $\mathsf A\land \hQ$, $\mathsf A\land \neg\hQ$, $\neg\mathsf A\land \hQ$, bzw. $\neg\mathsf A\land\neg\hQ$ gilt, außer für die Fälle $\hDisjCoNP\land\hQ$, $\hTFNP\land\hQ$, $\hSAT\land\hQ$.
\end{corollary}

\subsection*{Offene Fragen}

Zum Schluss sind hier noch die offenen Fragen aufgelistet, welche sich im Verlauf dieser Arbeit ergeben haben, und welche eine Basis für zukünftige Forschung bilden könnten:
\begin{enumerate}[label=\arabic*.,midpenalty=0]
    \item Die zentrale offene Frage dieser Arbeit ist jene zwischen Karp- und Levin-Vollständigkeit, bzw. der Hypothese $\mathsf{KvL}$ (vgl. Frage~\ref{question:kvl}). Um mehr Evidenz für die Gültigkeit von $\mathsf{KvL}$ zu sammeln, wären weitere hinreichende natürliche Annahmen, z.B. kryptographischer Art, wünschenswert.
        Das betrifft u.a. auch die Hypothese $\hQ$. Lässt sich die Äquivalenz $\mathsf{KvL}\Leftrightarrow \neg\hQ$ zeigen, wie in Abschnitt~\ref{sec:karp-vs-levin} vermutet?
        Allgemein gesprochen wäre hierzu auch ein besseres Verständnis der Zusammenhänge und Beziehungen zwischen den Hypothesen $\mathsf{KvL}$, $\hQ$, $\hSAT$ und $\mathsf{SAT^{eff}}$ erstrebenswert, sowie angrenzend nähere Untersuchungen zum Begriff der effektiven $\P$-Simulation; die vorliegende Arbeit hat hierzu nur den ersten Anfang gemacht.

    \item Konkret auch auf das Pudláksche Programm bezogen, wären weitere Orakelkonstruktionen dienlich, die $\mathsf{KvL}$ von den anderen Hypothesen trennen, um so $\mathsf{KvL}$ besser situieren zu können. Folgende Trennungen wurden in Abschnitt~\ref{sec:pudlak-overview} als besonders interessant erachtet: konstruiere je ein Orakel, sodass relativ zu diesem $\mathsf{KvL}\land\neg\mathsf{SAT^{eff}}$, $\mathsf{KvL}\land\neg\hSAT$ bzw. $\neg\hQ\land\neg\mathsf{KvL}$ gilt.
        Schon allein ein Orakel $O$ mit $\mathsf{KvL}^O$ wäre erstrebenswert, denn damit könne man relativierende Beweise für $\neg\mathsf{KvL}$ ausschließen.


    \item Konkret in Bezug auf die Hypothese $\hQ$ könnte der Begriff der Levin-Paddability intensiver betrachtet werden, wie er z.B. in der Charakterisierung der Hypothese $\hQ$ eingesetzt wird (Definition~\ref{def:levin-paddable}). Besonders interessant wäre, ob sich diese zusätzliche Bedingung in der Charakterisierung durch Satz~\ref{thm:q}(9) streichen lässt. In anderen Worten: gilt $\hQ$ genau dann wenn für eine $\leqlp$-vollständige NP-Relation $R$ das Standardbeweissystem $\mathit{std}_R$ für jede $N$ mit $L(N)=\Proj(R)$ das Beweissystem $\mathit{std}_N$ $\P$-simulieren kann?
        Zeige hierfür beispielsweise, dass (nicht) jede $\leqlp$-vollständige NP-Relation auch Levin-paddable ist (vgl. Frage~\ref{question:levin-paddable}).
    \item Angrenzend an die Karp-vs-Levin-Frage bzw. Suche-vs-Entscheiden-Frage wäre ein besseres Verständnis zur Selbstreduzierbarkeit unter den Suchproblemen erstrebenswert. Wie schon in Abschnitt~\ref{sec:search-vs-decision} beschrieben, ist die Forschungslage hierzu noch sehr dünn, insbesondere auf den nicht-totalen NP-Suchproblemen. Lassen sich die Erkenntnisse der Selbstreduzierbarkeit von \emph{Entscheidungsproblemen} hierzu produktiv machen? Hierzu auch angrenzend wäre die Betrachtung schwächerer Reduzierbarkeits-Begriffe auf NP-Relationen (z.B. ähnlich zu Turing- oder Truth-Table-Reduktionen), und damit verbunden eine Abschwächung der Hypothese $\mathsf{KvL}$, vielversprechend.
\end{enumerate}
