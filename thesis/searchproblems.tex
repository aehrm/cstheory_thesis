%! TEX root = ./thesis.tex
\chapter{Zur Konzeptualisierung und Ordnung von Suchproblemen}\label{chap:searchproblems}

In diesem Kapitel werden wir grundsätzlich überlegen, wie NP-Suchprobleme in der Komplexitätstheorie erfasst werden können, wie wir diese in ihrer Schwierigkeit untereinander vergleichen können und in welcher  Beziehung sie zu den NP-Entscheidungsproblemen stehen.
In Abschnitt~\ref{sec:searchproblems-def} werden wir eine formal präzise Definition von NP-Suchproblemen erarbeiten, wie sie bereits in der Einleitung intuitiv vorgestellt wurden. Das umfasst auch die Unterklasse TFNP der totalen NP-Suchprobleme.

In Abschnitt~\ref{sec:search-vs-decision} gehen wir auf die Beziehung zwischen NP-Suchproblemen und den entsprechenden NP-Entscheidungsproblemen ein. Insbesondere zeigen wir, in welchen Situationen das Entscheidungsproblem „gleich schwer“ wie das Suchproblem ist (das ist das Argument \emph{search reduces to decision}), und in welchen nicht.

Um die Schwierigkeit der unterschiedlichen NP-Suchproblemen zu vergleichen, werden wir – analog wie auf den Entscheidungsproblemen – ein Begriff der \emph{Levin}-Reduzierbarkeit definieren. In Abschnitt~\ref{sec:levin} definieren wir diesen Reduzierbarkeits-Begriff präzise und betrachten Eigenschaften des entsprechenden Vollständigkeitsbegriffs.
In diesem Zusammenhang betrachten wir auch die bekannteren verwandten \emph{sparsamen} („parsimonious“) Reduktionen.

\section{Definition von Suchproblemen}\label{sec:searchproblems-def}

Wir geben hier noch einmal die Definition von Suchproblemen wieder, welche schon in der Einleitung erarbeitet wurde.
Als Suchprobleme verstehen wir das algorithmische Problem, gegeben eine Probleminstanz $x\in\Sigma^*$, entweder eine entsprechende positive Lösungsinstanz $y\in\Sigma^*$ zu berechnen, oder negativ abzulehnen.  Hier noch einmal das Beispiel (7) aus der Einleitung: gegeben eine aussagenlogische Formel $\phi$, berechne entweder eine Belegung $y$ welche $\phi$ erfüllt, oder gebe „unerfüllbar“ aus.
Die wesentliche Einschränkung, welche wir auch schon in der Einleitung festgelegt haben, ist die Einschränkung auf \emph{NP-Suchprobleme}. Zur Erinnerung: wir meinen damit, dass
\begin{itemize}
    \item die Lösungen nur polynomiell länger als die Probleminstanzen sind, und
    \item effizient in Polynomialzeit verifiziert werden kann, ob zu einer gegebenen Probleminstanz $x$ ein beliebiges Wort $y$ tatsächlich eine (positive) Lösung im Sinne des Suchproblems darstellt oder nicht.
\end{itemize}
(Wir fordern im Übrigen nicht, dass negatives Ablehnen effizient verifiziert werden kann.)
Um das Beispiel wieder aufzugreifen: Zum einen haben Formeln $\phi$, welche überhaupt erfüllbar sind, eine erfüllende Belegung in der Länge von $\phi$. Zum anderen kann effizient geprüft werden, ob $y$ tatsächlich eine erfüllbare Belegung von $\phi$ ist.

Wir können die beiden obigen Punkte noch einmal in eine formale Definition gießen:
\begin{definition}[NP-Relation, $\FNP$]\label{def:np-relation}
    Eine \emph{NP-Relation} ist eine binäre Relation $R\subseteq \Sigma^*\times\Sigma^*$, sodass diese
    \begin{enumerate}
        \item in Polynomialzeit entscheidbar ist, d.h. $R\in\P$, bzw. genauer $\{\langle x, y\rangle \mid (x,y)\in R\}\in\P$, und
        \item polynomiell längenbeschränkt ist, d.h. es existiert ein Polynom $q$, sodass
            \begin{equation}\label{eq:zertifikatsschranke}
                (x,y)\in R \implies |y|\leq q(|x|) \quad\text{für alle $x,y\in\Sigma^*$}.
            \end{equation}
    \end{enumerate}
    Die Wörter der ersten Komponente nennen wir \emph{Probleminstanzen} oder \emph{Instanzen} oder \emph{Probleme} von $R$, die Wörter der zweiten Komponente nennen wir die $R$-\emph{Zertifikate} (oder manchmal $R$-\emph{Lösungen}). Wir sagen dann für $(x,y)\in R$, dass $y$ ein $R$-Zertifikat \emph{für $x$} ist. In diesem Sinne sagt \eqref{eq:zertifikatsschranke}  aus, dass Zertifikate $y$ für $x$ nicht superpolynomiell länger als $x$ sein dürfen.
    Das Polynom $q$ nennen wir auch die \emph{Zertifikatsschranke} zu $R$. 

    Wir schreiben $\FNP$ für die Klasse aller NP-Relationen. \qedhere
\end{definition}

Das oben diskutierte allgemeine Suchproblem zu einer NP-Relation $R$ kann jetzt wie folgt formal formuliert werden:
\begin{quote}
    \textbf{Suchproblem zur NP-Relation $R$:}
    \begin{description}[nosep]
        \item[Gegeben:] Instanz $x\in\Sigma^*$.
        \item[Gesucht:] Zertifikat $y\in\Sigma^*$ mit $(x,y)\in  R$ falls ein solches $y$ überhaupt existiert, sonst „keine Lösung“ ausgeben.
    \end{description}
\end{quote}
Zur Erinnerung:
\[ \Proj(R) = \{ x \mid \exists y\in\Sigma^*, (x,y)\in R \}\in\NP. \]
Die Menge $\Proj(R)$ ist also die Menge der Probleminstanzen, für welche mindestens ein zugehöriges Zertifikat existiert; damit entspricht $\Proj(R)$ derjenigen Menge, die üblicherweise bei algorithmischen Entscheidungsproblemen betrachtet wird. 
Um die beiden Varianten noch einmal gegenüberzustellen: das entsprechende Entscheidungsproblem einer NP-Relation $R$ lautet
\begin{quote}
    \textbf{Entscheidungsproblem zur NP-Relation $R$:}
    \begin{description}[nosep]
        \item[Gegeben:] Instanz $x\in\Sigma^*$.
        \item[Gesucht:] Akzeptieren falls ein Zertifikat $y\in\Sigma^*$ mit $(x,y)\in  R$ existiert, sonst ablehnen.
    \end{description}
\end{quote}
Das entspricht genau dem Entscheiden der Sprache $\Proj(R)$ im sonst üblichen Sinn. Damit wird auch klar, dass das entsprechende Entscheidungsproblem bzw. die Sprache $\Proj(R)$ nicht von der konkreten Relation $R$ abhängig ist. Vielmehr: zur Sprache $L$ existieren ggf. unendlich viele NP-Relationen $R$ mit $\Proj(R)=L$.  Für eine Sprache $L$ sagen wir dann auch, dass $R$ eine NP-Relation \emph{für $L$} ist, wenn $\Proj(R)=L$.

Die Zugehörigkeit des entsprechenden Suchproblems zu $\NP$ folgt hierbei unmittelbar aus der Definition von NP-Relationen. (Rate nichtdeterministisch ein Zertifikat und akzeptiere wenn dieses korrekt ist.)
Im nächsten Abschnitt wird die Beziehung zwischen NP-Suchproblemen bzw. NP-Relationen einerseits, und Entscheidungsproblemen bzw. Mengen aus $\NP$ andererseits, weiter behandelt.
Festhalten können wir hier aber schon, dass das Suchproblem offenbar „schwieriger“ ist als das alleinige Entscheidungsproblem: wenn das Suchproblem lösbar ist, dann ist auch das Entscheidungsproblem lösbar.

Im Folgenden werden eine Beispiele von natürlichen NP-Relationen angegeben. Um diese von den sonst üblicherweise verwendeten Labels für Mengen bzw. Suchprobleme abzugrenzen, sind im Verlauf dieser Arbeit NP-Relationen zu natürlichen Suchproblemen immer mit einem $\mathtt{r}$ am Anfang gekennzeichnet.
\label{page:natural-searchproblems}
\begin{itemize}[midpenalty=0]
\item $\mathtt{rLARGESTMATCHING} \defeq \{ (G, M) \mid \text{$G$ ist ein Graph, $M$ ein größtes Matching auf $G$} \}$. Das korrespondiert zum Suchproblem (3) aus der Einleitung.
\item $\mathtt{rSAT} \defeq \{ (\phi, w) \mid $ $\phi$ ist eine aussagenlogische Formel, $w$ erfüllende Belegung für $\phi$$\}$. Das korrespondiert zum Suchproblem (7) aus der Einleitung.
\item $\mathtt{rCLIQUE} \defeq \{ ((G, k), C) \mid $ $G$ ist ein Graph, $C$ eine Clique mit $\leq k$ Knoten$\}$. Das korrespondiert zu Suchproblem (6) aus der Einleitung.
\item $\mathtt{rHAMCYCLE} \defeq \{ (G, P) \mid $ $G$ ist ein Graph, $P$ ein Zyklus, welcher jeden Knoten von $G$ genau einmal berührt$\}$.
\item $\mathtt{rANOTHERHAMCYCLE} \defeq \{ ((G, P), P') \mid $ $G$ ist ein Graph, $P, P'$ je ein Zyklus der jeden Knoten genau einmal berührt, $P\neq P' \}$.
\item $\mathtt{rFACTORIZATION} \defeq \{ (n, (p_1,p_2,\dots, p_k)) \mid n\in\mathbb N, n>1$, alle $p_i>1$ Primzahlen, und $n\defeq p_1\cdots p_k \}$. Das korrespondiert zum Suchproblem (1) aus der Einleitung.
\item $\mathtt{rFACTOR} \defeq \{ (n, p) \mid n\in\mathbb N$ , $n>1$ ist nicht prim, und $p$ ist ein Primfaktor von $n\}$.
\item $\mathtt{rSMALLFACTOR} \defeq \{ ((n, a), p) \mid n\in\mathbb N$, $n>1$ ist nicht prim, und $p\leq a$ ist ein Primfaktor von $n$$\}$.
\item $\mathtt{rGI} \defeq \{ ((G, H), \sigma) \mid G, H$ sind Graphen mit gleicher Knotenmenge, und $\sigma$ ist ein Graphisomorphismus von $G$ nach $H\}$.
\end{itemize}
Jede dieser Relationen ist auch eine NP-Relation. Beachte, dass die Menge der Primzahlen in Polynomialzeit entscheidbar ist \parencite{agrawal_primes_2004}.
Bei jeder der obigen natürlichen Relationen gilt, dass die Projektion auch der sonst üblichen Sprache aus NP zum Entscheidungsproblem entspricht. Wir haben z.B.
\[ \Proj(\mathtt{rCLIQUE}) = \{ (G, k) \mid \text{ex. Clique $C$ von Graph $G$ mit $\leq k$ Knoten} \} \defeq \mathtt{CLIQUE}. \]

Die Definition von Suchproblemen als NP-\emph{Relationen} lässt es zu, Suchprobleme bzw. NP-Relationen als „partielle Multifunktionen“ zu verstehen.
\textcite{selman_taxonomy_1994} definiert in seiner Taxonomie der Funktionsklassen die Klasse $\mathrm{NPMV_g}$ als die Klasse derjenigen Multifunktionen $f\in\mathrm{NPMV}$, für die (der Graph) $f$ in $\P$ liegt.
Es lässt sich leicht sehen, dass die hier definierte Klasse $\FNP$ identisch zu \citeauthor{selman_taxonomy_1994} definierten Klasse $\mathrm{NPMV_g}$ ist, solange man Multifunktionen mit binären Relationen identifiziert.

Mit dieser Perspektivierung lässt sich auch formal definieren, was mit „Suchproblem lösen“ gemeint ist. Wir machen hierbei Gebrauch von Verfeinerungen von Multifunktionen.
Wir sagen, dass das Suchproblem zur NP-Relation $R$ \emph{in Polynomialzeit lösbar ist}, wenn $R\inc \mathrm{FP}$.
Diese Aussage bedeutet ja, dass eine Verfeinerung $f$ von $R$ existiert, und $f$ ist dabei eine (partielle) in Polynomialzeit berechenbare Funktion. Es existiert also ein deterministischer Polynomialzeit-Transduktor $T$, welcher $f$ berechnet.
Für eine Eingabeinstanz $x$ wird also entweder $T(x)$ einen Wert $y$ ausgeben für den $y=f(x)\in\fset{}R(x)$ gilt, bzw. in anderen Worten, eine $R$-Lösung $y$ für $x$ im Sinne der Definition~\ref{def:np-relation}.
Oder, falls $T(x)$ ablehnt, dann ist $x\not\in\dom(f)=\Proj(R)$, heißt „$f(x)$ lehnt ab“ bedeutet, dass $x$ keine Lösung hat.

Unter dieser Definition von Lösrbarkeit ist klar, dass das NP-Suchproblem „schwieriger“ ist als das entsprechende NP-Entscheidungsproblem, in dem Sinne dass sich das NP-Entscheidungsproblem auf das NP-Suchproblem reduzieren lässt:
\begin{observation}\label{obs:search-stronger-than-decision}
    Sei $R$ eine NP-Relation. Falls $R\inc\FP$, dann gilt $\Proj(R)\in\P$.
\end{observation}
%\begin{proof}
%    Nachdem $R\inc\FP$, existiert ein PTM-Transduktor $M$, welcher eine Verfeinerung von $R$ berechnet. Dieser Transduktor entscheidet schon $\Proj(R)$: wenn $x\in\Proj(R)$, dann wird $M(x)$ (mit einer entsprehenden Lösung) akzeptieren, sonst ablehnen.
%\end{proof}

Der aktuelle Stand zur Lösbarkeit der oben genannten natürlichen Suchprobleme ist:\label{page:lösbarkeit}
\begin{itemize}\raggedright
    \item $\mathtt{rLARGESTMATCHING}\inc\FP$.
    \item $\NP=\P \iff \mathtt{rSAT}\inc\FP \iff \mathtt{rCLIQUE}\inc\FP \iff \mathtt{rHAMCYCLE}\inc\FP \iff \mathtt{rANOTHERHAMCYCLE}\inc\FP$.
    \item Unklar, ob $\mathtt{rSMALLFACTOR}, \mathtt{rFACTOR}, \mathtt{rFACTORIZATION}\stackrel{\smash{?}}{\inc}\FP$. Wir haben aber $\UP\cap\coUP=\P \implies \mathtt{rSMALLFACTOR}\inc \FP \iff \mathtt{rFACTOR}\inc\FP \iff \mathtt{rFACTORIZATION}\inc\FP$.
    \item Unklar, ob $\mathtt{rGI}\stackrel{\smash{?}}{\inc}{\FP}$.
\end{itemize}

Bevor nun im nächsten Abschnitt die Suchprobleme den Entscheidungsproblemen näher gegenübergestellt werden, schließen wir diesen Abschnitt noch mit einer kurzen Diskussion zu \emph{totalen} Suchproblemen ab.

\subsection*{Totale NP-Suchprobleme}

Die oben formulierte Definition von $\FNP$ ist genau diejenige, die von \textcite{megiddo_total_1991} zuerst in dieser Form und Bezeichnung definiert wurde. Ihre Motivation war hierbei, insbesondere die \emph{totalen} NP-Suchprobleme in den Blick zu nehmen. Also solche NP-Suchprobleme, bei denen zu jeder Probleminstanz immer mindestens ein Zertifikat bzw. Lösung existiert. Die Faktorisierung ist beispielsweise ein solches totales Suchproblem, da sich ja jede natürliche Zahl faktorisieren lässt.

Das sind – entsprechend dieser Definition von $\FNP$ bzw. Konzeptualisierung von Suchproblemen – genau jene NP-Relationen, welche (links-)total sind: für jedes $x\in\Sigma^*$ existiert ein $y\in\Sigma^*$ mit $(x,y)\in R$. In anderen Worten, $\Proj(R)=\Sigma^*$.
Die oben definierten NP-Relationen $\mathtt{rFACTORIZATION}$ und $\mathtt{rFACTOR}$ sind nicht total; nachdem die negativen Instanzen aber einfach zu entscheiden sind, können für beide NP-Relationen effektiv äquivalente Relationen angegeben werden, die total sind:
\begin{itemize} \item $\mathtt{rFACTORIZATION}' \defeq \mathtt{rFACTORIZATION} \cup \{ (n, \text{\emph{„ungültig“}}) \mid n\leq 1 \}$.
    \item $\mathtt{rFACTOR}' \defeq \mathtt{rFACTOR} \cup \{ (n, \text{\emph{„ungültig“}}) \mid n\leq 1 \text{ oder $n$ ist prim} \}$.
\end{itemize}
Herbei soll \emph{„ungültig“} ein String sein, welcher eine Lösung darstellt.
\textcite{megiddo_total_1991} fassen diese totalen NP-Relationen zur Klasse $\mathrm{TFNP}$ zusammen:
\begin{definition}[$\mathrm{TFNP}$]
    Die Klasse $\mathrm{TFNP}$ ist die Teilmenge von $\mathtt{FNP}$ derjenigen NP-Relationen $R$, welche linkstotal sind, heißt zu jedem $x\in\Sigma^*$ existiert ein $y\in\Sigma^*$ mit $(x,y)\in R$.
\end{definition}
Hierzu gehören die oben genannten Varianten $\mathtt{rFACTORIZATION}'$ und $\mathtt{rFACTOR}'$.
Für \citeauthor{megiddo_total_1991} befinden sich in $\mathrm{TFNP}$ eine Vielzahl von interessanten und schwierigen Suchproblemen, bei denen die Frage der Lösbarkeit in Polynomialzeit noch offen ist.
Das betrifft u.a. zahlentheoretische Probleme aus der Kryptographie wie Faktorisierung oder diskreter Logarithmus. Beachte, dass $\TFNP$ nicht identisch ist zur Klasse $\NPMVt$: Die Klasse $\TFNP$ ist eine Teilmenge von $\NPMVt$ jener totalen Multifunktionen $f\in\NPMVt$, für die der (der Graph) $f$ in $\P$ liegt. Man könnte also $\TFNP$ äquivalent als $\mathrm{(NPMV_{t})_{g}}$ schreiben.
Beachte, dass $\mathrm{TFNP}$ sogar eine echte Teilmenge von $\NPMVt$ ist, außer $\P=\NP$:
\begin{observation}[{\cite[vgl.][Prop.~5]{fenner_inverting_2003}}]\label{obs:npmvt-properin-tfnp}
    Wenn für alle $f\in\NPMVt$ auch (der Graph) $f\in \P$ ist, dann gilt $\P=\NP$.
\end{observation}
% \begin{proof}
%     Betrachte folgenden NPTM-Transduktor $N$ auf Eingabe $\phi\in\Sigma^*$: zunächst spaltet sich die Berechnung nichtdeterministisch auf. In der ersten Rechnung wird sofort 1 ausgegeben. In der zweiten Rechnung wird eine Belegung $w$ für die aussagenlogische Formel $\phi$ geraten, und 2 ausgegeben wenn $w$ die Formel $\phi$ erfüllt. Sei $f$ die Multifunktion, welche von $N$ berechnet wird.
%     Damit gilt:
%     \[ \fset{}f(x) = \begin{cases} \{ 1, 2\} & \text{falls $x\in\mathtt{SAT}$,} \\ \{ 1\} & \text{sonst. } \end{cases} \]
%     und $f\in\NPMVt$.
%     Nach Annahme ist $f\in \P$. Nun kann aber $\mathtt{SAT}$ in Polynomialzeit entschieden werden, denn $\phi\in\mathtt{SAT}$ genau dann wenn $(\phi, 2) \in f$.
% 
%     Die Aussage relativiert, wenn anstelle $\mathtt{SAT}$ z.B. das kanonische vollständige Problem gewählt wird.
% \end{proof}
Wie in der Einleitung angesprochen, lässt sich die Aussage „alle Suchprobleme in $\TFNP$ sind effizient lösbar“ auf einfache Weise äquivalent als Hypothese $\hQ$ formulieren.
\begin{observation}[\cite{fenner_inverting_2003}]\label{obs:tfnp-q}
    Folgende Aussagen sind äquivalent:
    \begin{enumerate}
        \item Aussage $\hQ$: für jede totale NPTM $N$ (d.h. $L(N)=\Sigma^*$) existiert eine Funktion $g\in\FP$ sodass für alle $x$ das Bild $g(x)$ ein akzeptierender Rechenweg von $N(x)$ ist. 
        \item $\TFNP\subseteqc \FP$.
    \end{enumerate}
\end{observation}
\begin{proof}
\begin{prooflist}
\item (1)$\implies$(2): Sei $R\in\TFNP$ mit Zertifikatsschranke $p$. Definiere die NPTM $N$ die auf Eingabe $x$ erst ein Zertifikat $y\in\Sigma^{\leq p(|x|)}$ rät, und genau dann akzeptiert wenn $(x,y)\in R$. Es ist klar, dass $L(N)=\Proj(R)=\Sigma^*$. Nach (1) existiert also $g\in\FP$ sodass $g(x)$ ein akzeptierender Rechenweg von $N(x)$ ist. Aus diesem Rechenweg lässt sich nun effizient das geratene Zertifikat extrahieren: sei $g'(x)$ definiert als der geratene Zeuge $y$. Dann ist $(x,g'(x))\in R$ und damit $g'$ eine Verfeinerung von $R$. Da $g'\in\FP$ ist $R\inc\FP$.

\item (2)$\implies$(1): Sei $N$ eine NPTM mit $L(N)=\Sigma^*$. Klar ist, dass die Relation $R\defeq \{ (x,\alpha) \mid N(x)$ akz. auf Rechenweg $\alpha\}$ eine totale NP-Relation ist.
    Mit (2) existiert also eine Verfeinerung $g\in\FP$ von $R$. Für $x\in\Sigma^*$ ist also $(x, g(x))\in R$ und nach Definition akzeptiert also $N(x)$ auf Rechenweg $g(x)$, wie gewünscht.
\end{prooflist}
\end{proof}

Aus der Beschäftigung mit TFNP-Problemen kam es ferner zu einer umfassenden Theoriebildung. So kam z.B. eine verfeinerte Betrachtung durch Aufmachen bestimmter Unterklassen von TFNP hinzu. Diese Unterklassen verinnerlichen hierbei jeweils das kombinatorische Prinzip, „warum“ ein Suchproblem total ist \parencite[vgl. den Überblick von][]{goldberg_towards_2018}. Exemplarisch seien hier zwei Unterklassen skizziert:
\begin{itemize}
    \item Die Unterklasse PLS („polynomial local search“) umfasst die Suchprobleme, welche in die Form eines Suchgraphen mit polynomiellen Grad gebracht werden können, worauf ein lokales Optimum gesucht ist.
        Das zugrunde liegende kombinatorische Prinzip zur Totalität wäre „{Endliche Suchgraphen haben immer ein lokales Optimum}“ oder allgemeiner „{Jeder endliche gerichtete azyklische Graph hat eine Senke}“.

        Ein Beispiel hierfür wäre die Suche nach einem lokal optimalen Schnitt in einem Graphen; hier meint „lokal optimal“, dass kein Flip eines Knotens zu mehr Kantenschnitten führt. Nachdem es nur exponentiell viele Schnitte gibt, muss mindestens einer davon lokal optimal sein. Beachte außerdem, dass lokale Optimalität in Polynomialzeit überprüft werden kann, denn es muss nur getestet werden, ob einer der linear vielen möglichen Flips eines Knotens zu einer Verbesserung führt.

    \item Die Unterklasse PPP („polynomial pigeon principle“) umfasst Suchprobleme, welche aufgrund des kombinatorischen Schubfachprinzip total sind. 

        Ein Beispiel hierfür ist das Gleiche-Summe-Suchproblem: gegeben $n$ positive ganze Zahlen die sich zu $<2^n-1$ aufsummieren, finde zwei unterschiedliche nichtleere Teilmengen dieser Zahlen welche die gleiche Summe haben. Diese zwei Teilmengen existieren immer nach Schubfachprinzip: es existieren $2^n-1$ viele nichtleere Teilmengen, jede davon mit Summe $<2^n-1$, die Summen können also nicht alle unterschiedlich sein.
\end{itemize}
Auf die weitere Theorie der TFNP-Probleme wird in dieser Arbeit nicht weiter eingegangen. Wir werden aber in Abschnitt~\ref{sec:levin} noch Reduktionen auf NP-Relationen definieren; dieser Reduktionsbegriff identisch zu dem sonst üblichen Reduktionsbegriff auf den TFNP-Problemen \parencite{megiddo_total_1991}.


\section{Suchprobleme vs. Entscheidungsprobleme}\label{sec:search-vs-decision}

Wie in der Einleitung schon ausgeführt, konzentriert sich die algorithmische Komplexitätstheorie primär auf die Entscheidungsprobleme und weniger auf die Suchprobleme. Das ist durchaus fundiert: es geht mit einer Vereinfachung der Konzepte, Definitionen und Theorien einher, und gleichzeitig lassen sich viele NP-Suchproblem auf das entsprechende Entscheidungsproblem reduzieren. Gemeint ist damit: gegeben ein effizienter Algorithmus $A$, welcher das korrespondierende NP-Entscheidungsproblem löst, kann auch ein effizienter Algorithmus $A'$ angegeben werden, welcher das NP-Suchproblem löst.
Dieses Argument wird als \emph{search reduces to decision} beschrieben. In diesen Fällen ist dann die Lösbarkeit der Entscheidungsvariante äquivalent zur Lösbarkeit der Suchvariante. 

In diesem Abschnitt werden detailliert Suchprobleme und Entscheidungsprobleme gegenübergestellt und Forschungsergebnisse hierzu aus der Literatur präsentiert.
Zum einen wird die eben genannte Reduzierbarkeit und das \emph{search-reduces-to-decision}-Argument ausgeführt, und zum anderen werden Ergebnisse vorgestellt, die darauf hinweisen dass genau dieses Argument nicht für alle Suchprobleme zutrifft.

Zunächst sei auf die Beziehungen zwischen NP-Relationen und NP-Sprachen hingewiesen.
Tatsächlich haben wir bereits gesehen, dass wir über die Projektion jedem NP-Suchproblem bzw. NP-Relation ein korrespondierendes Entscheidungsproblem aus $\NP$ zuordnen konnten. Tatsächlich lässt sich diese Zuordnung auch umkehren: zu jeder Sprache bzw. Entscheidungsproblem $L\in \NP$ existiert eine NPTM $N$, welche $L$ entscheidet. Diese induziert eine NP-Relation für $L$ (akz. Rechenweg auf $N(x)$ ist Lösung für $x$). Das ist die übliche „Zertifikats-Charakterisierung“ von $\NP$ aus den Lehrbüchern.
\begin{observation}[Zertifikats-Definition von NP]\label{obs:np-certificate-def}
\[ \NP = \{ \Proj(R) \mid \text{$R$ ist eine NP-Relation} \}.\]
\end{observation}
%\begin{proof}
%    Wir müssen nur noch die Inklusion von links nach rechts zeigen. Sei hierfür $L\in\NP$ eine Sprache und $N$ eine NPTM die $L$ entscheidet, wobei die Laufzeit durch das Polynom $p$ beschränkt ist. Definiere nun die Relation
%    \[ R_N  \defeq \{ (x, \alpha) \mid \text{$N(x)$ akz. mit RW $\alpha$, $\alpha$ hat $\leq p(|x|)$ viele Schritte} \} \]
%    Diese Relation ist eine NP-Relation. Der Test ist offenbar in Polynomialzeit möglich, und die Relation ist polynomiell längenbeschränkt, ist $|\alpha|\in O($\# Schritte von $\alpha)\in O(p(|x|))$.
%    Aus Definition geht hervor dass $L(N) = \Proj(R_N)$.
%\end{proof}
Damit ist im Übrigen die obige Definition von NP-Relationen auch nicht neu sondern schon immer mitgedacht. Die eben formulierte Charakterisierung findet sich in allen üblichen Einführungswerken zur Komplexitätstheorie. 
%Dagegen machen die unterschiedlichen Lehrbücher ihren Zugang manchmal stärker von der Perspektive der Suchprobleme abhängig, und manchmal stärker von der typischen Herangehensweise über Entscheidungsprobleme.
%Vgl. z.B. \textcite{goldreich_computational_2008} welcher in seinem Lehrbuch die $\P$-vs.-$\NP$-Frage zunächst als die äquivalente Frage der Beziehung zwischen den „\emph{efficiently solvable search problems}“ und den „\emph{search problems with efficiently checkable solutions}“ (letzteres sind genau die NP-Relationen) formuliert. Erst später wird mittels \emph{search-reduces-to-decision}-Argumenten dafür argumentiert, NP-Entscheidungsprobleme als die zentralen Untersuchungsobjekte der Komplexitätstheorie anzusehen.

%Beachte auch, dass sich die vorige Beobachtung auch analog für $\UP$ formulieren lässt:
%\begin{observation}[Zertifikats-Definition von NP]\label{obs:up-certificate-def}
    %$\UP = \{ \Proj(R) \mid \text{$R$ ist eine rechtseindeutige NP-Relation} \}.$
%\end{observation}

Auch $\UP$ lässt eine solche Zertifikats-Definition zu, in der $\UP$ mit der Menge der Projektionen von rechtseindeutigen NP-Relationen übereinstimmt.

Zumindest für die $\P$-$\NP$-Frage ist es irrelevant, ob man sich auf Suchprobleme von NP-Relationen oder auf Entscheidungsproblemen von NP-Mengen bezieht. Jedes NP-Suchproblem ist in Polynomialzeit lösbar genau dann wenn jede Menge in NP in deterministischer Polynomialzeit entscheidbar ist.
\begin{lemma}\label{lemma:equiv-p-np-question}
    $\FNP\subseteqc \FP \iff \P = \NP.$
\end{lemma}
\begin{proof}
    Die Richtung von links nach rechts ist klar, folgt ja aus der Lösbarkeit des Suchproblems die Lösbarkeit des Entscheidungsproblems (Beobachtung~\ref{obs:search-stronger-than-decision} mit Beobachtung \ref{obs:np-certificate-def}).

    Die Richtung von rechts nach links zeigen wir mittels Präfixsuche. Sei $R$ eine beliebige NP-Relation mit Zertifikatsschranke $q$. Wir zeigen dass $R\inc\FP$.
    Betrachte folgende Menge
    \[ A_R \defeq \{ (x,z) \mid \exists y\in\Sigma^{\leq q(|x|)}, (x,y)\in R, z\sqsubseteq y \}. \]
    Es ist leicht zu sehen dass $A_R\in\NP$. Also gilt nach Annahme auch $A\in\P$.
    Gegeben eine Instanz $x$ kann nun iterativ ein Präfix eines Zertifikats verlängert werden:\\
    \begin{algorithm}[H]
        $z\gets\epsilon$\;
        \While(\tcp*[h]{Invariante: wenn eine $R$-Lösung für $x$ ex., dann existiert eine $R$-Lösung $y$ mit $z\sqsubseteq y$}){$|z|\leq q(|x|)$}
        {
            \uIf{$(x,z)\in R$}{\AcceptWith{$z$}}
            \uElseIf{$(x,z0)\in A_R$}{$z\gets z0$}
            \uElseIf{$(x,z1)\in A_R$}{$z\gets z1$}
            \Else{\Reject}
        }
        \Reject
    \end{algorithm}
    \noindent
    Es ist klar, dass der obige Algorithmus eine Funktion definiert, welche eine Verfeinerung von $R$ ist.
    Unter Annahme $A_R\in\P$ ist auch klar, dass diese Funktion von einem PTM-Transduktor berechnet werden kann. Damit $R\inc\FP$.
\end{proof}

\subsection*{\emph{Search reduces to decision}}

Es ist leicht zu sehen, dass der Suchalgorithmus von obigem Beweis so geändert werden kann, dass anstelle der Entscheidung von $A_R$ auch Orakelfragen an ein (externes) Orakel $A_R$ gestellt werden können, d.h. das Suchproblem von $R$ kann \emph{à la Cook} auf das Entscheidungsproblem von $A_R$ reduziert werden. In anderen Worten, $R\inc \FP^{A_R}$. Das generalisiert sogar, wenn statt $A_R$ ein beliebiges Orakel gewählt wird, welches $\leq_\mathrm{m}^\mathrm{T}$-vollständig für $\NP$ ist. Ist also $\Proj(R)$ $\leq_\mathrm{m}^\mathrm{T}$-vollständig für $\NP$, dann gilt trivialerweise der Spezialfall $R\inc \FP^{\Proj(R)}$. Das ist genau das \emph{search-reduces-to-decision}-Argument: ist das Entscheidungsproblem zu $R$ über ein Algorithmus $A$ effizient lösbar, dann kann dieser Algorithmus so eingesetzt werden, dass auch das Suchproblem zu $R$ effizient durch einen Algorithmus $A'$ gelöst werden kann.

\begin{corollary}[\emph{Search reduces to decision} für die NP-Vollständigen]\label{cor:search-to-decision}
    Sei $R$ eine NP-Relation, für die $\Proj(R)$ auch $\leq_\mathrm{m}^\mathrm{T}$-vollständig für $\NP$ ist.
    Dann gilt $R\inc \FP^{\Proj(R)}$.
\end{corollary}
%\begin{proof}
%    Wir zeigen die Aussage mit einem Relativierbarkeits-Argument.
%
%    Relativ zum Orakel $\Proj(R)$ gilt $\P=\NP$, ist ja $\Proj(R)$ vollständig für $\NP$. Damit gilt mit vorigem Lemma~\ref{lemma:equiv-p-np-question} auch $\FNP \subseteqc \FP$ relativ zu $\Proj(R)$.
%    Da $R\in \FNP$, gilt also auch $R\inc \FP$ relativ zu $\Proj(R)$.
%\end{proof}

Damit bleibten insbesondere diejenigen Situationen offen, in denen $R$ eine NP-Relation ist, aber $\Proj(R)$ nicht $\leqmp$-vollständig für $\NP$ ist, also $\Proj(R)$ ein NP-Intermediate ist.

%Für die NP-Intermediates, also Entscheidungsprobleme aus NP, die weder in P liegen, noch $\leqmp$-vollständig für $\NP$ sind, ist aber unklar, ob immer das Suchproblem auf das Entscheidungsproblem reduziert werden kann.

Wie beim Suchalgorithmus aus obigem Beweis ist klar, dass Suchprobleme zumindest immer auf eine Präfix- bzw. Bisektion-Entscheidungsvariante reduziert werden können.
Im allgemeinen Fall: für jede NP-Relation $R$ mit Laufzeitschranke $q$ gilt
\[ R \inc \FP^{L_R} \text{ wobei } L_R = \{ (x, z) \mid \exists y\in\Sigma^{\leq q(|x|)} \mid (x, y)\in R, z\sqsubseteq y \}\in\NP \]
und 
\[ R \inc \FP^{L'_R} \text{ wobei } L'_R = \{ (x, z) \mid \exists y\in\Sigma^{\leq q(|x|)} \mid (x, y) \in R, y\leq z \}\in\NP, \]
jeweils mit einer polynomiellen Anzahl an Orakelfragen.


Konkret ist das zum Beispiel der Fall bei der NP-Relation $\mathtt{rSMALLFACTOR}$. Zur Erinnerung, wir haben
\[ \Proj(\mathtt{rSMALLFACTOR}) = \{ (n,a)\mid \text{$n>1$ nicht prim, ex. Faktor $p>1$ von $n$ mit $p\leq a$} \}. \]
Wollen wir für gegebene $n, a$ einen Faktor $p\leq a$ finden, können wir diesen mit binärer Suche un polynomiell vielen Orakelfragen an $\Proj(\mathtt{rSMALLFACTOR})$ finden.
In anderen Worten, es gilt $\mathtt{rSMALLFACTOR} \inc \FP^{\Proj(\mathtt{rSMALLFACTOR})}$.

%Es lässt sich im Übrigen zeigen, dass $\Proj(\mathtt{rSMALLFACTOR})\in\UP\cap\coUP$.
%Die Projektion ist einerseits in $\UP$: gegeben $(n, a)$, $n$ nicht prim, rate zunächst eine Primfaktorzerlegung von $n$ mit aufsteigenden Primfaktoren. Akzeptiere genau dann wenn ein Faktor $p\leq a$ in dieser Zerlegung erhalten ist. Der Fundamentalsatz der Arithmetik sichert, dass die (aufsteigend geordnete) Primfaktorzerlegung eindeutig ist, heißt der oben skizzierte nichtdeterministische Polynomialzeit-Algorithmus akzeptiert auf höchstens einem Rechenweg.
%Auf ähnliche Weise lässt sich zeigen dass die Projektion in $\coUP$ liegt.
Mittels der Eindeutigkeit der Primfaktorzerlegun lässt sich im Übrigen zeigen, dass $\Proj(\mathtt{rSMALLFACTOR})\in\UP\cap\coUP$.
Damit ergibt sich auch der auf S.~\pageref{page:lösbarkeit} angegebene Stand, dass ein Kollaps von $\UP\cap\coUP$ mit $\P$ zur Folge hätte, dass $\mathtt{rSMALLFACTOR}\inc\FP$.

%Beachte insbesondere dass $\Proj(\mathtt{rSMALLFACTOR})\in\
%Auf ähnliche Weise kann auch eine geeignete Variante zum Faktorisierungsproblem angegeben werden. Zur Erinnerung, die natürliche NP-Relation $\mathtt{rFACTORIZATION}$ reliert die Zahlen $\geq 2$ mit ihren Primfaktorzerlegungen, und damit haben wir $\Proj(\mathtt{rFACTORIZATION})=\{2,3,\dots\}$; Ein Orakel für $\Proj(\mathtt{rFACTORIZATION})$ hat also keinen Effekt.
%Betrachte nun folgendes NP-Entscheidungsproblem
%\[ 
%\begin{split} A \defeq \{ (n, z) \mid &n\geq 2, \text{ und sei $y=\langle p_1, p_2, \dots, p_k\rangle$ eine Primfaktorzerlegung von $n$} \\ &\text{mit $p_1, \ldots, p_k$ prim, $n=p_1p_2\cdots p_k$, $p_1\leq p_2\leq\cdots \leq p_k$}\\ &\text{und $y$ startet mit $z$} \}.\end{split}
%\]
%Dann haben wir nach obiger Argumentation $\mathtt{rFACTORIZATION}\inc\FP^{A}$.
%Insbesondere lässt sich zeigen, dass $A\in\mathrm{UP\cap coUP}$




Das \emph{search-reduces-to-decision}-Argument hat aber auch Grenzen:
Diese Technik scheitert insbesondere, wenn wir wirklich immer die exakte Projektion als Entscheidungsproblem verstehen. Betrachte zum Beispiel die NP-Relation zur linearen Teilbarkeit:
\[ \mathtt{rLINDIV} \defeq \{ ((a, b), k) \mid a,b,k\in\mathbb N, a\cdot k + 1\text{ teilt } b\}. \]
Unter Annahme von $\NP\neq\coNP$ gilt  $\Proj(\mathtt{rLINDIV})\not\in \P$  \parencite{adleman_reducibility_1977}; ob $\Proj(\mathtt{rLINDIV})$ auch $\leqmp$-vollständig für $\NP$ ist, bleibt unklar.
Bei dieser NP-Relation wäre nun nicht ersichtlich, wie das Suchproblem auf das Entscheidungsproblem reduziert werden könnte; eine triviale binäre Suche wie oben ist ja nicht möglich.

Für andere Suchprobleme existieren aber nichttriviale Möglichkeiten  das Suchproblem auf das (natürliche) Entscheidungsproblem zu reduzieren, auch wenn das Entscheidungsproblem nicht in der Form einer Bisektion bzw. Präfixsuche ist. Hierbei wird die spezifische Struktur des Problems ausgenutzt. Ein Beispiel ist $\mathtt{rSAT}$: Gegeben Formel $\phi$, teste mittels dem Orakel, ob $\phi[x_1/0]\in\mathtt{SAT}$ oder $\phi[x_1/1]\in\mathtt{SAT}$. Hier meint $\phi[x_1/0]$ die Formel, welche entsteht wenn alle Vorkommen von Variable $x_1$ in $\phi$ mit $0$ ersetzt werden, $\phi[x_1/1]$ analog. Sollte jetzt $\phi[x_1/0]\in\mathtt{SAT}$ stimmen, dann wissen wir dass es eine Belegung für $\phi$ existiert die $\phi$ erfüllt und gleichzeitig $x_1$ auf $0$ setzt. Wir können dann iterativ auf dem gleichen Weg eine Belegung für die nächste Variable $x_2$ bestimmen usw. (Der Fall dass $\phi[x_1/1]\in\mathtt{SAT}$ ist analog.) Es gilt daher $\mathtt{rSAT}\in\FP^{\Proj(\mathtt{rSAT})}$.
(Beachte aber, dass $\Proj(\mathtt{rSAT})=\mathtt{SAT}$ schon $\leqmp$-vollständig ist. Damit folgt $\mathtt{rSAT}\in \FP^{\mathtt{SAT}}$ schon aus Korollar~\ref{cor:search-to-decision}.)

Ein weiteres nichttriviales Beispiel wäre die NP-Relation $\mathtt{rGI}$. Zur Erinnerung: dieses Suchproblem sucht nach einem Graphisomorphismus zwischen zwei gegebenen Graphen.
Deren Projektion ist mutmaßlich nicht $\leqmp$-vollständig für $\NP$. 
Gleichzeitig gilt $\mathtt{rGI}\inc \FP^{\Proj(\mathtt{rGI})}$: es lässt sich ein Graphisomorphismus zwischen $G$ und $H$ bestimmen, indem mehrmals mittels des Orakels bei (anderen) Paaren von  Graphen getestet wird, ob diese isomorph sind \parencite[vgl.][S. 65, 100]{goldreich_computational_2008}.
Ob eine solche nichttriviale Reduktion von $\mathtt{rLINDIV}$ auf $\Proj(\mathrm{rLINDIV})$ möglich ist, scheint in der Literatur nicht untersucht zu sein.

Abschließend wollen wir noch drei theoretische Resultate diskutieren. Das erste charakterisiert diejenigen Sprachen, für welche sich das Suchproblem auf das Entscheidungsproblem reduzieren lässst.
Das zweite und drite Resultat gibt hinreiche plausible Bedingungen an, unter denen eine NP-Relation $R$ existiert, für welche sich das Suchproblem nicht auf das Entscheidungsproblem reduzieren lässt.

Zunächst folgende Definition:

\begin{definition}
    \begin{enumerate}
        \item Eine deterministische OTM heißt \emph{robust für $A$} falls $L(M^O)=A$ für alle Orakel $O$.
        \item Eine Menge $A$ heißt \emph{selbsthelfend} falls eine OTM $M$ existiert, welche robust für $A$ ist, und für die $\mathrm{time}_M^A(x)$ polynomiell in abh. von $|x|$ wächst, d.h. zusammen mit dem Orakel $A$ ist $M^A$ ist eine POTM.\qedhere
    \end{enumerate}
\end{definition}

\citeauthor{balcazar_self_1989} fasst die Intuition hinter dieser Definition wie folgt zusammen: man will die Situation abbilden, dass ein Entscheidungsalgorithmus existiert, der, mit genug Zeit, immer zu einem korrekten Ergebnis kommt, aber auch mit einem externen „Helfer“ interagieren darf, welcher dem Algorithmus helfen kann, schneller fertig zu rechnen.

Folgendes Resultat charakterisiert diejenigen Sprachen $L\in\NP$, die zumindest \emph{eine} NP-Relation $R$ für $L$ haben, sodass das Suchproblem (bzgl. $R$) auf das Entscheidungsproblem reduzierbar ist.

\begin{theorem}[\cite{balcazar_self_1989}]
    Sei $A\in \NP$. Folgende Aussagen sind äquivalent:
    \begin{enumerate}
        \item $A$ ist selbsthelfend.
        \item Es existiert eine NP-Relation $R$ sodass $\Proj(R)=A$ und $R\inc \FP^{A}$.
    \end{enumerate}
\end{theorem}

Andererseits existieren unter geeigneten Bedingungen eine NP-Relation $R$, für welche \emph{search reduces to decision} fehlschlägt.

\begin{theorem}[{\cites{impagliazzo_1991}[Thm.~5]{borodin_comments_1976}}]\label{thm:impagliazzo-search}
    Angenommen $\mathrm{E\neq NE}$ oder $\P\neq \NP\cap\coNP$. Dann existiert eine NP-Relation $R$ mit $\Proj(R)\in \NP-\P$, für die $R\not\in_\mathrm c\FP^{\Proj(R)}$ gilt.
\end{theorem}

Unter stärkeren Bedingungen lässt sich sogar zeigen, dass sogar \emph{Mengen} $L\in\NP$ existieren, für die das Suchproblem \emph{jeder} NP-Relation für $L$ nicht auf das Entscheidungsproblem reduziert werden kann. In anderen Worten, unabhängig davon wie ein „Zertifikatssystem“ für $L$ aussieht, ist keins so einfach dass Zertifikate mit Hilfe eines Orakels für das Suchproblem $L$ gefunden werden können.

\begin{theorem}[{\cites[Thm.~1.1]{bellare_complexity_1994}{impagliazzo_1991}}]
Angenommen $\mathrm{EE\neq NEE}$ oder $\mathrm{NE\neq coNE}$. Dann existiert eine Menge $L\in\NP-\P$ sodass $R\not\in_\mathrm c \FP^L$ für jede NP-Relation $R$ für $L$, d.h. für die $\Proj(R)=L$ gilt. (Bzw. ist $L$ nicht selbsthelfend.)
\end{theorem}
Beachte, dass für jede dieser Relationen $R$ aus den beiden vorigen Sätzen die entsprechende Projektion $\Proj(R)$ ein NP-Intermediate ist; $\Proj(R)$ kann nicht $\leqmp$-vollständig für $\NP$ sein, denn das wäre ein Widerspruch zu Korollar~\ref{cor:search-to-decision}.

%\begin{definition}
    %Eine Menge $A$ heißt \emph{selbsthelfend} falls eine POTM $M$ existiert mit $M^X\subseteq A$ für alle $X$, und insbesondere $M^A=A$.
%\end{definition}
%(Die hier angegebene Definition von \emph{self-helping} unterscheidet sich von der ursprünglichen Fassung von \citeauthor{ko_helping_1987}, ist aber nach \citeauthor[180]{balcazar_self_1989} äquivalent.)

\subsection*{Selbstreduzierbarkeit in TFNP}

Für \emph{totale} Suchprobleme, also genau jene aus $\TFNP$, kann nicht sinnvoll gefragt werden, ob hier das Suchproblem auf das Entscheidungsproblem reduziert werden kann, ist ja für $R\in\TFNP$ das entsprechende Entscheidungsproblem $\Proj(R)=\Sigma^*$ trivial.

Stattdessen können wir uns aber fragen, ob das Suchproblem eines Zertifikats zu $x$ einfacher wird, wenn wir Lösungen zu „kleineren“ Instanzen $x'$ gratis abfragen dürfen.
Hierzu schlagen \citeauthor{harsha_downward_2023} folgenden Begriff der Selbstreduzierbarkeit vor:

Betrachte hierbei zunächst folgende Variante eine POTM-Transduktors relativ zu $R\in\TFNP$: Dieser Transduktor ist wie ein üblicher PTM-Transduktor, hat zusätzlich aber Zugriff auf ein \emph{funktionales} Orakel, in dem Sinn dass die PTM Orakelfragen der Form „\emph{gib mir ein Zertifikat $y$ für $x'$}“ stellen kann. Das Orakel antwortet dann mit einem solchen Zertifikat $y$ mit $(x', y)\in R$. Das existiert, ist ja $R$ total. 

Es ist klar, dass mit einem solchen Transduktor relativ zu $R$ auch das Suchproblem zu $R$ lösbar ist. (Gegeben $x$, stelle einfach die Frage „gib mir Zertifikat für $x$“.) Deshalb nehmen wir folgende Einschränkung vor: der Transduktor darf bei Eingabe $x$ in den Orakelfragen nur nach Zertifikaten für $x'$ fragen, die kürzer sind als $x$.
Falls selbst unter dieser Einschränkung der Fragen das Suchproblem durch einen solchen Transduktor relativ zu $R$ gelöst werden kann, sagen wir, dass $R$ \emph{nach unten selbstreduzierbar ist}.

Zum Verständnis erinnern wir uns an die TFNP-Relation $\mathtt{rFACTOR}'$, welche nach einem nichttrivialem Faktor für $n$ sucht, oder „$n$ ist prim“ ausgibt. Wäre nun $\mathtt{rFACTOR}'$  nach unten selbstreduzierbar, dann würde das bedeuten, dass ein Faktor von zusammengesetztem $n\in\mathbb N$ effizient gefunden werden kann, wenn wir nach Faktoren von Zahlen $\leq n/2$ fragen dürfen.
Welche TFNP-Probleme nach unten selbstreduzierbar sind, ist erstaunlich wenig untersucht, und eine Beforschung in dieser präzisen Formulierung wurde erst durch \textcite{harsha_downward_2023} angetreten.
Sie zeigen die Selbstreduzierbarkeit nach unten für das TFNP-Problem „Iterate with source“, welches als ein kanonischer Repräsentant für die Unterklasse $\mathrm{PLS}$ (zur Erinnerung: \emph{polynomial local search}) gilt.
\begin{quote}
    \textbf{Iterate with Source:}
    \begin{description}[nosep]
        \item[Gegeben:] $n\in\mathbb N$, und ein Schaltkreis $S\colon\Sigma^n\to\Sigma^n$ polynomieller Größe abhängig von $n$, und ein $u\in\Sigma^n$ mit $u<S(u)$. 
            %(Der Schaltkreis $S$ induziert einen gerichteten Graphen auf den Knoten $\Sigma^n$, d.i. $S(v)$ ist einziger Nachfolger von $v$. Dabei entspricht $v$ interpretiert als Zahl dem Gewicht von $v$.)
        \item[Gesucht:] Knoten $v\in\Sigma^n$ sodass $v<S(v)\not < S(S(v))$ gilt.
            %(D.h. gesucht ist ein Knoten $S(v)$ welcher höheres Gewicht als sein Nachfolger hat. Dieser existiert immer, denn es existieren nur endlich viele Knoten, bzw. ist Gewicht nach oben beschränkt.)
    \end{description}
\end{quote}
Die Selbstreduktion macht dabei nur Orakelfragen mit kleineren Schaltkreisen $S'\colon\allowbreak\Sigma^{n-1}\to\Sigma^{n-1}$ (die auch eine kürzere Repräsentation haben) und Startknoten $u\in \Sigma^{n-1}$.

Für natürliche  TFNP-Probleme ist offen, welche davon nach unten selbstreduzierbar sind. 
\citeauthor{harsha_downward_2023} fragen explizit danach, ob z.B. die Suche nach einem maximalen Schnitt in der Flip-Umgebung auch nach unten abgeschlossen ist.
%\[ \begin{split} \mathtt{rITERWITHSOURCE} = \{ ((S, s), v) \mid &\text{$S\colon \Sigma^n\to\Sigma^n$ ist ein Schaltkreis,   \]


%Ein Beispiel für eine nach unten selbstreduzierbares TFNP-Problem ist das bereits oben ausgeführte Problem des lokal größten Schnitts.
%Der formale Beweis wird hier ausgelassen, aber im Wesentlichen reicht es aus, sich zunächst auf Graphen $G$ zu beschränken, die Grad $>1$ haben. Wähle einen Knoten $v$, und berechne einen lokal größten Schnitt $(U,W)$ auf der kleineren Instanz $G-v$. Einer der Schnitte $(U\cup\{v\}, W)$ oder $(U, W\cup\{v\})$ ist dann ein lokal größter Schnitt (das ist der Kern vom Beweis), und dieser kann nach Definition in Polynomialzeit erkannt werden.
%Wir definieren hierfür das Problem zunächst formal:
%Sei $d_\mathrm{H}(w, w')$ die Hammingdistanz zwischen zwei Wörtern $w, w'\in \Sigma^n$. Gegeben einen Graphen $G=(V,E)$ mit Knotenmenge $V=\{0, 1, \ldots, n-1\}$ soll ein Wort $w\in\Sigma^n$ die Zuordunung der Knoten zu der jeweiligen Partition markieren. Entsprechend definieren wie das Gewicht vom Schnitt als 
%\[ \mathrm{val}_G(w) = \sum_{u,v\in E} |w[u] - w[v]|. \]
%Ein Schnitt hatten wir als lokal maximal verstanden, wenn kein Flip zu einem höhergewichtigem Schnitt führt. Auf Wörtern bedeutet das also, dass der Schnitt $w$ lokal maximal ist, wenn kein Schnitt $w'$ mit Hammingdistanz 1 ein höhergewichtiger Schnitt ist.
%Damit lässt sich das eigentliche Suchproblem über folgende NP-Relation definieren:
%\[ \begin{split} \mathtt{rLOCALMAXCUT} = \{ (G, w) \mid \,&\text{$G$ ist ein Graph mit Knotenmenge $\{0,\dots, n-1\}$, $w\in\Sigma^n$},\\
%& \mathrm{val}_G(w) \geq \mathrm{val}_G(w') \text{ für alle $w'\in\Sigma^n, d_\mathrm{H}(w, w')=1$} \} \end{split} \]


Zumindest im Bezug auf die Faktorisierung zeigen \citeauthor{harsha_downward_2023}, dass diese wahrscheinlich nicht nach unten selbstreduzierbar ist.
\begin{theorem}[{\cite{harsha_downward_2023}}]
    Die NP-Relation $\mathtt{rFACTOR}'\in\TFNP$ ist nicht nach unten selbstreduzierbar, außer $\mathtt{rFACTOR}'\in\mathrm{PLS}$.
\end{theorem}
Es ist offen ob $\mathtt{rFACTOR}'\stackrel{\smash{\text{\tiny ?}}}{\in}\mathrm{PLS}$ und zumindest unplausibel, weil unklar ist wie Faktorisierung als Suche nach einem lokalen Optimum repräsentiert werden kann.
Tatsächlich zeigen die Autorinnen sogar die stärkere Konsequenz $\mathtt{rFACTOR}'\in\mathrm{UEOPL}$, was auch noch $\mathtt{rFACTOR}'\in\mathrm{PPAD}$ zur Folge hätte. Auch die Frage, ob $\mathtt{rFACTOR}'\stackrel{\smash{\text{\tiny ?}}}{\in}\mathrm{PPAD}$ gilt, ist offen und wurde breit untersucht. Eine positive Antwort wäre daher sehr überraschend (vgl. \cite[67:15]{harsha_downward_2023}; siehe ebd. auch für eine Def. von $\mathrm{UEOPL}$, $\mathrm{PPAD}$).\label{page:self-reducibility}
Insgesamt ist die Forschung bezüglich Selbstreduzierbarkeit für Suchprobleme, sowohl TFNP-Probleme als auch FNP-Probleme, sehr klein. \citeauthor{harsha_downward_2023} stellen fest: „\foreignlanguage{english}{Almost all the study of downward self-reducibility, to date, has been focused in the decisional landscape}“ \parencite*{harsha_downward_2023}. 
Es bedarf auf jeden Fall weiterer Untersuchungen, unter anderem auch im Richtung einer Konzeptualisierung von „kleinerer Instanz“, die robuster als „kürzerer String“ ist. Hier könnte eine Konzeptualisierung wie bei der \emph{disjunktiver Selbstreduzierbarkeit} auf Entscheidungsproblemen \parencites(vgl.)(){meyer_frequency_1979}{balcazar_self_1989}{selman_natural_1988}[Abschn. 9.5]{wechsung_vorlesungen_2000} produktiv gemacht werden, welche die Ordnungsrelation „kürzer“ über eine beliebige polynomiell wohlfundierte und längenbeschränkte Halbordnung auf den Wörtern verallgemeinert.

\section{Levin-Reduzierbarkeit}\label{sec:levin}

Ähnlich wie auf den üblichen Entscheidungsproblemen können wir auch von Reduzierbarkeiten zwischen verschiedenen Suchproblemen sprechen. In der Literatur hat sich folgender Begriff von Reduzierbarkeit zwischen NP-Relationen als Analog zur Many-one-Reduktion herausgebildet \parencites(vgl.)()[229]{papadimitriou_computational_1994}[61]{goldreich_computational_2008}[50]{arora_computational_2009}:
\begin{marginfigure}[-4cm]
    \centering\includegraphics[page=11]{figures.pdf}\vspace*{2ex}
    \caption{Schematische Skizze einer Levin-Reduktion $Q\leqlp R$ über Reduktionsfunktion $f$ und Translationsfunktion $g$. Beachte, wie $f$ Instanzen ohne Lösungen in $R$ zu Instanzen ohne Lösungen in $Q$ reduziert.}
\end{marginfigure}

\begin{definition}[{Levin-Reduzierbarkeit\protect\footnotemark}]\label{def:levin-reduction}
    Seien $Q, R$ zwei NP-Relationen. Wir sagen dass sich \emph{$Q$ auf $R$ (Polynomialzeit-)Levin-reduzieren lässt}, bzw. $Q\leqlp R$, wenn zwei Funktionen $f:\Sigma^*\to\Sigma^*$, $g:\Sigma^*\times\Sigma^*\to\Sigma^*$, $f, g\in \FP$ existieren sodass
    \begin{enumerate}
        \item $x\in\Proj(Q) \iff f(x)\in\Proj(R)$, und
        \item $(f(x), y)\in R \implies (x, g(x,y))\in Q$.
    \end{enumerate}
    Punkt (1) sagt also nur aus, dass $f$ eine Many-one-Polynomialzeit-Reduktion zwischen den entsprechenden Entscheidungsproblemen ist.
    Punkt (2) sagt nun aus, dass wenn $y$ ein $R$-Zertifikat für die Instanz $f(x)$ ist, dann lässt sich aus $y$ wieder ein $Q$-Zertifikat $g(x,y)$ für die originale Instanz $x$ berechnen.

    Die Funktion $f$ nennen wir \emph{Reduktionsfunktion}, die Funktion $g$ nennen wir \emph{Translationsfunktion}.

    Wir schreiben $Q\leq_\mathrm{L,1}^\mathrm p R$ falls $f$ zusätzlich injektiv ist. Wir schreiben $Q\leq_\mathrm{L,1,i}^\mathrm p R$ falls $f$ zusätzlich injektiv und $\P$-invertierbar ist. Klar ist:
    \[ Q\leq_\mathrm{L,1,i}^\mathrm p R \implies Q \leq_\mathrm{L,1}^\mathrm p R \implies Q \leqlp R. \qedhere\]
\end{definition}
\footnotetext[-4cm]{Die Bezeichnung \emph{Levin}-Reduktion ist hier in Anlehnung an bisherige Verwendung gewählt, und bezieht sich darauf, dass in der Etablierung der NP-Vollständigkeit durch \textcite{karp_reducibility_1972}, \textcite{cook_complexity_1971} und \textcite{levin_universal_1973} gerade Levin die Suchprobleme in den Blick genommen hat, während Karp und Cook sich auf Entscheidungsprobleme konzentriert haben. Die Formalisierung von NP-Suchproblemen durch NP-Relationen (Definition~\ref{def:np-relation}) findet sich in Grundzügen schon in Levins Präsentation. Es sei aber darauf hingewiesen, dass sich die hier genannte Definition der Levin-Reduzierbarkeit (Definition~\ref{def:levin-reduction}) eine schwächere Form der Reduzierbarkeit ist als die eigentliche von Levin vorgeschlagene. Die hier genannte Definition ist jedoch hinreichend für alle relevanten Eigenschaften, sowie für die Aussagen aus Levins eigener Publikation.}

Beachte dass $\leqlp$-Reduktionen eine Verstärkung von $\leqmp$-Reduktionen auf den jeweiligen Projektionen darstellt:
\begin{observation}
    Seien $Q$ und $R$ zwei NP-Relationen.
    Wenn $Q\leqlp R$ dann gilt $\Proj(Q)\leqmp \Proj(R)$.
\end{observation}


Die Relationen $\leqlp$, $\leq_\mathrm{L,1}^\mathrm p$ und $\leq_\mathrm{L,1,i}^\mathrm p$ sind reflexiv und transitiv, bilden also eine Quasiordnung.
%Intuitiv formt die Levin-Reduktion $\leqlp$ auf den Suchproblemen das Analog der Many-one-Reduktion $\leqmp$ auf den Entscheidungsproblemen.
% 
Genau so wie wir es bei der üblichen $\leqmp$-Reduktion auf den Suchproblemen gewohnt sind, ordnet $\leqlp$ die Suchprobleme der NP-Relationen nach ihrer „Schwierigkeit“: wenn $Q\leqlp R$ dann ist $Q$ höchstens so „schwer“ wie $R$; gegeben einen Lösungsalgorithmus für $R$ lässt sich auch $Q$ effizient lösen (und das sogar mit nur einer einzigen Anfrage an den Lösungsalgorithmus). Damit folgt: wenn das Suchproblem zu $R$ effizient gelöst werden kann, dann kann auch das Suchproblem zu $Q$ gelöst werden. 
Formal ausgedrückt, ist $\FP$ nach unten abgeschlossen unter der $\leqlp$-Ordnung:

\begin{lemma}
    Seien $Q$ und $R$ zwei NP-Relationen.
    Wenn $Q\leqlp R$ und $R\inc\FP$ dann ist ist $Q\inc \FP$.
\end{lemma}
\begin{proof}
    Seien $f,g$ die Reduktions- bzw. Translationsfunktion, welche $Q\leqlp R$ realisieren, und sei $r\in\FP$ eine Verfeinerung von $R$.
    Definiere nun
    \[ q(x) \defeq \begin{cases} g(x, r(f(x))) & \text{falls $r(f(x))\neq \bot$} \\ \bot & \text{sonst}.\end{cases} \]
    Offenbar ist $q\in\FP$. Wir zeigen nun dass $q$ eine Verfeinerung von $Q$ ist.
    Zum einen gilt $\dom(q)=\Proj(Q)$:
    \begin{gather*} x\in\Proj(Q) \iff f(x)\in\Proj(R) \iff f(x)\in\dom(r) \\
    \iff r(f(x))\neq\bot \iff \iff q(x)\neq\bot \iff x\in\dom(q). \end{gather*}
    Hierbei folgt die erste Äquivalenz nach Definition~\ref{def:levin-reduction}(1).
    Zum anderen haben wir
    \begin{gather*}
        x\in\Proj(Q) \implies f(x)\in\Proj(R) \implies (f(x), r(f(x)))\in R\\ \implies (x, g(x, r(f(x))))\in Q \implies q(x)\in\fset{}Q(x),
    \end{gather*}
    wobei dritte Implikation nach Definition~\ref{def:levin-reduction}(2) folgt. Also ist $q(x)$ ein $R$-Zertifikat für $x$, wie gewünscht.
\end{proof}

\begin{marginfigure}
    \centering\includegraphics[page=10]{figures.pdf}\vspace*{2ex}
    \caption{Skizze, wie eine Levin-Reduktion $Q\leqlp R$ von NP-Relation $Q$ auf NP-Relation $R$ genutzt werden kann, einen Algorithmus für $Q$ mithilfe eines Algorithmus für $R$ anzugeben. Hierbei ist $f$ die Reduktions- und $g$ die Translationsfunktion.}
\end{marginfigure}

Genau so wie bei der Many-one-Reduktion auf den Suchproblemen können wir nach größten Elementen auf der $\leqlp$-Ordnung fragen.

\begin{definition}
    Sei $\mathcal F$ eine Klasse von Multifunktionen (z.B. $\FNP$ oder $\mathrm{TFNP}$), und sei $R\in\mathcal F$.
    Wir nennen $R\in\mathcal F$ \emph{$\leqlp$-vollständig für $\mathcal F$} wenn $R$ ein größtes Element von $\mathcal F$ geordnet über $\leqlp$ ist:
    Für alle $Q\in\mathcal F$ gilt $Q\leqlp R$.

    Die $\leq_\mathrm{L,1}^\mathrm p$- und $\leq_\mathrm{L,1,i}^\mathrm p$-Vollständigkeit ist analog definiert.
\end{definition}
Wie schon bei $\leqmp$-Reduktionen geschehen, lassen wir die Angabe der Klasse $\mathcal F$ gelegentlich auch weg, wenn $\mathcal F$ klar aus dem Kontext hervorgeht.

Es existiert eine $\leqlp$-vollständige NP-Relation für $\FNP$. Diese ist im Wesentlichen die natürliche Erweiterung der kanonischen $\leqmp$-vollständigen Menge $\mathtt{KAN}$:
\begin{definition}
\[ \begin{split} \mathtt{rKAN} \defeq \{ ((N,x,1^n), \alpha) \mid\, &\text{$N$ ist NPTM,}\\ &\text{$\alpha$ ist ein akz. Rechenweg auf $N(x)$ mit $\leq n$ Schritten} \}.\qedhere\end{split} \]
\end{definition}
Beachte dass $\Proj(\mathtt{rKAN})=\mathtt{KAN}$.
Über die Identifikation von NP-Relationen mit NPTMs (Beobachtung~\ref{obs:np-certificate-def}) ist klar, dass $\mathtt{rKAN}$ vollständig ist.
\begin{theorem}
    Die kanonische NP-Relation $\mathtt{rKAN}$ 
    ist $\leq_\mathrm{L,1,i}^\mathrm p$-vollständig für $\FNP$.
\end{theorem}
% \begin{proof}
%     Es ist leicht zu sehen dass $\mathtt{rKAN}\in\P$, da „$\alpha$ ist Rechenweg für $N(x)$“ in Polynomialzeit (abh. von $|\alpha|, |N|, |x|, n$) verifiziert werden kann. Um zu sehen, dass  $\mathtt{rKAN}$ zu polynomiell längenbeschränkt ist, können wir voraussetzen, dass jeder Schritt eines Rechenwegs $\alpha$ als Konfiguration codiert ist, also Inhalt der Bänder, Kopfpositionen und Zustände. Damit ist jede Konfiguration als ein polynomiell langes Wort abhängig von $|N|$, $|x|$, $n$ beschreibbar, also auch $\alpha$ als Liste von $\leq n$ vielen Konfigurationen nur polynomiell länger als $|N|$, $|x|$, $n$.
%     Damit $\mathtt{rKAN}\in\FNP$.
% 
%     Sei $R$ eine beliebige NP-Relation mit Zertifikatsschranke $r$, heißt $(x,y)\in R\implies |y|\leq r(|x|)$. Sei $M$ die PTM welche $R$ entscheidet, mit Laufzeitschranke $p$. Sei $N$ eine NPTM welche auf Eingabe $x$ zunächst ein Zertifikat $y, |y|\leq r(|x|)$ rät, und dann testet ob $M(x,y)$ akzeptiert. Die Laufzeit von $N$ ist beschränkt auf $p(|(x,y)|)\in O(p(r(|x|)))$ (hier nutzen wir die effiziente Listencodierung von Abschnitt \ref{sec:notation} aus). Sei daher $q$ ein Polynom, welches die Laufzeit von $N$ beschränkt.
% 
%     Definiere die Reduktionsfunktion $f(x)\defeq(N, x, 1^{q(|x|)})$. Wir zeigen zunächst dass
%     \[ x\in \Proj(R)\iff f(x)\in \Proj(\mathtt{rKAN}). \]
%     Wenn $x\in\Proj(R)$, dann existiert ein $y, |y|\leq r(|x|)$ sodass $(x,y)\in R$. Dann wird auch $N(x)$ akzeptieren, nämlich auf jenem Pfad welcher $y$ rät. Es existiert also ein Rechenweg $\alpha$ mit $|\alpha|\leq q(|x|)$ sodass $N(x)$ auf $\alpha$ akzeptiert. Dann gilt aber auch $(f(x), \alpha)=((N,x,1^{q(|x|)}),\alpha)\in \mathtt{rKAN}$.
%     Die Rückrichtung $x\not\in \Proj(R)\implies f(x)\not\in\Proj(R)$ folgt analog.
%     Es ist klar, dass $f$ injektiv ist, dass $f$ Polynomialzeit-berechenbar und -invertierbar ist. 
% 
%     Es lässt sich außerdem einfach eine Translationsfunktion $g\in \FP$ angeben, die  aus $\alpha$ das entsprechende geratene Zertifikat $y$ aus $\alpha$ berechnen kann, also $g(f(x), \alpha)=y$.
% \end{proof}

Die Ordnung $\leqlp$ und dessen Vollständigkeitsbegriff verhält sich auch sonst wie bei dem Analog $\leqmp$ gewohnt.
\begin{lemma}\label{lemma:fnp-completeness}
    Sei $R$ eine $\leqlp$-vollständige NP-Relation für $\FNP$. Es gelten folgende Aussagen:
    \begin{enumerate}
        \item $\Proj(R)$ ist eine $\leqmp$-vollständige Menge für $\NP$.
        \item Wenn $Q$ eine NP-Relation ist und $R\leqlp Q$, dann ist auch $Q$ $\leqlp$-vollständig für $\FNP$.
        \item $R\inc \FP \iff \FNP \subseteqc \FP \iff \NP=\P$.
    \end{enumerate}
\end{lemma}

Es existieren auch natürliche NP-Relationen, die $\leqlp$-vollständig für $\FNP$ sind.
Das Bekannteste ist $\mathtt{rSAT}$. Zur Erinnerung:
\[ \mathtt{rSAT} = \{ (\phi, w) \mid \text{$\phi$ ist eine aussagenlogische Formel, $w$ erfüllende Belegung für $\phi$} \}. \]
Die Aussage „$\mathtt{rSAT}$ ist $\leqlp$-vollständig“ entspricht dem üblichen Cook–Levin-Satz. %, und wird hier nur wiederholt:
\begin{theorem}[Satz von Cook und Levin]
    Die NP-Relation $\mathtt{rSAT}$ ist $\leq_\mathrm{L,1,i}^\mathrm p$-vollständig für $\FNP$.
\end{theorem}
%\begin{proof}[Skizze.]
%    Wir zeigen nur, dass $\mathtt{rCSAT}=\{(C, w) \mid C$ ist SAT-Schaltkreis und $C(w)=1\}$, also Schaltkreiserfüllbarkeit, $\leq_\mathrm{L,1,i}^\mathrm p$-vollständig ist. Es ist leicht zu sehen dass $\mathtt{rCSAT}\leq_\mathrm{L,1,i}^\mathrm p \mathtt{rSAT}$ mit den gleichen Argumenten wie $\mathtt{CSAT}\leqmp \mathtt{SAT}$. Klar ist, dass $\mathtt{rCSAT}\in\FNP$.
%
%    Ein üblicher Beweis des Satzes von Cook und Levin (welcher auf der Seite der Entscheidungsprobleme operiert) zeigt als erstes, dass eine PTM $M$ und eine „Eingabegröße“ $n$ effizient als Schaltkreis $C_{M,n}$ repräsentiert werden kann, sodass $C_{M,n}(x)=1 \iff M(x)$ akzeptiert, für alle $x\in\Sigma^{n}$.
%    %(Die Formel $\phi_{M,n}$ ist hierbei das Raum-Zeit-Tableau der PTM $M$ auf Eingaben der Größe $n$, welche gegeben einer Eingabe $x$ durch eine Variablenbelegung $w$ „konsistent“ aufgefüllt werden muss.)
%
%    Sei nun $R$ eine beliebige NP-Relation mit Laufzeitschranke $q$. Ohne Beschränkung können wir in diesem Beweis annehmen, dass alle Zertifikate für $x$ bezüglich $R$ genau die Länge $q(|x|)$ haben. Dann existiert eine PTM $M$ die $R$ entscheidet. Sei $x\in\Sigma^*$ gegeben. Für geeignetes $n$ haben wir
%    \[ \forall y\in\Sigma^{q(|x|)}.\quad C_{M,n}(x,y)=1 \iff M(x,y)\text{ akz.} \iff (x,y)\in R. \]
%    Wenn wir jetzt $x$ in $C_{M,n}$ „hart verdrahten“ erhalten wir einen Schaltkreis $C_{M,n,x}$ und es gilt
%    \[ \forall y\in\Sigma^{q(|x|)}.\quad C_{M,n,x}(y)=1 \iff C_{M,n}(x,y)=1 \iff (x,y)\in R. \]
%    Und damit ist $f(x) \defeq C_{M,n,x}$, $f\in\FP$ eine Reduktionsfunktion von $R$ nach $\mathtt{rCSAT}$:
%    \begin{gather*}
%    x\in \Proj(R) \iff \exists  y\in\Sigma^{q(|x|)}.(x,y)\in R \iff \exists y\in\Sigma^{q(|x|)}.C_{M,n,x}(y)=1 \\\iff C_{M,n,x}\in\Proj(\mathtt{rCSAT}) \iff f(x)\in\Proj(\mathtt{rCSAT}). \end{gather*}
%    Eine Translationsfunktion $g$ kann auch einfach angegeben werden: wenn $(f(x), y)\in \mathtt{rCSAT}$ dann gilt $C_{M,n,x}(y)=1$ und damit  $(x,y)\in R$.
%
%    Es ist leicht zu sehen dass $f$ injektiv ist, denn wenn die Schaltkreise $f(x), f(x')$ identisch sind, dann muss auch $x=x'$. Genauso ist klar, dass aus $f(x)=C_{M,n,x}$ einfach das „hineincodierte“ $x$ wieder ausgelesen werden kann, und damit ist $f$ auch $\P$-invertierbar.
%\end{proof}

Andererseits kann gefragt werden, ob TFNP-Relationen existieren, die $\leqlp$-vollständig für $\TFNP$ sind. In der Literatur \parencite[vgl.][]{pudlak_incompleteness_2017} wird hierauf eine negative Antwort vermutet:
\begin{conjecture}[\hTFNP]
    Es existiert keine NP-Relation $R\in\TFNP$ die $\leqlp$-vollständig für $\TFNP$ ist.
\end{conjecture}
Auch hier ist ein Beweis für diese Vermutung mindestens so schwer wie ein Beweis für $\NP\neq\coNP$. Die Beziehung dieser Vermutung mit weiteren Vermutungen betreffend Promise-Probelemen wird in Kapitel~\ref{chap:pudlak} erarbeitet.

Bevor wir uns der Beziehung zwischen Karp- und Levin-Reduktion nähern, sei an dieser Stelle noch kurz hingeweisen, dass sich die Levin-Reduzierbarkeit auf natürliche Weise abschwächen lassen kann, so wie Many-one- auf Turing-Reduzierbarkeit. Intuitiv ist mit einer solchen Turing-Reduktion von NP-Suchproblem $Q$ auf $R$ gemeint, dass sich $Q$ effizient lösen lassen kann, wenn ein effizienter Algorithmus für eine Verfeinerung von $R$ existiert. Das lässt sich zum Beispiel formal über die Idee der funktionalen Orakel umsetzen, ähnlich wie bei der Selbstreduzierbarkeit nach unten (S.~\pageref{page:self-reducibility}). Insbesondere dürfen hierbei mehrere Orakelfragen gestellt werden. (Eine Beschränkung auf höchstens eine Orakelfrage wäre äquivalent zur Levin-Reduzierbarkeit.) 

Klar ist, dass diese Turing-Reduzierbarkeit auf Suchproblemen die Lösbarkeit nach unten überträgt. Zum Beispiel lässt sich $\mathtt{rFACTORIZATION}$ auf $\mathtt{rFACTOR}$ Turing-reduzieren: gegeben $n$, frage nach einem Primfaktor $p_1$ von $n$, frage dann nach einem Primfaktor $p_2$ von $n/p_1$, usw. Damit folgt
$\mathtt{rFACTOR}\inc \FP \implies \mathtt{rFACTORIZATION}\inc\FP$, womit dann auch die angegebenen Äquivalenzen auf S.~\pageref{page:lösbarkeit} gelten. 

\subsection*{Karp- vs. Levin-Reduktionen}

Die bekannten NP-Relationen $R$ mit $\leqmp$-vollständiger Projektion $\Proj(R)\in\NP$ sind auch Levin-vollständig. Typische Präsentationen von $\leqmp$-Vollständigkeit erfolgten über die $\leqmp$-Reduktion von einer bereits bekannten NP-vollständige Menge, z.B. $\mathtt{SAT}\leqmp \mathtt{CLIQUE}$. Diese Beweise geben uns üblicherweise nicht nur eine $\leqmp$-Reduktionsfunktion $f$ von Instanzen $x$ der einen Menge zu Instanzen $f(x)$ der anderen Menge, sondern beinhalten meist im Beweis eine (implizit mitgedachte) effiziente Übersetzung von Zertifikaten von $x$ nach $f(x)$ und umgekehrt. 
Die Reduktionsfunktion $f$ mit der Rückübersetzung der  Zertifikate für $f(x)$ nach Zertifikaten für $x$ reichen dann aus, um eine $\leqlp$-Reduktion zu realisieren, und damit Levin-Vollständigkeit zu zeigen.

\begin{marginfigure}[-4cm]
    \centering\includegraphics[page=12]{figures.pdf}
\caption{Schema der Reduktion von $\mathtt{SAT}$ auf $\mathtt{CLIQUE}$, hier für die CNF-Formel $\smash{\phi=(a\lor b\lor c) \land (\overline{a} \lor\overline{b}\lor c)} \land \smash{(b\lor\overline{c} \lor d)}$. Diese wird auf eine $\mathtt{CLIQUE}$-Instanz $f(\phi)=(G, 3)$ reduziert. Die Knoten von $G$ entsprechen hierbei den Literalen der Klauseln, und genau dann inzident wenn diese in unterschiedlichen Klauseln sind, außer $x$ und $\overline{x}$. Damit hat $G$ eine Clique $C$ der Größe 3 genau dann wenn $\phi$ erfüllbar ist. Aus der Konstruktion wird klar, dass sich aus jeder solchen Clique $C$ auch eine erfüllende Belegung für $\phi$ bestimmen lassen kann, je nach dem welche Knoten bzw. Klauseln in $C$ enthalten sind.}\label{fig:sat-to-clique}
\end{marginfigure}

Beispielsweise gilt im Lehrbuch-Beweis von $\mathtt{SAT}\leqmp \mathtt{CLIQUE}$ (siehe Abbildung~\ref{fig:sat-to-clique}), dass einerseits jede erfüllende Belegung $w$ der (CNF-)Formel $\phi$ einer Clique $C$ im Graph der Instanz $f(\phi)$ entspricht, und umgekehrt lässt sich an jeder Clique $C$ eine erfüllende Belegung $w$ für $\phi$ trivial ablesen. Damit lässt sich leicht eine entsprechende Translationsfunktion $g$ angeben, sodass $f,g$ die Levin-Reduktion $\mathtt{rSAT}\leqlp\mathtt{rCLIQUE}$ auf den entsprechenden NP-Relationen realisieren. 


Nach \textcite[104]{goldreich_computational_2008} sind die folgenden NP-Relationen jedenfalls definitiv $\leq_\mathrm{L,1,i}^\mathrm p$-vollständig: $\mathtt{rSAT}$, $\mathtt{rSETCOVER}$, $\mathtt{rCLIQUE}$, $\mathtt{rVERTEXCOVER}$, $\mathtt{r3COLORABILITY}$.
Die von \textcite[193-198]{papadimitriou_computational_1994} angegebene Reduktion $\mathtt{SAT}\leqmp\mathtt{HAMCYCLE}$ lässt sich leicht zu $\mathtt{rSAT}\leqlp\mathtt{rHAMCYCLE}$ erweitern, womit $\mathtt{rHAMCYCLE}$ auch $\leqlp$-vollständig ist.
Ebenso ist $\mathtt{rANOTHERHAMCYCLE}$ auch $\leqlp$-vollständig \parencite*[232]{papadimitriou_computational_1994}. Damit gilt mit Lemma~\ref{lemma:fnp-completeness}(3) auch die auf S.~\pageref{page:natural-searchproblems} angegebene Äquivalenz „$\NP=\P$ genau dann wenn $\mathtt{rSAT}$, $\mathtt{rCLIQUE}$, $\dots$ $\inc \FP$“.

NP-Relationen $R$ für welche die Projektion $\Proj(R)$ zwar $\leqmp$-vollständig für $\NP$ ist, aber $R$ nicht $\leqlp$-vollständig für $\FNP$ sind, scheinen nicht bekannt zu sein.
Aus dieser empirischen Beobachtung ergibt sich die Frage, ob das auch für \emph{alle} NP-Relationen $R$ gilt: 
\begin{question}\label{question:kvl}
Sei $R$ eine NP-Relation. Wenn $\Proj(R)$ eine $\leqmp$-vollständige Menge für $\NP$ ist, ist dann auch $R$ eine $\leqlp$-vollständige NP-Relation für $\FNP$?
\end{question}
Aus Präsentationsgründen wollen wir die pessimistische negative Antwort auf diese Frage als Vermutung formulieren:
\begin{conjecture}[Karp-vs-Levin-Vermutung; $\mathsf{KvL}$]\label{conj:kvl}
    Es existiert eine NP-Relation $R$ sodass $\Proj(R)$ $\leqmp$-vollständig für $\NP$ ist, aber $R$ ist nicht $\leqlp$-vollständig für $\FNP$.
\end{conjecture}
Obwohl diese Frage erstaunlich auf natürlich scheint, und zwei umfassende Reduktionsbegriffe der Komplexitätstheorie in Beziehung setzen versucht, gibt es erstaunlicherweise kaum Forschung, welche sich dieser Hypothese annähert.
Ein Beweis von $\mathsf{KvL}$ ist jedenfalls mindestens so schwer wie die P-NP-Frage, denn $\mathsf{KvL}\Rightarrow \P\neq \NP$.
In Abschnitt~\ref{sec:karp-vs-levin} werden wir diese Hypothese und dessen Beziehung zu anderen Hypothesen erarbeiten. %Dort werden dann auch Argumente geliefert, die für diese negative Antwort sprechen.

%Eine natürliche Möglichkeit, die oben genannte Frage abzuschwächen, wäre die, 
Die oben genannte Frage lässt sich auf natürliche Weise abschwächen, indem man von der konkreten NP-Relation abstrahiert: Wenn $L$ eine $\leqmp$-vollständige Menge für $\NP$ ist, existiert dann zumindest \emph{eine} NP-Relation $R_L$ für $L$ (d.h. $\Proj(R_L)=L$) sodass $R_L$ $\leqlp$-vollständig ist? In anderen Worten, existiert ein hinreichend ausdrucksstarkes „Zertifikatssystem“ $R$ für $L$ sodass $R$ $\leqlp$-vollständig ist?
Diese abgeschwächte Frage lässt sich positiv beantworten, falls man die Berman–Hartmanis-Vermutung $\mathsf{IC}$ annimmt:
\begin{observation}[\cite{buhrman_functions_1998}]\label{obs:isomorphs-sind-leqlp-vollst}
    Für jede Menge $L\in \NP$ die $\P$-isomorph zu $\mathtt{SAT}$ ist, existiert eine NP-Relation $R_L$ sodass $\Proj(R_L)=L$ und $R_L$ auch $\leqlp$-vollständig für $\FNP$ ist.
\end{observation}
\begin{proof}
    Nach Voraussetzung haben wir eine bijektive $\P$-invertierbare Funktion $h\in\FP$ mit $x\in L \iff h(x) \in \mathtt{SAT}$.
    Definiere nun
    \[ R_L \defeq \{ (x,w) \mid (h(x), w)\in \mathtt{rSAT}\}. \]
    Es ist leicht zu sehen, dass $R_L$ eine NP-Relation ist. Es ist auch leicht zu sehen dass $\Proj(R_L)=L$.

    Wir zeigen nun, dass $R_L$ auch $\leqlp$-vollständig ist. Sei hierfür $Q$ eine beliebige NP-Relation. Nachdem $\mathtt{rSAT}$ ja $\leqlp$-vollständig ist, existieren Reduktions- und Translationsfunktionen $f,g$ die $Q\leqlp \mathtt{rSAT}$ realisieren.
    Definiere nun
    \[ f'(x) \defeq h^{-1}(f(x)). \]
    Insbesondere ist $h^{-1}(\cdot)$ wohldefiniert, ist ja $h$ surjektiv.
    Damit gilt zum einen für $f'$
    \[ x\in \Proj(Q) \iff f(x) \in \mathtt{SAT} \iff h(\underbrace{h^{-1}(f(x))}_{f'(x)}) \in \mathtt{SAT} \iff f'(x) \in \Proj(R_L), \]
    und zum anderen gilt
    \begin{gather*} (f'(x), w) \in R_L \implies (h(h^{-1}(f(x))), w)\in \mathtt{rSAT} \\\implies (f(x), w)\in \mathtt{rSAT} \implies (x, g(x, w))\in Q.  \end{gather*}
    Damit erfüllen also $f'$ und $g$ die Voraussetzungen an eine Reduktions- bzw. Translationsfunktion und $Q\leqlp R_L$, wie gewünscht.
\end{proof}
(Es ist leicht zu sehen, dass diese Aussage relativiert, wenn anstelle $\mathtt{rSAT}$ eine andere beliebige $\leqlp$-vollständige Relation $R$ gewählt wird.)

Damit haben (im unrelativierten Fall) insbesondere alle \emph{bekannten} NP-vollständigen Mengen, d.h. die zu $\mathtt{SAT}$ $\P$-isomorphen Mengen, eine entsprechende $\leqlp$-vollständige NP-Relation.
Es muss aber gleichzeitig darauf hingewiesen werden, dass die Zertifikate in den entsprechenden Relationen $R_L$ nicht natürlich sind; die Zertifikate sind nur Belegungen für die Formeln $h(x)$ und haben an sich keinen Bezug zur Interpretierbarkeit gegenüber der Instanz $x$.

%Abschließend wollen wir noch Vollständigkeit für $\TFNP$ diskutieren. Zum einen ist klar, dass unter dem hier vorgeschlagenen Reduktionsbegriff keine Reduktion von einer nicht-totalem NP-Relation auf eine (totale) TFNP-Relation möglich ist: 
%\[ R \not\leq_{\mathrm L}^{\mathrm p} Q, \text{ für alle } Q\in\TFNP, R\in\FNP, \Proj(R)\neq\Sigma^*, \]
%denn jede potentielle Reduktion würde Definition~\ref{def:levin-reduction}(1), also  die Many-one-Reduktion auf den jeweiligen Projektionen, verletzen.
%
%Andererseits kann gefragt werden, ob TFNP-Relationen existieren die $\leqlp$-vollständig für $\TFNP$ sind. In der Literatur \parencite[vgl.][]{pudlak_incompleteness_2017} wird hierauf eine negative Antwort vermutet:
%\begin{conjecture}[\hTFNP]
%    Es existiert keine NP-Relation $R\in\TFNP$ die $\leqlp$-vollständig für $\TFNP$ ist.
%\end{conjecture}
%Auch hier ist ein Beweis für diese Vermutung mindestens so schwer wie ein Beweis für $\NP\neq\coNP$. Die Beziehung dieser Vermutung mit weiteren Vermutungen betreffend Promise-Probelemen wird in Kapitel~\ref{chap:pudlak} erarbeitet.

\subsection*{Sparsame Reduktionen}

Wir wollen hier noch auf den Begriff der \emph{sparsamen („parsimonious“)} Reduktionen eingehen. Diese ist einerseits verwandt mit der oben betrachteten Levin-Reduktion, andererseits aber wesentlich breiter bekannt und untersucht, insbesondere aufgrund der Relevanz von sparsamen Reduktion in der Komplexitätstheorie des Zählens.

%Kurz zur Entwicklung: aus den Erfahrungen in den Entdeckungen von NP-vollständigen Mengen bzw. der Entwicklung der hierfür notwendigen Reduktionen unter NP-Entscheidungsproblemen wurde intuitiv deutlich, dass die NP-vollständigen Mengen sich viele nicht-offensichtliche Eigenschaften teilen, die über „Equi-Lösbarkeit“ ($A\in \P$ genau dann wenn $B\in \P$ für $\leq_\mathrm T^\mathrm p$-vollständige Mengen $A,B$) hinaus geht.
%Das beginnt schon bei der üblichen Many-one-Reduktion. Hier wird auf die fundamental ähnliche Struktur von zwei $\leqmp$-äquivalenten Mengen $A, B$ hingewiesen: eine Instanz $x$ von $A$ kann als ein „äquivalenter“ Fall $f(x)$ von $B$ repräsentiert werden.

%Hinzu kommt die Beobachtung, dass unter den Mengen $L\in\NP$ bisher jeder Beweise der $\leqmp$-Vollständigkeit von $L$ zu Beweisen der $\leq_\mathrm{1,i}$-Vollständigkeit verstärkt werden konnte.
%Dies führte schlussendlich zur (Berman–Hartmanis-\nolinebreak)\linebreak[1]Isomorphievermutung $\mathsf{IC}$, in der postuliert wird, dass es im Wesentlichen nur \emph{eine} NP-vollständige Menge gibt, und die verschiedenen Ausprägungen unterschiedlicher NP-vollständiger Mengen nur triviale Umcodierungen des selben Problems sind.

%Obwohl die Forschung zu NP-Suchproblemen im Vergleich zu den entsprechenden Entscheidungsproblemen im Hintergrund blieb, wurde in der Forschung aufgrund der oben genannten Erfahrungen und Intuitionen die Beobachtung gemacht, dass viele der entsprechenden Suchprobleme eine inhärente strukturelle Ähnlichkeit untereinander haben \parencite[vgl. auch die Diskussion von][]{hemaspaandra_take-home_1998}.

Aus den Erfahrungen in der Entwicklung der NP-Vollständigkeit von Entscheidungsproblmen wurde schnell deutlich, dass die NP-vollständigen Mengen sich viele nicht-offensichtliche Eigenschaften teilen. 
In stärkster Ausprägung ist das z.B. die (Berman–Hartmanis-\nolinebreak)\linebreak[1]Isomorphievermutung $\mathsf{IC}$, in der postuliert wird, dass es im Wesentlichen nur \emph{eine} NP-vollständige Menge gibt, und die verschiedenen Ausprägungen unterschiedlicher NP-vollständiger Mengen nur triviale Umcodierungen des selben Problems sind.
Diese Forschung betrachtete aber insbesondere primär die NP-Entscheidungsprobleme, und nicht NP-Suchprobleme.
Nichtsdestotrotz wurde intuitiv aber die Beobachtung gemacht, dass auch viele der entsprechenden \emph{Suchprobleme} eine inhärente strukturelle Ähnlichkeit untereinander haben, auch wenn diese oft nicht explizit mitgedacht wurden \parencite[vgl. auch die Diskussion von][]{hemaspaandra_take-home_1998}.

Hieraus entwickelte sich u.a. der Begriff der sparsamen Reduktionen.
\textcite[83]{simon_central_1975} machte beispielsweise die Beobachtung, dass die ihm bekannten Reduktionsfunktionen $f\colon A\to B$ in den Beweisen zur NP-Vollständigkeit so gebaut sind, dass die Instanz $x$ genau $k$ „Lösungen“ bezüglich $A$ hat genau dann wenn $f(x)$ genau $k$ „Lösungen“ bezüglich $B$ hat. „Lösung“ hier in Anführungszeichen weil auf Mengen überhaupt kein Begriff von Lösungen bzw. Zertifikaten existiert; \citeauthor{simon_central_1975} dachte in seinen Überlegungen die zugrunde liegende kombinatorischen (Such-)Probleme zu $A$ und $B$ nur unausgesprochen mit.

Auf NP-Relationen lässt sich sein Reduktionsbegriff aber formal präzise formulieren:
\begin{definition}[Sparsame  Reduktionen]
    Seien $Q, R$ NP-Relationen. Wir sagen dass sich $Q$ auf $R$ (in Polynomialzeit) \emph{sparsam} (\emph{„parsimonious“}) reduzieren lässt, bzw. $Q\leq_\mathrm{pars}^\mathrm p R$ wenn eine Funktion $f\in\FP$ existiert mit
    \[ |\fset{}Q(x)|=|\fset{}R(f(x))|. \qedhere \]
\end{definition}

Beachte, dass sparsame Reduktionen eine Many-one-Reduktion realisieren: wir haben 
\[ x\in\Proj(Q) \iff |\fset{}Q(x)|>0 \iff |\fset{}R(f(x))|>0 \iff f(x)\in\Proj(R). \]
Sparsame Reduktionen wurden insbesondere in der Komplexitätstheorie des Zählens aufgegriffen \parencites{simon_central_1975}{valiant_complexity_1979}. Typische algorithmische Probleme sind z.B. „\emph{wie viele Belegungen $w$ erfüllen die aussagenlogische Formel $\phi$}“ oder, kanonischer, „\emph{Auf wie vielen Rechenwegen akzeptiert die Berechnung $N(x)$ der NPTM $N$?}“. Es ist einfach zu sehen, dass sich sparsame Reduktionen $\mathtt{rSAT}\leq_\mathrm{pars}^\mathrm p \mathtt{rKAN}$ und $\mathtt{rKAN}\leq_\mathrm{pars}^\mathrm p \mathtt{rSAT}$ angeben lassen können. Damit kann das Zählproblem zur $\mathtt{rSAT}$-Instanz $x$ als $\mathtt{rKAN}$-Instanz $f(x)$ repräsentiert werden und umgekehrt – die beiden Zählprobleme sind relativ zum jeweils anderem gleich schwer. Tatsächlich sind $\mathtt{rSAT}$ und $\mathtt{rKAN}$ sogar $\leq_\mathrm{pars}^\mathrm p$-vollständig für $\FNP$. Auf eine weitere Präsentation der Komplexitätstheorie des Zählens muss hier aber verzichtet werden \parencites(siehe)()[Kap.~7]{wechsung_vorlesungen_2000}[Chap.~17]{arora_computational_2009}.

Beachte, dass sparsame Reduktionen nicht mit Levin-Reduktionen vergleichbar sind. Levin-Reduktionen erhalten im Allgemeinen nicht die Anzahl an Zertifikaten, während umgekehrt sparsame Reduktionen keine effektive Übersetzung zwischen den Zertifikaten für $f(x)$ auf Zertifikate für $x$ zulassen.

An dieser Stelle sei noch auf zwei weitere verwandte Arbeiten verwiesen, die hier nur kurz skizziert werden sollen: zum einen gehen \textcite{wiedermann_witness-isomorphic_1995} noch einen Schritt über sparsame Reduktionen hinaus, und erforschen \emph{„witness-isomorphic reductions“} zwischen NP-Relationen. Hier erhält die Reduktionsfunktion von den $A$-Instanzen auf die $B$-Instanzen nicht nur die \emph{Anzahl} der Zertifikate, sondern es werden zusätzlich die $A$-Zertifikate für $x\in\Proj(A)$ mit den $B$-Zertifikaten für $f(x)\in \Proj(B)$ in eine effiziente in Polynomialzeit berechenbare und invertierbare Eins-zu-Eins-Korrespondenz gesetzt. Damit sind sind \emph{witness-isomorphic reductions} insbesondere eine Verstärkung von Levin-Reduktionen und sparsamen Reduktionen.

\begin{figure}[tb]
    \centering\includegraphics[page=2]{figures.pdf}
    \caption{Implikationen zwischen den  Vollständigkeitsbegriffen, wobei $R$ eine beliebige aber feste NP-Relation ist. Ein unterbrochener Pfeile von $\mathsf{A}$ nach $\mathsf{B}$ sagt aus, dass ein Gegenbeispiel $Q$ für die Implikation $\mathsf{A\Rightarrow B}$ existiert, also eine NP-Relation $Q$ die $\mathsf{A}$ erfüllt und gleichzeitig $\neg\mathsf{B}$ erfüllt.}\label{fig:reduktionsbegriffe}
    \forcerectofloat
\end{figure}

Zum anderen geben \textcite{agrawal_universal_1992} strukturelle Kriterien an, die zur $\leqlp$-Vollständigkeit von NP-Relationen $R$ ausreichen. Diese Kritieren lauten intuitiv, dass sich bezogen auf das Suchproblem $R$ gewisse „Gadgets“ mit geeigneten Eigenschaften konstruieren lassen können, wie wir sie auch aus einigen NP-Vollständigkeitsbeweisen kennnen. \citeauthor{agrawal_universal_1992} nennen dann $R$ \emph{universell}. 
Rein aus diesen \emph{strukturellen} Eigenschaften lässt sich dann nachweisen, dass Universalität von $R$ hinreichend für $\leqlp$-Vollständigkeit von $R$ ist. 
\textcite{agrawal_universal_1992} überprüfen insbesondere die Universalität den ihnen bekannten vollständigen Entscheidungsproblemen, und können so in einer uniformen Weise die $\leqlp$-Vollständigkeit vieler natürlicher NP-Suchbrobleme ableiten.

Nun wollen wir abschließend noch die einzelnen Vollständigkeitsbegriffe in Beziehung setzen.
Zusammen mit Beobachtung~\ref{obs:isomorphs-sind-leqlp-vollst} sehen wir bereits die eingezeichneten Implikationen aus Abbildung~\ref{fig:reduktionsbegriffe}.

Nun zu den eingezeichneten Trennungen: zunächst halten wir erstens fest, dass die $\leqlp$-Vollständigkeit eine Eigenschaft ist, welche sogar bezüglich Problemen gilt, die mutmaßlich nicht $\P$-isomorph sind.
Angenommen, es existiert eine Einwegfunktion $f\in\FP$, das heißt $f$ ist injektiv, aber $f$ ist nicht $\P$-invertierbar.
Unter der \emph{Encrypted Complete Set Conjecture} ($\mathsf{ECSC}$) wird die Vermutung genannt, nach der die Menge
\[ f(\mathtt{SAT}) \defeq \{ f(\phi) \mid \phi\in\mathtt{SAT} \}\in\NP \]
nicht paddable ist, damit also auch nicht $\P$-isomorph zu $\mathtt{SAT}$ ist.
Gleichzeitig ist $\mathtt{SAT}\leqmp f(\mathtt{SAT})$ über Reduktionsfunktion $f$, und damit $f(\mathtt{SAT})$ auch $\leqmp$-vollständig für $\NP$.
Damit ist $f(\mathtt{SAT})$, zu verstehen als eine „verschlüsselte“ Variante zu $\mathtt{SAT}$; ein vermutetes Gegenbeispiel für die Berman–Hartmanis-Isomorphievermutung $\mathsf{IC}$.
Gleichzeitig ist leicht zu sehen, dass eine entsprechende natürliche NP-Relation
\[ \mathtt{rSAT}_f \defeq \{ (z, (\phi, w)) \mid \text{$z=f(\phi)$, und $w$ ist erfüllende Belegung für $\phi$} \} \]
$\leqlp$-vollständig ist. Wir haben $\mathtt{rSAT}\leqlp\mathtt{rSAT}_f$ über Reduktionsfunktion $f$ und über Translationsfunktion $g(\phi, (\phi, w))= w$.




Nun werden wir uns auf die sparsamen Reduktionen konzentrieren. Die Suche nach einem maximalen Schnitt ist ein triviales Beispiel einer $\leqlp$-vollständigen NP-Relation, welche nicht unter sparsamen Reduktionen vollständig ist.

Zu einem Graphen $G$ mit Knotenmenge $\{0,1,\dots, n-1\}$ können wir einen \emph{Schnitt} als einen String $w\in\Sigma^n$ schreiben, wobei $V_0 \defeq \{ i \mid i<n, w[i]=0\}$ und $V_1 \defeq \{ i \mid i<n, w[i]=0\}$ den Graphen in zwei Teile partitioniert. Einem Schnitt $w$ können wir dann ein Gewicht zuordnen: die Anzahl an Kanten in $G$ die zwischen $V_0$ und $V_1$ laufen.
Sei nun
\[ \begin{split} \mathtt{rMAXCUT} \defeq \{ ((G, r), w) \mid {}&\text{$G$ ist Graph mit Knotenmenge $\{0,1,\dots,n-1\}$,} \\ &\text{und $w\in\Sigma^n$ ist ein Schnitt mit Gewicht $\geq r$} \}.\end{split} \]
 Die $\leqlp$-Vollständigkeit von lässt sich leicht aus den üblichen $\leqmp$-Reduktionen verstärken.
Wir behaupten nun dass $\mathtt{rSAT} \not\leq_\mathrm{pars}^\mathrm p \mathtt{rMAXCUT}$. Angenommen es existiert eine solche sparsame Reduktion $f$. Beachte dass die SAT-Instanz $\phi={}$„$x_1$“ genau eine erfüllende Belegung hat. Dann wäre
\[ 1=|\fset{}\mathtt{rSAT}(\phi)|=|\fset{}\mathtt{rMAXCUT}(f(\phi))|. \]
Es lässt sich aber leicht sehen, dass $|\fset{}\mathtt{rMAXCUT}(x)|$ für jede $\mathtt{rMAXCUT}$-Instanz gerade sein muss: ist $w$ Schnitt mit Gewicht $\geq r$, dann ist auch der komplementäre String $\overline{w}$ auch ein Schnitt mit Gewicht $\geq r$; die Mengen $V_0$ und $V_1$ werden einfach vertauscht.
Damit erhalten wir den Widerspruch.

An dieser Stelle muss aber kritisch hervorgehoben werden, dass dieses Gegenbeispiel auf einem kontingenten „Hütchenspielertrick“ aufbaut: Die Schnitte $w$ und $\overline{w}$ werden als unterschiedliche Zertifikate gehandhabt, \emph{repräsentieren} doch aber die \emph{identische} Partitionierung des Graphen.
Das Problem löst sich auf, wenn anstelle der naiven Formulierung von $\mathtt{rMAXCUT}$ folgende Verfeinerung gewählt wird:
\[ \begin{split} \mathtt{rMAXCUT'} \defeq \{ ((G, r), w) \mid {}&\text{$G$ ist Graph mit Knotenmenge $\{0,1,\dots,n-1\}$,} \\ &\text{$w\in\Sigma^n$ ist ein Schnitt mit Gewicht $\geq r$, und startet mit $0$.} \}.\end{split} \]
In anderen Worten, ein Schnitt für eine $\mathtt{rMAXCUT'}$-Instanz hat immer den Knoten $0\in V_0$.
Dann ist auch möglich, eine sparsame Reduktion von $\mathtt{rSAT}$ auf $\mathtt{rMAXCUT'}$ anzugeben.\marginnote{\todo{prüfen! Vgl. Garey, Johnson, Stockmeyer}}

Ein filigraneres Beispiel ist Kantenfärbung:  Wir werden zeigen dass das Problem der 4-Kantenfärbung nicht vollständig unter sparsamen Reduktionen ist, außer $\P=\NP$.

Zu einem Graphen $G$ mit Kantenmenge $\{0,1,\dots, m-1\}$ können wir eine $k$-\emph{Kantenfärbung} als String $w$ der Länge $m$ über dem Alphabet $\{1,2,\dots k\}$ darstellen, wobei Kante $j$ die Farbe $w[j]$ erhält.
Wir wollen im Folgenden die Anzahl der möglichen Kantenfärbungen \emph{als Partitionierung} zählen, und sind dabei insbesondere nicht an redundanten Lösungen interessiert, die aus reiner Permutation der Farben entsteht. Ähnlich zu $\mathtt{rMAXCUT}'$ setzen wir für eine \emph{gültige} Färbung $w$ daher voraus, dass $w$ die unter Permutationen lexikographisch kleinste Färbung ist, in dem Sinne dass keine Permutation $\pi$ auf $\{1,2,\dots,k\}$ existiert sodass $\pi(w)$ lexikographisch kleiner ist als $w$. (Beachte: wir suchen \emph{nicht} nach einer „global“ lexikographisch kleinsten Färbung von $G$.)
Definiere nun
\[ \begin{split} \mathtt{r4CHROMINDEX} \defeq \{ ((G, k), w) \mid {}&\text{$G$ ist Graph mit Kantenmenge $\{0,1,\dots,m-1\}$} \\& \text{$G$ hat maximalem Grad 4,} \\ &\text{und $w\in\{1,2, 3,4\}^m$ ist gültige Färbung mit 4 Farben} \}.\end{split} \]
\begin{theorem}[{\cite{cai_complexity_2020} nach Edward und Welsh\protect\footnotemark}]\ifbook\def\a{\footnotetext[-5cm]}\expandafter\a\else\expandafter\footnotetext\fi{Dieses Beispiel geht auf ein unpubliziertes Preprint von Edward und Welsh mit dem Titel „On the Complexity of Uniqueness Problems“ welches offenbar in den 1980ern zirkuliert ist; viele der Arbeiten aus diesem Abschnitt nehmen auf genau dieses Preprint Bezug. Tatsächlich ist überliefert, dass die Autoren über die Kantenfärbbarkeit sogar ein „Gegenbeispiel“ zur Berman–Hartmanis-Isomorphievermutung gefunden hätte. Hierbei gingen Edward und Welsh aber von einer wesentlichen stärkeren abweichenden Interpretation der Isomorphievermutung aus: neben der Isomorphie zwischen allen NP-vollständigen Entscheidungsproblemen würde ihre Interpretation der Isomorphievermutung auch eine Isomorphie auf den jeweiligen Zertifikatsmengen umfassen. Das würde (mindestens) eine sparsame Interreduzierbarkeit zwischen allen NP-vollständigen Suchproblemen implizieren. Diese Aussage ist nun aber so stark, dass diese durch eben das Beispiel der Kantenfärbbarkeit widerlegt werden kann.\par Vgl. \textcites{hemaspaandra_take-home_1998}{wiedermann_witness-isomorphic_1995}{cai_complexity_2020}[118]{welsh_complexity_1993}.}
    Die NP-Relation {\normalfont\texttt{r4CHROMINDEX}} ist nicht $\leq_\mathrm{pars}^\mathrm p$-vollständig, außer $\P=\NP$.
\end{theorem}
\begin{proof}[Skizze.]
    Sei $\chi'(G)$ die minimale Anzahl an Farben, die zur Kantenfärbung eines Graphen $G$ benötigt werden.
    \citeauthor{cai_complexity_2020} können  
    sämtliche Graphen charakterisieren, welche eine eindeutige (modulo Permutationen der Farben) 4-Kantenfärbung haben:
    %zum Ergebnis, dass die Graphen mit eindeutiger 4-Kantenfärbung (modulo Permutationen der Farben) in Polynomialzeit erkannt werden können:
    \begin{itemize}[nosep,beginpenalty=0]
        \item Unter den Graphen mit $\chi'(G)=4$ ist $K_{1,k}$ der einzige Graph mit eindeutiger 4-Kantenfärbung. (Das ist der Satz von \cite{thomason_hamiltonian_1978}.)
        \item Unter den Graphen mit $\chi'(G)=3$ sind $C_3$ und $K_{1,3}$ die einzigen Graphen mit eindeutiger 4-Kantenfärbung.
        \item Unter den Graphen mit $\chi'(G)=2$ ist $K_{1,2}$ der einzige Graph mit eindeutiger 4-Kantenfärbung.
        \item Unter den Graphen mit $\chi'(G)=1$ ist $K_{1,1}$ der einzige Graph mit eindeutiger 4-Kantenfärbung.
    \end{itemize}
    In allen Fällen können isolierte Knoten ignoriert werden. Sei hier nur der Beweis für den Fall $\chi'(G)=3$ skizziert. Sei hierfür $G$ ein solcher Graph, dann existiert also mindestens eine Kantenfärbung $C$ von $G$ mit drei Farben. Sei $C_i$ die Teilmenge der Kanten in Farbe $i$.
    Wir haben ohne Beschränkung also $C_1,C_2,C_3\neq\emptyset, C_4=\emptyset$.
    In je $C_1,C_2,C_3$ ist dann auch nur genau eine Kante enthalten, denn andernfalls könnte die zweite Kante auch in Farbe $4$ gefärbt sein; das widerspräche der eindeutigen 4-Kantenfärbung.
    Damit folgt schon mal, dass $G$ aus genau drei Kanten besteht.
    Gleichzeitig müssen alle Kanten paarweise zueinander inzident sein: wenn $e\in C_i$ nicht mit $f\in C_j$ inzident ist, könnten wir auch $e$ mit der Farbe $j$ färben; wieder Widerspruch zur eindeutigen 4-Kantenfärbung.
    Also kann $G$ nur die Form eines Kreises $C_3$ oder eines Sterns $K_{1,3}$ haben.

    Die Fälle $\chi'(G)=2$ und $\chi'(G)=1$ gehen analog.
    Insgesamt ergibt sich also, dass in Linearzeit überprüft werden, ob ein gegebener Graph $G$ eine eindeutige 4-Kantenfärbung zulässt. Sei $A\in \P$ diese Menge der eindeutig färbbaren Graphen.

    Mit diesem Fakt zeigen wir nun die Aussage.
    Angenommen, \texttt{r4CHROMINDEX} ist $\leq_\mathrm{pars}^\mathrm p$-vollständig, dann existiert auch eine sparsame Reduktion $f$ von $\mathtt{rSAT}$ auf \texttt{r4CHROMINDEX}.
    Sei $\phi$ eine beliebige SAT-Formel, in der nur die Variablen $x_1, \dots, x_n$ vorkommen.
    Wir werden nun in Polynomialzeit entscheiden ob $\phi\in \mathtt{SAT}$. 
    Definiere eine zweite SAT-Formel
    \[ \phi' \defeq (\neg y \land \phi) \lor (y\land \neg x_1 \land \neg x_2 \land\cdots\land x_n), \]
    wobei $y$ ein neues Variablensymbol ist. Es ist leicht zu sehen, dass $\phi'$ genau eine erfüllende Belegung mehr als $\phi$ hat.

    Wir haben nun
    \begin{gather*} \phi\not\in\mathtt{SAT} \iff |\fset{}\mathtt{rSAT}(\phi)|=0 \iff |\fset{}\mathtt{rSAT}(\phi')|=1 \\ \iff |\fset{}\mathtt{r4CHROMINDEX}(f(\phi'))|=1 \iff f(\phi') \in A, \end{gather*}
    und damit $\mathtt{SAT}\in\P$.
\end{proof}

\textcite{leven_np_1983} zeigen, dass die Menge $\Proj(\mathtt{r4CHROMINDEX})$ $\leqmp$-vollständig für $\NP$ ist. 
Mit den Konstruktionen aus deren Beweis ist es leicht zu sehen, dass die NP-Relation \texttt{r4CHROMINDEX} auch $\leqlp$-vollständig ist. 
Es ist auch leicht zu sehen, dass $\Proj(\mathtt{r4CHROMINDEX})$ paddable ist, also  auch $\P$-isomorph zu $\mathtt{SAT}$ ist.
Wir kommen zum Resultat:
\begin{corollary}
    Die NP-Relation $\mathtt{r4CHROMINDEX}$ ist $\leqlp$-vollständig für $\FNP$, und die zugehörige Projektion  $\Proj(\mathtt{r4CHROMINDEX})$ ist $\leqmp$-vollständig für $\NP$, und $\P$-isomorph zu $\mathtt{SAT}$.
    Diese NP-Relation ist insbesondere nicht $\leq_\mathrm{pars}^\mathrm p$-vollständig für $\FNP$, außer $\P=\NP$.
\end{corollary}

