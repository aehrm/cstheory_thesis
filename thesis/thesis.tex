%\documentclass[a4paper,DIV=8]{scrbook}

%\usepackage{fontspec}
%\usepackage{unicode-math}
%\setmainfont{TeX Gyre Pagella}
%\setmathfont{TeX Gyre Pagella Math}

%\usepackage[ngerman]{babel}
%\usepackage{blindtext}


%\begin{document}

%\blindmathpaper
%\blindmathpaper

%\end{document}

%\documentclass[11pt,twoside,BCOR=12mm,DIV=9,headings=small]{scrreprt}
\newif\ifbook
\ifbook
\documentclass[11pt,a4paper,twoside]{report}
\else
\documentclass[10pt,a4paper]{report}
\fi

\usepackage[ngerman]{babel}
\usepackage[fleqn]{mathtools}
\usepackage{fontspec}
%\setsansfont[BoldFont={Fira Sans Medium}]{Fira Sans}
\setmonofont[Scale=0.80]{JetBrains Mono Light}
\usepackage{unicode-math}
\ifbook
\setmainfont[Numbers={OldStyle}]{TeX Gyre Pagella}
\setmathfont{TeX Gyre Pagella Math}
\usepackage{setspace}
\usepackage[DIV=9]{typearea}
\fi

\setlength{\emergencystretch}{1em}

\usepackage{titlesec}		% header customization
\usepackage{titletoc}		% toc customization
\usepackage{titleps}		% toc customization
\usepackage[titles]{tocloft}% lof lot lol customizatios

%	header formatting
\ifbook
\titlespacing*{\part}{.2\linewidth}{.3\textheight}{0pt}
\titlespacing*{\chapter}{0pt}{-10pt}{25pt}
\titleformat{\part}%[hang]
	{\thispagestyle{empty}\huge\raggedleft}
	{\Large\lsstyle\MakeUppercase{\partname}~~\thepart\normalfont}
	{0em}
	{\vskip 12pt \setstretch{0.9}\huge\lsstyle\sffamily\uppercase}
% \let\oldpart\part
% \renewcommand{\part}{
% }
%\titleformat{\chapter}[block]
	%{\begin{wide}\thispagestyle{empty}\itshape\huge}
	%{\normalfont\bfseries\huge\thechapter}
	%{1em}
	%{\setstretch{0.9}\raggedright\huge\itshape}[\end{wide}]
\titleformat{\chapter}
    {\thispagestyle{empty}\itshape\huge}
	{\normalfont\bfseries\huge\thechapter}
	{1em}
	{\setstretch{0.9}\raggedright\huge\itshape}
\titleformat{\section}
	{\itshape\Large}
	{\normalfont\thesection}
	{1em}
	{\setstretch{0.9}\raggedright\itshape}
\titleformat{\subsection}
	{\itshape\normalfont}
	{\raggedright\normalfont\thesubsection}
	{1em}
	{\setstretch{0.9}\itshape}
\titleformat{\subsubsection}[runin]
	{}
	{\itshape\thesubsubsection}
	{1ex}
	{\itshape}[~~$\cdot$]

\contentsmargin[1cm]{0cm}
\titlecontents*{part}[0em]
	{\large\scshape}
	{\scshape\thecontentslabel}
	{\vskip 4pt\large\scshape\lowercase}
	{}[\vskip 0pt]
\titlecontents{chapter}[0em]
	{\large\vspace{12pt}}
	{\makebox[-1em][r]{\thecontentslabel}\large\hspace{1em}}
	{\large}
	{~~~~{\contentspage}}[\vskip 2pt]
\titlecontents*{section}[0em]
	{\small}
	{\thecontentslabel~~}
	{\small}
	{~~{\contentspage}}
	[\\]

%	toc offset
\let\oldtoc\tableofcontents
\renewcommand{\tableofcontents}{%
	\begin{quote}\setstretch{1.0}
	\oldtoc
	\end{quote}
}
%	toc depth
\AtBeginDocument{\setcounter{tocdepth}{1}}

\newpagestyle{main}{
\sethead[\thepage][][\sffamily\scshape{\chaptertitle}] % even
{\sffamily\scshape\sectiontitle}{}{\thepage}} % odd
\pagestyle{main}

\else

\titleformat{\chapter}
{\normalfont\Large\bfseries}{\thechapter}{1em}{}
\titleformat{\section}
{\normalfont\large\bfseries}{\thesection}{1em}{}

\titlespacing*{\chapter} {0pt}{3.5ex plus 1ex minus .2ex}{2.3ex plus .2ex}
\titlespacing*{\section} {0pt}{3.25ex plus 1ex minus .2ex}{1.5ex plus .2ex}

\fi



\usepackage{amsthm}
\usepackage{thmtools}
\newtheorem{theorem}{Satz}[chapter]
\newtheorem{proposition}[theorem]{Proposition}
\newtheorem{lemma}[theorem]{Lemma}
\newtheorem{claim}[theorem]{Behauptung}
\newtheorem{corollary}[theorem]{Korollar}
\newtheorem{observation}[theorem]{Beobachtung}
\newtheorem{conjecture}[theorem]{Vermutung}
\newtheorem*{remark}{Anmerkung}

%\theoremstyle{definition}
%\newtheorem{definition}[theorem]{Definition}
\declaretheorem[style=definition,qed={$\triangleleft$},sibling=theorem]{definition}
\makeatletter
\renewenvironment{proof}[1][\proofname]{\par
  \pushQED{\qed}%
  \normalfont\footnotesize \topsep6\p@\@plus6\p@\relax
  \list{}{\leftmargin=4em
          \rightmargin=\leftmargin
          \settowidth{\itemindent}{\itshape#1}%
          \labelwidth=\itemindent
          % the following line is not needed with amsart, but might be with other classes
          \parsep=0pt \listparindent=\parindent 
  }
  \item[\hskip\labelsep
        \itshape
    #1\@addpunct{.}]\ignorespaces
}{%
  \popQED\endlist\@endpefalse
}
\makeatother
\newcounter{prooflist}
\newenvironment{prooflist}
{\setcounter{prooflist}{0}\renewcommand\item{\medskip\par\noindent \stepcounter{prooflist}\arabic{prooflist}. }}
{}
\newenvironment{prooflist*}
{\renewcommand\item{\medskip\par\noindent}}
{}

\ifbook
\AtBeginDocument{\setlength{\abovedisplayskip}{5pt}\setlength{\abovedisplayshortskip}{5pt}%
\setlength{\belowdisplayskip}{3pt}\setlength{\belowdisplayshortskip}{3pt}}
\fi


\usepackage{enumitem}
\setlist[enumerate]{label=(\arabic*), itemsep=0pt}
\setlist[itemize]{itemsep=0pt}
\setlist{beginpenalty=10000, midpenalty=10000}

%\usepackage[authordate,url=false,dashed=false]{biblatex-chicago}
\usepackage[style=myauthoryear,backend=biber,url=false,mergedate=false,maxbibnames=99]{biblatex}
\addbibresource{zoterolib.bib}
\DeclareSourcemap{
  \maps[datatype=bibtex]{
     \map{
        \step[fieldsource=doi,final]
        \step[fieldset=isbn,null]
        }
      }
}
\setlength{\bibitemsep}{10pt}



\usepackage[vlined,linesnumbered,nofillcomment,german,onelanguage]{algorithm2e}
\DontPrintSemicolon
\SetArgSty{}
\SetCommentSty{itshape}
\SetKwComment{tcc}{(}{)}
\SetKw{KwOr}{\thesis@algorithmor}
\SetKw{KwAnd}{\thesis@algorithmand}
\SetKw{KwNot}{\thesis@algorithmnot}
\SetKw{Accept}{Akzeptieren}
\SetKw{Reject}{Ablehnen}
\SetKwInput{Input}{\thesis@algorithminput}
\SetKwInput{Output}{\thesis@algorithmoutput}
\setlength{\algomargin}{6ex}

\usepackage[pdfusetitle,pdfencoding=auto,psdextra,bookmarksopen]{hyperref}

\usepackage{xcolor}
\usepackage{ragged2e}
\usepackage{sidenotes}
\setlength{\marginparwidth}{90pt}
\renewcommand*{\marginfont}{\normalfont\footnotesize\sffamily}

\makeatletter
\RenewDocumentCommand\sidenotetext{ o o +m }{%
    \IfNoValueOrEmptyTF{#1}{%
        \@sidenotes@placemarginal{#2}{\noindent\marginfont\RaggedRight{\thesidenote{}.~#3}}%
        \refstepcounter{sidenote}%
    }{%
        \@sidenotes@placemarginal{#2}{\noindent\marginfont\RaggedRight{#1.~#3}}%
    }%
}
\makeatother

\newcommand{\todo}[1]{\textcolor{red}{{\sffamily TODO: #1}}}
\newcommand{\note}[1]{\textcolor{green!30!black}{{\sffamily #1}}}



\def\P{\ensuremath{\mathrm{P}}}
\def\NP{\ensuremath{\mathrm{NP}}}
\def\NE{\ensuremath{\mathrm{NE}}}
\def\NEE{\ensuremath{\mathrm{NEE}}}
\def\FP{\ensuremath{\mathrm{FP}}}
\def\UP{\ensuremath{\mathrm{UP}}}
\def\DisjNP{\ensuremath{\mathrm{DisjNP}}}
\def\DisjCoNP{\ensuremath{\mathrm{DisjCoNP}}}
\def\DisjUP{\ensuremath{\mathrm{DisjUP}}}
\def\DisjCoUP{\ensuremath{\mathrm{DisjCoUP}}}
\def\coNP{\ensuremath{\mathrm{coNP}}}
\def\coNE{\ensuremath{\mathrm{coNE}}}
\def\coNEE{\ensuremath{\mathrm{coNEE}}}
\def\coUP{\ensuremath{\mathrm{coUP}}}
\def\NPcoNP{\ensuremath{\mathrm{NP}\cap\mathrm{coNP}}}
\def\TFNP{\ensuremath{\mathrm{TFNP}}}
\def\TALLY{\ensuremath{\mathrm{TALLY}}}
\def\NPMV{\ensuremath{\mathrm{NPMV}}}
\def\NPMVt{\ensuremath{\mathrm{NPMV_t}}}
\def\NPSV{\ensuremath{\mathrm{NPSV}}}
\def\NPSVt{\ensuremath{\mathrm{NPSV_t}}}
\def\NPbV{\ensuremath{\mathrm{NPbV}}}
\def\NPbVt{\ensuremath{\mathrm{NPbV_t}}}
\def\NPkV{\ensuremath{\mathrm{NP}k\mathrm{V}}}
\def\NPkVt{\ensuremath{\mathrm{NP}k\mathrm{V_t}}}
\def\TAUT{\ensuremath{\mathrm{TAUT}}}
\def\SAT{\ensuremath{\mathrm{SAT}}}
\def\PF{\ensuremath{\mathrm{PF}}}
\DeclareMathOperator{\dom}{dom}
\DeclareMathOperator{\img}{img}
\DeclareMathOperator{\supp}{supp}
\def\hQ{\ensuremath{\mathsf{Q}}}
\def\hUP{\ensuremath{\mathsf{UP}}}
\def\hDisjNP{\ensuremath{\mathsf{DisjNP}}}
\def\hDisjCoNP{\ensuremath{\mathsf{DisjCoNP}}}
\def\hNPcoNP{\ensuremath{\mathsf{NP}{}\cap{}\mathsf{coNP}}}
\def\hTAUT{\ensuremath{\mathsf{TAUT}}}
\def\hSAT{\ensuremath{\mathsf{SAT}}}
\def\hTFNP{\ensuremath{\mathsf{TFNP}}}
\def\leqmpp{\ensuremath{\leq_\mathrm{m}^\mathrm{pp}}}
\def\leqlp{\ensuremath{\leq_\mathrm{L}^\mathrm{p}}}
\def\leqmp{\ensuremath{\leq_\mathrm{m}^\mathrm{p}}}
\DeclareMathOperator{\tower}{tower}
\DeclareMathOperator{\Proj}{Proj}
\DeclareMathOperator{\Sol}{Sol}
\def\mhyphen{\text{-}}
\def\fset{\ensuremath{\mathit{set}\mhyphen}}
\AtBeginDocument{\renewcommand{\phi}{\varphi}}
\AtBeginDocument{\renewcommand{\epsilon}{\varepsilon}}

\begin{document}
\title{Komplexität von Suchproblemen und Beweissystemen}
\author{Anton Ehrmanntraut}
\maketitle
\tableofcontents
\clearpage

\chapter{Definitionen}

\begin{itemize}
    \item Vllt. sollten wir NP-Relationen als FNP-Probleme definieren. Andererseits wird in der aktuellen Literatur (T)FNP primär auf Funktionen verstanden.
    \item NP-Relation definieren: Zertifikate mind. so lange wie Probleme, alternativ streng monoton steigend
    \item Levin-Reduktion definieren
\end{itemize}

\begin{theorem}
    Die kanonische NP-Relation
    \[ \mathtt{rKAN} = \{ ((N,x,1^n), \alpha) \mid \text{$\alpha$ ist ein akz. Rechenweg auf $N(x)$ und $|\alpha|\leq n$} \} \]
    ist $\leq_\mathrm{m,inv}^\mathrm L$-vollständig.
\end{theorem}
\begin{proof}
    Sei $R$ eine beliebige NP-Relation mit Zertifikatsschranke $r$, i.e. $(x,y)\in R\implies |y|\leq r(|x|)$. Sei $M$ die PTM welche $R$ entscheidet, mit Laufzeitschranke $p$. Sei $N$ eine NPTM welche auf Eingabe $x$ zunächst ein Zertifikat $y, |y|\leq r(|x|)$ rät, und dann testet ob $M(\langle x,y\rangle )$ akzeptiert. Die Laufzeit von $N$ ist beschränkt auf $p(|\langle x,y\rangle|)\in O(p(r(|x|))$; sei daher $q$ ein Polynom, welches die Laufzeit von $N$ beschränkt.

    Definiere die Reduktionsfunktion $f(x)=(N, x, 1^{q(|x|)})$. Es gilt $x\in \Proj(R)\implies f(x)\in \Proj(\mathtt{rKAN})$.
    Wenn $x\in\Proj(R)$, dann existiert ein $y, |y|\leq r(|x|)$ sodass $(x,y)\in R$. Dann wird auch $N(x)$ akzeptieren, nämlich auf jenem Pfad welcher $y$ rät. Es existiert also ein Rechenweg $\alpha$ mit $|\alpha|\leq q(|x|)$ sodass $N(x)$ auf $\alpha$ akzptiert. Dann gilt aber auch $(f(x), \alpha)=((N,x,1^{q(|x|)}),\alpha)\in \mathtt{rKAN}$.
    Die Rückrichtung $x\not\in \Proj(R)\implies f(x)\not\in\Proj(R)$ folgt analog.
    Es ist klar, dass $f$ Polynomialzeit-berechenbar und -invertierbar ist. 

    Es lässt sich außerdem einfach eine Rekonstruktionsfunktion $g\in \FP$ angeben, die für $g(f(x), \alpha)=y$ aus $\alpha$ das entsprechende geratene Zertifikat $y$ aus $\alpha$ berechenen kann.
\end{proof}

\chapter{Suchprobleme und die Hypothese Q}

\begin{itemize}
    \item Will Q in den Pudlák-Baum einordnen: dafür ist es notwendig, diese ordentlich zu relativieren. Insb. will ich zeigen, dass einige bisherige Resultate natürlicherweise auf „Standardbeweissysteme“ vollständiger Mengen übertragen (nicht nur das Standardbeweissystem für SAT).
\end{itemize}

\begin{definition}[Levin-Paddability]
    Eine NP-Relation $R$ ist \emph{Levin-paddable} wenn 
    Funktionen $\mathit{pad}\in\FP$ und $\mathit{padsol}\in\FP$ existieren sodass
    \begin{enumerate}
        \item $x\in \Proj(R) \iff \mathit{pad}(x, 1^n) \in \Proj(R)$,
        \item $(\mathit{pad}(x, 1^n), y)\in R \implies (x, \mathit{padsol}(x, 1^n, y)) \in R$,
        \item $|\mathit{pad}(x, 1^n)|\geq n$.
    \end{enumerate}
\end{definition}

\begin{lemma}\label{lemma:stdps-q}
    Sei $R$ eine NP-Relation die Levin-paddable ist. Folgende Aussagen sind äquivalent:
    \begin{enumerate}
        \item Für alle NPTM $N$ mit $L(N)=\Proj(R)$ lassen sich akzepierende Rechenwege von $N$ in Zertifikate umrechnen: es existiert eine Funktion $h\in\mathrm{FP}$ sodass
            \[ N(x) \text{ akz. mit RW $\alpha$} \implies (x,h(x,\alpha))\in R. \]
        \item Das Standard-Beweissystem $\mathit{std}_R$ bzgl. $R$ ist p-optimal. Dieses ist
            \[ \mathit{std}_R(w) = \begin{cases} x & \text{wenn $w=(x,y)$ und $(x,y)\in R$}\\
            \top & \text{sonst}\end{cases}. \] 
    \end{enumerate}
\end{lemma}

\begin{lemma}\label{lemma:q-generalized}
%Sei $A\in\mathrm{NP}$ mit folgender Eigenschaft von Vollständigkeit: es existiert eine Menge $B\in\mathrm P$ sodass $A=\{x\mid \exists y, |y|\leq p(|x|), (x,y)\in B\}$ und für alle Mengen $A'\in\mathrm{NP}$, $A'=\{x\mid \exists y, |y|\leq p'(|x|), (x,y)\in B'\}$ existieren zwei Funktionen $r,r^{-1},t\in\mathrm{FP}$ sodass
%\[ x\in A' \iff r(x) \in A, \quad (r(x),z)\in B \implies (x, t(x,z)) \in B'. \]
%($A'\leq_m^p A$ via invertierbarem $r$, Funktion $t$ bildet Zertifikate für $r(x)\in A$ auf Zertifikate für $x\in A'$ ab. Vgl. Reduktionsbegriff unter TFNP-Problemen. Vgl. Levin-Reduktionsbegriff.)
Sei $R$ eine $\leqml$-vollständige NP-Relation, mit der zusätzlichen Eigenschaft dass für die jeweilige entsprechende Problem-Reduktionsfunktion $f\colon Q\to R$ für $Q\leqml R$ immer gilt, dass $f$ ehrlich ist. %$\img f = \{f(x)\mid x\in\Sigma^*\}\in \P$.
Folgende Aussagen sind äquivalent:
\begin{enumerate}
    \item Für alle NPTM $N$ mit $L(N)=\Proj(R)$ lassen sich akzepierende Rechenwege von $N$ in Zertifikate umrechnen: es existiert eine Funktion $h\in\mathrm{FP}$ sodass
        \[ N(x) \text{ akz. mit RW $\alpha$} \implies (x,h(x,\alpha))\in R. \]
    \item Für alle NPTM $N$ mit $L(N)=\Sigma^*$ lassen sich aus Eingabe $x$ Rechenwege von $N(x)$ effizient bestimmen: es existiert $r\in\mathrm{FP}$ sodass $N(x)$ auf Rechenweg $r(x)$ akzeptiert. (Das ist die Aussage Q.)
\end{enumerate}
\end{lemma}


\begin{lemma}
    Die in Lemma~\ref{lemma:stdps-q} und~\ref{lemma:q-generalized} genannten Voraussetzungen an die NP-Relation $R$ werden von allen solchen $R$ erfüllt, die $\leqml$-vollständig sind und Levin-paddable sind.
\end{lemma}
\begin{proof}
    Es ist sofort klar, dass $R$ die Voraussetzungen von Lemma~\ref{lemma:stdps-q} erfüllt.
    Es bleibt nur zu zeigen, dass für jede NP-Relation $Q$ eine $\leqml$-Reduktion angegeben werden kann, bei dem die Problem-Reduktionsfunktion ehrlich ist.
    Wir nutzen hierbei aus, dass $R$ eine Levin-paddable Relation ist.

    Nachdem $R$ vollständig ist, gilt $Q\leqml R$; sei $f,g\in\FP$ die Reduktions- bzw. Rekonstruktionsfunktion welche diese Reduktion realisieren. Wir werden nun Funktionen $f', g'\in\FP$ angeben, welche die gleiche Reduktion realisieren, aber $f'$ ist nichtverkürzend, und damit ehrlich, wie gewünscht.

    Sei $\mathit{pad}, \mathit{padsol}$ die zu $R$ zugehörigen Padding-Funktionen. Definiere
    \[ f'(x) = \mathit{pad}(f(x), 1^{|x|}). \]
    Es gilt
    \[ x\in\Proj(Q) \iff f(x)\in \Proj(R) \iff \mathit{pad}(f(x), 1^{|x|})=f'(x)\in\Proj(R), \]
    wobei erste Implikation die Eigenschaft der Reduktionsfunktion $f$ ist, und die zweite aus der Definition von Levin-Paddability folgt.
    Aus der Definition von  Levin-Paddability $f'$ folgt auch $|f'(x)|\geq |x|$ und damit ist $f'$ nichtverkürzend.

    Definiere
    \[ g'(x, z) = g(x, \mathit{padsol}(f(x), 1^{|x|}, z)). \]
    Sei nun $(f'(x), z)\in R$. Die Funktion $g'$ berechnet nun ein Zertifikat $y$ für $x$: Wir haben $(\mathit{pad}(f(x), 1^{|x|}), z)\in R$, also gilt nach Levin-Paddability dass \[(f(x), \mathit{padsol}(f(x), 1^{|x|}, z))\in R,\] 
    und nach Definition der Rekonstruktionsfunktion $g$ gilt dann
    \[(x, g(x, \mathit{padsol}((f(x), 1^{|x|}, z)))\in Q,\]
    und das ist genau $(x, g'(x, z))\in Q$, wie gewünscht.
\end{proof}

\begin{observation}
    Jede universelle Relation ist Levin-paddable. Dieses Resultat gilt nur im unrelativierten Fall.
\end{observation}
\begin{proof}
    Sei $R$ eine universelle Relation, mit zugehörigem Polynom $q$, welches die Zertifikatsgröße spezifiziert. Zur Erinnerung, dieses Polynom ist streng monoton steigend, und aus $(x,y)\in R$ folgt $|y|=q(|x|)$.
    Wir zeigen zunächst, wie wir für beliebige Instanz $x$ und $n\in\mathbb N$ auf eine Instanz $x'$ der Länge $\geq n$ hochpadden.

    %Nach Satz ?? existiert ein \emph{building block} $\mathit{block}_R$ und drei Indices $b_1,b_2,b_3\in\mathbb N$ sodass 
    %\[ \{ y[b_1b_2b_3] \mid (\exists y)\, (\mathit{block}_R, y)\in R \} = \Sigma^3-\{000\}. \]
    %Damit muss auch $q(|\mathit{block}_R|)\geq 1$.

    %Ebenso nach Satz ?? existiert eine Join-Funktion $\mathit{join}_R$.
    %Sei nun
    %\[ (z,\alpha) = \mathit{join}_R(\underbrace{\mathit{block}_R, \mathit{block}_R, \mathit{block}_R, \ldots}_{\text{$n$ mal}}) \]
    %Nach Definition haben wir
    %\[ \{ y[\alpha] \mid (\exists y)\, (z,y)\in R\} = \{ y_1y_2\cdots y_n \mid (\forall k\leq n). (\mathit{block}_R, y_i) \} \]
    %Zunächst können wir festhalten, dass $|\Sol_R(x)|\leq 2^{q(|x|)}$: jedes Zertifikat zu $x$ hat Länge $q(|x|)$, also sind auch nur $2^{q(|x|)}$ viele Zertifikate für $x$ in $R$ möglich.

    %Nachdem $R$ universell ist, existiert eine zertifikatserhaltende Reduktion $f$ von $\mathtt{rKAN}$ auf $R$. 
    %Es ist leicht zu sehen, wie eine positive Instanz $w=(N, \epsilon, 1^{q(n)})$ für $\mathtt{rKAN}$ konstruiert werden kann, welche $2^{q(n)}$ viele Zertifikate hat (z.B. indem $N$ auf jedem der $2^{q(n)}$ vielen Rechenwege akzeptiert).
    %Sei $(z,\alpha)=f(w)$, d.h. $w$ ist auf eine $R$-Instanz $z$ reduziert worden, welche die $2^{q(n)}$ vielen Zertifikate von $w$ erhält.
    %In anderen Worten, wir haben nun
    %\[ \{ y[\alpha] \mid (z,y)\in R \} = \{ y \mid (w, y) \in \mathtt{rKAN} \}. \]
    %Damit sind auch die beiden Mengen gleich groß und wir haben
    %\[ |\{ y[\alpha] \mid (z,y)\in R \}| = 2^{q(n)}. \]
    %Also muss auch $|\alpha|=q(n)$ sein, und alle Indizes in $\alpha$ sind paarweise verschieden.
    %Damit müssen auch alle $y$ mit $(z,y)\in R$ eine Länge von $\geq q(n)$ habe. Damit ist ersichtlich dass $q(|z|)\geq q(n)$. Aus Monotonie folgt $|z|>n$, das heißt die 

    Nachdem $R$ universell ist, existiert eine zertifikatserhaltende Reduktion $f\in FP$ von $\mathtt{rSAT}$ auf $R$. 
    Es lässt sich leicht eine Familie $\phi_1, \phi_2, \ldots$ an positiven Instanzen für $\mathtt{rSAT}$ erdenken, für die $|\phi_i|\geq i$ und $\phi_i$ effizient auf Eingabe $1^i$ konstruiert werden kann.
    Zur Erinnerung: wir haben für $\mathtt{rSAT}$ das zugehörige Polynom der Zertifikatsgröße so gewählt, dass die Zertifikate genau so lange wie die Probleminstanzen sind. Damit gilt
    für $(\phi_i, w)\in \mathtt{rSAT}$ dann auch $|w|=|\phi_i|\geq i$.

    Sei nun $(z,\alpha)=f(\phi_{q(n)})$, d.h. $\phi_i$ ist auf eine $R$-Instanz $z$ reduziert worden. Nach Definition von zertifikatserhaltenden Reduktionen gilt $|\alpha|=|\phi_{q(n)}|\geq {q(n)}$, und $\alpha$ besteht aus paarweise verschiedenen Indizes. Wir haben nun nach Definition
    \[ \{ y'[\alpha] \mid y'\in \Sigma^{q(|z|)}, (z,y')\in R \} = \{ y\mid y\in \Sigma^{|\phi_{q(n)}|}, (\phi_{q(n)},y)\in\mathtt{rSAT}\} \neq \emptyset, \]
    in anderen Worten, aus Zertifikaten $y'$ für $z$ können wir durch geeignete Projektion via $\alpha$ die Zertifikate für $w$ rekonstruieren. Davon existiert mindestens eins.
    Alle Zertifikate $y'$ haben eine feste Länge $q(|z|)$, und da $\alpha$ aus $q(n)$ paarweise verschiedenen Indizes besteht, 
    gilt $q(|z|) \geq q(n)$. % und aus Monotonie folgt $|z|\geq n$.

    Nach Satz ?? (nur im unrelativierten Fall!) ist die Relation $R$ auch \emph{joinable}, das heißt wir haben eine Funktion $\mathit{join}\in\FP$. Sei $(x',\delta)=\mathit{join}(x, z)$. Nach Definition ?? gilt
    \[ |\delta|=q(|x|)+q(|z|)\geq q(n). \]
    Wieder gilt, dass alle Zertifikate $y'$ für $x'$ die feste Länge $q(|x'|)$ haben. Nach ähnlicher Argumentatn folgt $q(|x'|)\geq q(n)$ und aus Monotonie folgt $|x'|\geq n$.

    Sei nun $\mathit{pad}$ genau jene polynomialzeit-berechenbare Funktion, die aus $x$ und $1^n$ die Instanz $x'$ konstruiert:
    \[ \mathit{pad}(x, 1^n) = x' \quad\text{ wobei } (x', \cdot) = \mathit{join}(x, \underbrace{f(\phi_{q(n)})}_{z}). \]
    Dann gilt schon sofort, dass $|\mathit{pad}(x, 1^n)|=|x'|\geq n$ wie gewünscht.

    Ferner, nach Definition des $\mathit{join}$-Operators, haben wir
    \begin{multline*} \{ y[\delta] \mid y\in\Sigma^{l'}, (\mathit{pad}(x, 1^n), y)\in R \}\\ =  \{ y_1y_2 \mid y_1\in\Sigma^{q(|x|)}, y_2\in\Sigma^{q(|z|)}, (x, y_1)\in R \land (z, y_2)\in R\}.\end{multline*}
    Gilt also $x\not\in\Proj(R)$, dann ist die rechte Menge leer, also auch die linke Menge und damit $\mathit{pad}(x, 1^n)\not\in \Proj(R)$.
    Falls anders herum $x\in\Proj(R)$, dann ist die rechte Menge nicht leer, existieren ja sowohl ein Zertifikat $y_1$ für $x$, als auch (nach Konstruktion) immer ein Zertifikat $y_2$ für $z$. Also ist auch die linke Menge nicht leer, damit $\mathit{pad}(x, 1^n)\in \Proj(R)$.

    Die noch verbleibende Funktion $\mathit{padsol}$ ist durch die bitweise Projektion durch $\delta$ leicht möglich:
    \[
        \mathit{padsol}(x, 1^n, y) = y[\delta[1:q(|x|)]] \quad\text{ wobei } (\cdot, \delta) = \mathit{join}(x, \underbrace{f(\phi_{q(n)})}_z). \]
    Wir verifizieren: Sei $(\mathit{pad}(x, 1^n), y)\in R$, dann ist $y[\delta]=y_1y_2$ wobei $y_1,y_2\in\Sigma^*$, $|y_1|=q(|x|)$, $|q_2|=q(|z|)$, $(x,y_1), (z, y_2)\in R$. Es gilt $y[\delta[1:k]] = (y_1y_2)[1:k]$
    und mit $k=q(|x|)$ haben wir
    \[ \mathit{padsol}(x, 1^n, y) = y[\delta[1:q(|x|)]] = (y_1y_2)[1:q(|x|)] = y_1, \]
    und damit $(x, \mathit{padsol}(x, 1^n, y)) = (x, y_1)\in R$, wie gewünscht.
\end{proof}





\end{document}
