%! TEX root = ./thesis.tex
\chapter{Orakel mit $\hDisjNP$, $\hUP$ und $\hQ$}\label{chap:orakel}

Ziel dieses Kapitels ist die Konstruktion eines Orakels $O$, relativ zu dem einerseits die Hypothesen $\hDisjNP$ und $\hUP$ gelten, andererseits auch die Hypothese $\hQ$ gilt.
Damit wird ausgeschlossen, dass $\neg\hQ$ nicht mit relativierenden Mitteln beweisen werden kann, selbst unter Annahme der starken, aber wahrscheinlichen Annahme $\hDisjNP\land\hUP$.
Dies verbessert auch ein Ergebnis von \textcite[Cor.~3.3]{dose_oracle_2020}, welcher ein ähnliches Orakel konstruiert, relativ zu diesem $\hDisjNP\land\hUP\land\neg\hSAT$ gilt.
Präzise werden wir folgenden Satz beweisen:

\begin{theorem}\label{thm:myoracle}
    Es existiert ein Orakel $O$ sodass folgende Aussagen gelten:
    \begin{enumerate}
        \item Für jede totale NPTM $N$ existiert eine Funktion $g\in\FP^O$ sodass $g(x)$ ein akzeptierender Rechenweg von $N^O(x)$ ist (was äquivalent zu $\hQ$ relativ zu $O$ ist).
        \item Es existiert kein $\leq_\mathrm{m}^{\mathrm{pp},O}$-hartes Paar in $\DisjNP^O$ für $\DisjUP^O$ (was $\hDisjNP$ relativ zu $O$ impliziert).
        \item Es existiert keine $\leq_\mathrm{m}^{\mathrm{p},O}$-vollständige Menge für $\UP^O$ (was äquivalent zu $\hUP$ relativ zu $O$ ist).
    \end{enumerate}
\end{theorem}

Die Orakelkonstruktion bedient sich zwei zentralen Ideen. 
Die erste Idee betrifft vor allem die Konstruktion. 
Wir wenden hierbei ein Verfahren von \textcite{dose_np-completeness_2019} an, welche sich als ein starkes Framework erwiesen hat, schrittweise Orakel zu konstruieren, die simultan mehrere Eigenschaften erfüllen. Dieses Framework baut maßgeblich auf drei Zutaten auf: erstens, partiell definierten Orakeln, zweitens, ein Begriff von Gültigkeit, und drittens, eine stufenweise nicht-konstruktive Definitionsvorschrift des Orakels.
Das wird später beim Einsatz in der Konstruktion näher erläutert.

Die zweite Idee besteht intuitiv darin, von einem PSPACE-vollständigem Orakel $D$ auszugehen und dieses zu modifizieren, um die gewünschten Eigenschaften zu erhalten. Durch geschicktes Rückgreifen auf $D$ lassen sich dann gewisse nichttriviale Berechnungen effizient umsetzen. Diese Idee geht auf \textcite{baker_relativizations_1975} zurück, die ein Orakel konstruiert haben, relativ zu diesem $\P=\NP\cap\coNP\neq\NP$. Hierbei wird das PSPACE-vollständige Orakel $D$ mit einem zweiten, speziell konstruierten Orakel $B$ kombiniert; in der Originalfassung zu $D\cup B$. Das Orakel $B$ sorgt für die Trennung zwischen $\P$ und $\NP$, ist aber gleichzeitig so einfach zu entscheiden, dass ein komplementär akzeptierendes Maschinenpaar einer Sprache $L\in\NP\cap\coNP$ eine relevante Portion von $B$ entscheiden kann, und dann mittels dem PSPACE-vollständige Orakel $D$ dann abfragen kann, ob die entsprechende Eingabe akzeptiert wird.

Wir werden im Folgenden etwas Ähnliches mittels \emph{Relativierungen} von Orakelkonstruktionen umsetzen.
In der Literatur hat es sich als dienlich gezeigt, anstelle mit einem konkreten PSPACE-vollständigem Orakel zu starten und zu ergänzen, stattdessen mit der \emph{Annahme} $\P=\mathrm{PSPACE}$ zu starten. (Dieses Vorgehen findet sich zum Beispiel bei \cites{blum_generic_1987}{fortnow_separability_2002}{fenner_oracle_2003}.) 
Wir können dann (mit üblichen Methoden) ein Orakel $E$ konstruieren, relativ zu diesem die oberen Eigenschaften gelten. Im Verlauf dieses Kapitels zeigen wir die folgende Implikation:
\begin{theorem}\label{thm:myoracle-work}
    Angenommen $\P=\mathrm{PSPACE}$. Dann existiert ein Orakel $E$ sodass folgende Aussagen gelten:
    \begin{enumerate}
        \item Für jede totale NPTM $N$ existiert eine Funktion $g\in\FP^E$ sodass $g(x)$ ein akzeptierender Rechenweg von $N^E(x)$ ist (was äquivalent zu $\hQ$ relativ zu $E$ ist).
        \item Es existiert kein $\leq_\mathrm{m}^{\mathrm{pp},E}$-hartes Paar in $\DisjNP^E$ für $\DisjUP^E$ (was $\hDisjNP$ relativ zu $E$ impliziert).
        \item Es existiert keine $\leq_\mathrm{m}^{\mathrm{p},E}$-vollständige Menge für $\UP^E$ (was äquivalent zu $\hUP$ relativ zu $E$ ist).
    \end{enumerate}
\end{theorem}
Die Konstruktion von $E$ ist dabei solcherart, dass diese Konstruktion auf ein zweites Orakel relativiert. 
Intuitiv gemeint ist damit, dass wir der Aussage ein zweites Orakel hinzufügen können, ohne Gültigkeit zu verlieren. Hieraus ergibt sich insbesondere, dass für jedes Orakel $A$ folgende Implikation gilt:
\[ \P^X=\mathrm{PSPACE}^X \stackrel{\text{\ref{thm:myoracle-work}, rel.}}{\implies} \text{ex. $E$ sodass $E\oplus X$ den Eig. von Satz~\ref{thm:myoracle} genügt.} \]
Mit Wahl von $X$ als dem PSPACE-vollständigen $D$ erreichen wir dann wie gewünscht, dass $O=E\oplus D$ den Eigenschaften von Satz~\ref{thm:myoracle} genügt.
Diese Relativierungen von Orakelkonstruktionen werden im Verlauf dieses Kapitels noch formal präzise spezifiziert.


Die Hauptarbeit liegt also in der Konstruktion eines Orakels $E$ aus Satz~\ref{thm:myoracle-work}.
Die zentrale Idee sei hier kurz skizziert:
Wir starten mit einem leeren Orakel.
Für die Aussage $\hUP$ werden wir eine Familie von Zeugensprachen $C_m\in\UP^O$ definieren, welche sensitiv gegenüber dem Inhalt des Orakels $E$ sind.
Damit lassen sich dann Wörter so zu $E$ hinzufügen, dass für jede $\UP^E$-Maschine und jeden $\FP^E$-Transduktor die Reduktion auf eine entsprechende Zeugensprache $C_m\in\UP^E$ ausgeschlossen wird.
Das läuft über ein klassisches Diagonalargument, ähnlich wie die $\P\neq\NP$-Konstruktion von \textcite{baker_relativizations_1975}, und kann analog für die Aussage $\DisjNP$ umgesetzt werden.

Nun zur Aussage $\hQ$. Unter Annahme von $\P=\PSPACE$ gilt trivialerweise $\hQ$ relativ zum leeren Orakel. 
Hätten wir nun die Möglichkeit, effizient $E$ auszurechnen, hätten wir auch $\hQ$ relativ zu $E$.
Das ist zwar nicht möglich, aber für jede totale NPTM $N$ ist es zumindest möglich, eine relevante Portion von $E$ zu berechnen, sodass sich diese Portion in $N$ hineincodieren lässt. Wir erhalten dann eine äquivalente NPTM $N'$ welche keine Orakelfragen stellen muss. Unter Annahme $\P=\PSPACE$ lässt sich dann ein akzeptierender Rechenweg für $N'$ (bzw. $N$) finden.


Im folgenden Abschnitt~\ref{sec:oracle-relativization} erarbeiten wir detailliert die Relativierungen von Orakelkonstruktionen.
Anschließend spezifizieren wir in Abschnitt~\ref{sec:oracle-notation} noch einige spezielle Notationen, die wir zur Orakelkonstruktion benötigen.
Aufbauend auf der obigen Skizze können wir dann in Abschnitt~\ref{sec:oracle-definition} die konkrete Konstruktionsvorschrift des Orakels formulieren.
In Abschnitt~\ref{sec:oracle-correctness} wird abschließend nachgewiesen, dass diese Konstruktionsvorschrift zum einen wohldefiniert ist, und zum anderen, dass sich hieraus die gewünschten Eigenschaften wie in Satz~\ref{thm:myoracle} ergeben.


\section{Relativierende Orakelkonstruktionen}\label{sec:oracle-relativization}

Um den Begriff der relativierenden Orakelkonstruktion zu definieren, generalisieren wir zunächst den Begriff der (1-)Orakel-Turing-Maschine $M^A$ auf 2-Orakel-Turing-Maschinen $M^{A,B}$. Anstelle eines Fragezustands haben 2-Orakel-TM zwei ausgezeichnete Fragezustände $q_1$ und $q_2$. 
Im Zustand $q_1$ wird getestet ob der Inhalt $q$ des Orakelbands in $A$ liegt (und entsprechend zu $q_{\text{yes}}$ oder $q_{\text{no}}$ übergegangen), und analog in Zustand $q_2$ getestet, ob $q$ in $B$ liegt.
Damit rechnet $M$ relativ dem Orakel-Paar $(A,B)$. 
Definitionen und Aussagen können damit nicht nur, wie bisher, auf \emph{ein} Orakel relativiert werden, sondern auch auf zwei Orakel $(A,B)$ relativiert werden.
Zum Beispiel meint $\hQ^{A,B}$, dass zu jedem 2-Orakel-NPTM $N$, welcher total relativ zu $(A,B)$ arbeitet, auch ein 2-Orakel-PTM-Transduktor existiert, der auf Eingabe $x$ relativ zu $(A,B)$ einen akzeptierenden Rechenweg der Berechnung $N^{A,B}(x)$ ausgibt.

Die zentrale Einsicht zu den relativierenden Orakelkonstruktionen ist nun, dass diese genauso wie die üblichen Beweismethoden \emph{relativieren}.
Das ist insofern klar, weil in der Konstruktion von Orakeln nicht in die diskutierten TM „hineingeschaut“ wird, sondern diese nur anhand von Akzeptierverhaltens einiger TM definiert werden.
Es lässt sich also in solchen relativierenden Orakelkonstruktionen jede 1-Orakel-TM durch eine 2-Orakel-TM austauschen, wobei das hinzugefügte Orakel fest, aber frei wählbar ist. Die Konstruktion selbst ist ja „blind“ gegenüber dem zweiten Orakel.
Hätten wir also beispielsweise durch eine relativierende Orakelkonstruktion von $A$ die Aussage $\neg\hQ^A$, dann gilt auch $\neg\hQ^{A,X}$ für jedes Orakel $X$.

Eine solches Paar $(A,B)$ von Orakeln können wir dann wieder zurück in \emph{ein} äquivalentes Orakel $A\oplus B\subseteq\Sigma^*$ umwandeln. Zur Erinnerung:
\[ A\oplus B = \{0a\mid a\in A\} \cup \{1b\mid b\in B\}. \]
Es ist leicht zu sehen, dass sich zu jeder 2-Orakel-(N)PTM $N$ sich eine 1-Orakel-(N)PTM $N'$ angeben lassen kann, sodass sich $N$ relativ zu $(A,B)$ äquivalent verhält wie $N'$ zu $A\oplus B$, und umgekehrt.
In diesem Sinne hätten wie mit obigen Beispiel, dass auch $\neg\hQ^{A\oplus X}$ für alle $X$ gilt.

Vor diesem Hintergrund können wir noch einmal den Plan der nun folgenden Orakelkonstruktion angeben.
Wir zeigen erstens die Implikation von Satz~\ref{thm:myoracle-work} bzw.
\[  \P^\emptyset =\mathrm{PSPACE}^\emptyset \implies \text{ex. $E$ sodass $E$ den Eig. von Satz~\ref{thm:myoracle} genügt.}  \]
Diese Konstruktion relativiert insbesondere. Wir können daher auf beliebige Orakel $X$ relativieren:
\[  \P^{\emptyset,X} =\mathrm{PSPACE}^{\emptyset,X} \implies \text{ex. $E$ sodass $(E, X)$ den relativierten Eig. von Satz~\ref{thm:myoracle} genügt.}  \]
Wählen wir nun ein PSPACE-vollständiges $D$ als $X$, dann gilt trivialerweise die Voraussetzung $\P^{\emptyset,D}=\PSPACE^{\emptyset,D}$. Dann existiert also auch ein $E$, sodass $(E,D)$ den Eigenschaften von Satz~\ref{thm:myoracle} genügt, wobei die Eigenschaften auf 2-Orakel-TM relativiert wurden.
Wie bereits diskutiert, erfüllt $E\oplus D$ dann auch diese Eigenschaften, relativiert auf 1-Orakel-TM.
Setzen wir also $O=E\oplus D$ sind wir fertig und haben Satz~\ref{thm:myoracle} bewiesen.







\section{Notation zur Orakelkonstruktion}\label{sec:oracle-notation}

Nun definieren wir noch einige Notationen, die wir bei der Orakelkonstruktion, und insbesondere für den Beweis von Satz~\ref{thm:myoracle-work} benötigen.

Zunächst seien $\{M_i\}_{i\in \mathbb N}$, $\{N_i\}_{i\in \mathbb N}$, $\{F_i\}_{i\in\mathbb N}$ sogenannte \emph{Standardaufzählungen} der Orakel-PTMs, Orakel-NPTMs, bzw. Orakel-PTM-Tranduktoren, welche folgende Eigenschaften haben
\begin{enumerate}[label=\arabic*.,nosep]
    \item Die Mengen $\{M_i \mid i\in\mathbb N\}$, $\{N_i \mid i\in\mathbb N\}$, $\{F_i \mid i\in\mathbb N\}$ sind in Polynomialzeit entscheidbar, heißt es ist effizient entscheidbar, ob der gegebene Maschinencode einer TM der Standardaufzählung entspricht.
    \item Für jedes Orakel $D$ terminiert $M_i^D(x)$ nach höchstens $p_i(|x|)\defeq |x|^i+i$ Schritten; analog für $N_i$ und $F_i$.
    \item Für jede Orakel-PTM $M$ existiert ein $i$ sodass $L(M^D_i)=L(M^D)$, 
        für jede Orakel-NPTM $N$ existiert ein $i$ sodass $L(N^D_i)=L(N^D)$, und
        für jeden Orakel-PTM-Transduktor $T$ existiert ein $i$ sodass $T^D(x)=F_i^D(x)$.

        Insbesondere kann jeder akzeptierender Rechenweg $\alpha$ der Berechnung $N_i(x)$ in einen akzeptierenden Rechenweg $\alpha'$ der Berechnung $N(x)$ effizient übersetzt werden.
\end{enumerate}
Solche Standardaufzählungen existieren. Eine Möglichkeit, diese zu konstruieren ist beispielsweise, jede nichtdeterministische TM mit einem Timer auszustatten, die nach polynomieller Laufzeit die Berechnung abbricht. Konkret: gegeben $i$, sei zunächst $N$ die nichtdeterministische TM mit Codierung $i$. Die TM $N_i$ führt dann auf Eingabe $x$ parallel zur Rechnung $N(x)$ auf jedem Rechenweg einen Timer aus, der nach $>|x|^i+i$ Schritten den Rechenweg ablehnend abbricht.
Damit ist die Laufzeit von $N_i$ polynomiell beschränkt auf $p_i$.

Wie bereits angesprochen, werden wir ein Konstruktionsverfahren von \citeauthor{dose_np-completeness_2019} anwenden. Die erste wesentliche Zutat dieses Frameworks ist der Begriff von „partiell definierten“ Orakeln, bei denen also für gewisse $x$ noch nicht endgültig festgelegt ist, ob $x\in O$ oder $x\not\in O$ gelten soll.
Diese werden mittels finiten Wörtern $w\in\Sigma^*$ formalisiert:
ein finites Wort $w\in\Sigma^*$ können wir im Folgenden auch als die Menge $\{ i \mid i<|w|, w[i] = 1 \}$ verstehen (aber $|w|$ immer als die Länge von $w$).
Die intendierte Interpretation ist, dass gegenüber der \emph{Menge} $w$ die Zugehörigkeit aller Zahlen (bzw. äquivalent Wörtern) $x$ mit $x<|w|$ final spezifiziert ist, nicht dagegen für $x\geq|w|$ (und nur ersatzweise $x\not\in w$ gilt).

Diese Interpretation von finiten Wörtern $w$ als Orakel macht es einfacher, unsere Orakelkonstruktionen präzise und knapp zu beschreiben. Üblicherweise werden wir $w$ so erweitern, dass die Zugehörigkeit des kleinsten $x\in\mathbb N$ spezifiziert wird, welche noch nicht final spezifiziert ist. Dieses $x$ ist genau $|w|$ und wir legen die Zugehörigkeit final fest, indem wir an $w$ entweder $0$ anhängen (und $x$ ist final nicht im Orakel $w0$) oder $1$ anhängen (und $x$ ist final im Orakel $w1$).


Wir können relativ partiell definierten Orakeln auch rechnen. Für $w\in\Sigma^*$ definieren wir $M^w(x)$ entsprechend als $M^{\{i\mid w(i)=1\}}(x)$ (heißt, Orakel-Anfragen, für die $w$ nicht definiert ist, werden negativ beantwortet).
Dies ermöglicht es uns auch, folgenden Begriff zu definieren: Wir sagen, dass NPTM $M^w(x)$ \emph{definit} ist, wenn alle Anfragen auf allen Rechenwegen $<|w|$ sind (oder äquivalent: $w[q]$ ist für alle Anfragen $q$ auf allen Rechenwegen definiert); wir sagen, dass $M^w(x)$ \emph{definitiv akzeptiert} (bzw. \emph{definitiv ablehnt}), wenn $M^w(x)$ definit ist und akzeptiert (bzw. ablehnt). Intuitiv beschreibt der Begriff „definit“ Berechnungen, die sich nicht ändern, wenn das jeweilige partiell definierte Orakel erweitert wird, weil die Anfragen zu kurz sind.
\begin{observation}\label{obs:partialoracles}
    \begin{enumerate}
        \item Wenn $M^w(x)$ eine definite Berechnung ist und $v\sqsupseteq w$ ist, dann ist $M^v(x)$ definit. Die Berechnung $M^v(x)$ akzeptiert genau dann, wenn $M^w(x)$ akzeptiert.
        \item Wenn $w$ für alle Wörter der Länge $p_i(|x|)=|x|^i+i$ definiert ist, dann ist $M_i^w(x)$ definit.
        \item Wenn $M^w(x)$ auf einem Rechenweg mit der Menge der Orakel-Anfragen $Q$ akzeptiert, und $w$, $v$ auf $Q$ übereinstimmen, dann akzeptiert $M^v(x)$ auf dem gleichen Rechenweg und mit der gleichen Menge der Anfragen $Q$.
    \end{enumerate}
\end{observation}

Für ein partielles Orakel $w$, einen PTM-Transduktor $F$ und eine (N)PTM $M$ schreiben wir manchmal $M^w(F^w(x))$ als die \emph{eine} Berechnung der (N)PTM $M\circ F$ auf Eingabe $x$ relativ zu $w$.
Entsprechend sagen wir dann auch, dass $M^w(F^w(x))$ definit ist (bzw. definitiv akzeptiert, oder definitiv ablehnt) wenn $M\circ F$ definit auf Eingabe $x$ relativ zu $w$ ist (bzw. definitiv akzeptiert, definitiv ablehnt).
Sollte es aus dem Kontext klar hervorgehen, lassen wir üblicherweise auch den Zusatz „partiell“ weg, wenn wir z.B. von einem (partiellen) Orakel $w\in\Sigma^*$ sprechen.

Vor der Konstruktion machen wir nun noch folgende bekannte kombinatorische Aussage:
\begin{lemma}\label{lemma:bipartite}
    Sei $G$ ein gerichteter bipartiter Graph mit den Knotenmengen $A$ und $B$.
    Das heißt, jede Kante in $G$ führt entweder von einem Knoten in $A$ zu einem Knoten in $B$ oder umgekehrt.
    Sei $\Delta$ eine obere Schranke für den Ausgangsgrad jedes Knotens in $G$.

    Wenn $|A|,|B|>2\Delta$ gilt, dann gibt es ein $a\in A$ und $b\in B$, sodass weder $(a,b)$ noch $(b,a)$ eine Kante in $G$ ist.
\end{lemma}
\begin{proof}
    Sei $n=\min\{|A|,|B|\}>2\Delta$.
    Entferne Knoten aus $A$ und $B$, bis beide Knotenmengen jeweils genau $n$ Knoten haben, um einen gerichteten bipartiten Graphen $G'$ mit den Knotenmengen $A'$ und $B'$ zu bilden.
    Sei $G''$ der zugrunde liegende ungerichtete Graph von $G'$.
    In $G'$ gibt es $\leq |A'|\cdot \Delta + |B'|\cdot\Delta<n^2$ viele ungerichtete Kanten,
    aber $n^2$ viele ungerichtete Kanten im vollständigen bipartiten ungerichteten Graphen $K_{n,n}$.

    Das bedeutet, dass es $a\in A'\subseteq A$, $b\in B'\subseteq B$ gibt, die in $G''$ nicht adjazent sind; damit sind sowohl $(a,b)\not\in E(G')$ als auch $(b,a)\not\in E(G')$ im induzierten gerichteten bipartiten Teilgraphen $G'$.
    Also gilt für den ursprünglichen Graphen $G$ sowohl $(a,b)\not\in E(G)$ als auch $(b,a)\not\in E(G)$, wie gewünscht.
\end{proof}

\section{Definition}\label{sec:oracle-definition}

Nun werden wir den Beweis von Satz~\ref{thm:myoracle-work} erarbeiten. Dafür starten wir mit der Annahme $\P=\PSPACE$.

Wie zum Teil anfangs des Kapitels skizziert, möchten wir in unseren Orakelkonstruktion abzählbar unendlich viele \emph{Ebenen} $n$, das heißt, Wörter gleicher Länge $n$, für eine abzählbar unendliche Familie von Zeugensprachen mit zunehmend großen Lücken injektiv reservieren und zuordnen.
Hierfür sei $e(0) \defeq 2$, $e(i) \defeq 2^{e(i-1)}$.
Es gibt eine polynomialzeit-berechenbare, polynomialzeit-invertierbare injektive Funktion $f$, die von $(m,h)\in\mathbb N\times\mathbb N$ auf $\mathbb N$ abbildet.
Definiere nun $H_m \defeq \{ e(f(m,h)) \mid h\in\mathbb N \}$ als die Menge der für die Zeugensprache $m$ reservierten Ebenen.
Diese Definition stellt nun Folgendes sicher:
\begin{observation}\label{obs:leveldefinitions}
    \begin{enumerate}
        \item Die Menge $H_m$ ist abzählbar unendlich, eine Teilmenge der geraden Zahlen, und alle $H_0, H_1, \dots$ sind paarweise disjunkt.
        \item Die Folge $\min H_0, \min H_1, \dots$ ist nach oben unbegrenzt.
        \item Wenn $n\in H_m$, dann gilt für jedes $a\in\mathbb N$: $n<a<2^{n} \implies a\not\in H_0, H_1, \dots$.
        \item Jede Menge $H_m\in \P$ für alle $m\in\mathbb N$.
    \end{enumerate}
\end{observation}


Wie bereits angesprochen, werden wir in der Konstruktion Wörter der Länge $n\in H_m$ in das Orakel $E$ einsetzen, sodass $E$ die gewünschten Eigenschaften hat.
Wörter anderer Länge werden nicht in das Orakel eingesetzt.
Wir definieren folgende Zeugensprachen, welche vom Inhalt der Ebenen abhängig sind. Sei hierfür $m\in\mathbb N$ und $w\in\Sigma^*\cup\Sigma^{\omega}$ beliebig:
\begin{gather*}
    A_m^w \defeq \{ 0^n \mid n\in H_m, \text{existiert $x\in \Sigma^{n}$ mit } x\in w \text{ und $x$ endet mit $0$} \}\\
    B_m^w \defeq \{ 0^n \mid n\in H_m, \text{existiert $x\in \Sigma^{n}$ mit } x\in w \text{ und $x$ endet mit $1$} \}\\
    C_m^w \defeq \{ 0^n \mid n\in H_m, \text{existiert $x\in \Sigma^{n}$ mit } x\in w  \}
\end{gather*}
Sind die Ebenen der Höhe $H_m$ in $w$ auf geeignete Weise gefüllt, lässt sich leicht sehen dass diese entsprechenden Zeugensprachen in $\DisjUP^w$ bzw. $\UP^w$ fallen:
\begin{claim}\label{claim:witnesslanguages}
    Sei $w\in\Sigma^*\cup\Sigma^\omega$ ein beliebiges Orakel.
    \begin{enumerate}
        \item Wenn $|w\cap \Sigma^{n}|\leq 1$ für alle $n\in H_m$, dann $(A_m^w, B_m^w)\in\DisjUP^w$.
        \item Wenn $|w\cap \Sigma^n|\leq 1$ für alle $n\in H_m$, dann $C_m^w \in \UP^w$.
    \end{enumerate}
\end{claim}

\subsection*{Idee und Vorschau der Konstruktion}

Die Konstruktion verläuft im Wesentlichen so, dass simultan und ineinander verflochten die drei Eigenschaften von Satz~\ref{thm:myoracle-work} erarbeitet werden:
\begin{enumerate}[label=\arabic*.,midpenalty=0,endpenalty=0]

    \item Erarbeitung (1), bzw. $\hQ$: Für alle $j\in\mathbb N$ versucht die Konstruktion die Ebenen so einzurichten, dass $N_j$ nicht total ist.
        Falls dies nicht möglich ist, dann wird $N_j$ inhärent total akzeptieren. Diese Eigenschaft können wir dann algorithmisch ausnutzen, um eine relevante Portion von $E$ zu bestimmen, und, zusammen mit der Voraussetzung $\P=\PSPACE$ so einen akzeptierenden Rechenweg für $N_j(x)$ bestimmen.

    \item Erarbeitung von (2), bzw. $\hDisjNP$: Für alle $a\neq b$ versucht die Konstruktion die Ebenen so einzurichten, dass $N_a, N_b$ beide eine Eingabe $x$ akzeptieren, womit $x\in L(N_a)\cap L(N_b)$ und $(L(N_a), L(N_b))\not\in \DisjNP$.
        Falls dies nicht möglich ist, ist $(N_a, N_b)$ inhärent ein disjunktes $\NP$-Paar.
        In diesem Fall fixieren wir ein $m$, stellen sicher, dass $(A_m,B_m)$ ein disjunktes $\UP$-Paar ist, und diagonalisieren gegen jeden PTM-Transduktor $F_r$, sodass $F_r$ die Reduktion $(A_m, B_m)\leqmpp (L(N_a), L(N_b))$ nicht realisiert.
        Dies wird folgendermaßen erreicht: (i) Für alle $n\in H_m$ fügen wir höchstens ein Wort der Länge $n$ in $E$ ein (und somit $(A_m, B_m)\in\DisjUP$), und (ii) für jedes $r$ gibt es ein $n\in H_m$ so, dass $0^{n}\in A_m$, aber $N_a(F_r(0^n))$ ablehnt (oder analog $0^n\in B_m$, aber $N_b(F_r(0^n))$ ablehnt).

    \item Erarbeitung von (3), bzw. $\hUP$: Für alle $a$ versucht die Konstruktion die Ebenen so einzurichten, dass $N_a$ eine Eingabe $x$ auf zwei Rechenwegen akzeptieren, womit  $L(N_a)\not\in \UP$.
        Falls dies nicht möglich ist, $L(N_a)$ inhärent eine $\UP$-Sprache.
        In diesem Fall fixieren wir ein $m$, stellen sicher, dass $C_m$ eine Sprache in $\UP$ ist, und diagonalisieren gegen jeden PTM-Transduktor $F_r$, sodass $F_r$ die Reduktion $C_m\leqmpp L(N_a)$ nicht realisiert.
        Dies wird folgendermaßen erreicht: (i) Für alle $n\in H_m$ fügen wir höchstens ein Wort der Länge $n$ in $E$ ein (und somit $C_m\in\DisjUP$), und (ii) für jedes $r$ gibt es ein $n\in H_m$ so, dass $0^{n}\in C_m$ genau dann wenn $N_a(F_r(0^n))$ ablehnt.
\end{enumerate}
Wir weisen diesen Arbeitsschritten folgende Symbole zu, welche die einzelnen Tasks repräsentieren sollen: $\tau^1_j$, $\tau^2_{a,b}$, $\tau^2_{a,b,r}$, $\tau^3_a$, $\tau^3_{a,r}$.
Symbol $\tau^1_j$ repräsentiert den (versuchsweisen) Ausschluss der Totalität von $N_j$. Symbol $\tau^2_{a,b}$ repräsentiert analog den Ausschluss der Disjunktheit von $L(N_a), L(N_b)$, Symbol $\tau^2_{a,b,r}$ dann die Diagonalisierung dieses Paares gegen den Transduktor $F_r$. Analog für $\UP$ und $\tau^3_a, \tau^3_{a,r}$.

Intuitiv gesprochen verläuft die Konstruktion nun wie folgt: wir starten mit einem partiell definierten Orakel $w$, und wollen dieses nun schrittweise erweitern. In jeder Erweiterung nehmen wir uns einen noch nicht bearbeiteten Task $\tau$, und werden die Erweiterung so wählen, dass $\tau$ relativ zu dieser Erweiterung erfüllt wird. Wir passen dabei auf, dass auch in zukünftigen Erweiterungen $\tau$ erhalten bleibt.
Führen wir das nun unendlich oft durch, enden wir im Limit mit einem $\omega$-langen Orakel, welcher alle unsere Tasks $\tau$ abgearbeitet hat.

Beachte, wie wir in der oberen Vorschau einige Tasks $\tau$ Voraussetzungen an das finale Orakel stellen. Kann zum Beispiel im Fall von Task $\tau^3_a$ nicht erwirkt werden, dass die NPTM $N_a$ auf mehr als einem Rechenweg akzeptiert, so muss in der Konstruktion dafür gesorgt werden, dass auch die Zeugensprache $C_m$ in $\UP$ liegt, indem selbst in den Erweiterungen nach Bearbeiten von Task $\tau^3_a$ für alle $n\in H_m$ höchstens ein Wort der Länge $n$ in $E$ eingesetzt wird.

Solche „globalen“ Voraussetzungen behandelt das Framework von \citeauthor{dose_np-completeness_2019} durch den Begriff der \emph{Gültigkeit} von partiellen Orakeln. Dies ist die zweite wesentliche Zutat.
Hierbei werden alle solche Eigenschaften, die im Verlauf der Konstruktion aufrecht erhalten müssen, in einer partiellen Funktion $t$ zusammengefasst.
Wenn ein Orakel $w\in\Sigma^*$ dieses Bündel $t$ an Eigenschaften erfüllt, dann nennen wir dies $t$-gültig.
Gültigkeit ist also eine binäre Relation zwischen den partiellen Orakeln aus $\Sigma^*$ und partiellen Funktionen.
%\todo{Idee skizzieren; Verweise darauf dass das Zutat Nr. 2 von Dose/Glaßer ist}

Für unsere Zwecke definieren wir uns folgende Funktionenklasse $\mathcal T$. Die Klasse $\mathcal T$ ist eine Teilmenge der partiellen Funktionen, und es gilt $t\in\mathcal T$ genau dann wenn 
\begin{itemize}[nosep]
    \item der Definitionsbereich nur aus Tasks $\tau^1_j, \tau^2_{a,b}, \tau^3_a$ besteht, 
    \item die Bildmenge nur aus natürlichen Zahlen besteht, und
    \item $t$ injektiv auf den positiven Funktionswerten ist, also $t(\tau)=t(\tau')>0 \implies \tau=\tau'$.
\end{itemize}
Die intendierte Interpretation ist nun wie folgt: wenn $t$ so gewählt ist dass $t(\tau^3_a)=m>0$, dann soll damit notiert werden dass Task $\tau^3_a$ die Ebenen der Menge $H_m$ für die Zeugensprache $C_m$ reserviert hat, also in jede diese Ebenen der Wortlänge $n$ nur höchstens ein Wort der Länge $n$ eingesetzt werden darf. Analog für $\tau^2_{a,b}$.

Wir setzen nun folgenden Begriff von Gültigkeit für diese Konstruktion fest.
Ein Orakel $w\in\Sigma^*$ ist $t$-gültig wenn $t\in\mathcal T$ und folgendes gilt:
\begin{enumerate}[label={V\arabic*}]
    \item Wenn $x<|w|$ und $|x|\not\in \img(e)$, dann gilt $x\not\in w$.\\
        (Bedeutung: Orakel $w$ enthält keine Wörter der Länge $\neq e(\cdot)$.)
    \item Für alle $i$ gilt $|w\cap \Sigma^{e(i)}|\leq 2$.\\
        (Bedeutung: Orakel $w$ ist dünn auf den Ebenen der Länge $e(\cdot)$.)
    \item Wenn $t(\tau^1_j)=0$, dann existiert ein $z$ sodass $N_j^w(z)$ definitiv ablehnt.\\
        ($L(N_j)\neq \Sigma^*$ relativ zum finalen Orakel.)
    \item Wenn $t(\tau^2_{a,b})=0$, dann existiert ein $z$ sodass $M_a^w(z)$ und $M_b^w(z)$ definitiv akzeptieren.\\
        (Bedeutung: wenn $t(\tau^2_{a,b})=0$, dann $L(M_a)\cap L(M_b)\neq \emptyset$ relativ zum finalen Orakel.)
    \item Wenn $0<t(\tau^2_{a,b})=m$, dann gilt für alle $n\in H_m$ dass $|\Sigma^{n}\cap w|\leq 1$.\\
        (Bedeutung: wenn $0<t(\tau^2_{a,b})=m$, dann $(A_m,B_m)\in\DisjNP$ relativ zum finalen Orakel.)
    \item Wenn $t(\tau^3_{a})=0$, dann existiert ein $z$ sodass $M_a^w(z)$ definitiv auf zwei Rechenwegen akzeptiert.\\
        (Bedeutung: wenn $t(\tau^3_{a})=0$, dann $L(M_a)\not\in \UP$ relativ zum finalen Orakel.)
    \item Wenn $0<t(\tau^3_{a})=m$, dann gilt für alle $n\in H_m$ dass $|\Sigma^n\cap w|\leq 1$.\\
        (Bedeutung: wenn $0<t(\tau^3_{a})=m$, dann $C_m\in\UP$ relativ zum finalen Orakel.)
\end{enumerate}


\subsection*{Induktive Definition des Orakels}

%Diese Idee formalisiern wir nun.
Sei $T$ eine abzählbare Aufzählung der oben genannten Tasks sodass $\tau^2_{a,b,r}$ immer nach $\tau^2_{a,b}$ kommt, sowie $\tau^3_{a,r}$ immer nach $\tau^3_a$ kommt.
Die Orakelkonstruktion erfolgt nun in abzählbar unendlich vielen \emph{Stufen}. In jeder Stufe bearbeiten wir den kleinsten Task $\tau$ in der durch $T$ festgelegten Reihenfolge. Anschließend entfernen wir $\tau$ aus $T$, und entfernen möglicherweise zusätzlich höhere Tasks aus $T$.
In der nächsten Stufe fahren wir mit dem nächsten Task fort, die noch nicht aus $T$ entfernt wurde. (In jeder Stufe existiert immer mindestens ein Task, die noch nicht entfernt wurde, da wir in keiner Stufe alle Tasks aus $T$ entfernen werden.)

Eine Stufe $s<\omega$ identifizieren wir hierbei mit einem Orakel $w_s\in\Sigma^*$, einer Gültigkeits-Funktion $t_s\in\mathcal T$, und einem Task (bzw. mehreren Tasks) welcher in dieser Stufe bearbeitet wurde.
Wir werden die einzelnen $w_0, w_1, \dots$ und $t_0, t_1, t_2, \dots$ so definieren, dass
\[ w_0\sqsubsetneq w_1 \sqsubsetneq w_2 \sqsubsetneq \cdots, \]
und 
\[ t_0 \subseteq t_1 \subseteq t_2 \subseteq \cdots, \]
heißt insbesondere das $t_j$ eine Fortsetzung von $t_i$ ist, wenn immer $j\geq i$.
Außerdem werden wir sichern, dass jedes $w_s$ ein $t_s$-gültiges Orakel ist.
Es ist klar, dass jeder Task $\tau\in T$ letztlich in irgendeiner Stufe $s$ bearbeitet wird.


Nun zur Definition von $w_s, t_s$: wir werden die einzelnen Stufen induktiv definieren. Als Basisklausel setzen wir $w_0 = \epsilon$ und $t_0 = \emptyset$.
Für unsere induktive Klausel sei $s>0$. Die Definition für $w_s, t_s$ ergibt sich nun aus der bereits definierten Funktion $t_{s-1}\in\mathcal T$, und dem bereits definierten $t_{s-1}$-gültigen Orakel $w_{s-1}$, sowie dem zu bearbeitenden kleinsten verbleibenden Task $\tau$ in $T$.
Zur Erinnerung: dieser wird unmittelbar nach der Bearbeitung aus $T$ entfernt. In der Bearbeitung wird das Orakel strikt verlängert.
Es gibt nun fünf Fälle, je nach dem welche Form der bearbeitete Task $\tau$ hat.
\begin{description}[leftmargin=\parindent]
    \item[Task $\tau^1_j$:] Setze $t'\defeq t_{s-1}\cup\{\tau^1_j\mapsto 0\}$. Existiert ein $t'$-gültiges Orakel $v\sqsupsetneq w_{s-1}$, dann setze $t_s\defeq t'$ und $w_s\defeq v$.

        Ansonsten setze $t_s\defeq t_{s-1}$ und setze $w_s\defeq w_{s-1}y$ für geeignetes $y\in\{0,1\}$ sodass $w_s$ auch $t_s$-gültig ist. Das ist möglich nach Behauptung~\ref{claim:myoracle-up-extension}.

        (Bedeutung: wenn das Orakel $w_s$ so eingerichtet werden kann, dass $N_j$ nicht mehr total arbeitet, dann erweitere genau so, vgl. V2.) 

    \item[Task $\tau^2_{a,b}$:] Setze $t'\defeq t_{s-1}\cup\{\tau^2_{a,b}\mapsto 0\}$. Existiert ein $t'$-gültiges Orakel $v\sqsupsetneq w_{s-1}$, dann setze $t_s\defeq t'$ und $w_s\defeq v$. Entferne außerdem alle Tasks der Form $\tau^2_{a,b,r}$ von $T$.

        Ansonsten wähle ein hinreichend großes $m\not\in \img(t_s)$ sodass $w_s$ kein Wort der Länge $\min H_m$ definiert. Setze $t_s\defeq t_{s-1}\cup \{ \tau^2_{a,b}\mapsto m \}$; damit ist $w_{s-1}$ auch $t_s$-gültig. Setze $w_s\defeq w_{s-1}y$ für geeignetes $y\in\{0,1\}$ sodass $w_s$ auch $t_s$-gültig ist. Das ist möglich nach Behauptung~\ref{claim:myoracle-up-extension}.

        (Bedeutung: wenn das Orakel $w_s$ so eingerichtet werden kann, dass $N_a, N_b$ nicht mehr disjunkt akzeptieren, dann erweitere genau so, vgl. V4. Ansonsten, falls das nicht möglich ist, wähle ein geeignetes frisches $m$ und setze ab dieser Stufe voraus, dass $(A_m, B_m)\in \DisjUP$.) 

    \item[Task $\tau^2_{a,b,r}$:] Wir wissen dass $t_{s-1}(\tau^2_{a,b})=m>0$. Setze $t_s=t_{s-1}$ und wähle ein $t_s$-gültiges Orakel $w_s\sqsupsetneq w_{s-1}$ sodass bezüglich einem $n\in\mathbb N$ eine der folgenden Aussagen gilt:
        \begin{itemize}[nosep,endpenalty=10000]
            \item $0^n\in A_m^v$ für alle $v\sqsupseteq w_s$ und $M_a(F_r(0^n))$ lehnt relativ zu $w_s$ definitiv ab.
            \item $0^n\in B_m^v$ für alle $v\sqsupseteq w_s$ und $M_b(F_r(0^n))$ lehnt relativ zu $w_s$ definitiv ab.
        \end{itemize} Das ist möglich nach Behauptung~\ref{lemma:myoracle-disjnp}.

        (Bedeutung: erweitere zu $w_s$, sodass $F_r$ nicht die Reduktion $(A_m, B_m)\not\leq_\mathrm{m}^\mathrm{pp} (L(N_a), L(N_b))$ realisiert.) 

    \item[Task $\tau^3_{a}$:] Setze $t'\defeq t_{s-1}\cup\{\tau^3_{a,b}\mapsto 0\}$. Existiert ein $t'$-gültiges Orakel $v\sqsupsetneq w_{s-1}$, dann setze $t_s\defeq t'$ und $w_s\defeq v$. Entferne außerdem alle Tasks der Form $\tau^3_{a,b,r}$ von $T$.

        Ansonsten wähle ein hinreichend großes $m\not\in \img(t_s)$ sodass $w_s$ kein Wort der Länge $\min H_m$ definiert. Setze $t_s\defeq t_{s-1}\cup \{ \tau^3_{a,b}\mapsto m \}$; damit ist $w_{s-1}$ auch $t_s$-gültig. Setze $w_s\defeq w_{s-1}y$ für geeignetes $y\in\{0,1\}$ sodass $w_s$ auch $t_s$-gültig ist. Das ist möglich nach Behauptung~\ref{claim:myoracle-up-extension}.

        (Bedeutung: wenn das Orakel $w_s$ so eingerichtet werden kann, dass $N_a$ auf mehr als einem Rechenweg akzeptiert, dann erweitere genau so, vgl. V6. Ansonsten, falls das nicht möglich ist, wähle ein geeignetes frisches $m$ und setze ab dieser Stufe voraus, dass $(C_m)\in \UP$.) 

    \item[Task $\tau^3_{a,r}$:] Wir wissen dass $t_{s-1}(\tau^3_{a})=m>0$. Setze $t_s\defeq t_{s-1}$ und wähle ein $t_s$-gültiges Orakel $w_s\sqsupsetneq w_{s-1}$ sodass bezüglich einem $n\in\mathbb N$ eine der folgenden Aussagen gilt:
        \begin{itemize}[nosep,endpenalty=10000]
            \item $0^n\in C_m^v$ für alle $v\sqsupseteq w_s$ und $M_a(F_r(0^n))$ lehnt relativ zu $w_s$ definitiv ab.
            \item $0^n\not\in C_m^v$ für alle $v\sqsupseteq w_s$ und $M_a(F_r(0^n))$ akzeptiert relativ zu $w_s$ definitiv.
        \end{itemize} Das ist möglich nach Behauptung~\ref{lemma:myoracle-up}.

        (Bedeutung: erweitere zu $w_s$, sodass $F_r$ nicht die Reduktion $(C_m)\not\leq_\mathrm{m}^\mathrm{p} (L(N_a))$ realisiert.) 
\end{description}

Beobachte, dass $t_s$ immer als Element von $\mathcal T$ definiert ist.
Beachte auch den „Existenzquantor“ in Task $\tau^1_j$ (bzw. ähnlich bei $\tau^2_{a,b}, \tau^3_a$), welcher in der Konstruktion „testet“, ob eine gültige Erweiterung des Orakels existiert, welche die Totalität der NPTM $N_j$ definitiv ausschließt. 
Hierzu zwei Bemerkungen: Erstens handelt es sich hierbei insbesondere um ein nicht-konstruktives Argument. Falls eine solche Erweiterung existieren sollte, dann können wir diese auch auswählen (z.B. indem die lexikographisch kleinste gewählt wird), ohne dass diese in expliziter Weise angegeben wird.
Zweitens, falls eine solche gültige Erweiterung nicht existiert, dann ist die NPTM $N_j$ sogar „besonders total“, in dem Sinn dass $N_j$ sogar unter \emph{jeder} gültigen Erweiterung total ist. 
Diese Konstruktionstaktik ist die dritte Zutat des Verfahrens von \citeauthor{dose_np-completeness_2019}.

Mit den oben genannten Tasks ist die die Definition der Stufe $s$ abgeschlossen, und somit die beiden Folgen $\{w_s\}_{s<\omega}$ $\{t_s\}_{s<\omega}$.
Später werden wir das finale Orakel $E$ als $\bigcup_{s\in\mathbb N} w_s = \bigcup_{s\in\mathbb N} \{ i \mid w_s[i]=1, i<|w_s|\} $ definieren, heißt die Vereinigung über alle partiellen Orakel, oder äquivalent, der Grenzwert $\lim_{n\to\omega} w_s = E\in\Sigma^\omega$ der Folge von finiten Wörtern $w_0\sqsubsetneq  w_1\sqsubsetneq \cdots$.

Wir müssen nun zwei Aussagen zeigen: erstens, dass diese Konstruktion tatsächlich möglich ist, indem wir die in der Definition angekündigten Lemmata angeben und beweisen.
Zweitens müssen wir zeigen, dass die Konstruktion das leistet was wir uns wünschen, also dass $E$ die behaupteten Eigenschaften des Satzes~\ref{thm:myoracle} erfüllen.
Beides werden wir nun im folgenden Abschnitt beweisen.

\section{Korrektheit}\label{sec:oracle-correctness}

\subsection*{Existenz}

Wir zeigen im Folgenden zunächst, dass die oben definierte induktive Definition tatsächlich wohldefiniert ist, in dem Sinne dass jeder Task umgesetzt werden kann und die induktive Definition nicht „abbricht“.

Zunächst beweisen wir, dass sich ein $t$-gültiges Orakel $w$ immer um ein Bit verlängern lassen kann, ohne Gültigkeit zu verletzen.

\begin{lemma}\label{claim:myoracle-up-extension}
    Sei $t\in\mathcal T$ und $w$ ein $t$-gültiges Orakel, und sei $z=|w|$. 
    (Denke $z$ als das kleinste Wort, für welche die Zuordnung zum Orakel noch nicht endgültig feststeht.)

    Dann ist $w0$ auch $t$-gültig. (Das heißt wir verletzen nicht die Gültigkeit, wenn wir das partielle Orakel so setzen, dass $z$ nicht enthalten ist.)
\end{lemma}
\begin{proof}
    Sei $z=|w|$. 
    Angenommen, $w0$ ist nicht $t$-gültig, dann muss eine der Bedingungen V1–V7 verletzt sein.

    Angenommen V3 ist verletzt, weil $N_j^{w0}(x)$ nicht definit ablehnt und gleichzeitig $t(\tau_j^1)=0$. Da nach Voraussetzung $w$ aber $t$-gültig ist, wird $N_j^{w}(x)$ definitiv ablehnen.
    Nach Beobachtung~\ref{obs:partialoracles}(3) wissen wir aber, dass dann auch $N_j^{w0}(x)$ definitiv ablehnen wird. Widerspruch.
    Also kann V3 nicht verletzt sein. Mit analoger Argumentation sehen wir auch, dass V4 und V6 nicht verletzt sein können.

    Angenommen V1 ist verletzt, weil für $x<|w0|$, $|x|\not\in\img(e)$ die Zugehörigkeit $x\in w$ gilt. Dann kann $x$ nicht $<|w|$ sein, denn V1 gilt hier nach $t$-Gültigkeit von $w$. Also muss $x=z$.
    Nun gilt aber nach Definition $z\not\in w$; Widerspruch zur Wahl von $x$. Also kann V1 nicht verletzt sein.

    Es kann also nur V2, V5, oder V7 verletzt sein.
    Angenommen V5 ist verletzt, weil für $n\in H_m$ gilt $|w0\cap\Sigma^n|> 1$.
    Nach Definition wissen wir, dass $w0$ und $w$ auf allen Wörtern übereinstimmen.
    Das bedeutet dass auch $|w\cap\Sigma^n|>1$.
    Das widerspricht der $t$-Gültigkeit von $w$.
    Die Bedingung V5 kann also nicht verletzt sein.
    Auf analoge Weise wie lässt sich auch zeigen, dass V2 und V7 nicht verletzt sein können.

    Insgesamt kann also keine Bedingung V1–V7 verletzt sein; $w0$ ist $t$-gültig wie gewünscht.
\end{proof}

Damit haben wir die ersten beiden Tasks aus der Definition schon gesichert. Nun zeigen wir, dass die Bearbeitung von $\tau^2_{a,b,r}$ möglich ist.

\begin{lemma}\label{lemma:myoracle-disjnp}
    Die Bearbeitung eines Tasks $\tau^2_{a,b,r}$ ist möglich: gilt $t_s=t_{s-1}, t_{s}(\tau^2_{a,b})=m>0$, dann lässt sich $w_{s}$ so zu $t_{s}$-gültigem $w\sqsupsetneq w_{s}$ erweitern, sodass eine der folgenden Aussagen gilt:
    \begin{enumerate}[nosep,endpenalty=10000]
        \item $0^n\in A_m^v$ für alle $v\sqsupseteq w$ und $M_a(F_r(0^n))$ lehnt relativ zu $w$ definitiv ab.
        \item $0^n\in B_m^v$ für alle $v\sqsupseteq w$ und $M_b(F_r(0^n))$ lehnt relativ zu $w$ definitiv ab.
    \end{enumerate}
\end{lemma}
\begin{proof}[Skizze.]
    Widerspruchsbeweis. Erweitere $w_{s-1}$ so weit zu $u$, dass genau alle Wörter der Länge $<n=e(i)\in H_m$ definiert sind, wobei das $i$ hinreichend groß gewählt wird. Sei für jedes $X\subseteq \Sigma^n$ das Orakel $u(X)\sqsupsetneq w_{s-1}$ jenes Orakel was entsteht, wenn die Ebene $e(i)$ mit genau den Wörtern aus $X$ gefüllt wird, heißt $u(X)$ und $X$ stimmen auf $\Sigma^n$ überein. Beob. dass $u(X), |X|\leq 1$ auch $t_{s}$-gültig ist.

    Nach Annahme gilt
    \begin{itemize}[nosep]
        \item für $\alpha\in \Sigma^{n-1}0$ gilt $0^n\in A_m^{u(\{\alpha\})}$ und daher akzeptiert $M_a(F_r(0^n))$ relativ zu $u(\{\alpha\})$.
        \item für $\beta\in \Sigma^{n-1}1$ gilt $0^n\in B_m^{u(\{\alpha\})}$ und daher akzeptiert $M_b(F_r(0^n))$ relativ zu $u(\{\beta\})$.
    \end{itemize}
    Kombinatorische Standardmethoden zeigen dann, dass relativ zu $u(\{\alpha,\beta\})$ mit geeignetem $\alpha\in\Sigma^{n-1}0$, $\beta\in\Sigma^{n-1}1$ sowohl $M_a(F_r(0^n))$ also auch $M_b(F_r(0^n))$ relativ zu $u(\{\alpha,\beta\})$ akzeptieren.
    Damit wäre aber auch $u(\{\alpha,\beta\})$ ein geeignetes Orakel in der Bearbeitung von Task $\tau^2_{a,b}$ und wir hätten $t_{s}(\tau^2_{a,b})=0$.
\end{proof}
\begin{proof}
Wir fixieren die Werte von $a$, $b$ und $r$ im gesamten Beweis dieses Satzes.

Sei $\hat{s} < s$ die Stufe, die $\tau^2_{a,b}$ behandelt hat.
Eine solche Stufe existiert, da andernfalls $t_{s}(\tau^2_{a,b})$ undefiniert wäre.
Wir haben $m = t_{\hat{s}}(\tau^2_{a,b}) = t_{s}(\tau^2_{a,b})$; fixiere auch $m$ für den Rest des Beweises.

Wir nehmen an, dass für alle $t_{s}$-gültigen $w\sqsupsetneq w_{s-1}$ weder (1) noch (2) zutrifft.
Daraus werden wir einen Widerspruch ableiten, indem wir ein geeignetes Orakel $u'\sqsupsetneq w_{\hat{s}-1}$ konstruieren, das bezüglich $t' \defeq  t_{\hat{s}-1}\cup \{\tau^2_{a,b}\mapsto 0\}$ gültig ist. (Gemeint ist: relativ zu $u'$ wird $M_a$ und $M_b$ eine Eingabe definitiv akzeptieren.)
Dann folgt nach Definition, dass $u'$ eine mögliche $t'$-gültige Erweiterung von $w_{\hat{s}-1}$ in Stufe $\hat{s}$ ist, daher hätte die Bearbeitung von $\tau^2_{a,b}$ eben $t_{\hat{s}}=t'$ gesetzt, damit auch $t_{s}tau^2_{a,b})=t'(\tau^2_{a,b})=0$, was der Voraussetzung widerspricht.

Sei
\begin{equation*} \gamma(n) \defeq  \max(p_a(p_r(n))+p_r(n), p_b(p_r(n))+p_r(n)) \end{equation*}
das Polynom, das die Laufzeit von $M_a\circ F_r$, $M_b\circ F_r$ bezüglich der Eingabelänge $n$ relativ zu einem beliebigen Orakel beschränkt.
Das bedeutet immer dann, wenn ein partielles Orakel $u'$ für alle Wörter der Länge $\leq \gamma(n)$ definiert ist, auch $M_a^{u'}(F_r^{u'}(x))$, $M_b^{u'}(F_r^{u'}(x))$ für alle Eingaben $x\in\Sigma^n$ definit sind.
Sei $n\in\mathbb N$ eine geeignete Zahl sodass $n\in H_m$, und $w_{s-1}$ keine Wörter der Länge $\geq n$ definiert,
und
\begin{equation}\label{eq:myoracle1-expbound}
    2^n > \gamma(n),\quad  2^{n-1} > 2\gamma(n).
\end{equation}
Die erste Ungleichung von \eqref{eq:myoracle1-expbound} stellt sicher, dass kein Level $a$, $n<a\leq \gamma(n)$ für irgendeine Zeugensprache reserviert ist, das heißt, $a\not\in H_0, H_1, \dots$ (vgl. Beobachtung~\ref{obs:leveldefinitions}(3)). Die zweite Ungleichung stellt sicher, dass es genügend Wörter der Länge $n$ gibt, damit bestimmte kombinatorische Argumente funktionieren.
Die Ungleichung stellt sicher, dass es genügend Wörter der Länge $n$ gibt, damit bestimmte kombinatorische Argumente funktionieren.

Für den restlichen Beweis fixieren wir zusätzlich $n$.
Beachte, dass $\ell(Q)\leq\gamma(n)$ für $Q$ die Menge der Orakel-Queries ist, die jeweils von der Berechnung $M_a(F_r(0^n))$ oder der Berechnung $M_b(F_r(0^n))$ gestellt werden.
Wir definieren nun $u\sqsupseteq w_{s-1}$ als ein $t_{s}$-gültiges partielles Orakel, das genau für alle Wörter bis zur Länge $<n$ definiert ist. Ein solches Orakel existiert nach Lemma~\ref{claim:myoracle-up-extension}, indem man $w_{s-1}$ bitweise erweitert, so dass es $t_{s}$-gültig (bzw. identisch $t_{s-1}$-gültig) bleibt.

Für unseren Beweis betrachten wir nicht alle $t_{s}$-gültigen $w$, sondern vielmehr eine ausreichende Teilmenge davon.
Für $X \subseteq \Sigma^n, |X|\leq 2$ definieren wir ein partielles Orakel $u(X)\sqsupsetneq u$ als
%\[ u(X) = (u\cup X \cup C)\cap\Sigma^{\leq\gamma(n)} \]
\[
    u(X)[x] \defeq \begin{cases} u[x] & \text{falls $|x|< n$}\\
    1 & \text{falls $|x|=n, x\in X$}\\
    0 & \text{falls $|x|=n, x\not\in X$}\\
    0 & \text{falls $n<|x|\leq \gamma(n)$} \\ \bot & \text{sonst}, \end{cases}
\]
das für alle Wörter bis zur Länge $\leq\gamma(n)$ definiert ist, und so dass $u(X) \cap \Sigma^n = X$, das heißt, $u(X)$ und $X$ stimmen in $\Sigma^n$ überein.

Im Wesentlichen entsteht also $u(X)$ aus $u$, indem die Ebene $n$ mit $X$ gefüllt wird, und dann die höheren Ebenen leer gelassen werden.
Es ist leicht zu sehen, dass für $|X|\leq 1$ das Orakel $u(X)$ sogar $t_{s}$-gültig ist, indem iterativ $u$ erweitert wird.

\begin{claim}\label{claim:myoracle-validty-leq1}
    Sei $|X|\leq 1$. Dann ist $u(X)$ ein $t_{s}$-gültiges Orakel.
\end{claim}
\begin{proof}
    Angenommen, es ist kein $t_{s}$-gültiges Orakel. Dann ist eine der Bedingungen V1–V7 verletzt. Wir zeigen, dass diese Annahme zu einem Widerspruch führt.
    Klar ist, dass V3, V4, V6 nicht verletzt sein können, denn jede definite Berechnung relativ zu $u$ ist auch eine definite Berechnung relativ zu $u(X)$ nach Beobachtung~\ref{obs:partialoracles}(1).

    Angenommen, V1 ist verletzt. Dann muss ein $x$ mit $|x|\leq \gamma(n), |x|\neq e(\cdot)$ existieren, sodass $x\in u(X)$.
    Klar ist, dass $|x|\geq n$ sein muss, da ansonsten schon $u$ nicht $t_{s}$-gültig wäre; das wäre ein Widerspruch zur Konstruktion von $u$.
    Also $n<|x|\leq \gamma(n)$. Dann wissen wir aber nach Definition von $u(X)$ dass $x\not\in u(X)$; Widerspruch zur Wahl von $x$.

    Angenommen, V5 ist verletzt. Dann existiert ein $\tau'$ mit $t_{s}(\tau')=m'>0$ und $n'\in H_{m'}$ und $|\Sigma^{n'}\cap u(X)|>1$. Wieder klar ist, dass $n\leq n'\leq\gamma(n)<2^n$ sein muss, ansonsten wäre schon $u$ nicht mehr $t_{s}$-gültig.
    Dann aber muss auch $n=n'$, denn ansonsten wäre $n'\not\in H_{m'}$, nach Beobachtung~\ref{obs:leveldefinitions}.
    Jetzt haben wir aber $|\Sigma^{n'}\cap u(X)|=|\Sigma^n\cap u(X)| = |X|\leq 1$; das ist ein Widerspruch zur Annahme $|\Sigma^{n'}\cap u(X)|>1$. Also kann V5 nicht verletzt sein. 

    Auf analoge Weise ist ersichtlich, dass auch V2 und V7 nicht verletzt sein können.
    Damit ist keine Bedingung verletzt, $u(X)$ muss also $t_s$-gültig sein.
\end{proof}

Wir haben angenommen, dass für $t_{s}$-gültige Orakel nicht (1) oder (2) gilt. Angewendet auf $u(X), |X|\leq 1$ bedeutet das
\begin{itemize}[nosep]
    \item Für $\alpha\in \Sigma^{n-1}0$ gilt $0^n\in A_m^{u(\{\alpha\})}$ und daher akzeptiert $M_a(F_r(0^n))$ definitiv relativ zu $u(\{\alpha\})$ auf einem Rechenweg mit Menge $Q_\alpha$ an Orakelfragen.
    \item Für $\beta\in \Sigma^{n-1}1$ gilt $0^n\in B_m^{u(\{\alpha\})}$ und daher akzeptiert $M_b(F_r(0^n))$ definitiv relativ zu $u(\{\beta\})$ auf einem Rechenweg mit Menge $Q_\beta$ an Orakelfragen.
\end{itemize}
Wir wollen nun ein Orakel $u'$ konstruieren, sodass sowohl $M_a$ als auch $M_b$ akzeptieren.
Hierfür wollen wir die beiden jeweiligen akzeptierenden Rechenwege fixieren.
Wir stellen das sicher, indem wir $u'$ so wählen, dass $u'$ mit $u(\{\alpha\})$ auf $Q_\alpha$ übereinstimmt, dann wird auch $M_a(F_r(0^n))$ relativ zu $u'$ akzeptieren.
Symmetrisch stellen wir das auch für $M_b$ bzw. $\beta$ sicher, sodass auch $M_a(F_r(0^n))$ relativ zu $u'$ akzeptieren wird.

Hierzu müssen wir $\alpha\in\Sigma^{n-1}0$ und $\beta\in\Sigma^{n-1}1$ finden, die sich nicht gegenseitig „stören“.
Ein Wort $\alpha\in\Sigma^{n-1}0$ stört $\beta\in\Sigma^{n-1}1$ falls $\alpha\in Q_\beta$, 
symmetrisch stört ein Wort $\beta\in\Sigma^{n-1}0$ ein Wort  $\alpha\in\Sigma^{n-1}1$ falls $\beta\in Q_\alpha$.

Es existieren $\alpha\in\Sigma^{n-1}0$ und $\beta\in\Sigma^{n-1}1$ die sich nicht gegenseitig stören. Wir zeigen das über einen bipartiten „Störgraph“ $G$.
Setze 
\[ A=\Sigma^{n-1}0, B=\Sigma^{n-1}1 \]
als die zwei Knotenmengen von $G$.
Setze nun
\[ E = \{ (\alpha, \beta) \mid \alpha\in\Sigma^{n-1}0, \beta\in\Sigma^{n-1}1, \alpha\in Q_\beta\} \cup \{ (\beta, \alpha) \mid \alpha\in\Sigma^{n-1}0, \beta\in\Sigma^{n-1}1, \beta\in Q_\alpha\} \]
als Kantenmenge.
Beachte, dass der Ausgangsgrad $\Delta$ von $G$ höchstens $\gamma(n)$ sein kann, denn ein Rechenweg auf $M_a(F_r(0^n))$ (bzw. $M_b(\cdots)$) ist auf $\leq \gamma(n)$ viele Schritte beschränkt.

Nach \eqref{eq:myoracle1-expbound} wissen wir nun, dass $|A|, |B|>2\Delta$. Mittels Lemma~\ref{lemma:bipartite} existieren also $\alpha\in A = \Sigma^{n-1}0$, $\beta\in B=\Sigma^{n-1}1$ sodass weder $(a,b)\in E$ noch $(b,a)\in E$.
In anderen Worten, es gilt $\alpha\not\in Q_\beta$ und $\beta\not\in Q_\alpha$.

Betrachte nun $u'\defeq u(\{\alpha, \beta\})$. Wir zeigen nun zwei Dinge: erstens, dass $M_a(F_r(0^n))$ und $M_b(F_r(0^n))$ definitiv relativ zu $u(\{\alpha, \beta\})$ akzeptieren.
Zweitens anschließend, dass $u(\{\alpha,\beta\})$ auch $t'$-gültig ist.

Für die erste Aussage beobachten wir, dass $u(\{\alpha\})$ und $u(\{\alpha, \beta\})$ auf $Q_\alpha$ übereinstimmen. Angenommen sie stimmen nicht überein, dann müssen sie nach Konstruktion auf $\beta$ nicht übereinstimmen. Dann wäre aber $\beta\in Q_\alpha$, was ein Widerspruch zur Wahl von $\beta$ ist.
Symmetrisch sehen wir, dass $u(\{\beta\})$ und $u(\{\alpha, \beta\})$ auf $Q_\beta$ ist.
Also akzeptieren $M_a(F_r(0^n))$ und $M_b(F_r(0^n))$ definitiv relativ zu $u(\{\alpha, \beta\})$.

Nun zur zweiten Aussage. Zunächst zeigen wir $t_{\hat{s}-1}$-Gültigkeit.
\begin{claim}\label{claim:myoracle-validty-eq2}
    Das Orakel $u'=u(\{\alpha, \beta\})$ ist $t_{\hat{s}-1}$-gültig.
\end{claim}
\begin{proof}
    Angenommen, es ist kein $t_{\hat{s}-1}$-gültiges Orakel. Dann ist eine der Bedingungen V1–V7 verletzt. Wir zeigen, dass diese Annahme zu einem Widerspruch führt.
    Wieder klar ist, dass V3, V4, V6 nicht verletzt sein können, denn jede definite Berechnung relativ zu $t_{\hat{s}-1}$-gültigem $u$ ist auch eine definite Berechnung relativ zu $u'$ nach Beobachtung~\ref{obs:partialoracles}(1).
    Auch V1 kann nicht verletzt sein; das folgt aus dem gleichen Argument wie im Beweis von Behauptung~\ref{claim:myoracle-validty-leq1}.

    Angenommen, V5 ist verletzt. Dann existiert ein $\tau'$ mit $t_{\hat{s}-1}(\tau')=m'>0$ und $n'\in H_{m'}$ und $|\Sigma^{n'}\cap u(X)|>1$. 
    Wieder folgt $n=n'$.
    Insbesondere folgt daraus auch $m=m'$, denn $n$ ist nur in $H_m$ enthalten.
    Gleichzeitig haben wir $\tau'\neq \tau^2_{a,b}$, da $\tau^2_{a,b}\not\in\dom(t_{\hat{s}-1})$.
    Damit gilt aber $t_s(\tau')=t_{\hat{s}-1}(\tau')=m'=m=t_s(\tau^2_{a,b})$; Widerspruch zur Injektivität von $t_s\in\mathcal T$ auf dem Support.
    Also kann V5 nicht verletzt sein. 
    Auf analoge Weise ist ersichtlich, dass auch V7 nicht verletzt sein kann.

    Angenommen, V2 ist verletzt. Dann existiert ein $i$ mit $|u'\cap\Sigma^{n'}|>2$ für $n'=e(i)$.
    Wieder folgt $n=n'$.
    Jetzt haben wir aber $|\Sigma^{n'}\cap u'|=|\Sigma^n\cap u(\{\alpha, \beta\})| = |\{\alpha,\beta\}|=2$; das ist ein Widerspruch zur Annahme $|\Sigma^{n'}\cap u'|>2$. Also kann V2 nicht verletzt sein. 

    Damit ist keine Bedingung verletzt, $u'$ muss also $t_{\hat{s}-1}$-gültig sein.
\end{proof}
Wir erinnern uns daran dass $t'=t_{\hat{s}-1}\cup \{\tau^2_{a,b}\mapsto 0\}$.
Nun ist es leicht zu sehen, dass $u'\defeq u(\{\alpha,\beta\})$ auch $t'$-gültig ist. Das Orakel $u'$ ist nach voriger Behauptung $t_{\hat{s}-1}$-gültig, und bei der Erweiterung von $t_{\hat{s}-1}$ zu $t'$ kommt nur $\tau^2_{a,b}\mapsto 0$ hinzu. Das bedeutet, dass wir nur noch die Bedingung V4 bezüglich $a,b$ verifizieren müssen.
Sei nun $y=F_r(0^n)$ relativ zu $u'$.
Nach voriger Aussage oben wissen wir aber, dass $M_a(y)$ und $M_b(y)$ relativ zu $u'$ beide definitiv akzeptieren, wie von V4 verlangt.

Da nun also $u'\sqsupsetneq u \sqsupsetneq w_{\hat{s}-1}$ auch $t'$-gültig ist, sind wir fertig und erreichen einen Widerspruch, wie anfangs argumentiert: während der Bearbeitung von Task $\tau^2_{a,b}$ in Stufe $\hat{s}$ war das Orakel $u'$ eine mögliche $t'$-gültig Erweiterung von $w_{\hat{s}-1}$, denn es ist $t'$-gültig und $u'\sqsupsetneq w_{\hat{s}-1}$. Also wäre nach Definition des Tasks $\tau^2_{a,b}$ dann $t_{\hat{s}}=t'$ gesetzt worden.
Damit wäre dann auch $t_{s}(\tau^2_{a,b})=t'(\tau^2_{a,b})=0$, was der Voraussetzung dieses Lemma~\ref{lemma:myoracle-disjnp} widerspricht.
\end{proof}

Ohne größeren Veränderungen lässt sich auf fast gleiche Weise auch zeigen, dass die Tasks $\tau^3_{a,r}$ bearbeitet werden können.

\begin{lemma}\label{lemma:myoracle-up}
    Die Bearbeitung eines Tasks $\tau^3_{a,r}$ ist möglich: gilt $t_s=t_{s-1}, t_{s}(\tau^3_{a})=m>0$, dann lässt sich $w_{s}$ so zu $t_{s}$-gültigem $w\sqsupsetneq w_{s-1}$ erweitern, dass eine der folgenden Fälle eintritt:
        \begin{itemize}[nosep,endpenalty=10000]
            \item $0^n\in C_m^v$ für alle $v\sqsupseteq w$ und $M_a(F_r(0^n))$ lehnt relativ zu $w$ definitiv ab.
            \item $0^n\not\in C_m^v$ für alle $v\sqsupseteq w$ und $M_a(F_r(0^n))$ akzeptiert relativ zu $w$ definitiv.
        \end{itemize}
\end{lemma}
\begin{proof}
Wir verfahren wie im Beweis von Lemma~\ref{lemma:myoracle-disjnp}.
Um wieder einen Widerspruch abzuleiten, nehmen wir an, dass für alle $t_{s}$-gültigen $w\sqsupsetneq w_{s-1}$ weder (1) noch (2) zutrifft.
Definiere wieder identisch $u$ und $u(X)$.
Wieder gilt nach Behauptung~\ref{claim:myoracle-validty-leq1} dass $u(X)$ immer $t_{s}$-gültig ist wenn $|X|\leq 1$.
Wir haben also:
\begin{itemize}[nosep]
    \item Die Berechnung $M_a(F_r(0^n))$ lehnt relativ zu $u(\emptyset)$ definitiv ab.
    \item Für $\xi\in \Sigma^{n}$ gilt $0^n\in C_m^{u(\{\xi\})}$ und daher akzeptiert $M_a(F_r(0^n))$ definitiv relativ zu $u(\{\xi\})$ auf einem Rechenweg mit Menge $Q_\xi$ an Orakelfragen.
\end{itemize}

Beachte, dass $\xi\in Q_\xi$ ist, andernfalls stimmen $u(\emptyset)$ und $u(\{\xi\})$ auf $Q_\xi$ überein, damit folgt mit Beobachtung~\ref{obs:partialoracles}(3), dass $M_a(F_r(0^n))$ relativ zu $u(\emptyset)$ akzeptiert, was der Annahme widerspricht.

Wir fahren nun fort wie in Lemma~\ref{lemma:myoracle-disjnp}.
Seien $\alpha\in\Sigma^{n}$, $\beta\in\Sigma^{n}$ zwei unterschiedliche Wörter, die sich nicht gegenseitig stören. Das heißt, $\alpha\not\in Q_\beta$, $\beta\not\in Q_\alpha$.
Diese zwei Wörter existieren nach dem gleichen kombinatorischen Argument wie bei Lemma~\ref{lemma:myoracle-disjnp}.

Setze nun $u(\{\alpha, \beta\})$. Wir zeigen nun, dass $M_a(F^r(0^n))$ auf zwei unterschiedlichen Rechenwegen akzeptieren.
Wieder haben wir, dass $u(\{\alpha\})$ und $u(\{\alpha, \beta\})$ auf $Q_\alpha$ übereinstimmen, und symmetrisch $u(\{\beta\})$ und $u(\{\alpha, \beta\})$ auf $Q_\beta$ übereinstimmen.
Mittels Beobachtung~\ref{obs:partialoracles}(3) sehen wir also, dass es zwei Rechenwege von $M_a(F_r(0^n))$ relativ zu $u(\{\alpha, \beta\})$ gibt, welche definitiv akzeptieren: einer mit Orakelfragen $Q_\alpha$, und einer mit $Q_\beta$.
Diese zwei Rechenwege sind tatsächlich unterschiedlich:
einerseits gilt aus obiger Feststellung dass $\alpha\in Q_\alpha$, aber nach Wahl von $\beta$ ist $\alpha\not\in Q_\beta$. Auf einem Rechenweg wird also $\alpha$ erfragt, auf dem anderen nicht; also sind die zwei Rechenwege unterschiedlich.


Wie im Beweis von Lemma~\ref{lemma:myoracle-disjnp} können wir jetzt $t'\defeq t_{\hat{s}-1}\cup\{\tau^3_{a} \mapsto 0\}$ setzen.
Wieder gilt nach Behauptung~\ref{claim:myoracle-validty-eq2} dass $u'\defeq u(\{\alpha, \beta\})$ auch $t_{\hat{s}-1}$-gültig ist.
Dann lässt sich auch leicht sehen, dass $u'$ auch $t'$-gültig ist. Es kommt V6 bezüglich $a$ hinzu, aber wir haben ja eben gesehen, dass $M_a$ auf zwei unterschiedlichen Rechenwegen definitiv akzeptiert.

Da nun also $u'\sqsupsetneq u \sqsupsetneq w_{\hat{s}-1}$ auch $t'$-gültig ist, sind wir fertig und erreichen einen Widerspruch: während der Bearbeitung von Task $\tau^2_{a,b}$ in Stufe $\hat{s}$ war das Orakel $u'$ eine mögliche $t'$-gültige Erweiterung von $w_{\hat{s}-1}$. Also wäre nach Definition des Tasks $\tau^2_{a,b}$ dann $t_{\hat{s}}=t'$ gesetzt worden.
Damit wäre dann auch $t_{s}(\tau^2_{a,b})=t'(\tau^2_{a,b})=0$, was der Voraussetzung dieses Lemma~\ref{lemma:myoracle-up} widerspricht.
\end{proof}

Damit ist die Konstruktion möglich. Sei $E\defeq \bigcup_{s\in\mathbb N} w_s$.
Beachte dass $w_s\sqsubsetneq E$ für alle $w_s$.

Sei außerdem $t_\omega \defeq \bigcup_{s\in\mathbb N} t_s$. Beachte, dass nach Definition von $t_0, t_1, \dots$ auch $t_\omega$ eine (totale) Funktion ist.
Beachte außerdem, dass $|w_0|<|w_1|< |w_2|< \ldots$ eine nach oben unbeschränkte Folge ist.

\subsection*{Eigenschaften des konstruierten Orakels}

Nun müssen wir noch verifizieren, dass $E$ tatsächlich alle gewünschten Eigenschaften erfüllt, die in Satz~\ref{thm:myoracle-work} behauptet wurden.
Wir starten mit einer Beobachtung über den groben Aufbau von $E$, der sich sofort aus der Ebenen-Konstruktion und unserem Gültigkeitsbegriff ergibt.

\begin{claim}\label{claim:myoracle-structure}
    \begin{enumerate}[midpenalty=0,beginpenalty=0,endpenalty=0]
        \item $E$ enthält keine Wörter der Länge $\neq e(\cdot)$.
        \item Es gilt $|E\cup\Sigma^{e(i)}|\leq 2$ für alle $i$.
        \item Wenn $t_\omega(\tau^1_j)=0$, dann gilt $L(N_j^E)\neq\Sigma^*$.
        \item Wenn $t_\omega(\tau^2_{a,b})=m>0$, dann gilt $|E\cap \Sigma^n|\leq 1$ für alle $n\in H_m$. Insbesondere gilt damit $(A_m^E, B_m^E)\in\DisjUP^E$.
        \item Wenn $t_\omega(\tau^3_{a})=m>0$, dann gilt $|E\cap \Sigma^n|\leq 1$ für alle $n\in H_m$. Insbesondere gilt damit $(C_m^E)\in\UP^E$.
    \end{enumerate}
\end{claim}
\begin{proof}
    \begin{prooflist}
    \item Zu (1): Sei $x$ ein Wort mit $|x|\neq e(\cdot)$. Wir zeigen dass $x\not\in E$.  Wähle ein $s$ sodass $x<|w_s|$, heißt $w_s$ wird von $x$ definiert. Nun ist $w_s$ auch $t_s$-gültig, und mit V1 gilt insbesondere $x\not\in w_sC$.
        Da nun $w_s\sqsubsetneq E$, gilt auch $x\not\in E$ und wir sind fertig.
        
    \item Zu (2): Beweis läuft analog wie bei (1), unter Berufung auf V2.

    \item Zu (3): Nach Voraussetzung muss es ein $s$ geben, für das $t_s(\tau^1_j)=0$. Insbesondere ist $w_s$ auch $t_s$-gültig, und mit V3 existiert ein $z$ sodass $N_j^{w_s}(z)$ definitiv ablehnt.
        Da $w_s\sqsubseteq E$, folgt mit Beobachtung~\ref{obs:partialoracles}(1), dass $N_j^E(z)$ auch definitiv ablehnen wird.

    \item Zu (4): Angenommen, es existiert ein $n\in H_m$ für das $|E\cap\Sigma^n|>1$.
        Wähle ein $s$ sodass $w_s$ alle Wörter der Länge $\leq n$ definiert. Dieses $w_s$ ist insbesondere $t_s$-gültig.

        Nun gilt $t_s(\tau^2_{a,b})=t_\omega(\tau^2_{a,b})=m>0$, und mit V5 gilt insbesondere $|w_s\cap\Sigma^n|\leq 1$.
        Da nun $w_s$ und $E$ auf $\Sigma^{n}$ übereinstimmen, haben wir auch $|E\cap\Sigma^n|\leq 1$. Das widerspricht der Annahme.

        Also gilt $|E\cap\Sigma^n|\leq 1$ für alle $n\in H_m$. Dann folgt aus Beobachtung~\ref{claim:witnesslanguages}(1) auch schon sofort, dass $(A_m^E, B_m^E)\in\DisjUP$.

    \item Zu (5): Beweis läuft analog wie bei (4).
    \end{prooflist}
\end{proof}

Über die Definition von Task $\tau^2_{a,b}$ bzw. $\tau^3_{a,r}$ ist nun in Verbindung mit der vorigen Beobachtung auch ersichtlich, dass für $\DisjNP$ bzw. $\UP$ keine vollständigen Elemente existieren können.

\begin{claim}\label{claim:myoracle-completness}
    \begin{enumerate}
        \item Kein Paar aus $\DisjNP^E$ ist $\leqmpp$-hart für $\DisjUP^E$.
        \item Keine Menge aus $\UP^E$ ist $\leqmp$-vollständig für $\UP^E$.
    \end{enumerate}
\end{claim}
\begin{proof}
    Wir zeigen hier nur (1), der Beweis für Aussage (2) folgt analog.

    Angenommen es existiert ein solches $\leqmpp$-hartes Paar $(U_1, U_2)\in\DisjNP^E$.
    Dann existieren über die Wahl der Standardenumeration auch zwei NPTM sodass $L(M_a^E)=U_1$, $L(M_b^E)=U_2$.
    Betrachte den Task $\tau^2_{a,b}$; dieser wurde in der Konstruktion von $E$ in Stufe $\hat{s}$ bearbeitet.
    Wir erinnern uns, dass $w_{\hat{s}}\sqsubsetneq E$ ein $t_{\hat{s}}$-gültiges Orakel ist.

    Wenn $t_{\hat{s}}(\tau^2_{a,b})=0$ ist, dann besagt V4, dass eine Eingabe $x\in\Sigma^*$ existiert für die sowohl $M_a(x)$ als auch $M_b(x)$ definitiv relativ zu $w_{\hat{s}}$ akzeptieren. Da $w_{\hat{s}}\sqsubseteq E$ ist, akzeptieren $M_a(x)$ und $M_b(x)$ relativ zu $E$, nach Beobachtung~\ref{obs:partialoracles}(1).
    Das bedeutet aber $x\in U\cap U'$; Widerspruch zur Wahl von $U, U'$ als disjunkt.

    Daher können wir annehmen, dass $m\defeq  t_{\hat{s}}(\tau^2_{a,b})>0$ ist.
    Insbesondere ist daher $t_\omega(\tau^2_{a,b})>0$ und nach voriger Behauptung~\ref{claim:myoracle-structure}(4) gilt dann auch $(A^E_m, B^E_m)\in\DisjUP^E$. Also gilt nach Annahme auch $(A^E_m, B^E_m) \leqmpp (U_1, U_2)$ relativ zu $E$.
    
    Sei nun $r$ so gewählt, dass diese Reduktion von $F^E_r$ realisiert wird, und betrachte Task  $\tau^2_{i,j,r}$, welcher in einem bestimmten Schritt $s$ behandelt wird.
    Nach Definition dieses Task existiert ohne Beschränkung der Allgemeinheit ein $n\in\mathbb N$ sodass $0^n\in A_m^E$, und $M_a(F_r(0^n))$ definitiv relativ zu $w_s$ ablehnt. (Der andere Fall in dem $M_b$ ablehnt läuft symmetrisch.)

    Mit Beobachtung~\ref{obs:partialoracles}(1) erhalten wir dann, dass $M_a(F_r(0^n))$ relativ zu $E$ ablehnt. Also auch $F^E_r(0^n)\not\in U_1$. Dies widerspricht der Annahme, dass $F_r^E$ die Reduktion realisiert, denn wir haben ja $0^n\in A_m^E$.
\end{proof}

Damit erfüllt $E$ schon die Aussagen (1) und (2) von Satz~\ref{thm:myoracle}. Die letzte Aussage (3) ist aufwändiger zu beweisen.
Folgende Behauptung sagt aus, dass totale NPTM $N_j$ relativ zu $E$ „besonders total“ sind. Nimmt man (ab hinreichender Länge) Wörter der Länge $=e(\cdot)$ aus $E$ weg, und erhält $T\subseteq E$, bleibt die NPTM relativ zu $T$ immer noch total. Das ergibt auch Sinn, denn in der Bearbeitung von Task $\tau^1_j$ wurden sämtlichen geeigneten Erweiterungen ausprobiert, welche die Totalität von $N_j$ zerstören würden. Da keine solche Erweiterung existiert, wird auch $T$ nicht die Totalität von $N_j$ zerstören können.

\begin{claim}\label{claim:myoracle-totality}
    Sei $N_j$ eine totale NPTM, d.h. $L(N_j^E)=\Sigma^*$.
    Es existiert eine Länge $n$ mit folgender Eigenschaft: 
    falls $T\subseteq E$ mit $E$ auf Wörtern der Länge $\leq n$ übereinstimmt, dann $L(N_j^T)=\Sigma^*$.
\end{claim}
\begin{proof}
    Sei $s$ die Stufe bei der $\tau^1_j$ bearbeitet wurde, 
    %Angenommen $t_s(\tau^1_j)=0$. Dann hätten wir schon sofort einen Widerspruch: nach V3 wird $N_j$ eine Eingabe relativ zu $w_s$ definitiv ablehnen, mit Beobachtung~\ref{obs:partialoracles}(1) dann auch $L(N_j^O)\neq\Sigma^*$; Widerspruch zur Wahl von $N_j$.
    %Also gilt $t_s(\tau^1_j)>0$. 
    und setze $n=|w_{s}|$.
    Wir zeigen nun, dass dieses $n$ die behauptete Eigenschaft erfüllt.
    Angenommen, dies gilt nicht, dann existiert ein $T\subseteq E$ dass mit $E$ auf Wörtern der Länge $\leq n$ übereinstimmt, aber für ein Wort $z$ lehnt $N_j^T(z)$ ab.

    Sei $m\in\mathbb N$ so gewählt dass $m> n, p_j(|x|)$, und definiere das folgende partielle Orakel $v$, das genau für alle Wörter der Länge $\leq m$ definiert ist:
    \[ v[x] \defeq \begin{cases} T[x] & \text{falls $|x|\leq m$} \\ \bot & \text{sonst}. \end{cases}\]
    Es ist leicht zu sehen, dass $N_j^v(z)$ definitiv ablehnt, stimmen $v$ und $T$ ja auf allen Wörtern der Länge $\leq p_j(|x|)$ überein.
    Außerdem ist klar, dass $v\sqsupsetneq w_s \sqsupsetneq w_{s-1}$.

    Sei $t'=t_{s-1}\cup \{\tau^1_j\mapsto 0\}$
    Wir zeigen unten, dass $v$ auch $t_{s-1}$-gültig ist. Damit ist dann klar,
    dass $v$ auch $t'$-gültig ist: es kommt nur V3 bezüglich $N_j$ hinzu, aber wir haben ja bereits gesehen dass $N_j^v(z)$ definitiv ablehnt.

    %Damit wäre $v$ eine geeignete Erweiterung in Stufe $s$ und wir hätten $t_s=t'$. 
    Insbesondere ist dann während der Bearbeitung von Task $\tau^1_{j}$ in Stufe $s$ das Orakel $v$ eine mögliche $t'$-gültige Erweiterung von $w_{s-1}$, denn es ist $t'$-gültig und $v\sqsupsetneq w_{s-1}$. Also wäre nach Definition des Tasks $\tau^1_{j}$ dann $t_{s}=t'$ gesetzt worden,
    und wir hätten $t_s(\tau^1_j)=0$.
    Dann wäre aber auch $t_\omega(\tau^1_j)=0$, und nach voriger Behauptung~\ref{claim:myoracle-structure}(3) damit $L(N_j^E)\neq\Sigma^*$. Widerspruch zur Wahl von $N_j$.

    Es bleibt zu zeigen dass $v$ auch $t_{s-1}$-gültig ist. Angenommen, $v$ ist nicht $t_{s-1}$-gültig, dann muss eine Bedingung V1–V7 verletzt sein. Zur Erinnerung: $v\sqsupsetneq w_{s-1}$ und $w_{s-1}$ ist $t_{s-1}$-gültig.
    Wieder klar ist, dass V3, V4, V6 nicht verletzt sein können, denn jede definite Berechnung relativ zu $w_{s-1}$ ist auch eine definite Berechnung relativ zu $v$ nach Beobachtung~\ref{obs:partialoracles}(1).

    Angenommen V1 ist verletzt. Dann existiert ein $x<|v|$ und $|x|\neq e(\cdot)$ und $x\in v$.
    Nun stimmt $v$ und $T$ aber insbesondere auf $x$ überein, heißt wir haben
    \[ x\in v \iff x\in T \implies x\in E, \]
    das widerspricht aber der Beobachtung~\ref{claim:myoracle-structure}(1) dass $x\not\in E$.
    Widerspruch zur Annahme; V1 kann nicht verletzt sein.

    Angenommen, V5 ist verletzt. Dann existiert ein $\tau'$ mit $t_{s-1}(\tau')=m'>0$ und $n'\in H_{m'}$ und $|\Sigma^{n'}\cap v|>1$. Insbesondere gilt $t_\omega(\tau')=m'>0$.
    Außerdem muss $n'\leq m$ sein, da ansonsten $\Sigma^{n'}$ nicht durch $v$ definiert ist (und damit $v\cap\Sigma^{n'}=\emptyset$.
    Insbesondere stimmen damit $v$ und $T$ auf $\Sigma^{n'}$ überein.
    Damit gilt
    \[ |\Sigma^{n'}\cap v|=|\Sigma^{n'}\cap T|\leq |\Sigma^{n'}\cap E| \leq 1, \]
    wobei die erste Ungleichung aus $T\subseteq E$ folgt, und die zweite Ungleichung aus voriger Behauptung~\ref{claim:myoracle-structure}(4) bzw. (5) folgt.
    Das widerspricht der Annahme $|\Sigma^{n'}\cap v|>1$, also kann V5 nicht verletzt sein. Auf analoge Weise lässt sich zeigen, dass V2 und V7 nicht verletzt sein können.

    Damit ist keine Bedingung verletzt, $v$ muss also $t_{s-1}$-gültig sein.
\end{proof}

Nun können wir für die Hypothese $\hQ$ argumentieren.
Hierfür müssen wir, gegeben eine totale NPTM $N_j$, effizient einen akzeptierenden Rechenweg von $N_j^E(x)$ bestimmen.
Zur Erinnerung: wir haben in der Konstruktion angenommen, dass $\P=\PSPACE$.
Das Ausrechnen eines akzeptierenden Rechenwegs ist daher offensichtlich zumindest relativ zu $\emptyset$ (anstelle $E$) möglich, da unter unserer Annahme insbesondere $\P=\NP$ gilt.
%Beachte, dass wir $O$ als $C\cup D$ schreiben können, wobei 
%\[ D=O-C=\{ w \mid w\in O, \exists i. |w|=e(i) \}, \]
%und $C$ und $D$ sind insbesondere disjunkt.
%Beachte dass $C$ das ursprünglich gewählte PSPACE-vollständige Orakel ist, wobei die Ebenen der Höhe $e(\cdot)$ leer waren. Die Menge $D$ entspricht dann genau dem Inhalt dieser Lücken.

Hätten wir nun die Möglichkeit, effizient $E$ entscheiden zu können, dann wäre auch $\P^{E}=\NP^{E}$ und wir könnten einen akzeptierenden Rechenweg von $N_j^E(x)$ einfach ausrechnen.
Nun können wir zwar $E$ nicht effizient entscheiden, aber wir können mittels $N_j$ und Behauptung~\ref{claim:myoracle-totality} zumindest iterativ eine relevante finite Approximation $E'\subseteq E$ von $E$ bestimmen.
Das soll kurz skizziert werden: Gegeben unsere finite Approximation $T\subseteq E$, können wir relativ zu $T$ einen akzeptierenden Rechenweg $\alpha$ von $N_j(x)$ ausrechnen. Dieser existiert insbesondere nach Behauptung~\ref{claim:myoracle-totality}. Nun können wir die Orakelfragen auf $\alpha$ mit dem „echten“ Orakel $E$ abgleichen, und so überprüfen wo unsere Approximation fehlerhaft ist, und ggf. unsere Approximation aktualisieren. Nachdem $E$ dünn ist,  enden wir spätestens nach polynomiell vielen Iterationen bei einer Approximation $T$, die auf allen relevanten Wörtern mit $E$ übereinstimmt.


\begin{claim}\label{claim:myoracle-q}
    Sei $N_j$ eine totale NPTM, d.h. $L(N_j^E)=\Sigma^*$. Dann existiert eine Funktion $g\in \FP^E$ sodass $g(x)$ einen akzeptierenden Rechenweg von $M^E_j(x)$ ausgibt. Damit gilt nach Definition die Hypothese $\hQ$ relativ zu $E$.
\end{claim}
\begin{proof}
    %Es reicht aus, dass $g\in\FP^O$ nur Wörter hinreichender Länge verarbeiten muss.
    Sei $n$ hinreichend groß, sodass diese vorige Behauptung~\ref{claim:myoracle-totality} erfüllt ist.
    Damit gilt
    \begin{equation} \begin{split}&T\subseteq E, \text{$T$ stimmt mit Wörtern der Länge $\leq n$ mit $E$ überein} \\ &\implies L(N_j^{T})=\Sigma^*.\end{split}\label{eq:myoracle-1a} \end{equation}

    Wir betrachten nun folgenden formalen Algorithmus relativ zu $E$ auf Eingabe $x$:\\
    \SetKwFor{Loop}{wiederhole}{}{}
    \begin{algorithm}[H]
        $T\gets \{ w\mid w\in E, |w|\leq n, \exists i.|w|=e(i)\}$ \tcc{Konstante, muss nicht berechnet werden}
        \Loop(\tcp*[h]{Invariante: $T\subseteq E$}){}
        {
            Sei $\alpha$ ein akzeptierender Rechenweg auf $N_j^{T}(x)$ und $Q$ die Menge an Orakelfragen\;
            \uIf{existiert eine Frage $q\in Q$ für die $q\in E$ aber $q\not\in T$}
            {
                $T\gets E \cup \{q\}$\;
            }
            \Else(\tcp*[h]{für alle $q\in Q$ gilt: $q\not\in E\lor q\in T$}){
                Gebe {$\alpha$} aus\;
            }
        }
    \end{algorithm}

    Korrektheit: Beobachte zunächst die Invariante dass $T\subseteq E$.
    Damit gilt nach \eqref{eq:myoracle-1a} auch $L(N_j^{T})=\Sigma^*$ und insbesondere existiert dann auch ein akzeptierender Rechenweg auf $N_j^{T}(x)$. Damit ist Z.~3 wohldefiniert.
    Terminiert nun der Algorithmus mit einem Rechenweg $\alpha$ in Z.~7, 
    dann stimmt aber $T$ mit $E$ auf $Q$ überein: 
    Falls $q\in T$, dann auch $q\in E$ (nach Invariante).
    Falls $q\not\in T$, dann $q\in E$ (nach Negation der If-Bedingung).
    Also wird auch $N_j^E(x)$ mit Rechenweg $\alpha$ akzeptieren, nach Beobachtung~\ref{obs:partialoracles}(3).

    Laufzeit: Wir zeigen dass der Algorithmus in polynomiell beschränkter deterministischer Zeit (abhängig von $|x|$) relativ zu $E$ arbeitet. 
    Wir wissen, dass für jede Orakelfrage $q\in Q$ gilt, dass $|q|\leq p_j(|x|)$.
    Zusammen mit oben genannten Invariante gilt $T\subseteq E\cap \{w \mid \exists i.|w|=e(i)\leq p_j(|x|)\}$.
    Nun ist aber $E$ dünn. Im Speziellen gilt mit Behauptung~\ref{claim:myoracle-structure}(2) dass
\[ \ell(T) \leq \sum_{\mathclap{\substack{i\in\mathbb N\\ e(i)\leq p_j(|x|)}}} \ell(E\cap\Sigma^{e(i)}) = \sum_{\mathclap{\substack{i\in\mathbb N\\ e(i)\leq p_j(|x|)}}} e(i)\cdot |E\cap\Sigma^{e(i)}| \leq \sum_{\mathclap{\substack{i\in\mathbb N\\ e(i)\leq p_j(|x|)}}} 2e(i) \leq 2p_j(|x|)^2. \]
    Das heißt, dass nach polynomiell vielen Iterationen wird kein weiteres Wort zu $T$ hinzugefügt und der Algorithmus terminiert.

    Wir zeigen nun abschließend, dass Zeile 3 in polynomiell beschränkter deterministischer Zeit berechnet werden kann.
    Hierfür werden wir das finite $T$ in $N_j$ „hineincodieren“.
    Gemeint ist, $T$ in das „Programm“ $N_j$ hineincodieren, sodass für Queries $q, |q|=e(\cdot)$ nur getestet wird ob $q\in T$ anstelle einer echten Orakelfrage.
    Es existiert also eine Orakel-NPTM $N'_j$, sodass $\smash{N'_j(x)}$ (ohne Orakel) äquivalent zu $N_j^{T}(x)$ arbeitet. Und da $\smash{\P=\mathrm{PSPACE}}$, können wir in $\FP$ auch einen akzeptierenden Rechenweg $\beta'$ von $N'_j(x)$ effizient bestimmen.
    Das geht dann auch in $\FP^E\supseteq \FP$.
    Also können wir auch einen akzeptierender Rechenweg von $N_j^{E}(x)$ in $\FP^E$ berechnen.
    Beachte insbesondere, dass das Hineincodieren in Polynomialzeit möglich ist, ist ja $\ell(E)$ polynomiell in Abhängigkeit von $|x|$ beschränkt.
\end{proof}

Damit ist der Beweis von Satz~\ref{thm:myoracle-work} abgeschlossen: unter Annahme von $\P=\PSPACE$ erfüllt das Orakel $E$ alle Eigenschaften.
Relativ zu $E$ existiert nach Behauptung~\ref{claim:myoracle-completness} (1) kein $\leqmpp$-hartes Paar in $\DisjNP$ für $\DisjUP$, und (2) kein $\leqmp$-vollständige Menge für $\UP$.
Nach Behauptung~\ref{claim:myoracle-q} kann für jede totale NPTM $N_j$ eine Funktion $g\in\FP$ angegeben werden, für die $g(x)$ ein akzeptierender Rechenweg für $N_j(x)$ ist. Mit den Eigenschaften der Standardenumeration ist klar, dass dass auch für \emph{alle} totalen NPTM $N$ gilt. (Übersetze $N$ in äquivalente Standard-NPTM $N_j$, übersetze dann den akzeptierenden Rechenweg $g(x)=\alpha$ von $N_j(x)$ in einen akzeptierenden Rechenweg $\alpha'$ von $N(x)$ zurück.) Damit gilt relativ zu $E$ also auch die Hypothese $\hQ$ (3).

Es lässt sich einfach überprüfen, dass die vorhergegangene Konstruktion relativiert.
Sei nun $D$ ein $\PSPACE$-vollständiges Orakel.
Wie anfangs dieses Kapitels argumentiert, gilt nun mit $O\defeq E\oplus D$ auch Satz~\ref{thm:myoracle} \emph{ohne} die Annahme $\P=\PSPACE$.

Wir schließen dieses Kapitel mit einer Zusammenfassung weiterer Eigenschaften von $O$, die sich unmittelbar aus relativierenden Implikationen ergeben.
Sei hierfür $\mathrm{NP}b\mathrm{V}$ diejenige Teilmenge der Multifunktionen $f\in\NPMV$ für die $\fset{f}(x)\subseteq\{0,1\}$ für alle $x$ gilt, 
und sei $\mathrm{NP}k\mathrm{V}$diejenige Teilmenge der Multifunktionen $f\in\NPMV$ für die $|\fset{f}(x)|\leq k$ für alle $x$ gilt.
\begin{corollary}
    Folgende Aussagen gelten relativ zum Orakel $O$:
    \begin{enumerate}[noitemsep,midpenalty=0]\raggedright
        \item $\P=\UP\cap\coUP=\NP\cap\coNP\subsetneq \UP \subsetneq \NP$.
        \item $\mathrm{NP, UP, NE, NEE}$ sind nicht abgeschlossen unter Komplement. 
        \item $\mathrm{P, E, EE}$ sind nicht abgeschlossen unter Nichtdeterminismus.
        \item $\UP\not\subseteq \coNP$.
        \item $\NPSVt\subseteq\FP$.
        \item $\mathrm{NP}b\mathrm{V_t} \subseteq_{\mathrm{c}}\FP$.
        \item $\mathrm{NP}k\mathrm{V_t} \subseteq_{\mathrm{c}}\FP$ für alle $k\geq 1$.
        \item $\NPMVt \subseteq_{\mathrm{c}} \FP$.
        \item $\NPMVt \subseteq_{\mathrm{c}} \NPSVt$.
        \item $\NPMVt \subseteq_{\mathrm{c}} \TFNP$.
        \item $\mathrm{NPMV} \not\subseteq_{\mathrm{c}} \NPSV$.
        \item $\NPMVt \not\subseteq_{\mathrm{c}} \TFNP$.
        \item $\TFNP\subseteq_{\mathrm{c}} \FP$.
        \item Hypothese $\neg\hQ$.
        \item Hypothese $\neg\hQ'$.
        \item Es existiert eine NP-Relation $R$ mit $\leqmp$-vollständigem $\Proj(R)\in\NP$ und $R$ ist nicht $\leqlp$-vollständig für $\FNP$.
        \item $\NP\cap\coNP$ hat keine $\leqmp$-vollständige Menge.
        \item $\UP$ hat keine $\leqmp$-vollständige Menge.
        \item $\DisjNP$ hat kein $\leqmpp$-vollständiges Paar.
        \item $\NPSV$ hat keine $\leqmp$-vollständige Multifunktion.
        \item $\NPMVt$ hat eine $\leqmp$-vollständige Multifunktion.
        \item $\TFNP$ hat eine $\leqmp$-vollständige Multifunktion.
        \item $\DisjCoNP$ hat ein $\leqmpp$-vollständiges Paar.
        \item Für keine $\leqmp$-vollständige Menge $A\in\NP$ existiert ein $\P$-optimales Beweissystem.
        \item Für alle Mengen $A\in\coNP$ existiert ein $\P$-optimales Beweissystem.
        \item Es existiert ein $\P$-inseparables $\DisjNP$-Paar.
        \item Jedes $\DisjCoNP$-Paar ist $\P$-separierbar.
        \item Weder $\NP$ noch $\coNP$ haben die \emph{shrinking property} \parencite{glaser_shrinking_2011}.
        \item $\NP$ hat die \emph{separation property} \parencite{glaser_shrinking_2011}.
        \item $\coNP$ hat nicht die \emph{separation property}.
    \end{enumerate}
\end{corollary}
\begin{proof}
Es ist klar, dass sich aus Satz~\ref{thm:myoracle} unmittelbar (13), (16) und (17) ergeben.
Die meisten der anderen Aussagen erschließen sich sofort mit Zuhilfenahme der in Abbildung~\ref{fig:figure-implications} eingezeichneten Äquivalenzen und Implikationen, zusammen mit Satz~\ref{thm:figure-implications}.

Folgende Aussagen bleiben noch offen:
\begin{prooflist}[nosep,midpenalty=0]
\item Zu (1): $\P=\UP\cap\coUP=\NP\cap\coNP\subsetneq\UP$ klar aus der Abbildung, die Echtheit der Inklusion $\UP\subsetneq\NP$ folgt daraus, dass $\NP$ eine vollständige Menge hat, $\UP$ aber nicht, nach (16).
\item Zu (2): Angenommen $\UP=\coUP$ dann auch $\UP=\UP\cap\coUP=\P$, was ein Widerspruch zu (1) ist. Restliche Klassen sind ersichtlich über die Abbildung.
\item Zu (4): Angenommen $\UP\subseteq\coNP$, dann wäre $\UP\subseteq \UP\cap\coNP=\P$, was ein Widerspruch zu (1) ist.
\item Zu (6): Ist nach \textcite[Thm.~4]{fenner_inverting_2003} äquivalent zu (15).
\item Zu (7): Ist nach \textcite[Thm.~14]{fenner_inverting_2003} äquivalent zu (6).
\item Zu (9)–(10): Aus (8) folgt insbesondere $\NPMVt\subseteqc \FP_\mathrm{t}$ und $\FP_\mathrm{t}\subseteq \NPSVt, \TFNP$.
\item Zu (11): Es ist leicht zu sehen, dass (relativ zu jedem Orakel) eine kanonische $\leqmp$-vollständige Multifunktion $f$ für $\NPMV$ existiert. Angenommen $\mathrm{NPMV} \subseteq_{\mathrm{c}} \NPSV$, dann wäre auch $f\inc \NPSV$ und die Verfeinerung insbesondere auch $\leqmp$-vollständig für $\NPSV$.
    Das widerspricht (20). 
\item Zu (12): Angenommen $\NPMVt\subseteq \TFNP$, dann gilt mit Beobachtung~\ref{obs:npmvt-properin-tfnp} auch $\P=\NP$; das widerspricht (1).
\item Zu (28): Folgt aus (2) und (19) durch Ergebnisse von \textcite{glaser_shrinking_2011}. Wenn es kein $\leqmpp$-vollständiges Paar für $\DisjNP$ gibt, dann hat $\NP$ nicht die \emph{shrinking property}. Außerdem hat $\coNP$ die \emph{shrinking property} genau dann wenn $\NP=\coNP$.
\item Zu (29): Ist nach \textcite{glaser_shrinking_2011} äquivalent zu (6).
\item Zu (30): Folgt nach \textcite{glaser_shrinking_2011} aus (4).
\end{prooflist}
\end{proof}




