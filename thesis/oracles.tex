%! TEX root = ./thesis.tex
\chapter{Orakelkonstruktionen}\label{chap:orakel}

\todo{Intro}

\subsection*{Notation zur Orakelkonstruktion}

Zunächst sei $\{M_i\}_{i\in \mathbb N}$, $\{N_i\}_{i\in \mathbb N}$, $\{F_i\}_{i\in\mathbb N}$ eine sogenannte \emph{Standardaufzählung} der Orakel-PTMs bzw. Orakel-NPTMs, welche folgende Eigenschaften hat:
\begin{enumerate}[label=\arabic*.,nosep]
    \item Die Menge $\{N_i \mid i\in\mathbb N\}$ ist in Polynomialzeit erkennbar.
    \item Für jedes Orakel $D$ terminiert $N_i^D(x)$ nach höchstens $p_i(|x|)\defeq |x|^i+i$ Schritten.
    \item Für jede Orakel-NPTM $N$ existiert ein $i$ sodass $L(N^D_i)=L(N^D).$.
\end{enumerate}
Analoge Eigenschaften gelten für die Aufzählung $\{M_i\}$ der PTMs und $\{T_i\}$ der PTM-Transduktoren.

Um nun die beiden Orakel zu konstruieren, wenden wir ein Verfahren von \textcite{dose_np-completeness_2019} an, welche sich als ein starkes Framework erwiesen hat, Orakel zu konstruieren die simultan mehrere Eigenschaften erfüllen.
Dieses Framework werden wir im nächsten Abschnitt gleich beim Einsatz der ersten Konstruktion detailiert erläutern; in der zweiten Konstruktion auf eine detailierte Beschreibung aber weitestgehend verzichten.

Wesentliche Zutat dieses Frameworks sind „partiell definierte“ Orakel, bei denen also z.B. für gewisse $x$ noch nicht endgültig festgelegt ist, ob $x\in O$ oder $x\not\in O$ gelten soll.
Wir formalisieren solche partiell definierten Orakel mittels finiten Wörten $w\in\Sigma^*$:
ein finites Wort $w\in\Sigma^*$ können wir im Folgenden auch als die Menge $\{ i \mid i<|w|, w[i] = 1 \}$ verstehen (aber $|w|$ immer als die Länge von $w$).
Die intendierte Interpretation ist, dass gegenüber der \emph{Menge} $w$ die Zugehörigkeit aller Zahlen (bzw. äquivalent Wörtern) $x$ mit $x<|w|$ final speizifiziert ist, nicht dagegen für $x\geq|w|$ (und nur ersatzweise $x\not\in w$ gilt).

Diese Interpretation von finiten Wörtern $w$ als Orakel macht es einfacher, unsere Orakelkonstruktionen präzise und knapp zu beschreiben. Üblicherweise werden wir $w$ so erweitern, dass die Zugehörigkeit des kleinsten $x\in\mathbb N$ spezifiziert wird, welche noch nicht final spezifiziert ist. Dieses $x$ ist genau $|w|$ und wir legen die Zugehörigkeit final fest, indem wir an $w$ entweder $0$ anhängen (und $x$ ist final nicht im Orakel $w0$) oder $1$ anhängen (und $x$ ist final im Orakel $w1$).


Wir können relativ diesem partiell definierten Orakel auch rechnen. Definiere $M^w(x)$ entsprechend als $M^{\{i\mid w(i)=1\}}(x)$ (heißt, Orakel-Anfragen, für die $w$ nicht definiert ist, werden negativ beantwortet).
Dies ermöglicht es uns auch, folgenden Begriff zu definieren: Wir sagen, dass NPTM $M^w(x)$ \emph{definit} ist, wenn alle Anfragen auf allen Rechenwegen $<|w|$ sind (oder äquivalent: $w[q]$ ist für alle Anfragen $q$ auf allen Rechenwegen definiert); wir sagen, dass $M^w(x)$ \emph{definitiv akzeptiert} (bzw. \emph{definitiv ablehnt}), wenn $M^w(x)$ definit ist und akzeptiert (bzw. ablehnt). Intuitiv beschreibt der Begriff „definit“ Berechnungen, die sich nicht ändern, wenn das jeweilige partiell definierte Orakel erweitert wird, denn die Anfragen sind zu kurz.
\begin{observation}\label{obs:partialoracles}
    \begin{enumerate}
        \item Wenn $M^w(x)$ eine definite Berechnung ist und $v\sqsupseteq w$ ist, dann ist $M^v(x)$ definit. Die Berechnung $M^v(x)$ akzeptiert genau dann, wenn $M^w(x)$ akzeptiert.
        \item Wenn $w$ für alle Wörter der Länge $p_i(|x|)=|x|^i+i$ definiert ist, dann ist $M_i^w(x)$ definit.
        \item Wenn $M^w(x)$ auf einem Rechenweg mit der Menge der Orakel-Anfragen $Q$ akzeptiert, und $w$, $v$ auf $Q$ übereinstimmen, dann akzeptiert $M^v(x)$ auf dem gleichen Rechenweg und mit der gleichen Menge der Anfragen $Q$.
    \end{enumerate}
\end{observation}

Für ein partielles Orakel $w$, einen PTM-Transduktor $F$ und eine (N)PTM $M$ schreiben wir manchmal $M^w(F^w(x))$ als die \emph{eine} Berechnung der Maschine $M\circ F$ auf Eingabe $x$ relativ zu $w$.
Entsprechend sagen wir dann auch, dass $M^w(F^w(x))$ definit ist (bzw. definitiv akzeptiert, oder definitiv ablehnt) when $M\circ F$ definit auf Eingabe $x$ relativ zu $w$ ist (bzw. definitiv akzeptiert, defitiv ablehnt).

In unseren Orakelonstruktionen möchten wir abzählbar unendlich viele \emph{Ebenen} $n$, das heißt, Wörter gleicher Länge $n$, für eine abzählbar unendliche Familie von Zeugensprachen mit zunehmend großen Lücken injektiv reservieren und zuordnen.
Hierfür sei $e(0) \defeq 2$, $e(i) \defeq 2^{e(i-1)}$.
Es gibt eine polynomialzeit-berechenbare, polynomialzeit-invertierbare injektive Funktion $f$, die von $(m,h)\in\mathbb N\times\mathbb N$ auf $\mathbb N$ abbildet.
Definieren nun $H_m \defeq \{ e(f(m,h)) \mid h\in\mathbb N \}$ als die Menge der für die Zeugensprache $m$ reservierten Levels.
Diese Definition stellt nun Folgendes sicher:
\begin{observation}\label{obs:leveldefinitions}
    \begin{enumerate}
        \item Die Menge $H_m$ ist abzählbar unendlich, eine Teilmenge der geraden Zahlen, und alle $H_0, H_1, \dots$ sind paarweise disjunkt.
        \item Die Folge $\min H_0, \min H_1, \dots$ ist nach oben unbegrenzt.
        \item Wenn $n\in H_m$, dann gilt für jedes $a\in\mathbb N$: $n<a<2^{n} \implies a\not\in H_0, H_1, \dots$.
        \item Jede Menge $H_m\in \P$ für alle $m\in\mathbb N$.
    \end{enumerate}
\end{observation}
Vor der Konstruktion machen wir nun noch folgende kombinatorische Aussage:
\begin{lemma}\label{lemma:bipartite}
    Sei $G$ ein gerichteter bipartiter Graph mit den Knotenmengen $A$ und $B$.
    Das heißt, jede Kante in $G$ führt entweder von einem Knoten in $A$ zu einem Knoten in $B$ oder umgekehrt.
    Sei $\Delta$ eine obere Schranke für den Ausgangsgrad jedes Knotens in $G$.

    Wenn $|A|,|B|>2\Delta$ gilt, dann gibt es ein $a\in A$ und $b\in B$, sodass weder $(a,b)$ noch $(b,a)$ eine Kante in $G$ ist.
\end{lemma}
\begin{proof}
    Sei $n=\min\{|A|,|B|\}>2\Delta$.
    Entfernen Knoten aus $A$ und $B$, bis beide Knotenmengen jeweils genau $n$ Knoten haben, um einen gerichteten bipartiten Graphen $G'$ mit den Knotenmengen $A'$ und $B'$ zu bilden.
    Sei $G''$ der zugrunde liegende ungerichtete Graph von $G'$.
    In $G'$ gibt es $\leq |A'|\cdot \Delta + |B'|\cdot\Delta<n^2$ viele ungerichtete Kanten,
    aber $n^2$ viele ungerichtete Kanten im vollständigen bipartiten ungerichteten Graphen $K_{n,n}$.

    Das bedeutet, dass es $a\in A'\subseteq A$, $b\in B'\subseteq B$ gibt, die in $G''$ nicht adjazent sind; damit sind sowohl $(a,b)\not\in E(G')$ als auch $(b,a)\not\in E(G')$ im induzierten gerichteten bipartiten Teilgraphen $G'$.
    Also gilt für den ursprünglichen Graphen $G$ sowohl $(a,b)\not\in E(G)$ als auch $(b,a)\not\in E(G)$, wie gewünscht.
\end{proof}
\section{Orakel mit $\hDisjNP$, $\hUP$ und $\hQ$}

\todo{Fluff}

\begin{theorem}
    Es existiert ein Orakel $O$ sodass folgende Aussagen gelten:
    \begin{enumerate}
        \item Es existiert kein $\leqmpp$-vollständiges Paar für $\DisjNP^O$ (was $\hDisjNP$ relativ zu $O$ impliziert).
        \item Es existiert keine $\leqmp$-vollständige Menge für $\UP^O$ (was äquivalent zu $\hUP$ relativ zu $O$ ist).
        \item Für jede totale NPTM $N$ existiert eine Funktion $g\in\FP^O$ sodass $g(x)$ ein akzeptierender Rechenweg von $N^O(x)$ ist (was äquivalent zu $\hQ$ relativ zu $O$ ist).
    \end{enumerate}
\end{theorem}

\todo{Fluff}


Starte mit $\mathrm{PSPACE}$-vollständiger Menge $C$ welche keine Wörter der Länge $e(\cdot)$ enthält.
Definiere folgende Zeugensprachen:
\begin{gather*}
    A_m^O \defeq \{ 0^n \mid n\in H_m, \text{existiert $x\in \Sigma^{n}$ mit } x\in O \text{ und $x$ endet mit $0$} \}\\
    B_m^O \defeq \{ 0^n \mid n\in H_m, \text{existiert $x\in \Sigma^{n}$ mit } x\in O \text{ und $x$ endet mit $1$} \}\\
    C_m^O \defeq \{ 0^n \mid n\in H_m, \text{existiert $x\in \Sigma^{n}$ mit } x\in O  \}
\end{gather*}
%Fakt: wenn $|O\cap \Sigma^{n}|\leq 1$ für alle $n\in H_m$, dann $(A_m^O, B_m^O)\in\DisjUP^O$.\\
%Wenn $|O\cap \Sigma^n|\leq 1$ für alle $n\in H_m$, dann $C_m^O \in \UP^O$.
Sind die Ebenen der Höhe $H_m$ in $O$ auf geeignete Weise gefüllt, lässt sich leicht sehen dass diese entsprechenden Zeugensprachen in $\DisjUP$ bzw. $\UP$ fallen:
\begin{claim}
    \begin{enumerate}
        \item Wenn $|O\cap \Sigma^{n}|\leq 1$ für alle $n\in H_m$, dann $(A_m^O, B_m^O)\in\DisjUP^O$.
        \item Wenn $|O\cap \Sigma^n|\leq 1$ für alle $n\in H_m$, dann $C_m^O \in \UP^O$.
    \end{enumerate}
\end{claim}

\subsection*{Idee und Vorschau der Konstruktion}
\todo{Fluff}

Ein Orakel $w\in\Sigma^*$ ist $t$-valide wenn $t\in\mathcal T$ und folgendes gilt:
\begin{enumerate}[label={V\arabic*}]
    \item Wenn $x<|w|$ und $|x|\not\in \img(e)$, dann gilt $x\in w\iff x\in C$.\\
        (Orakel $w$ und $C$ stimmen auf Wörtern mit Länge $\neq e(\cdot)$ überein.)
    \item Für alle $i$ gilt $|w\cap \Sigma^{e(i)}|\leq 2$.\\
        (Orakel $w$ ist dünn auf den Ebenen der Länge $e(\cdot)$.)
    \item Wenn $t(\tau^1_j)=0$, dann existiert ein $z$ sodass $M_j^w(z)$ definitiv ablehnt.\\
        ($L(M_j)\neq \Sigma^*$ relativ zum finalen Orakel.)
    \item Wenn $t(\tau^2_{a,b})=0$, dann existiert ein $z$ sodass $M_a^w(z)$ und $M_b^w(z)$ definitiv akzeptieren.\\
        (Wenn $t(\tau^2_{a,b})=0$, dann $L(M_a)\cap L(M_b)\neq \emptyset$ relativ zum finalen Orakel.)
    \item Wenn $0<t(\tau^2_{a,b})=m$, dann gilt für alle $n\in H_m$ dass $|\Sigma^{n}\cap w|\leq 1$.\\
        (Wenn $0<t(\tau^2_{a,b})=m$, dann $(A_m,B_m)\in\DisjNP$.)
    \item Wenn $t(\tau^3_{a})=0$, dann existiert ein $z$ sodass $M_a^w(z)$ definitiv auf zwei Rechenwegen akzeptiert.\\
        (Wenn $t(\tau^3_{a})=0$, dann $L(M_a)\not\in \UP$ relativ zum finalen Orakel.)
    \item Wenn $0<t(\tau^3_{a})=m$, dann gilt für alle $n\in H_m$ dass $|\Sigma^n\cap w|\leq 1$.\\
        (Wenn $0<t(\tau^3_{a})=m$, dann $C_m\in\UP$.)
\end{enumerate}


\subsection*{Induktive Definition des Orakels}
\todo{Fluff}

Sei $T$ eine abzählbare Aufzählung der o.g. Tasks sodass $\tau^2_{a,b,r}$ immer nach $\tau^2_{a,b}$ kommt, sowie $\tau^3_{a,r}$ immer nach $\tau^3_a$ kommt.

\todo{Üblicher Text zur stufenweisen Erweiterung von $w_s$ und $t_s$}

Wir definieren nun Stufe $s>0$, diese startet mit einem $t_{s-1}\in\mathcal T$ und eine $t_{s-1}$-validen Orakel $w_{s-1}$ welche nun den kleinsten Task bearbeitet, welcher noch in $T$ ist. Dieser wird unmittelbar nach der Bearbeitung aus $T$ entfernt. In der Bearbeitung wird das Orakel strikt verlängert.
\begin{itemize}

    \item $\tau^1_j$: Setze $t'=t_{s-1}\cup\{\tau^1_j\mapsto 0\}$. Existiert ein $t'$-valides Orakel $v\sqsupsetneq w_{s-1}$, dann setze $t_s\defeq t'$ und $w_s\defeq v$.

        Ansonsten setze $t_s\defeq t_{s-1}$ und setze $w_s\defeq w_{s-1}y$ für geeignetes $y\in\{0,1\}$ sodass $w_s$ auch $t_s$-valide ist. (Das ist möglich nach Behauptung~\ref{claim:oracle-up-extension}.)

    \item $\tau^2_{a,b}$: Setze $t'=t_{s-1}\cup\{\tau^2_{a,b}\mapsto 0\}$. Existiert ein $t'$-valides Orakel $v\sqsupsetneq w_{s-1}$, dann setze $t_s\defeq t'$ und $w_s\defeq v$. Entferne außerdem alle Tasks der Form $\tau^2_{a,b,r}$ von $T$.

        Ansonsten wähle ein hinreichend großes $m\not\in \img(t_s)$ sodass $w_s$ kein Wort der Länge $\min H_m$ definiert. Setze $t_s\defeq t_{s-1}\cup \{ \tau^2_{a,b}\mapsto m \}$; damit ist $w_{s-1}$ auch $t_s$-valide. Setze $w_s\defeq w_{s-1}y$ für geeignetes $y\in\{0,1\}$ sodass $w_s$ auch $t_s$-valide ist. (Das ist möglich nach Behauptung~\ref{claim:oracle-up-extension}.)

    \item $\tau^2_{a,b,r}$: Wir wissen dass $t_{s-1}(\tau^2_{a,b})=m>0$. Setze $t_s=t_{s-1}$ und wähle ein $t_s$-valides Orakel $w_s\sqsupsetneq w_{s-1}$ sodass bezüglich einem $n\in\mathbb N$ eine der folgenden Aussagen gilt:
        \begin{itemize}[nosep,endpenalty=10000]
            \item $0^n\in A_m^v$ für alle $v\sqsupseteq w_s$ und $M_a(F_r(0^n))$ lehnt relativ zu $w_s$ definitiv ab.
            \item $0^n\in B_m^v$ für alle $v\sqsupseteq w_s$ und $M_b(F_r(0^n))$ lehnt relativ zu $w_s$ definitiv ab.
        \end{itemize} (Das ist möglich nach Behauptung 1.2.)

    \item $\tau^3_{a}$: Setze $t'=t_{s-1}\cup\{\tau^3_{a,b}\mapsto 0\}$. Existiert ein $t'$-valides Orakel $v\sqsupsetneq w_{s-1}$, dann setze $t_s\defeq t'$ und $w_s\defeq v$. Entferne außerdem alle Tasks der Form $\tau^3_{a,b,r}$ von $T$.

        Ansonsten wähle ein hinreichend großes $m\not\in \img(t_s)$ sodass $w_s$ kein Wort der Länge $\min H_m$ definiert. Setze $t_s\defeq t_{s-1}\cup \{ \tau^3_{a,b}\mapsto m \}$; damit ist $w_{s-1}$ auch $t_s$-valide. Setze $w_s\defeq w_{s-1}y$ für geeignetes $y\in\{0,1\}$ sodass $w_s$ auch $t_s$-valide ist. (Das ist möglich nach Behauptung 1.1.)

    \item $\tau^3_{a,r}$: Wir wissen dass $t_{s-1}(\tau^3_{a})=m>0$. Setze $t_s=t_{s-1}$ und wähle ein $t_s$-valides Orakel $w_s\sqsupsetneq w_{s-1}$ sodass bezüglich einem $n\in\mathbb N$ eine der folgenden Aussagen gilt:
        \begin{itemize}[nosep,endpenalty=10000]
            \item $0^n\in C_m^v$ für alle $v\sqsupseteq w_s$ und $M_a(F_r(0^n))$ lehnt relativ zu $w_s$ definitiv ab.
            \item $0^n\not\in C_m^v$ für alle $v\sqsupseteq w_s$ und $M_a(F_r(0^n))$ akzeptiert relativ zu $w_s$ definitiv.
        \end{itemize} (Das ist möglich nach Behauptung 1.3.)
\end{itemize}

\subsection*{Korrektheit der Definition}

Wir zeigen im Folgenden zunächst, dass die oben definierte induktive Definition tatsächlich wohldefiniert ist.
Anschließend verifizieren wir die Eigenschaften des konstruierten finalen Orakels, um so den Beweis von Satz~\ref{} abzuschließen.

Zunächst zeigen wir, dass sich ein $t$-valides Orakel $w$ immer um ein Bit verlängern lassen kann, ohne Validität zu verletzen.

\begin{lemma}\label{claim:oracle-up-extension}
    Sei $t\in\mathcal T$ und $w$ ein $t$-valides Orakel, und sei $z=|w|$. 
    (Denke $z$ als das nächste Wort, für welche wir die Zuordnung zum Orakel festlegen wollen, d.h. wahlweise $z\not\in w0$ oder $z\in w1$.)
    Dann existiert ein $b\in\{0,1\}$ sodass $wb$ auch $t$-valide ist.
    Insbesondere gilt:
    \begin{enumerate}
        \item Falls $|z|\not\in \img(e)$, dann ist $w0$ $t$-valide wenn $z\not\in C$, und $w1$ $t$-valide wenn $z\in C$.
        \item Falls $|z|=e(i)$ für ein $i$, und $|w\cap\Sigma^{e(i)}|> 1$, dann ist $w0$ $t$-valide.
        \item Falls ein $\tau=\tau^2_{a,b}$ oder ein $\tau=\tau^3_a$ existiert mit $t(\tau)=m>0$, und ein $n\in H_m$ mit $|z|=n$, und $w\cap\Sigma^{n}\neq\emptyset$, dann ist $w0$ $t$-valide.
        \item In allen anderen Fällen (heißt, wenn weder (1) noch (2) noch (3) zutreffen) ist $w0$ und $w1$ $t$-valide.
    \end{enumerate}
\end{lemma}
\begin{proof}
    Sei $z=|w|$. Beachte, dass wir mit $b$ kontrollieren, ob $z\in wb$ gilt.
    Angenommen, $wb$ ist nicht $t$-valide, dann muss eine der Bedingungen V1--V7 verletzt sein.
    Wir zeigen nun für jede dieser Bedingungen, dass eine Verletzung zu eine Widerspruch führt.

    Angenommen V3 ist verletzt, weil $M_j^{wb}(x)$ nicht definit ablehnt und gleichzeitig $t(\tau_j^1)=0$. Da nach Voraussetzung $w$ aber $t$-valide ist, wird $M_j^{w}(x)$ definitiv ablehnen.
    Nach Beobachtung~\ref{obs:partialoracles}(3) wissen wir aber, dass dann auch $M_j^{wb}(x)$ definitiv ablehnen wird. Widerspruch.
    Angenommen V4 oder V6 ist verletzt. Wir sehen auf analoge Weise wie bei V3 einen Widerspruch zur $t$-Validität von $w$.

    Angenommen V1 ist verletzt, weil für $x<|w0|$, $|x|\not\in\img(e)$ die Äquivalenz von V1 nicht gilt, heißt $x\in wb \not\leftrightarrow x\in C$. Dann kann $x$ nicht $<|w|$ sein, denn V1 gilt hier nach $t$-Validität von $w$. Also muss $x=z$.
    Damit auch $|z|\not\in\img(e)$ und wir haben für Fall (2)--(4) schon einen Widerspruch zur Voraussetzung.
    Wir sind also in Fall (1) und können voraussetzen, dass $b=1$ genau dann wenn $z\in C$.
    Dann gilt aber genau $x\in wb\leftrightarrow x\in C$ und wir erhalten einen Widerspruch.

    Es kann also nur V2, V5, oder V7 verletzt sein.
    Angenommen V2 ist verletzt, weil $|wb\cap\Sigma^{n}|>2$ für ein $i$ mit $n=e(i)$.
    Nach Definition wissen wir, dass $wb$ und $w$ auf allen Wörtern $\neq z$ übereinstimmen.
    Dann muss $z\in\Sigma^n$ sein, da sonst $wb$ und $w$ sogar auf den Wörtern $\Sigma^n$ übereinstimmen.
    Das bedeutet dass auch $|w\cap\Sigma^n|>2$, und widerspricht der $t$-Validität von $w$.
    \begin{itemize}[nosep]
        \item Fall (1) kann nicht zutreffen, da $|z|\in\img(e)$.
        \item Im Fall (2) haben wir $b=0$, und damit auch $|w\cap\Sigma^n|=|wb\cap\Sigma^n|>2$; wieder Widerspruch zur $t$-Validität von $w$.
        \item Im Fall (3) symmetrisch wie bei (2).
        \item Im Fall (4) wissen wir, dass Fall (2) nicht zutrifft, also muss $|w\cap\Sigma^n|\leq 1$.
            Dann haben wir $|wb\cap\Sigma^n|\leq |w\cap\Sigma^n|+1 \leq 2$; Widerspruch zur Annahme $|wb\cap\Sigma^{n}|>2$ oben.
    \end{itemize}

    Die Argumentation bei V5 läuft analog:
    angenommen V5 ist verletzt, weil für ein $\tau^2_{a,b}$ mit $m=t(\tau^2_{a,b})>0$ und einem $n\in H_m$ gilt, dass $|wb\cap\Sigma^n|> 1$.
    Zur Erinnerung: $wb$ und $w$ stimmen auf allen Wörtern $\neq z$ überein.
    Damit muss wieder $z\in\Sigma^n$ sein.
    \begin{itemize}[nosep]
        \item Fall (1) kann nicht zutreffen, da $|z|\in\img(e)$.
        \item Im Fall (2) haben wir $b=0$, und damit auch $|w\cap\Sigma^n|=|wb\cap\Sigma^n|>1$; wieder Widerspruch zur $t$-Validität von $w$.
        \item Im Fall (3) symmetrisch wie bei (2).
        \item Im Fall (4) wissen wir, dass Fall (3) nicht zutrifft, also muss $w\cap\Sigma^n=\emptyset$.
            Dann haben wir $|wb\cap\Sigma^n|\leq |w\cap\Sigma^n|+1 = 1$; Widerspruch zur Annahme $|wb\cap\Sigma^n|> 1$.
    \end{itemize}
    Auf analoge Weise lässt sich zeigen, dass Verletzung von V7 zum Widerspruch führt.

    Insgesamt also kann keine Bedingung V1--V7 verletzt sein; $wb$ ist $t$-valide wie gewünscht.
\end{proof}

Damit haben wir die ersten beiden Fälle in der Definition schon gesichert. Nun zeigen wir, dass die Bearbeitung von $\tau^2_{a,b,r}$ möglich ist.

\begin{lemma}
    Die Bearbeitung eines Tasks $\tau^2_{a,b,r}$ ist möglich: gilt $t_{s-1}(\tau^2_{a,b})=m>0$, dann lässt sich $w_{s-1}$ so zu $t_{s-1}$-validem $w\sqsupsetneq w_{s-1}$ erweitern, sodass eine der folgenden Aussagen gilt:
    \begin{enumerate}[nosep,endpenalty=10000]
        \item $0^n\in A_m^v$ für alle $v\sqsupseteq w$ und $M_a(F_r(0^n))$ lehnt relativ zu $w$ definitiv ab.
        \item $0^n\in B_m^v$ für alle $v\sqsupseteq w$ und $M_b(F_r(0^n))$ lehnt relativ zu $w$ definitiv ab.
    \end{enumerate}
\end{lemma}
\begin{proof}[Skizze.]
    Widerspruchsbeweis. Erweitere $w_{s-1}$ so weit zu $u$, dass genau alle Wörter der Länge $<n=e(i)\in H_m$ definiert sind, wobei das $i$ hinreichend groß gewählt wird. Sei für jedes $X\subseteq \Sigma^n$ das Orakel $u(X)\sqsupsetneq w_{s-1}$ jenes Orakel was entsteht, wenn die Ebene $e(i)$ mit genau den Wörtern aus $X$ gefüllt wird, heißt $u(X)$ und $X$ stimmen auf $\Sigma^n$ überein. Beob. dass $u(X), |X|\leq 1$ auch $t_{s-1}$-valide ist.

    Nach Annahme gilt
    \begin{itemize}[nosep]
        \item für $\alpha\in \Sigma^{n-1}0$ gilt $0^n\in A_m^{u(\{\alpha\})}$ und daher akzeptiert $M_a(F_r(0^n))$ relativ zu $u(\{\alpha\})$.
        \item für $\beta\in \Sigma^{n-1}1$ gilt $0^n\in B_m^{u(\{\alpha\})}$ und daher akzeptiert $M_b(F_r(0^n))$ relativ zu $u(\{\beta\})$.
    \end{itemize}
    Kombinatorische Standardmethoden zeigen dann, dass relativ zu $u(\{\alpha,\beta\})$ mit geeignetem $\alpha\in\Sigma^{n-1}0$, $\beta\in\Sigma^{n-1}1$ sowohl $M_a(F_r(0^n))$ also auch $M_b(F_r(0^n))$ relativ zu $u(\{\alpha,\beta\})$ akzeptieren.
    Damit wäre aber auch $u(\{\alpha,\beta\})$ ein geeignetes Orakel in der Bearbeitung von Task $\tau^2_{a,b}$ und wir hätten $t_{s-1}(\tau^2_{a,b})=0$.
\end{proof}
\begin{proof}
Wir fixieren die Werte von $a$, $b$ und $r$ im gesamten Beweis dieses Satzes.

Sei $\hat{s} < s$ die Stufe, die $\tau^2_{a,b}$ behandelt hat.
Eine solche Stufe existiert, da andernfalls $t_{s-1}(\tau^2_{a,b})$ undefiniert wäre.
Wir haben $m = t_{\hat{s}}(\tau^2_{a,b}) = t_{s-1}tau^2_{a,b})$; fixiere auch $m$ für den Rest des Beweises.

Wir nehmen an, dass für alle $t_{s-1}$-gültigen $w\sqsupsetneq w_{s-1}$ weder (1) noch (2) zutrifft.
Daraus werden wir einen Widerspruch ableiten, indem wir ein geeignetes Orakel $u'\sqsupsetneq w_{\hat{s}-1}$ konstruieren, das bezüglich $t' \defeq  t_{\hat{s}-1}\cup \{\tau^2_{a,b}\mapsto 0\}$ valide ist. (Gemeint ist: relativ zu $u'$ wird $M_a$ und $M_b$ eine Eingabe definitiv akzeptieren.)
Dann folgt nach Definition, dass $u'$ eine mögliche $t'$-valide Erweiterung von $w_{\hat{s}-1}$ in Stufe $\hat{s}$ ist, daher hätte die Bearbeitung von $\tau^2_{a,b}$ eben $t_{\hat{s}}=t'$ gesetzt, damit auch $t_{s-1}tau^2_{a,b})=t'(\tau^2_{a,b})=0$, was der Voraussetzung widerspricht.

Sei
\begin{equation*} \gamma(n) \defeq  \max(p_a(p_r(n))+p_r(n), p_b(p_r(n))+p_r(n)) \end{equation*}
das Polynom, das die Laufzeit von $M_a\circ F_r$, $M_b\circ F_r$ bezüglich der Eingabelänge $n$ relativ zu einem beliebigen Orakel beschränkt.
Das bedeutet immer dann, wenn ein partielles Orakel $u'$ für alle Wörter der Länge $\leq \gamma(n)$ definiert ist, auch $M_a^{u'}(F_r^{u'}(x))$, $M_b^{u'}(F_r^{u'}(x))$ für alle Eingaben $x\in\Sigma^n$ definit sind.
Sei $n\in\mathbb N$ eine geeignete Zahl sodass $n\in H_m$, und $w_{s-1}$ keine Wörter der Länge $\geq n$ definiert,
und
\begin{equation}\label{eq:oracle1-expbound}
    2^n > \gamma(n),\quad  2^{n-1} > 2\gamma(n).
\end{equation}
Die erste Ungleichung von \eqref{eq:oracle1-expbound} stellt sicher, dass kein Level $a$, $n<a\leq \gamma(n)$ für irgendeine Zeugensprache reserviert ist, das heißt, $a\not\in H_0, H_1, \dots$ (vgl. Beobachtung~\ref{obs:leveldefinitions}(3)). Die zweite Ungleichung stellt sicher, dass es genügend Wörter der Länge $n$ gibt, damit bestimmte kombinatorische Argumente funktionieren.
Die Ungleichung stellt sicher, dass es genügend Wörter der Länge $n$ gibt, damit bestimmte kombinatorische Argumente funktionieren.

Für den restlichen Beweis fixieren wir zusätzlich $n$.
Beachte, dass $\ell(Q)\leq\gamma(n)$ für $Q$ die Menge der Orakelqueries ist, die jeweils von der Berechnung $M_a(F_r(0^n))$ oder der Berechnung $M_b(F_r(0^n))$ gestellt werden.
Wir definieren nun $u\sqsupseteq w_{s-1}$ als ein $t_{s-1}$-gültiges partielles Orakel, das genau für alle Wörter bis zur Länge $<n$ definiert ist. Ein solches Orakel existiert nach Lemma~\ref{claim:oracle-up-extension}, indem man $w_{s-1}$ bitweise erweitert, so dass es $t_{s-1}$-gültig bleibt.

Für unseren Beweis betrachten wir nicht alle $t_{s-1}$-gültigen $w$, sondern vielmehr eine ausreichende Teilmenge davon.
Für $X \subseteq \Sigma^n, |X|\leq 2$ definieren wir ein partielles Orakel $u(X)\sqsupsetneq u$ als
%\[ u(X) = (u\cup X \cup C)\cap\Sigma^{\leq\gamma(n)} \]
\[
    u(X)[x] \defeq \begin{cases} u[x] & \text{falls $|x|\leq n$}\\
    X[x] & \text{falls $|x|=n$}\\
    C[X] & \text{falls $n<|x|\leq \gamma(n)$} \\ \bot & \text{sonst}, \end{cases}
\]
das für alle Wörter bis zur Länge $\leq\gamma(n)$ definiert ist, und so dass $u(X) \cap \Sigma^n = X$, das heißt, $u(X)$ und $X$ stimmen in $\Sigma^n$ überein.

Im Wesentlichen entsteht also $u(X)$ aus $u$, indem die Ebene $n$ mit $X$ gefüllt wird, und dann in den höheren Ebenen mit dem PSPACE-Orakel $C$ weitergemacht wird.
Es ist leicht zu sehen, dass für $|X|\leq 1$ das Orakel $u(X)$ sogar $t_{s-1}$-valide ist, indem iterativ $u$ erweitert wird.

\begin{claim}
    Sei $|X|\leq 1$. Dann ist $u(X)$ ein $t_{s-1}$-valides Orakel.
\end{claim}
\begin{proof}
    Seien $u_0, u_1, u_2, \dots, u_l$ eine nach Länge geordnete Aufzählung aller Präfixe von $u(X)$, die für mindestens alle Wörter der Länge $<n$ definiert sind, das heißt, 
    \[ u = u_0 \sqsubsetneq u_1 \sqsubsetneq u_2 \sqsubsetneq \dots \sqsubsetneq u_l = u(X) \]
    für ein geeignetes $l\in\mathbb N$.
    Wir zeigen induktiv, dass jedes $u_k$, $0\leq k\leq l$, $t_{s-1}$-valide ist.
    Daraus folgt, dass $u_l=u(X)$ auch $t_{s-1}$-valide ist.
    Der Basisfall ist klar, da $u_0 = u$ nach Konstruktion $t_{s-1}$-valide ist.

    Für den Induktionsfall von $u_k$ zu $u_{k+1}$ sei $z=|u_k|$ und $b=u(X)[z]$. Es gilt also $u_kb=u_{k+1}\sqsubseteq u(X)$, das heißt, $b$ ist das letzte Bit von $u_{k+1}$.
    Beachte dass $n\leq |z|\leq \gamma(n)$.
    Das Orakel $u_k$ ist nach Induktionshypothese $t_{s-1}$-valide.
    Wir zeigen nun aufgeteilt in zwei Fälle, dass $u_kb=u_{k+1}$ auch $t_{s-1}$-valide ist. Hierfür werden wir Lemma~\ref{claim:oracle-up-extension} in Bezug auf das $t_{s-1}$-valide $u_k$ verwenden.

    Gilt $n<|z|\leq \gamma(n)$, dann wissen wir nach Definition dass $b=C[z]$. Da nach \eqref{eq:oracle1-expbound} insbesondere $n<|z|<2^n$ gilt, wissen wir auch dass $|z|\neq e(\cdot)$. Damit ist Lemma~\ref{claim:oracle-up-extension}(1) anwendbar, und es folgt dass $u_kb$ auch $t_{s-1}$-valide ist (gilt ja $b=1$ genau dann wenn $z\in C$).

    Gilt ansonsten $|z|=n$, dann gilt nach Definition dass $b=X[z]$. %Insbesondere gilt $|z|=n\in H_m$, wobei $m=t_{s-1}(\tau^2_{a,b})$ nach Wahl. %Damit ist Lemma~\ref{claim:oracle-up-extension}(2) anwendbar.
    Da $n\in H_m$ gilt sofort, dass $|z|=e(\cdot)$ und Fall (1) von Lemma~\ref{claim:oracle-up-extension} nicht anwendbar ist.
    Wenn nun $b=0$, dann sind wir sofort fertig, da in Fällen (2), (3) und (4) je $u_k0=u_kb$ $t_{\hat{s}-1}$-valide ist.

    Sei also nun $b=1$. Dann muss auch $u_k\cap\Sigma^n=\emptyset$ gelten. Andernfalls wäre
    \[ |u_{k}1\cap\Sigma^{n}|>1 \implies |u(X)\cap\Sigma^{n}|>1 \implies |X|>1, \]
    was ein Widerspruch zur Wahl von $X$ ist.
    Damit können Fälle (2) und (3) nicht auftreten. Heißt Lemma~\ref{claim:oracle-up-extension}(4) ist anwendbar und $u_k1=u_kb$ ist $t_{s-1}$-valide wie gewünscht.
\end{proof}

Wir haben angenommen, dass für $t_{s-1}$-valide Orakel nicht (1) oder (2) gilt. Angewendet auf $u(X), |X|\leq 1$ bedeutet das
\begin{itemize}[nosep]
    \item Für $\alpha\in \Sigma^{n-1}0$ gilt $0^n\in A_m^{u(\{\alpha\})}$ und daher akzeptiert $M_a(F_r(0^n))$ definitiv relativ zu $u(\{\alpha\})$ auf einem Rechenweg mit Orakelmenge $Q_\alpha$.
    \item Für $\beta\in \Sigma^{n-1}1$ gilt $0^n\in B_m^{u(\{\alpha\})}$ und daher akzeptiert $M_b(F_r(0^n))$ definitiv relativ zu $u(\{\beta\})$ auf einem Rechenweg mit Orakelmenge $Q_\beta$.
\end{itemize}
Wir wollen nun ein Orakel $u'$ konstruieren, sodass sowohl $M_a$ als auch $M_b$ akzeptieren.
Hierfür wollen wir die beiden jewieligen akzeptierenden Rechenwege fixieren.
Wir stellen das sicher, indem wir $u'$ so wählen, dass $u'$ mit $u(\{\alpha\})$ auf $Q_\alpha$ übereinstimmt, dann wird auch $M_a(F_r(0^n))$ relativ zu $u'$ akzeptieren.
Symmetrisch stellen wir das auch für $M_b$ bzw. $\beta$ sicher, sodass auch $M_a(F_r(0^n))$ relativ zu $u'$ akzeptieren wird.

Hierzu müssen wir $\alpha\in\Sigma^{n-1}0$ und $\beta\in\Sigma^{n-1}1$ finden, die sich nicht gegenseitig „stören“.
Ein Wort $\alpha\in\Sigma^{n-1}0$ stört $\beta\in\Sigma^{n-1}1$ falls $\alpha\in Q_\beta$, 
symmetrisch stört ein Wort $\beta\in\Sigma^{n-1}0$ ein Wort  $\alpha\in\Sigma^{n-1}1$ falls $\beta\in Q_\alpha$.

Es existieren $\alpha\in\Sigma^{n-1}0$ und $\beta\in\Sigma^{n-1}1$ die sich nicht gegenseitig stören. Wir zeigen das über einen bipartiten „Störgraph“ $G$.
Setze 
\[ A=\Sigma^{n-1}0, B=\Sigma^{n-1}1 \]
als die zwei Knotenmengen von $G$.
Setze nun
\[ E = \{ (\alpha, \beta) \mid \alpha\in\Sigma^{n-1}0, \beta\in\Sigma^{n-1}1, \alpha\in Q_\beta\} \cup \{ (\beta, \alpha) \mid \alpha\in\Sigma^{n-1}0, \beta\in\Sigma^{n-1}1, \beta\in Q_\alpha\} \]
als Kantenmenge.
Beachte, dass der Ausgangsgrad $\Delta$ von $G$ höchstens $\gamma(n)$ sein kann, denn ein Rechenweg auf $M_a(F_r(0^n))$ (bzw. $M_b(\cdots)$) ist auf $\leq \gamma(n)$ viele Schritte beschränkt.

Nach \eqref{eq:oracle1-expbound} wissen wir nun, dass $|A|, |B|>2\Delta$. Mittels Lemma~\ref{lemma:bipartite} exisieren also $\alpha\in A = \Sigma^{n-1}0$, $\beta\in B=\Sigma^{n-1}1$ sodass weder $(a,b)\in E$ noch $(b,a)\in E$.
In anderen Worten, es gilt $\alpha\not\in Q_\beta$ und $\beta\not\in Q_\alpha$.

Betrachte nun $u(\{\alpha, \beta\})$. Wir zeigen nun zwei Dinge: erstens, dass $M_a(F_r(0^n))$ und $M_b(F_r(0^n))$ definitiv relativ zu $u(\{\alpha, \beta\})$ akzeptieren.
Zweitens anschließend, dass $u(\{\alpha,\beta\})$ auch $t'$-valide ist.

Für die erste Aussage beobachten wir, dass $u(\{\alpha\})$ und $u(\{\alpha, \beta\})$ auf $Q_\alpha$ übereinstimmen. Angenommen sie stimmen nicht überein, dann müssen sie nach Konstruktion auf $\beta$ nicht übereinstimmen. Dann wäre aber $\beta\in Q_\alpha$, was ein Widerspruch zur Wahl von $\beta$ ist.
Symmetrisch sehen wir, dass $u(\{\beta\})$ und $u(\{\alpha, \beta\})$ auf $Q_\beta$ ist.
Also akzeptieren $M_a(F_r(0^n))$ und $M_b(F_r(0^n))$ definitiv relativ zu $u(\{\alpha, \beta\})$.

Nun zur zweiten Aussage. Zunächst zeigen wir $t_{\hat{s}-1}$-Validität.
\begin{claim}
    Das Orakel $u(\{\alpha, \beta\})$ ist $t_{\hat{s}-1}$-valide.
\end{claim}
\begin{proof}
    Seien wieder $u_0, u_1, u_2, \dots, u_l$ eine nach Länge geordnete Aufzählung aller Präfixe von $u(\{\alpha, \beta\})$, die für mindestens alle Wörter der Länge $<n$ definiert sind, das heißt, 
    \[ u = u_0 \sqsubsetneq u_1 \sqsubsetneq u_2 \sqsubsetneq \dots \sqsubsetneq u_l = u(\{\alpha,\beta\}) \]
    für ein geeignetes $l\in\mathbb N$.
    Wir zeigen induktiv, dass jedes $u_k$, $0\leq k\leq l$, $t_{s-1}$-valide ist.
    Der Basisfall ist klar, da $u_0 = u$ nach Konstruktion $t_{s-1}$-valide ist, und damit ist leicht zu sehen dass $u$ auch $t_{\hat{s}-1}$-valide ist.

    Für den Induktionsfall von $u_k$ zu $u_{k+1}$ sei $z=|u_k|$ und $b=u(X)[z]$. 
    Beachte wieder dass $n\leq |z|\leq \gamma(n)$.
    Das Orakel $u_k$ ist nach Induktionshypothese $t_{\hat{s}-1}$-valide.

    Gilt $n<|z|\leq \gamma(n)$, dann wissen wir nach Definition dass $z=C[b]$. Wie bei Beweis von Behauptung~\ref{} ist Lemma~\ref{claim:oracle-up-extension}(1) anwendbar, und es folgt dass $u_kb$ auch $t_{\hat{s}-1}$-valide ist.

    Gilt ansonsten $|z|=n$, dann gilt nach Definition dass $b=X[z]$. 
    Da $n\in H_m$ gilt sofort, dass $|z|=e(\cdot)$ und Fall (1) von Lemma~\ref{claim:oracle-up-extension} nicht anwendbar ist.
    Wenn nun $b=0$, dann sind wir sofort fertig, da in Fällen (2), (3) und (4) je $u_k0=u_kb$ $t_{\hat{s}-1}$-valide ist.

    Sei also nun $b=1$. Wir zeigen, dass Fall (4) von Lemma~\ref{claim:oracle-up-extension} zur Anwendung kommt und daher $u_kb$ auch $t_{\hat{s}-1}$-valide ist.
    Hierfür zeigen wir, dass Fälle (2) und (3) nicht auftreten können.

    Angenommen, Fall (2)  von Lemma~\ref{claim:oracle-up-extension} kommt zur Anwendung, dann gilt insbesondere $|u_k\cap\Sigma^{n}|\geq 2$.
    Dann aber wäre
    \[ |u_k1\cap\Sigma^n|>2 \implies |u(\{\alpha,\beta\})\cap\Sigma^n|>2 \]
    was der Konstruktion von $u(\{\alpha,\beta\})$ widersprechen würde, da $u(\{\alpha,\beta\})\cap\Sigma^n=\{\alpha,\beta\}$.
    Also kann Fall (2) nicht zur Anwendung kommen.

    Angenommen, Fall (3) von Lemma~\ref{claim:oracle-up-extension} kommt zur Anwendung, dann existiert aber ein $\tau\in\dom(t_{\hat{s}-1})$, $\tau\neq\tau^2_{a,b}$ mit $t_{\hat{s}-1}(\tau)=m>0$, also auch $t_{s-1}(\tau)=m=t_{s-1}(\tau^2_{a,b})$, was einen Widerspruch zur Injektivität von $t_{s-1}\in\mathbb T$ auf dem Support ist. Also kann Fall (3) nicht zur Anwendung kommen.

    Also ist $u_kb=u_k1$ auch $t_{\hat{s}-1}$-valide mit Lemma~\ref{claim:oracle-up-extension}(4), wie gewünscht.
\end{proof}
Wir erinnern uns daran dass $t'=t_{s-1}\cup \{\tau^2_{a,b}\mapsto 0\}$.
Nun ist es leicht zu sehen, dass $u'\defeq u(\{\alpha,\beta\})$ auch $t'$-valide ist. Das Orakel $u'$ ist nach voriger Behauptung $t_{\hat{s}-1}$-valide, und bei der Erweiterung von $t_{\hat{s}-1}$ zu $t'$ kommt nur $\tau^2_{a,b}\mapsto 0$ hinzu. Das bedeutet, dass wir nur noch die Bedingung V4 bezüglich $a,b$ verifizieren müssen.
Sei nun $y=F_r(0^n)$ relativ zu $u'$.
Nach voriger Aussage oben wissen wir aber, dass $M_a(y)$ und $M_b(y)$ relativ zu $u'$ beide definitiv akzeptieren, wie von V4 verlangt.

Da nun also $u'\sqsupsetneq u \sqsupsetneq w_{\hat{s}-1}$ auch $t'$-valide ist, sind wir fertig und erreichen einen Widerspruch, wie anfangs argumentiert: während der Bearbeitung von Task $\tau^2_{a,b}$ in Stufe $\hat{s}$ war das Orakel $u'$ eine mögliche $t'$-valide Erweiterung von $w_{\hat{s}-1}$, denn es ist $t'$-valide und $u'\sqsupsetneq w_{\hat{s}-1}$. Also wäre nach Definition des Tasks $\tau^2_{a,b}$ dann $t_{\hat{s}}=t'$ gesetzt worden.
Damit wäre dann auch $t_{s-1}(\tau^2_{a,b})=t'(\tau^2_{a,b})=0$, was der Voraussetzung dieses Lemma~\ref{} widerspricht.
\end{proof}

\todo{Fluff}

\begin{lemma}
    Die Bearbeitung eines Tasks $\tau^3_{a,r}$ ist möglich: gilt $t_{s-1}(\tau^3_{a})=m>0$, dann lässt sich $w_{s-1}$ so zu $t_{s-1}$-validem $w\sqsupsetneq w_{s-1}$ erweitern, dass eine der folgenden Fälle eintritt:
        \begin{itemize}[nosep,endpenalty=10000]
            \item $0^n\in C_m^v$ für alle $v\sqsupseteq w$ und $M_a(F_r(0^n))$ lehnt relativ zu $w$ definitiv ab.
            \item $0^n\not\in C_m^v$ für alle $v\sqsupseteq w$ und $M_a(F_r(0^n))$ akzeptiert relativ zu $w$ definitiv.
        \end{itemize}
\end{lemma}
\begin{proof}
Wir verfahren wie im Beweis von Lemma~\ref{thm:disjnp}.
Um wieder einen Widerspruch abzuleiten, nehmen wir an, dass für alle $t_{s-1}$-gültigen $w\sqsupsetneq w_{s-1}$ weder (1) noch (2) zutrifft.
Definiere wieder identisch $u$ und $u(X)$.
Wieder gilt nach Behauptung~\ref{} dass $u(X)$ immer $t_{s-1}$-valide ist wenn $|X|\leq 1$.
Wir haben also:
\begin{itemize}[noitemsep]
    \item Die Berechnung $M_a(F_r(0^n))$ lehnt relativ zu $u(\emptyset)$ definitiv ab.
    \item Für $\xi\in \Sigma^{n}$ gilt $0^n\in C_m^{u(\{\xi\})}$ und daher akzeptiert $M_a(F_r(0^n))$ definitiv relativ zu $u(\{\xi\})$ auf einem Rechenweg mit Orakelmenge $Q_\xi$.
\end{itemize}

Beachte, dass $\xi\in Q_\xi$ ist, andernfalls stimmen $u(\emptyset)$ und $u(\{\xi\})$ auf $Q_\xi$ überein, damit folgt mit Beobachtung~\ref{obs:partialoracles}(3), dass $M_a(F_r(0^n))$ relativ zu $u(\emptyset)$ akzeptiert, was der Annahme widerspricht.

Wir fahren nun fort wie in Lemma~\ref{}.
Seien $\alpha\in\Sigma^{n}$, $\beta\in\Sigma^{n}$ zwei unterschiedliche Wörter, die sich nicht gegenseitig stören. Das heißt, $\alpha\not\in Q_\beta$, $\beta\not\in Q_\alpha$.
Diese zwei Wörter existieren nach dem gleichen kombinatorischen Argument wie bei Lemma~\ref{}.

Setze nun $u(\{\alpha, \beta\})$. Wir zeigen nun, dass $M_a(F^r(0^n))$ auf zwei unterschiedlichen Rechenwegen akzeptieren.
Wieder haben wir, dass $u(\{\alpha\})$ und $u(\{\alpha, \beta\})$ auf $Q_\alpha$ übereinstimmen, und symmetrisch $u(\{\beta\})$ und $u(\{\alpha, \beta\})$ auf $Q_\beta$ übereinstimmen.
Mittels Beobachtung~\ref{obs:partialoracles}(3) sehen wir also, dass es zwei Rechenwege von $M_a(F_r(0^n))$ relativ zu $u(\{\alpha, \beta\})$ gibt, welche definitiv akzeptieren: einer mit Orakelfragen $Q_\alpha$, und einer mit $Q_\beta$.
Diese zwei Rechenwege sind tatsächlich unterschiedlich:
einerseits gilt aus obiger Feststellung dass $\alpha\in Q_\alpha$, aber nach Wahl von $\beta$ ist $\alpha\not\in Q_\beta$. Auf einem Rechenweg wird also $\alpha$ erfragt, auf dem anderen nicht; also sind die zwei Rechenwege unterschiedlich.


Wie im Beweis von Lemma~\ref{} können wir jetzt $t'\defeq t_{\hat{s}-1}\cup\{\tau^3_{a} \mapsto 0\}$ setzen.
Wieder gilt nach Behauptung~\ref{} dass $u'\defeq u(\{\alpha, \beta\})$ auch $t_{\hat{s}-1}$-valide ist.
Dann lässt sich auch leicht sehen, dass $u'$ auch $t'$-valide ist. Es kommt V6 bezüglich $a$ hinzu, aber wir haben ja eben gesehen, dass $M_a$ auf zwei unterschiedlichen Rechenwegen definitiv akzeptiert.

Da nun also $u'\sqsupsetneq u \sqsupsetneq w_{\hat{s}-1}$ auch $t'$-valide ist, sind wir fertig und erreichen einen Widerspruch: während der Bearbeitung von Task $\tau^2_{a,b}$ in Stufe $\hat{s}$ war das Orakel $u'$ eine mögliche $t'$-valide Erweiterung von $w_{\hat{s}-1}$. Also wäre nach Definition des Tasks $\tau^2_{a,b}$ dann $t_{\hat{s}}=t'$ gesetzt worden.
Damit wäre dann auch $t_{s-1}(\tau^2_{a,b})=t'(\tau^2_{a,b})=0$, was der Voraussetzung dieses Lemma~\ref{} widerspricht.
\end{proof}

Damit ist die Konstruktion möglich. Sei $O\defeq \bigcup_{s\in\mathbb N} w_s$.

\subsection*{Eigenschaften vom konstruierten Orakel}

\begin{claim}
    \begin{enumerate}
        \item Kein Paar aus $\DisjNP^O$ ist $\leqmpp$-hart für $\DisjUP^O$.
        \item Keine Menge aus $\UP^O$ ist $\leqmp$-vollständig.
    \end{enumerate}
\end{claim}
\begin{proof}
    Wir zeigen hier nur (1), der Beweis für Aussage (2) folgt analog.

    Angenommen es existiert ein solches $\leqmpp$-hartes Paar $(U, U')\in\DisjNP^O$.
    Dann existieren über die Wahl der Standardenumeration auch zwei NPTM sodass $L(M_a^O)=U$, $L(M_b^O)=U'$.
    Betrachte den Task $\tau^2_{a,b}$; dieser wurde in der Konstruktion von $O$ in Stufe $\hat{s}$ bearbeitet.
    Wir erinnern uns, dass $w_{\hat{s}}\sqsubsetneq O$ ein $t_{\hat{s}}$-valides Orakel ist.

    Wenn $t_{\hat{s}}(\tau^2_{a,b})=0$ ist, dann besagt V4, dass eine Eingabe $x\in\Sigma^*$ existiert für die sowohl $M_a(x)$ als auch $M_b(x)$ definitiv relativ zu $w_{\hat{s}}$ akzeptieren. Da $w_{\hat{s}}\sqsubseteq O$ ist, akzeptieren $M_a(x)$ und $M_b(x)$ relativ zu $O$, nach Beobachtung~\ref{obs:partialoracles}(1).
    Das bedeutet aber $x\in U\cap U'$; Widerspruch zur Wahl von $U, U'$ als disjunkt.

    Daher können wir annehmen, dass $m\defeq  t_{\hat{s}}(\tau^2_{a,b})>0$ ist.
    Betrachte nun das Paar $(A^O_m, B^O_m)$. Wir zeigen mit V4, dass dieses Paar ein disjunktes UP-Paar ist.
    %Nehmen Sie andernfalls an, dass gemäß der Kontraposition von Beobachtung~\ref{obs:witnesses}(i) für einige $n$ gilt $|O\cap \Sigma^n|>1$.
    Andernfalls gilt mit Kontraposition von Beobachtung~\ref{}(1), dass $|O\cap\Sigma^n|>1$ für ein $n$.
    Wähle nun ein hinreichend großes $s'$, sodass $w_{s'}$ für alle Wörter der Länge $n$ definiert ist.
    Da $w_{s'}$ $t_{s'}$-gültig ist, ist V4 anwendbar und sagt aus, dass $|\Sigma^n\cap w_{s'}|\leq 1$ ist. Da $w_{s'}$ so gewählt ist, dass es für alle Wörter der Länge $n$ definiert ist, stimmen $w_{s'}$ und $O$ in $\Sigma^n$ überein. Daher erhalten wir $|\Sigma^n\cap O|\leq 1$, was im Widerspruch zur Annahme steht.

    Daher gilt $(A^O_m, B^O_m)\in\DisjUP^O$. Also gilt nach Annahme auch $(A^O_m, B^O_m) \leqmpp (U, U')$.
    Sei nun $r$ so gewählt, dass diese Reduktion von $F^O_r$ realisiert wird, und betrachte Task  $\tau^2_{i,j,r}$, welcher in einem bestimmten Schritt $s$ behandelt wird.
    Nach Definition dieses Task existiert ohne Beschränkung der Allgemeinheit ein $n\in\mathbb N$ sodass $0^n\in A_m^O$, und $M_a(F_r(0^n))$ definitiv relativ zu $w_s$ ablehnt.

    Mit Beobachtung~\ref{obs:partialoracles}(1) erhalten wir dann, dass $M_a(F_r(0^n))$ relativ zu $O$ ablehnt. Also auch $F^O_r(0^n)\not\in U$. Dies widerspricht der Annahme, dass $F_r^O$ die Reduktion realisiert, denn wir haben ja $0^n\in A_m^O$.
\end{proof}

\begin{claim}
    Es gilt $C\subseteq O$ und $O$ und $C$ stimmen auf allen Wörtern der Länge $\neq e(\cdot)$ überein.
\end{claim}


\begin{claim}
    Sei $M_j$ eine totale Maschine, d.h. $L(M_j^O)=\Sigma^*$.
    Es existiert eine Länge $n$ mit folgender Eigenschaft: 
    falls $T\subseteq O$ mit $O$ auf Wörtern der Länge $\neq e(\cdot)$ und Wörtern $\leq n$ übereinstimmt, dann $L(M_j^T)=\Sigma^*$.
\end{claim}
\begin{proof}
    Sei $s$ die Stufe bei der $\tau^1_j$ bearbeitet wurde, 
    %Angenommen $t_s(\tau^1_j)=0$. Dann hätten wir schon sofort einen Widerspruch: nach V3 wird $M_j$ eine Eingabe relativ zu $w_s$ definitiv ablehnen, mit Beobachtung~\ref{obs:partialoracles}(1) dann auch $L(M_j^O)\neq\Sigma^*$; Widerspruch zur Wahl von $M_j$.
    %Also gilt $t_s(\tau^1_j)>0$. 
    und setze $n=|w_{s}|$.
    Wir zeigen nun, dass dieses $n$ die behauptete Eigenschaft erfüllt.
    Angenommen, dies gilt nicht, dann existiert ein $T\subseteq O$ dass mit $O$ auf Wörtern der Länge $\neq e(\cdot)$ und Wörtern $\leq n$ übereinstimmt, aber für ein Wort $z$ lehnt $M_j^T(z)$ ab.

    Sei $m\in\mathbb N$ so gewählt dass $m> n, p_j(|x|)$, und definiere das folgende partielle Orakel $v$, das genau für alle Wörter der Länge $\leq m$ definiert ist:
    \[ v[x] \defeq \begin{cases} T[x] & \text{falls $|x|\leq m$} \\ \bot & \text{sonst}. \end{cases}\]
    Es ist leicht zu sehen, dass $M_j^v(z)$ definitiv ablehnt, stimmen $v$ und $T$ ja auf allen Wörtern der Länge $\leq p_j(|x|)$ überein.
    Außerdem ist klar, dass $v\sqsupsetneq w_s \sqsupsetneq w_{s-1}$.

    Sei $t'=t_{s-1}\cup \{\tau^1_j\mapsto 0\}$
    Wir zeigen nun, dass $v$ auch $t'$-valide ist. Damit wäre $v$ eine geeignete Erweiterung in Stufe $s$ und wir hätten $t_s=t'$. 



    Damit wäre während der Bearbeitung von Task $\tau^1_{j}$ in Stufe $s$ das Orakel $v$ eine mögliche $t'$-valide Erweiterung von $w_{s-1}$, denn es ist $t'$-valide und $v\sqsupsetneq w_{s-1}$. Also wäre nach Definition des Tasks $\tau^1_{j}$ dann $t_{s}=t'$ gesetzt worden,
    und wir hätten $t_s(\tau^1_j)=0$.

    Damit erreichen wir einen Widerspruch: nach V3 wird $M_j$ eine Eingabe relativ zu $w_s$ definitiv ablehnen, mit Beobachtung~\ref{obs:partialoracles}(1) dann auch $L(M_j^O)\neq\Sigma^*$; Widerspruch zur Wahl von $M_j$.

    Seien $v_0, v_1, v_2, \dots, v_l$ eine nach Länge geordnete Aufzählung aller Präfixe von $v$, die mindestens $w_{s-1}$ als Präfix haben, das heißt, 
    \[ w_{s-1} = v_0 \sqsubsetneq v_1 \sqsubsetneq v_2 \sqsubsetneq \dots \sqsubsetneq v_l = v \]
    für ein geeignetes $l\in\mathbb N$.
    Wir zeigen induktiv, dass jedes $v_k$, $0\leq k\leq l$, $t_{s-1}$-valide ist.
    Basisfall klar.
    Für den Induktionsfall von $v_k$ zu $v_{k+1}$ sei $z=|v_k|$ und $b=v[z]$. 
    Das Orakel $v_k$ ist nach Induktionshypothese $t_{s-1}$-valide.

    Gilt $|z|\not\in\img(e)$, dann wissen wir dass $b = v[z] = O[z]$, und nach voriger Behauptung~\ref{} dann auch $b = O[z] = C[z]$.
    Mit Lemma~\ref{claim:oracle-up-extension}(1) gilt dann sofort, dass $v_kb$ auch $t_{s-1}$-valide ist.

    Gilt $|z|=e(i)$ für ein $i$, dann 
    
\end{proof}

\begin{claim}
    Sei $M_j$ eine totale Maschine, d.h. $L(M_j^O)=\Sigma^*$. Dann existiert eine Funktion $g\in \FP^O$ sodass $g(x)$ einen akzeptierenden Rechenweg von $M^O_j(x)$ ausgibt. Damit gilt nach Definition die Hypothese $\hQ$ relativ zu $O$.
\end{claim}
\begin{proof}
    %Es reicht aus, dass $g\in\FP^O$ nur Wörter hinreichender Länge verarbeiten muss.
    Sei $n$ hinreichend groß, sodass diese vorige Behauptung 1.6 erfüllt ist.
    Damit gilt: 
\begin{equation} \begin{split}&T\subseteq O, \text{$T$ stimmt mit Wörtern der Länge $\neq e(\cdot)$ und der Länge $\leq n$ mit $O$ überein} \\ &\implies L(M_j^T)=\Sigma^*.\end{split}\label{eq:oracle-1a} \end{equation}

    Sei exemplarisch $x\in\Sigma^*$ gegeben. 
    Wir werden obige Eigenschaft ausnutzen und iterativ eine Menge $D\subseteq O$ an Orakelwörtern der Länge $e(\cdot)$ aufbauen, welche für die Berechnung $M_j^O(x)$ relevant ist, bis wir alle solchen relevanten Wörter gefunden haben.
    Wir starten hierbei mit der Menge aller Orakelwörter in $O$, welche Länge $\leq n$  und Länge $=e(\cdot)$ haben, d.h. setze zunächst
    \[ D\gets \{ w\mid w\in O, |w|\leq n, \exists i.|w|=e(i)\}. \]
Beob. dass damit $C\cup D\subseteq O$ und mit $O$ auf Wörtern der Länge $\neq e(\cdot)$ und Wörtern der Länge $\leq n$ übereinstimmt, also nach \eqref{eq:oracle-1a} auch $L(M_j^{C\cup D})=\Sigma^*$.
    Beachte dass $C$ und $D$ disjunkt sind. Damit können wir auch das (finite) $D$ sogar in $M_j$ hineincodieren.
    Gemeint ist, $D$ in das „Programm“ $M_j$ hineincodieren, sodass für Queries $q, |q|=e(\cdot)$ nur getestet wird ob $q\in D$ anstelle einer echten Orakelfrage.
    Es existiert also eine Orakel-NPTM $M'_j$, sodass $\smash{M'^C_j(x)}$ äquivalent zu $M_j^{C\cup D}(x)$ arbeitet. Und da $\smash{\P^C=\mathrm{PSPACE}^C}$, können wir in $\FP^C$ auch einen akzeptierenden Rechenweg von $M'^C_j(x)$ effizient bestimmen.
    Das geht dann auch in $\FP^O\supseteq \FP^C$. (Jeder $\FP^C$-Algorithmus kann simuliert werden, imdem Orakelfragen mit Länge $e(\cdot)$ negativ beantwortet werden.)
    Damit ergibt sich: gegeben $D$ (unter den genannten Eigenschaften), kann ein akzeptierender Rechenweg von $M_j^{C\cup D}(x)$ in $\FP^O$ berechnet werden.

    Wir betrachten nun folgenden formalen Algorithmus relativ zu $O$ auf Eingabe $x$:\\
    \SetKwFor{Loop}{wiederhole}{}{}
    \begin{algorithm}[H]
        $D\gets \{ w\mid w\in O, |w|\leq n, \exists i.|w|=e(i)\}$ \tcc{Konstante, muss nicht berechnet werden}
        \Loop(\tcp*[h]{Invariante: $D\subseteq O\cap \{ w \mid \exists i.|w|=e(i)\}$}){}
        {
            Sei $\alpha$ ein akzeptierender Rechenweg auf $M_j^{C\cup D}(x)$ und $Q$ die Menge an Orakelfragen\;
            \eIf{existiert eine Frage $q\in Q$ für die $q\in O-C$ aber $q\not\in D$}
            {
                $D\gets D \cup \{q\}$\;
            }
            {
                Gebe {$\alpha$} aus\;
            }
        }
    \end{algorithm}

    Korrektheit: Beobachte zunächst die Invariante dass $D\subseteq O\cap \{ w \mid \exists i.|w|=e(i)\}$.
    Wie oben argumentiert gilt außerdem, dass $C\cup D\subseteq O$ und mit $O$ auf Wörtern der Länge $\neq e(\cdot)$ und Wörtern der Länge $\leq n$ übereinstimmt. Damit nach \eqref{eq:oracle-1a} auch $L(M_j^{C\cup D})=\Sigma^*$ und insbesondere existiert dann auch ein akzeptierender Rechenweg auf $M_j^{C\cup D}(x)$. Damit ist Z.~3 wohldefiniert.
    Terminiert nun der Algorithmus mit einem Rechenweg $\alpha$ in Z.~7, wissen wir auch dass für alle Orakelfragen $q\in Q$ entweder $q\in C$ gilt oder $q\not\in O$ oder $q\in O, D$ gilt.
    Damit stimmt $C\cup D$ mit $O$ auf $Q$ überein, und auch $M_j^O(x)$ akzeptiert mit Rechenweg $\alpha$.

    Laufzeit: Wir zeigen dass der Algorithmus in polynomiell beschränkter deterministischer Zeit (abhängig von $|x|$) relativ zu $O$ arbeitet. 
    Wir wissen, dass für jede Orakelfrage $q\in Q$ gilt, dass $|q|\leq p_j(|x|)$.
    Zusammen mit oben genannten Invariante gilt $D\subseteq O\cap \{w \mid \exists i.|w|=e(i)\leq p_j(|x|)\}$.
    Nach V2 gilt damit $\ell(D)\leq p_j(|x|)\cdot p_j(|x|)\cdot 2$, denn in den je $\leq p_j(|x|)$ Ebenen der Länge $e(i)\leq p_j(|x|)$ existieren höchstens zwei Wörter der Länge $e(i)\leq p_j(|x|)$.
    Damit folgt auch unmittelbar, dass der Algorithmus nach höchstens polynomiell vielen Iterationen terminiert.

    Zeile 3 kann damit auch in polynomiell beschränkter deterministischer Zeit berechnet werden. Wie oben skizziert kann der Rechenweg in deterministisch polynomieller Zeit abh. von $|x|$ und $\ell(D)$ relativ zu $C$ bestimmt werden.
    Nach Konstruktion ist leicht zu sehen, dass dieser Rechenweg dann auch relativ zu $O$ bestimmt werden kann.
\end{proof}


\section{Orakel mit $\mathsf{KvL}$ und $\neg\hSAT$}




