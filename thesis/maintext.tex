%! TEX root = ./thesis.tex
\chapter{Einleitung}

\chapter{Zur Konzeptionalisierung und Ordnung von Suchproblemen}

\section{Suchprobleme und Levin-Reduktionen}

%\begin{itemize}
    %\item Vllt. sollten wir NP-Relationen als FNP-Probleme definieren. Andererseits wird in der aktuellen Literatur (T)FNP primär auf Funktionen verstanden.
    %\item NP-Relation definieren: Zertifikate mind. so lange wie Probleme, alternativ streng monoton steigend
    %\item Levin-Reduktion definieren
%\end{itemize}

\begin{definition}[NP-Relation]
    Eine \emph{NP-Relation} ist eine zweistellige Relation $R\subseteq \Sigma^*\times\Sigma^*$, sodass diese
    \begin{enumerate}
        \item in Polynomialzeit entscheidbar ist, d.h., $R\in\P$, und
        \item ein Polynom $q$ existiert, sodass
            \begin{equation}\label{eq:zertifikatsschranke}
                (x,y)\in R \implies |y|\leq q(|x|) \quad\text{für alle $x,y\in\Sigma^*$}.
            \end{equation}
    \end{enumerate}
    Die Wörter der ersten Komponente nennen wir \emph{Probleminstanzen} oder \emph{Instanzen} oder emph{Probleme} von $R$, die Wörter der zweiten Komponente nennen wir die \emph{Zertifikate} von $R$. Wir sagen dann für $(x,y)\in R$, dass \emph{$y$ ein Zertifikat für $x$} ist. In diesem Sinne sagt (\ref{eq:zertifikatsschranke})  aus, dass Zertifikate $y$ für $x$ nicht superpolynomiell länger als $x$ sein dürfen.
    Das Polynom $q$ nennen wir auch die \emph{Zertifikatsschranke} zu $R$. 

    Für eine NP-Relation definieren wir
    \[ \Proj(R) = \{ x \mid \exists y\in\Sigma^*, (x,y)\in R \} \in \NP \]
    Die Menge $\Proj(R)$ ist also die Menge der Probleminstanzen, für welche ein zugehöriges Zertifikat existiert; damit entspricht $\Proj(R)$ derjenigen Menge, die üblicherweise bei algorithmischen Entscheidungsproblemen betrachtet wird. 
    %Wir sagen in diesem Sinne auch, dass $R$ eine NP-Relation \emph{für $L$} ist, wenn $L=\Proj(R)$. 
    Die Zugehörigkeit $\Proj(R)\in\NP$ folgt unmittelbar aus den Punkten (1) und (2).

    Wir definieren 
    \[ R(x) = \{ y \mid y\in\Sigma^*, (x,y)\in R\}  \]
    als die Menge aller Zertifikate $y$ zu $x$. Damit gilt unmittelbar
    \[ x\in \Proj(R) \iff R(x) \neq\emptyset. \qedhere \] 
\end{definition}

\begin{definition}[Levin-Reduzierbarkeit]
    Seien $Q, R$ zwei NP-Relationen. Wir sagen dass \emph{$Q$ sich auf $R$ (Polynomialzeit-)Levin-reduzieren lässt}, bzw. $Q\leqlp R$ wenn zwei Funktionen $f, g\in \FP$ existieren sodass
    \begin{enumerate}
        \item $x\in\Proj(Q) \iff f(x)\in\Proj(R)$,
        \item $(f(x), y)\in R \implies (x, g(x,y))\in Q$.
    \end{enumerate}
    Punkt (1) sagt also nur aus, dass $f$ eine Many-one-Polynomialzeit-Reduktion zwischen den entsprechenden Entscheidungsproblemen ist.
    Punkt (2) sagt nun aus, dass wenn $y$ ein Zertifikat für die Instanz $f(x)$ aus $R$ ist, dann lässt sich aus $y$ wieder ein Zertifikat $g(x,y)$ für die originale Instanz $x$ berechnen.

    Die Funktion $f$ nennen wir \emph{Reduktionsfunktion}, die Funktion $g$ nennen wir \emph{Translationsfunktion}.

    Wir schreiben $Q\leq_\mathrm{L,1}^\mathrm p R$ falls $f$ zusätzlich injektiv ist. Wir schreiben $Q\leq_\mathrm{L,1,inv}^\mathrm p R$ falls $f$ zusätzlich injektiv und p-invertierbar ist. Klar ist:
    \[ Q\leq_\mathrm{L,1,inv}^\mathrm p R \implies Q \leq_\mathrm{L,1}^\mathrm p R \implies Q \leqlp R \implies \Proj(Q) \leqmp \Proj(R). \]

    Wir sagen dass $R$ \emph{$\leqlp$-vollständig} ist, wenn $Q\leqlp R$ für alle NP-Relationen $Q$ gilt. Die $\leq_\mathrm{L,1}^\mathrm p$- und $\leq_\mathrm{L,1,inv}^\mathrm p$-Vollständigkeit ist analog definiert.
\end{definition}

\begin{theorem}
    Die kanonische NP-Relation
    \[ \mathtt{rKAN} = \{ ((N,x,1^n), \alpha) \mid \text{$\alpha$ ist ein akz. Rechenweg auf $N(x)$ und $|\alpha|\leq n$} \} \]
    ist $\leq_\mathrm{L,1,inv}^\mathrm p$-vollständig.
\end{theorem}
\begin{proof}
    Sei $R$ eine beliebige NP-Relation mit Zertifikatsschranke $r$, i.e. $(x,y)\in R\implies |y|\leq r(|x|)$. Sei $M$ die PTM welche $R$ entscheidet, mit Laufzeitschranke $p$. Sei $N$ eine NPTM welche auf Eingabe $x$ zunächst ein Zertifikat $y, |y|\leq r(|x|)$ rät, und dann testet ob $M(x,y)$ akzeptiert. Die Laufzeit von $N$ ist beschränkt auf $p(|(x,y)|)\in O(p(r(|x|)))$ (hier nutzen wir die effiziente Listencodierung von \ref{} aus). Sei daher $q$ ein Polynom, welches die Laufzeit von $N$ beschränkt.

    Definiere die Reduktionsfunktion $f(x)=(N, x, 1^{q(|x|)})$. Wir zeigen zunächst dass
    \[ x\in \Proj(R)\iff f(x)\in \Proj(\mathtt{rKAN}). \]
    Wenn $x\in\Proj(R)$, dann existiert ein $y, |y|\leq r(|x|)$ sodass $(x,y)\in R$. Dann wird auch $N(x)$ akzeptieren, nämlich auf jenem Pfad welcher $y$ rät. Es existiert also ein Rechenweg $\alpha$ mit $|\alpha|\leq q(|x|)$ sodass $N(x)$ auf $\alpha$ akzptiert. Dann gilt aber auch $(f(x), \alpha)=((N,x,1^{q(|x|)}),\alpha)\in \mathtt{rKAN}$.
    Die Rückrichtung $x\not\in \Proj(R)\implies f(x)\not\in\Proj(R)$ folgt analog.
    Es ist klar, dass $f$ injektiv ist, dass $f$ Polynomialzeit-berechenbar und -invertierbar ist. 

    Es lässt sich außerdem einfach eine Translationsfunktion $g\in \FP$ angeben, die für $g(f(x), \alpha)=y$ aus $\alpha$ das entsprechende geratene Zertifikat $y$ aus $\alpha$ berechenen kann.
\end{proof}

\section{Zur gemeinsamen Struktur von vollständigen Suchproblemen}





\begin{observation}[\cite{buhrman_functions_1998}]
    Sei $L$ eine beliebige Menge und sei $R$ eine $\leqlp$-vollständige NP-Relation.
    Gilt $L \leq_\mathrm{1,inv}^\mathrm p\Proj(R)$, dann existiert eine $\leqlp$-vollständige NP-Relation $R$ mit $\Proj(R)=L$.
    \todo{Das kann zu universellem $R$ verstärkt werden!}
\end{observation}
\begin{proof}
    Nach Voruassetzung haben wir eine p-invertierbare Funktion $h\in\FP$ mit $x\in L \iff h(x) \in \Proj(R)$.
    Definiere nun
    \[ R_L = \{ (x,w) \mid x\in L, (h(x), w)\in R\}. \]
    Es ist leicht zu sehen, dass $R_L$ eine NP-Relation ist. Es ist auch leicht zu sehen dass $\Proj(R_L)=L$.

    Wir zeigen nun, dass $R_L$ auch $\leqlp$-vollständig ist. Sei hierfür $Q$ eine beliebige NP-Relation. Nachdem $R$ ja $\leqlp$-vollständig ist, existieren Reduktions- und Translationsfunktionen $f,g$ die $Q\leqlp R$ realisieren.
    Definiere nun
    \[ f'(x) = h^{-1}(f(x)). \]
    Damit gilt zum einen für $f'$
    \[ x\in \Proj(Q) \iff f(x) \in \Proj(R) \iff h(\underbrace{h^{-1}(f(x))}_{f'(x)}) \in \Proj(R) \iff f'(x) \in \Proj(R_L), \]
    und zum anderen gilt
    \[ (f'(x), w) \in L_R \implies (h(h^{-1}(f(x))), w)\in R \implies (f(x), w)\in R \implies (x, g(x, w))\in Q.  \]
    Damit erfüllen also $f'$ und $g$ die Voraussetzungen an eine Reduktions- bzw. Translationsfunktion und $Q\leqlp R_L$, wie gewünscht.
\end{proof}
Damit haben (im unrelativierten Fall) insbesondere alle zu $\mathtt{SAT}$ p-isomorphen Mengen 
eine entsprechende NP-Relation, auch wenn hierbei die Zertifikate nicht „natürlich“ sind.

\begin{definition}[Sparsame  Reduktionen]
    Seien $Q, R$ NP-Relationen. Wir sagen dass sich $Q$ auf $R$ (in Polynomialzeit) \emph{sparsam} (\emph{„parsimonious“}\sidenote{vgl. \textcite{rothe_komplexitatstheorie_2008}.}) reduzieren lässt, bzw. $Q\leq_\mathrm{pars}^\mathrm p R$ wenn eine Funktion $f\in\FP$ existiert mit
    \[ x\in \Proj(Q) \iff f(x) \in \Proj(R) \quad\text{und}\quad |Q(x)|=|R(f(x))|. \]
    In anderen Worte, $f$ realisiert eine Many-one-Reduktion von $Q$ auf $R$, und haben sowohl die originale $Q$-Instanz $x$ als auch die reduzierte $R$-Instanz $f(x)$ die gleiche Anzahl an Lösungen.
    Definiere $Q\leq_\mathrm{pars}^\mathrm p$-Vollständigkeit entsprechend.
\end{definition}

\begin{definition}[Strukturerhaltende Reduktion; \cite{lynch_structure_1978}]
    Seien $Q, R$ NP-Relationen. Wir sagen dass sich $Q$ auf $R$ (in Polynomialzeit) \emph{strukturerhaltend reduzieren lässt}, bzw. $Q\leq_\mathrm{st}^\mathrm p R$, zwei Funktionen $f, g\in\FP$ existieren, und
    \begin{enumerate}
        \item $(x,y)\in Q \implies (f(x), g(x,y))\in R$ (Vorwärts-Translation von Zertifikaten),
        \item $(f(x),z)\in R \implies \exists y.\, (x,y)\in R\land g(x,y)=z$ ($g$ ist quasi “surjektiv”),
        \item Falls $y_1,y_2\in Q(x)$ und $y_1\neq y_2$, dann ist auch $g(x,y_1)\neq g(x,y_2)$ ($g$ ist quasi “injektiv”).
    \end{enumerate}
    Definiere $Q\leq_\mathrm{st}^\mathrm p$-Vollständigkeit entsprechend.
\end{definition}



\begin{definition}[Zertifikats-Isomorphie; \cite{wiedermann_witness-isomorphic_1995}]
    Seien $Q, R$ NP-Relationen. Wir sagen dass sich $Q$ auf $R$ (in Polynomialzeit) \emph{zertifikats-isomorph reduzieren lässt}, bzw. $Q\leq_\mathrm{wi}^\mathrm p R$, wenn zwei Funktionen $f, g\in \FP$ existieren, die p-invertierbar (also auch injektiv\marginnote{\todo{das steht so direkt aber nicht bei FHT...}}) sind, und 
    \begin{enumerate}
        %\item $x\in \Proj(Q) \iff f(x)\in\Proj(R)$, (i.e. $\Proj(Q)\leqmp\Proj(R)$ via $f$),
        \item $(x,y)\in Q \implies g(x,y)=(f(x), z)\in R$ (Vorwärts-Translation von Zertifikaten),
        \item $(f(x), z)\in R \implies g^{-1}(f(x),z)=(x,y)\in R$ ($g$ ist quasi “surjektiv“ und implementiert eine effiziente Rückwärts-Translation),
        \item Falls $y_1,y_2\in Q(x)$ und $y_1\neq y_2$, dann ist auch $g(x, y_1)\neq g(x, y_2)$, das heißt, mit $g(x, y_1)=(f(x), z_1), g(x, y_2)=(f(x), z_2)$, dass $z_1\neq z_2$ ($g$ ist quasi “injektiv”)).
    \end{enumerate}
    Definiere $Q\leq_\mathrm{wi}^\mathrm p$-Vollständigkeit entsprechend.\marginnote{\todo{Brauchen wir die wi-Isomorphie?}}
    %Wir sagen, dass die NP-Relationen $Q$ und $R$ zertifikats-isomorph sind, bzw. $Q\equiv_\mathrm{wi}^\mathrm p R$, wenn 
    %$Q\leq_\mathrm{wi}^\mathrm p R$ über $f,g$ und $R\leq_\mathrm{wi}^\mathrm p Q$ über $f^{-1}, g^{-1}$.
\end{definition}

\todo{An dieser Stelle die Definition von Fischer et al. mit der von Lynch et al. gegenüberstellen. Zeige insbesondere dass beide Definionen im Wesentlichen „gleich“ sind, nur dass Lynch/Lipton keine Invertierbarkeit fordert.}

\begin{definition}[Universelle Relationen; \cite{agrawal_universal_1992}]
    Sei $R$ eine NP-Relation mit Zertifikatsschranke $q$. 
    Wir nennen $R$ \emph{streng} wenn für $R$ gilt, dass $(x,y)\in R\implies |y|=q(|x|)>0$. In anderen Worten, jedes Zertifikat $y$ für $x$ ist nicht $\epsilon$ und hat genau die Länge $q(|x|)$.

    Eine Funktion $f\in \FP$ ist eine \emph{projektive Levin-Reduktion} einer strengen Relation $Q$ zu einer zweiten strengen Relation $R$ wenn diese die folgenden Bedingungen erfüllen
    \begin{enumerate}
        \item $f(x)=(z,\alpha)$ wobei $x,z\in\Sigma^*$ und $\alpha\in\mathbb N_{>0}^{q(|x|)}$ ist eine Sequenz von positiven paarweise verschiedenen  Indizes der Länge $q(|x|)$.
        \item $\{ y[\alpha] \mid y\in R(z) \} = Q(x)$
    \end{enumerate}

    Wir nennen eine strenge Relation $R$ \emph{universell} wenn sie vollständig bezüglich projektiven Levin-Reduktionen ist. In anderen Worten, wenn für jede strenge Relation $Q$ eine projektive Levin-Reduktion von $Q$ auf $R$ existiert.
\end{definition}
\begin{lemma}
    \begin{enumerate}
        \item Seien $Q, R$ strenge NP-Relationen. Ist $Q$ auf $R$ reduzierbar über eine projektive Levin-Reduktion, dann ist $Q\leqlp R$.
        \item Ist $R$ universell, dann ist $R$ auch $\leqlp$-vollständig. Damit gilt insbesondere auch $Q\leqlp R$ für NP-Relationen $Q$ die \emph{nicht} streng sind.
    \end{enumerate}
\end{lemma}

\begin{definition}[\cite{agrawal_universal_1992}]
    Sei $R$ eine NP-Relation mit Zertifikatsschranke $q$, wobei zusätzlich für $R$ gilt, dass $(x,y)\in R\implies |y|=q(|x|)$. In anderen Worten, jedes Zertifikat $y$ für $x$ hat genau die Länge $q(|x|)$.
    Wir definieren nun bezüglich einer solchen Relation $R$:
    \begin{enumerate}
        \item Die Relation $R$ hat einen \emph{building block}, wenn es ein Element $\mathit{block}\in\Proj(R)$ gibt, sowie paarweise verschiedene $b_1,b_2,b_3\in\mathbb N_{>0}$ sodass
            \[\{ y[b_1b_2b_3] \mid y\in R(\mathit{block}) \} = \Sigma^3 - \{000\} \]
        \item Die Relation $R$ ist \emph{joinable} wenn es eine Funktion $\mathit{join}\in \FP$ gibt sodass
            \begin{multline*}  \mathit{join}(x_1, \dots, x_n) = (z, \alpha) \text{ wobei } x_1, \ldots, x_n, z\in \Sigma^* \\\text{ und } \sum_{k=1}^n q(|x_k|)=|\alpha|\leq q(|z|), \end{multline*}
            wobei $\alpha\in \mathbb N_{>0}^*$ eine Sequenz von \emph{paarweise verschiedenen} Indizes ist, und
            \begin{equation*} \{ y'[\alpha] \mid y' \in R(z)\} = \{ y_1\circ y_2 \circ \cdots \circ y_n \mid (\forall k\leq n).\, y_k \in R(x_k) \}. \end{equation*}
        \item Die Relation $R$ ist \emph{coupable} wenn es eine Funktion $\mathit{cpl}\in \FP$ gibt sodass
            \begin{multline*} \mathit{cpl}(x, (i_1, \ldots, i_n), (j_1, \ldots, j_n)) = (z, \alpha) \text{ wobei } x\in\Sigma^*,\\ 1\leq i_1, \ldots, i_n, j_1,\ldots, n \leq q(|x|) \text{ und } |\alpha|=q(|x|),\end{multline*}
            wobei wieder $\alpha\in \mathbb N_{>0}^*$ eine Sequenz von \emph{paarweise verschiedenen} Indizes ist, und
            \begin{equation*} \{ y'[\alpha] \mid y'\in R(z) \} = \{ y \mid y\in R(x) \text{ und } (\forall k\leq n)(y[i_k]\neq y[j_k]) \}. \end{equation*}
    \end{enumerate}
\end{definition}

\begin{theorem}
    Sei $R$ eine strenge NP-Relation.
    Folgende Aussagen sind äquivalent:
    \begin{enumerate}
        \item $R$ ist eine universelle Relation.
        \item $R$ hat einen \emph{building block}, ist \emph{joinable} und ist \emph{coupable}.
        %\item Es existiert eine Menge $S\subseteq \Proj(R)$, und
            %\begin{enumerate}
                %\item $R$ hat einen \emph{building block} in $S$, 
                %\item und ist \emph{joinable} wobei die Funktion $\mathit{join}$ nur für Elemente aus $S$ definiert ist, und der Wertebereich in $S$ liegt,
                %\item und ist \emph{coupable}, wobei die Funktion $\mathit{cpl}$ nur für Elemente aus $S$ definiert ist, und der Wertebereich in $S$ liegt.
            %\end{enumerate}
    \end{enumerate}
    Diese Äquivalenz gilt nur im unrelativierten Fall.
\end{theorem}

\chapter{Suchprobleme und die Hypothese $\mathsf{Q}$ im Kontext des Pudlákschen Programms}

\begin{itemize}
    \item Will $\mathsf{Q}$ in den Pudlák-Baum einordnen: dafür ist es notwendig, diese ordentlich zu relativieren. Insb. will ich zeigen, dass einige bisherige Resultate natürlicherweise auf „Standardbeweissysteme“ vollständiger Mengen übertragen (nicht nur das Standardbeweissystem für SAT).
\end{itemize}


\begin{definition}[Levin-Paddability]\label{def:levin-paddable}
    Eine NP-Relation $R$ ist \emph{Levin-paddable} wenn 
    Funktionen $\mathit{pad}\in\FP$ und $\mathit{padsol}\in\FP$ existieren, sowie ein Polynom $r$ sodass
    \begin{enumerate}
        \item $x\in \Proj(R) \iff \mathit{pad}(x, 1^n) \in \Proj(R)$,
        \item $(\mathit{pad}(x, 1^n), y)\in R \implies (x, \mathit{padsol}(x, 1^n, y)) \in R$,
        \item $r(|\mathit{pad}(x, 1^n)|)\geq n$. (Funktion $\mathit{pad}$ ist ehrlich bzgl. der zweiten Komponente.)\qedhere
    \end{enumerate}
\end{definition}

\begin{definition}[Standardbeweissystem]
    Sei $R$ eine NP-Relation. Wir definieren bezüglich $R$ das \emph{Standardbeweissystem} $\mathit{std}_R$ für $\Proj(R)$ wie folgt:
    \[ \mathit{std}_R(w) = \begin{cases} x & \text{wenn $w=(x,y)$ und $(x,y)\in R$,}\\
    \bot & \text{sonst}.\end{cases} \qedhere \] 
\end{definition}
\begin{observation}\label{obs:spps-honest}
    Für jede NP-Relation $R$ ist das Standardbeweissystem $\mathit{std}_R$ ehrlich.
\end{observation}
\begin{proof}
    Sei $q$ die Zertifikatsschranke von $R$, und sei $w=(x,y)$ gegeben sodass $\mathit{std}_R(x,y) = x$.
    An dieser Stelle müssen wir auf die konkrete Codierung von Beweisen $w=(x,y)$ eingehen.
    Wie in \ref{} beschrieben, codieren wir Tupel in einer solchen Weise sodass
    \[ |w| = |(x,y)| = 2(|x|+|y|+2) = 2|x|+ 2|y| + 4. \]
    Da $(x,y)\in R$ gilt für $y$ auch $|y|\leq q(|x|)$.
    Damit also
    \[ |w| \leq 2|x|+ 2q(|x|) + 4 \leq q'(|x|) = q'(|\mathit{std}_R(w)|). \]
    für ein geeignetes Polynom $q'$, wie gewünscht.
\end{proof}

Folgendes Lemma ist eine Generalisierung von \textcite[Thm. 5.2]{messner_simulation_2001}
\begin{lemma}\label{lemma:stdps-q}
    Sei $R$ eine NP-Relation die Levin-paddable ist. Folgende Aussagen sind äquivalent:
    \begin{enumerate}
        \item Das Standardbweissystem $\mathit{std}_R$ bzgl. $R$ ist p-optimal.
        \item Für alle NTM $N$ (ohne Laufzeitbeschränkung) mit $L(N)=\Proj(R)$ lassen sich akzepierende Rechenwege von $N$ in Zertifikate umrechnen: es existiert eine Funktion $h\in\mathrm{FP}$ sodass
            \[ N(x) \text{ akz. mit Rechenweg $\alpha$} \implies (x,h(x,\alpha))\in R. \]
        \item Für alle NPTM $N$ mit $L(N)=\Proj(R)$ lassen sich akzepierende Rechenwege von $N$ in Zertifikate umrechnen: es existiert eine Funktion $h\in\mathrm{FP}$ sodass
            \[ N(x) \text{ akz. mit Rechenweg $\alpha$} \implies (x,h(x,\alpha))\in R. \]
    \end{enumerate}
\end{lemma}
\begin{proof}
\begin{prooflist*}
\item (1)$\Rightarrow$(2): Für NTM $f_N$ können wir das assoziierte Beweissystem
    \[ f_N(x, \alpha) = \begin{cases} x & \text{$N(x)$ akz. mit Rechenweg $\alpha$} \\ \bot & \text{sonst} \end{cases} \]
    angeben. Es ist klar, dass $f_N$ ein Beweissystem für $\Proj(R)$ ist.
    Aus der p-Optimalität von $\mathit{std}_R$ gilt nun $\mathit{std}_R\leq^\mathrm p f_N$, bzw. p-simuliert das Standardbeweissystem das Beweissystem $f_N$. Damit existiert eine Funktion $h\in\FP$ sodass
    \[ \mathit{std}_R(h(x, \alpha)) = f_N(x, \alpha). \]
    Diese Funktion $h$ erfüllt genau die Eigenschaften von (2): 
    Wir haben
    \begin{gather*}
        N(x) \text{ akz. mit Rechenweg $\alpha$} \implies h(x, \alpha)=x\\
        \implies \mathit{std}_R(h(x, \alpha)) = x\\
        \implies (x, h(x, \alpha))\in R,
    \end{gather*}
    wie gewünscht.

\item (2)$\Rightarrow$(3): Klar.

\item (3)$\Rightarrow$(1): Angenommen (3) gilt. 
    Seien $\mathit{pad}$, $\mathit{padsol}$ die entsprechenden Funktionen, welche die Levin-Paddability von $R$ realisieren. Das Polynom $r$ sei so gewählt dass $r(|\mathit{pad}(x, 1^n)|)\geq n$ (vgl.~\ref{def:levin-paddable}(3)).

    Wir wollen nun zeigen, dass $\mathit{std}_R$ auch p-optimal ist. Sei hierfür $f$ ein beliebiges Beweissystem für $\Proj(R)$. Wir zeigen nun, dass $\mathit{std}_R \leq^\mathrm p f$. Seien $\mathit{pad}$, $\mathit{padsol}$ die entsprechenden Padding-Funktionen von $R$.
    Definiere nun
    \[ f'(w) = \begin{cases} \mathit{pad}(x, 1^{|w|}) & \text{falls $w=1z$ und $f(z) = x$,} \\
    x & \text{falls $w=0z$ und $\mathit{std}_R(z)=x$,} \\ \bot & \text{sonst.} \end{cases} \]
    Es ist leicht zu sehen, dass $f'$ ehrlich ist: es ist ehrlich für Eingaben $0z$, denn das Standardbeweissystem $\mathit{std}_R$ ist ehrlich nach Beobachtung~\ref{obs:spps-honest}. Es ist ehrlich für Eingaben $w=1z$, denn
    \[ |1z| = |w| \leq r(|\underbrace{\mathit{pad}(x, 1^{|w|})}_{f'(1z)}|) = r(|f'(|w|)|). \]
    Sei im Folgenden dann das Polyom $r'$ so gewählt, dass $|w|\leq r'(|f'(w)|)$ gilt.

    Definiere nun die NPTM $N_{f'}$ welche auf Eingabe $x$ erst nichtdeterministisch einen Beweis $w$, $|w|\leq r'(|x|)$ rät, und genau dann akzeptiert falls $f'(w)=x$.
    Es ist klar, dass $L(N_{f'}) = \Proj(R)$.
    Nach Voraussetzung (3) gibt es also nun eine Funktion $h\in\FP$ sodass 
    \begin{equation} N_{f'}(x) \text{ akz. mit Rechenweg $\alpha$} \implies (x,h(x,\alpha))\in R.  \label{eq:stdps-q-1}
    \end{equation}

    Jetzt können wir $\mathit{std}_R \leq^\mathrm p f$ zeigen: sei $z$ ein $f$-Beweis für $x$, d.h. $f(z)=x$.
    Wir wissen, dass $f'(1z)=\mathit{pad}(x, 1^{|1z|})=x'$.
    Daher können wir aus $z$ einen Rechenweg $\alpha_z$ konstruieren, sodass $N_{f'}(x')$ akzeptiert, nämlich jener der den $f'$-Beweis $1z$ rät.
    Die Abbildung $z\mapsto \alpha_z$ lässt sich in Polynomialzeit leisten.

    Nun gilt
    \begin{gather*}
        N_{f'}(x') \text{ akz. mit $\alpha_z$ } \implies (x', \underbrace{h(x', \alpha_z)}_{y'})\in R \text{ nach (\ref{eq:stdps-q-1})}\\
        \implies (\mathit{pad}(x, 1^{|1z|}), y')\in R \text{ mit $y'=h(x', \alpha_z)$ und obiger Def. von $x'$} \\
        \implies (x, \underbrace{\mathit{padsol}(x, 1^{|1z|}, y')}_y) \in R\\
        \implies \mathit{std}(x, y)=x \text{ mit $y=\mathit{padsol}(x, 1^{|1z|}, y')$}
    \end{gather*}
    und wir haben aus dem $f$-Beweis $z$ für $x$ einen $\mathit{std}_R$-Beweis $(x,y)$ für $x$ bestimmt.
    Es ist klar, dass die Übersetzung $z\mapsto (x,y)$ in Polynomialzeit möglich ist.
\end{prooflist*}
\end{proof}

Folgendes Lemma ist eine Generalisierung von \textcite[Thm.~2]{fenner_inverting_2003}
\begin{lemma}\label{lemma:q-generalized}
%Sei $A\in\mathrm{NP}$ mit folgender Eigenschaft von Vollständigkeit: es existiert eine Menge $B\in\mathrm P$ sodass $A=\{x\mid \exists y, |y|\leq p(|x|), (x,y)\in B\}$ und für alle Mengen $A'\in\mathrm{NP}$, $A'=\{x\mid \exists y, |y|\leq p'(|x|), (x,y)\in B'\}$ existieren zwei Funktionen $r,r^{-1},t\in\mathrm{FP}$ sodass
%\[ x\in A' \iff r(x) \in A, \quad (r(x),z)\in B \implies (x, t(x,z)) \in B'. \]
%($A'\leq_m^p A$ via invertierbarem $r$, Funktion $t$ bildet Zertifikate für $r(x)\in A$ auf Zertifikate für $x\in A'$ ab. Vgl. Reduktionsbegriff unter TFNP-Problemen. Vgl. Levin-Reduktionsbegriff.)
    Sei $R$ eine $\leqlp$-vollständige NP-Relation, mit der zusätzlichen Eigenschaft dass für die jeweilige entsprechende Problem-Reduktionsfunktion $f\colon Q\to R$ für $Q\leqlp R$ immer gilt, dass $f$ ehrlich ist.
Folgende Aussagen sind äquivalent:
\begin{enumerate}
    \item Für alle NPTM $N$ mit $L(N)=\Proj(R)$ lassen sich akzepierende Rechenwege von $N$ in Zertifikate umrechnen: es existiert eine Funktion $h\in\mathrm{FP}$ sodass
        \[ N(x) \text{ akz. mit Rechenweg $\alpha$} \implies (x,h(x,\alpha))\in R. \]
    \item Für alle NPTM $N$ mit $L(N)=\Sigma^*$ lassen sich aus Eingabe $x$ Rechenwege von $N(x)$ effizient bestimmen: es existiert $r\in\mathrm{FP}$ sodass $N(x)$ auf Rechenweg $r(x)$ akzeptiert. (Das ist die Aussage $\mathsf{Q}$.)
\end{enumerate}
\end{lemma}
\begin{proof}
\begin{prooflist*}
\item (2)$\Rightarrow$(1): 
    Sei $R$ eine beliebige NP-Relation mit Zertifikatsschranke $q$, und
    sei $N$ eine beliebige NPTM mit $L(N)=\Proj(R)$. Definiere nun die NPTM $N'(w)$ wie folgt:\\
    \begin{algorithm}[H]
        \lIf{$w$ nicht von der Form $(x, \alpha)$}{akzeptiere}
        $(x, \alpha)\gets w$\;
        \lIf{$N(x)$ akzeptiert \emph{nicht} auf Rechenweg $\alpha$}{akzeptiere}
        \Else{
        \tcc{Ab hier gilt $x\in \Pr(R)$, also auch $R(x)\neq\emptyset$}
        Rate nichtdeterministisch $y\in \Sigma^{\leq q(|x|)}$\;
        Akzeptiere genau dann wenn $(x,y)\in A$.
        }
    \end{algorithm}
    Es ist nun leicht zu sehen dass $L(N')=\Sigma^*$. Nach Voraussetzung (2) existiert eine Funktion $r\in\FP$ sodass für alle $x$ die Maschine $N(w)$ auf Rechenweg $r(w)$ akzeptiert.
    Nun gilt
    \begin{gather*}
        N(x) \text{ akz. mit Rechenweg $\alpha$}\\
        \implies N'(x,\alpha) \text{ akz. in Z.~6}\\
        \implies N'(x,\alpha) \text{ akz. mit Rechenweg $r(x,\alpha)$ in Z.~6},
    \end{gather*}
    und aus diesem Rechenweg $r(x,\alpha)$ kann effizient der geratene Zeuge $y\in R(x)$ aus Z.~5 ausgelesen werden.
    Da $r\in\FP$ existiert also auch ein $h\in\FP$ sodass $h(x,\alpha)$ genau diesen geratenen Zeugen $y$ berechnet.
    Wir haben dann also
    \begin{gather*}
        \implies (x, h(x, \alpha))\in R,
    \end{gather*}
    wie gewünscht.

\item (1)$\Rightarrow$(2): 
    Sei $R$ eine $\leqlp$-vollständige NP-Relation unter ehrlichen Problem-Reduktionsfunktionen, und Zertifikatsschranke $p$.
    Sei nun $N$ eine NPTM mit $L(N)=\Sigma^*$. Betrachte die entsprechende NP-Relation
    \[ R_N = \{ (x,\alpha) \mid N(x) \text{ akz. mit Rechenweg $\alpha$} \} \]
    Da $R$ ja vollständig ist, gilt $R_N\leqlp R$ via $f,g\in\FP$ und (nach Voraussetzung) ist $f$ ehrlich; es existiert ein Polynom $q$ sodass $q(|f(x)|)\geq |x|$.
    %Sein nun
    %\[ S = f(\Sigma^*) = \{ f(x) \mid x\in\Sigma^*\} \]
    %das Bild der Reduktionsfunktion $f$.

    Definiere nun die folgende NPTM $N'(w)$:\\
    \begin{algorithm}[H]
        Rate nichtdeterministisch $x\in \Sigma^{\leq q(|w|)}$\;
        \lIf{$f(x)=w$}{akzeptiere}
        \tcc{Ab hier kann man $x$ wegwerfen}
        Rate nichtdeterministisch $y\in \Sigma^{\leq p(|w|)}$\;
        Akzeptiere genau dann wenn $(w,y)\in R$.
    \end{algorithm}
    Wir zeigen nun, dass $L(N')=\Proj(R)$. Wir müssen hierfür nur die Fälle betrachten, wenn $N'(w)$ in Z.~2 akzeptiert.
    In diesem Fall gilt $f(x)=w$, und wir haben
    \[ x\in\Sigma^* \implies x\in\Proj(R_N) \implies f(x)\in\Proj(R) \implies w\in\Proj(R), \]
    wie gewünscht.

    Nach Voraussetzung (1) gilt nun also, dass eine Funktion $h\in\FP$ existiert sodass
    \[ N'(w) \text{ akz. mit Rechenweg $\alpha$} \implies (w,h(w,\alpha))\in R. \]
    Beobachte wie für $N'(f(x))$ immer ein trivialer akzeptierender Rechenweg $\alpha_x$ existiert: nämlich jener, welcher in Z.~1 das Urbild $x$ rät. Beobachte dass die Umformung $x\mapsto \alpha_x$ in Polynomialzeit möglich ist.

    Um nun (2) zu zeigen müssen wir aus $x\in\Sigma^*$ effizient einen akzeptierenden Rechenweg für $N$ bestimmen.
    Wir haben
    \begin{gather*}
        N'(f(x)) \text{ akz. mit Rechenweg $\alpha_x$} \implies (f(x),h(f(x),\alpha_x))\in R\\
        \implies (x, \underbrace{g(h(f(x), \alpha_x))}_{r(x)}) \in R_N \quad\text{nach Translationsfunktion $g$}\\
        \implies N(x) \text{ akz. mit Rechenweg } r(x)
    \end{gather*}
    mit $r\in \FP$, $r(x) = g(h(f(x), \alpha_x))$, wie gewünscht.
\end{prooflist*}
\end{proof}


\begin{lemma}
    Die in Lemma~\ref{lemma:stdps-q} und~\ref{lemma:q-generalized} genannten Voraussetzungen an die NP-Relation $R$ werden von allen solchen $R$ erfüllt, die $\leqlp$-vollständig sind und Levin-paddable sind.
\end{lemma}
\begin{proof}
    Es ist sofort klar, dass $R$ die Voraussetzungen von Lemma~\ref{lemma:stdps-q} erfüllt.
    Es bleibt nur zu zeigen, dass für jede NP-Relation $Q$ eine $\leqlp$-Reduktion angegeben werden kann, bei dem die Problem-Reduktionsfunktion ehrlich ist.
    Wir nutzen hierbei aus, dass $R$ eine Levin-paddable Relation ist.

    Nachdem $R$ vollständig ist, gilt $Q\leqlp R$; sei $f,g\in\FP$ die Reduktions- bzw. Translationsfunktion welche diese Reduktion realisieren. Wir werden nun Funktionen $f', g'\in\FP$ angeben, welche die gleiche Reduktion realisieren, aber $f'$ ehrlich, wie gewünscht.

    Sei $\mathit{pad}, \mathit{padsol}$ die zu $R$ zugehörigen Padding-Funktionen. Definiere
    \[ f'(x) = \mathit{pad}(f(x), 1^{|x|}). \]
    Es gilt
    \[ x\in\Proj(Q) \iff f(x)\in \Proj(R) \iff \mathit{pad}(f(x), 1^{|x|})=f'(x)\in\Proj(R), \]
    wobei erste Implikation die Eigenschaft der Reduktionsfunktion $f$ ist, und die zweite aus der Definition von Levin-Paddability folgt.
    Aus der Definition von  Levin-Paddability folgt auch $r(|f'(x)|)\geq |x|$ für ein geeignetes Polynom $r$, und damit ist auch $f'$ ehrlich.

    Definiere
    \[ g'(x, z) = g(x, \mathit{padsol}(f(x), 1^{|x|}, z)). \]
    Sei nun $(f'(x), z)\in R$. Die Funktion $g'$ berechnet nun ein Zertifikat $y$ für $x$: Wir haben $(\mathit{pad}(f(x), 1^{|x|}), z)\in R$, also gilt nach Levin-Paddability dass \[(f(x), \mathit{padsol}(f(x), 1^{|x|}, z))\in R,\] 
    und nach Definition der Translationsfunktion $g$ gilt dann
    \[(x, g(x, \mathit{padsol}((f(x), 1^{|x|}, z)))\in Q,\]
    und das ist genau $(x, g'(x, z))\in Q$, wie gewünscht.
\end{proof}

\section{Welche Suchprobleme sind paddable?}

\begin{observation}\label{obs:rkan-paddable}
    Die kanonische Levin-vollständige NP-Relation $\mathtt{rKAN}$ ist Levin-paddable.
\end{observation}

\begin{observation}\label{obs:invcomplete-sind-levinpaddable}
    %Jede $\leq_\mathrm{L,inv}^\mathrm{p}$-vollständige NP-Relation $R$ ist auch Levin-paddable.
    \begin{enumerate}
        \item Gilt $\mathtt{rKAN}\leq_\mathrm{L}^\mathrm{p} R$, und ist die zugehörige Reduktionsfuktion $f$ ehrlich, dann ist $R$ Levin-paddable
        \item Jede $\leq_\mathrm{L,inv}^\mathrm{p}$-vollständige NP-Relation $R$ ist auch Levin-paddable.
    \end{enumerate}
\end{observation}
\begin{corollary}
    Jede $\leq_\mathrm{L,inv}^\mathrm{p}$-vollständige Relation $R$ erfüllt die in 
    Lemma~\ref{lemma:stdps-q} und~\ref{lemma:q-generalized} genannten Voraussetzungen an die NP-Relation $R$.

    Das sind im unrelativierten Fall u.a. $\mathtt{rSAT}$, $\mathtt{rSETCOVER}$, $\mathtt{rVERTEXCOVER}$, $\mathtt{rCLIQUE}$, $\mathtt{r3COLORABILITY}$.\marginnote{\note{So das Textbook von \textcite{goldreich_computational_2008}.}}
\end{corollary}
\begin{proof}[Beweis zu Beobachtung~\ref{obs:invcomplete-sind-levinpaddable}]
    Aussage (2) folgt unmittelbar aus (1): Wir haben $\mathtt{rKAN}\leq_\mathrm{L,inv}^\mathrm{p} R$ und damit ist die entsprechende Reduktionsfunktion $f$ p-invertierbar, und damit ehrlich.

    Für (1) nutzen wir die Levin-Paddability von $\mathtt{rKAN}$ aus: übersetze Instanz $x$ von $R$ nach $\mathtt{rKAN}$, padde dort hoch, und überetze zu $R$-Instanz $x'$ zurück. Ist dann $y'$ ein Zertifikat für $x'$, dann lässt sich dies auf ähnlichem Weg wieder zu einem Zertifikat für $x$ zurückrechnen.

    Seien $f, g$ die Reduktions- bzw. Translationsfunktion, welche $\mathtt{rKAN}\leq_\mathrm{L}^\mathrm p R$ bezeugen, und seinen analog $f', g'$ jene Funktionen, welche $R\leq_\mathrm{L}^\mathrm p \mathtt{rKAN}$ bezeugen. Erstere existieren nach Voraussetzung, zweitere existieren weil $\mathtt{rKAN}$ $\leq_\mathrm{L}^\mathrm p$-vollständig ist.
    Nach Voraussetzung ist $f$ ehrlich. %, und da $f'$ p-invertierbar ist, ist auch $f'$ ehrlich. 
    Und nach Beobachtung~\ref{obs:rkan-paddable} existieren für $\mathtt{rKAN}$ Padding-Funktionen $\mathit{pad}_\mathtt{rKAN}$, $\mathit{padsol}_\mathtt{rKAN}$.
    Sei $q$ ein entsprechendes Polynom mit $q(|\mathit{pad}_\mathtt{rKAN}(x, 1^n)|)\geq n$, $q(|f(x)|) \geq |x|$.

    Definiere nun
    \[ \mathit{pad}_R(x, 1^n) = f(\mathit{pad}_\mathtt{rKAN}(f'(x), 1^n)). \]
    Die Zugehörigkeit zu $\Proj(R)$ bleibt erhalten:
    \begin{multline*}
        x\in \Proj(R) \iff f'(x) \in \mathtt{KAN} \iff \mathit{pad}_\mathtt{rKAN}(f'(x), 1^n) \in \mathtt{KAN}\\ \iff f(\mathit{pad}_\mathtt{rKAN}(f'(x), 1^n)) \in \Proj(R) \iff \mathit{pad}_R(x, 1^n) \in\Proj(R).
    \end{multline*}
    Ferner gilt
    \begin{align*} &q(q(|\mathit{pad}_R(x, 1^n)|)) \\&= q(q(|f(\mathit{pad}_\mathtt{rKAN}(f'(x), 1^n)|))\\&\geq q(|\mathit{pad}_\mathtt{rKAN}(f'(x), 1^n)|)\\ &\geq n.
    \end{align*}
    und damit ist $\mathit{pad}_R$ wie gewünscht ehrlich bzgl. $n$ (mit Polynom $q\circ q$).

    Es verbleibt noch die Funktion $\mathit{padsol}_R$. Nehme hierfür an dass wir ein $y'$ haben mit $(\mathit{pad}_R(x, 1^n), y')\in R$.
    Wir können über $g, g'$ das Zertifikat $y'$ zu Zertifikat $y$ mit $(x, y)\in R$ zurück übersetzen:
    Sei $p=\mathit{pad}_\mathtt{rKAN}(f'(x), 1^n)$, dann gilt
    \[ (f(p), y')\in R \implies (p, \underbrace{g(p, y')}_z)\in \mathtt{rKAN}. \]
    Definere $z=g(p, y')$.
    Nun haben wir
    \begin{gather*} (p, z)=(\mathit{pad}_\mathtt{rKAN}(f'(x), 1^n), z)\in\mathtt{rKAN}  \\\quad\implies (f'(x), \underbrace{\mathit{padsol}_\mathtt{rKAN}(f'(x), 1^n, z)}_{z'})\in\mathtt{rKAN} \end{gather*}
    und mit $z'=\mathit{padsol}_\mathtt{rKAN}(f'(x), 1^n, z)$ gilt
    \[ (f'(x), z') \in \mathtt{rKAN} \implies (x, \underbrace{g'(x, z')}_{y}) \in R. \]
    %\[ \mathit{padsol}_R(x, 1^n, y') = g(\mathit{pad}_\mathtt{rKAN}(f'(x), 1^n), y')
    %\[ q(q(q(|\mathit{pad}_R(x, 1^n)|))) = q(q(q(|f(\mathit{pad}_\mathtt{rKAN}(f'(x), 1^n))|)))
    Es ist leicht zu sehen, dass sich eine Funktion $\mathit{padsol}_R\in\FP$ angeben kann, die aus $x, 1^n$ dieses entsprechende $y$ berechnen kann.
\end{proof}

\begin{observation}\label{obs:joinable-sind-levinpaddable}
    %Jede universelle Relation ist Levin-paddable. Dieses Resultat gilt nur im unrelativierten Fall.
    Jede NP-Relation mit einem \emph{building block} und die \emph{joinable} ist, ist auch Levin-paddable.
\end{observation}
\begin{corollary}
    Im unrelativierten Fall erfüllt jede universelle Relation $R$ die in 
    Lemma~\ref{lemma:stdps-q} und~\ref{lemma:q-generalized} genannten Voraussetzungen an die NP-Relation $R$.

    Das sind u.a. $\mathtt{rSAT}, \mathtt{rHAM}, \mathtt{rINDSET}, \mathtt{rKNAPSACK}, \mathtt{rMAXCUT}$.
\end{corollary}
\begin{proof}[Beweis zu Beobachtung~\ref{obs:joinable-sind-levinpaddable}]
    Sei $R$ eine NP-Relation, mit zugehörigem Polynom $q$, welches die Zertifikatsgröße spezifiziert. Zur Erinnerung, dieses Polynom ist streng monoton steigend, und aus $(x,y)\in R$ folgt $|y|=q(|x|)$.
    Wir zeigen zunächst, wie wir für beliebige Instanz $x$ und $n\in\mathbb N$ auf eine Instanz $x'$ hochpadden, in dem Sinne dass $q(|x'|) \geq n$.

    %Nachdem $R$ universell ist, existiert eine zertifikatserhaltende Reduktion $f\in FP$ von $\mathtt{rSAT}$ auf $R$. 
    %Es lässt sich leicht eine Familie $\phi_1, \phi_2, \ldots$ an positiven Instanzen für $\mathtt{rSAT}$ erdenken, für die $|\phi_i|\geq i$ und $\phi_i$ effizient auf Eingabe $1^i$ konstruiert werden kann.
    %Zur Erinnerung: wir haben für $\mathtt{rSAT}$ das zugehörige Polynom der Zertifikatsgröße so gewählt, dass die Zertifikate genau so lange wie die Probleminstanzen sind. Damit gilt
    %für $(\phi_i, w)\in \mathtt{rSAT}$ dann auch $|w|=|\phi_i|\geq i$.

    %Sei nun $(z,\alpha)=f(\phi_{q(n)})$, d.h. $\phi_i$ ist auf eine $R$-Instanz $z$ reduziert worden. Nach Definition von zertifikatserhaltenden Reduktionen gilt $|\alpha|=|\phi_{q(n)}|\geq {q(n)}$, und $\alpha$ besteht aus paarweise verschiedenen Indizes. Wir haben nun nach Definition
    %\[ \{ y'[\alpha] \mid y'\in \Sigma^{q(|z|)}, (z,y')\in R \} = \{ y\mid y\in \Sigma^{|\phi_{q(n)}|}, (\phi_{q(n)},y)\in\mathtt{rSAT}\} \neq \emptyset, \]
    %in anderen Worten, aus Zertifikaten $y'$ für $z$ können wir durch geeignete Projektion via $\alpha$ die Zertifikate für $w$ rekonstruieren. Davon existiert mindestens eins.
    %Alle Zertifikate $y'$ haben eine feste Länge $q(|z|)$, und da $\alpha$ aus $q(n)$ paarweise verschiedenen Indizes besteht, 
    %gilt $q(|z|) \geq q(n)$. % und aus Monotonie folgt $|z|\geq n$.
    \strut\marginnote{\note{Es sieht nicht danach aus, dass wir über die Anzahl an Zertifikaten für $x'$ die Länge abschätzen können: ist $x\not\in\Proj(R)$ dann hätte auch $x'$ keine Zertifikate, und könnte sich erlauben entsprechend kurz zu sein.}}
    Nach Voraussetzung hat  die Relation $R$ einen \emph{building block} $\mathit{block}$. Es lässt sich leicht aus der Definition eines \emph{building block} ableiten, dass $|\mathit{block}|>0$ und $\mathit{block}\in\Proj(R)$. Damit gilt auch dass die Zertifikate $y$ zu $\mathit{block}$ die Länge $l=q(|\mathit{block}|)\geq |\mathit{block}|\geq 1$ haben.

    Nach Voraussetzungen  ist die Relation $R$ auch \emph{joinable}, das heißt wir haben eine Funktion $\mathit{join}\in\FP$. Sei 
    \[ (x',\delta)=\mathit{join}(x, \underbrace{\mathit{block}, \mathit{block}, \ldots, \mathit{block}}_{\text{$n$ mal}}).\]
    Wir werden nun über die Länge $|\delta|$ auf die Länge von Zertifikaten zu $x'$ schließen, und damit $|x'|$ beschränken.
    Nach Definition ?? gilt
    \[ |\delta|=q(|x|)+q(n)\cdot q(|\mathit{block}|)=q(|x|)+q(n)\cdot l\geq n. \]
    Beob. dass unter Definition ?? alle Zertifikate $y'$ für $x'$ die feste Länge $q(|x'|)$ haben. 
    Zur Erinnerung: wir haben
    \begin{multline}\label{eq:levinpad-join} \{ y'[\delta] \mid y'\in\Sigma^{q(|x'|)}, (x', y')\in R \} = \{ yy_1y_2\cdots y_{n} \mid y\in\Sigma^{q(|x|)}, y_1,y_2, \ldots \in \Sigma^{l},\\ (x,y),(\mathit{block}, y_1),(\mathit{block}, y_2), \ldots \in R\} \end{multline}
    Die Sequenz $\delta$ besteht nach Definition aus paarweise verschiedenen Indizes, daher können wir argumentieren, dass auch alle Zertifikate $y'$ (mit vorgegebener Länge $q(|x'|)$) mindestens die Länge $|\delta|$ haben.
    Damit gilt
    \[ q(|x'|) \geq |\delta| \geq n \]
    wie gewünscht.

    Sei nun $\mathit{pad}$ genau jene polynomialzeit-berechenbare Funktion, die aus $x$ und $1^n$ die Instanz $x'$ konstruiert:
    \[ \mathit{pad}(x, 1^n) = x' \quad\text{ wobei }
    (x',\delta)=\mathit{join}(x, \underbrace{\mathit{block}, \mathit{block}, \ldots, \mathit{block}}_{\text{$q(n)$ mal}}).\]
    Dann gilt schon sofort, dass $q(|\mathit{pad}(x, 1^n)|)=q(|x'|)\geq n$ wie gewünscht.

    Wir zeigen jetzt, dass die Zugehörigkeit zu $\Proj(R)$ erhalten bleibt:
    Gilt $x\not\in\Proj(R)$, dann ist die rechte Menge in (\ref{eq:levinpad-join}) leer, also auch die linke Menge und damit $x'=\mathit{pad}(x, 1^n)\not\in \Proj(R)$.
    Falls anders herum $x\in\Proj(R)$, dann ist die rechte Menge nicht leer, existiert ja ein Zertifikat $y$ für $x$ und je ein weiters $y_i$ für $\mathit{block}$. Also ist auch die linke Menge nicht leer, damit $\mathit{pad}(x, 1^n)\in \Proj(R)$.

    Die noch verbleibende Funktion $\mathit{padsol}$ ist durch die bitweise Projektion durch $\delta$ leicht möglich:
    \[
        \mathit{padsol}(x, 1^n, y') = y'[\delta[1:q(|x|)]] \enspace\text{ wobei } (\cdot, \delta) = \mathit{join}(x, \underbrace{\mathit{block}, \mathit{block}, \ldots, \mathit{block}}_{\text{$n$ mal}}).\]
    Wir verifizieren: Sei $(\mathit{pad}(x, 1^n), y')\in R$, dann ist nach (\ref{eq:levinpad-join}) $y'[\delta]=yy_1y_2\cdots$ wobei $y\in\Sigma^{q(|x|)}$, $(x, y)\in R$. 
    Wir haben
    \[ \mathit{padsol}(x, 1^n, y') = y'[\delta[1:q(|x|)]] = (yy_1y_2\cdots)[1:q(|x|)] = y \]
    und damit $(x, \mathit{padsol}(x, 1^n, y')) = (x, y)\in R$, wie gewünscht.
\end{proof}

\section{Hypothese $\mathsf{Q}$ und die Vollständigkeit von Suchproblemen}

\begin{theorem}[Äquivalente Formulierungen der Hypothese $\mathsf{Q}$; \cites{fenner_inverting_2003}{messner_simulation_2001}]\label{thm:q}
    Folgende Aussagen sind äquivalent:
    \begin{enumerate}
        \item Hypothese $\mathsf{Q}$: Für jede NPTM $N$ mit $L(N)=\Sigma^*$ existiert eine Funktion $g\in\FP$ sodass für alle $x$ das Bild $g(x)$ eine akzeptierende Berechnung von $N(x)$ ist.
        \item $\mathrm{NPMV}_t\subseteq_c \mathrm{FP}$
        \item $\P=\NP\cap\coNP$ und $\mathrm{NPMV}_t\subseteq_c \mathrm{NPSV}_t$
        \item Jede surjektive ehrliche Funktion $f\in\FP$ ist p-invertierbar. %Zur Erinnerung: p-Invertierbarkeit bedeutet dass eine Funktion $g\in\FP$ existiert mit $f(g(y))=y$.
        \item Für jede Menge $L\in \P$  und jede NPTM $N$ mit $L(N)=L$ existiert eine Funktion $h\in \FP$ mit 
            \[ x\in L \implies N(x) \text{ akz. mit Rechenweg $h(x)$}. \]
        \item Für jedes Paar von NP-Relationen $A, B$ und jede Funktion $f\in\FP$ gilt:
            \[ \Proj(A) \leqmp \Proj(B) \text{ via $f$} \iff A \leqlp B \text{ via Reduktionsfunktion $f$}. \]
        \item Für jedes Beweissystem $h$ gilt: $h$ ist optimal $\iff$ $h$ ist p-optimal. 
        \item Es existiert eine $\leqlp$-vollständige Levin-paddable NP-Relation $R$ sodass für alle NPTM $N$ mit $L(N)=\Proj(R)$ gilt: es existiert eine Funktion $h\in\mathrm{FP}$ mit
            \[ N(x) \text{ akz. mit Rechenweg $\alpha$} \implies (x,h(x,\alpha))\in R. \]
        %\item[(\arabic{enumi}\,$'\!$)] Für jede $\leqlp$-vollständige Levin-paddable NP-Relation $R$, und für alle NPTM $N$ mit $L(N)=\Proj(R)$ gilt: es existiert eine Funktion $h\in\mathrm{FP}$ mit
            %\[ N(x) \text{ akz. mit RW $\alpha$} \implies (x,h(x,\alpha))\in R. \]
        \item Es existiert eine $\leqlp$-vollständige Levin-paddable NP-Relation $R$ für welche das Standardbeweissystem $\mathit{std}_R$ p-optimal ist.
        %\item[(\arabic{enumi}\,$'\!$)] Für jede $\leqlp$-vollständige Levin-paddable NP-Relationen $R$ ist das Standardbeweissystem $\mathit{std}_R$ p-optimal.
        \item Es existiert eine $\leq_\mathrm{L,1,inv}^\mathrm p$-vollständige NP-Relation $R$ sodass für jede Menge $S\in \P$ mit $S\subseteq \Proj(R)$ gilt: es existiert eine Funktion $g\in\mathrm{FP}$ sodass
            \[ x\in S \implies (x, g(x))\in R. \]
        %\item[(\arabic{enumi}\,$'\!$)] Für jede $\leq_\mathrm{L,1,inv}^\mathrm p$-vollständige NP-Relation $R$ und jede Menge $S\in \P$ mit $S\subseteq \Proj(R)$ gilt: es existiert eine Funktion $g\in\mathrm{FP}$ sodass
            %\[ x\in S \implies (x, g(x))\in R. \]
    \end{enumerate}
\end{theorem}
\begin{proof}
\begin{prooflist}
\item (1)$\iff$(2)$\iff$(3)$\iff$(4)$\iff$(5): nach \textcite[Thm.~2]{fenner_inverting_2003}.

\item (1)$\iff$(8)$\iff$(9): nach Lemma~\ref{lemma:q-generalized} und~\ref{lemma:stdps-q}.

\item (5)$\implies$(10): Wir zeigen eine stärkere Variante von (10), welche sich über \emph{alle} NP-Relationen $R$ estreckt (und damit auch über $\leq_\mathrm{L,1,inv}^\mathrm p$-vollständige $\mathtt{rKAN}$, wie von (10) gefordert). Sei $R$ eine beliebige  NP-Relation, wobei Polynom $q$ die Zertifikatsgröße beschränkt. Sei nun $S\subseteq \Proj(R)$ mit $S\in \P$. Definiere die NPTM $N$, welche auf Eingabe $x$ folgendes leistet: teste zuerst ob $x\in S$; falls nicht, lehne sofort ab. Rate dann ein $y\in\Sigma^{\leq q(|x|)}$ und akzeptiere genau dann wenn $(x,y)\in R$. 

    Klar ist, dass $L(N)=S$. Nach (5) existiert nun eine Funktion $h\in \FP$, die für $x\in S$ einen akzeptierenden Rechenweg $h(x)$ von $N(x)$ ausgibt. Wir können sogar aus $h(x)$ das geratene Zertifikat $y$ extrahieren. Es ist daher leicht eine Funktion $g\in \FP$ anzugeben für die $(x,g(x))\in R$ für alle $x\in S$.

\item (10)$\implies$(5): Sei $L\in \P$ und sei $N$ eine NPTM mit $L(N)=L$, wobei das Polynom $q$ die Laufzeit beschränkt. Wir wollen eine Funktion $h\in\FP$ definieren sodass $h(x)$ ein akzeptierender Rechenweg von $N(x)$ für $x\in L$ ist. Definiere die NP-Relation
    \[ Q = \{ (x, y) \mid \text{$N(x)$ akzeptiert mit Rechenweg $y\in\Sigma^{\leq q(|x|)}$} \}. \]
    Nachdem (10) gilt, haben wir eine $\leq_\mathrm{L,1,inv}^\mathrm p$-vollständige NP-Relation $R$. Damit gilt $Q\leqlp R$ mittels Reduktions- bwz. Translationsfunktion $f, k\in FP$. Insbesondere existiert eine Inverse $f^{-1}\in\FP$ zu $f$.

    Sei $S=f(L)$ die Bildmenge der Elemente aus $L$, also 
    \[ S= \{f(x) \mid x\in L\}. \]
    Es ist leicht zu sehen dass $S\subseteq \Proj(R)$. Außerdem ist $S\in \P$: teste $z\in S$ indem getestet wird ob $f^{-1}(z)=x\neq\bot$ und ob $x\in L$.

    Damit sind die Voraussetzungen von (10) erfüllt, und es existiert eine Funktion $g\in \FP$ sodass $(z,g(z))\in R$ für alle $z\in S$. Damit gilt
    \begin{gather*}x\in L \implies f(x) \in S \implies (f(N,x), g(f(x)))\in R\\ \implies ((x), k(g(f(x))))\in Q\\
    \implies N(x) \text{ akz. mit Rechenweg $\underbrace{k(g(f(x)))}_{h(x)}$. }
\end{gather*}
    Definiere nun die gesuchte Funktion $h\in \FP$ mit $h(x) = k(g(f(x)))$. Damit gilt für alle $x\in L$ dass $N(x)$ mit Rechenweg $h(x)$ akzeptiert, wie gewünscht.

\item (1)$\implies$(6): Die Richtunng von rechts nach links ist klar. Für die andere Richtung sei $\Proj(A) \leqmp \Proj(B)$ mit $A,B$ NP-Relationen. Sei $q$ hierbei das Polynom was die Zertifikatslänge in $A$ begrenzt.
    Wir wollen nun eine Levin-Reduktion von $A$ auf $B$ angeben. Sei $f\in \FP$ die Funktion, welche die Reduktion $\Proj(A) \leqmp \Proj(B)$ realisiert.

    Definiere folgende NPTM $N$, die wie folgt auf Eingabe $w$ arbeitet: \\
    \begin{algorithm}[H]
        \lIf{$w$ nicht von der Form $(x, y')$}{akzeptiere}
        $(x, y')\gets w$\;
        \lIf{$(f(x), y')\not\in B$}{akzeptiere}
        \Else{
        \tcc{Ab hier gilt $f(x)\in \Pr(B)$ und $x\in \Pr(A)$}
        Rate nichtdeterministisch $y\in \Sigma^{q(|x|)}$\;
        Akzeptiere genau dann wenn $(x,y)\in A$.
        }
    \end{algorithm}
    Es ist leicht zu sehen dass $L(N)=\Sigma^*$. Nach (1) existiert nun also eine Funktion $g$ sodass, für alle $w$, $g(w)$ eine akzeptierender Rechwenweg von $N(w)$ ist.

    Damit lässt die Levin-Reduktion von $A$ auf $B$ angeben: wähle $f$ als Reduktionsfunktion, und definiere $g'$ als Translationsfunktion, welche aus dem akzeptierenden Rechenweg $g(x, y')$ das geratene Zertifikat $y$ von Zeile 5 ausliest. Dann gilt
    \begin{gather*}
        (f(x), y') \in B \implies N(x, y') \text{ akz. auf einem Rechenweg in Z.\,6, ratet $y$}\\
        \implies (x,y)=(x, g'(x, y'))\in A
    \end{gather*}
    wie gewünscht. Wir haben $A\leqlp$ via $f, g'$.
\item (6)$\implies$(8): Sei $R$ eine beliebige $\leqlp$-vollständige und Levin-paddable NP-Relation (diese existiert immer) und sei $N$ eine NPTM $N$ mit $L(N)=\Proj(R)$, wobei das Polynom $q$ die Laufzeit von $N$ beschränkt.  Definiere
    \[ Q = \{ (x, y) \mid \text{$N(x)$ akzeptiert mit Rechenweg $y\in\Sigma^{\leq q(|x|)}$} \}. \]

    Es ist klar, dass $\Proj(R)\leqmp \Proj(Q)=L(N)=\Proj(R)$ über die Identiätsfunktion.
    Nach (6) gilt nun auch $R\leqlp Q$. mit Identitätsfunktion als Reduktionsfunktion und $h\in \FP$ als Translationsfunktion.
    Damit gilt
    \[ \text{$N(x)$ akz. mit RW $\alpha$} \implies (x, \alpha)\in Q \implies (x, h(x, \alpha))\in R \]
    wie gewünscht.


\item (2)$\implies$(7): Die Richtung von rechts nach links ist klar. Sei für die andere Richtung $h$ ein optimales Beweissystem für eine Menge $L$. Wir wollen zeigen, dass $h$ auch p-optimal ist. Sei dafür $g$ ein weiteres Beweissystem für $L$. Nach Voraussetzung haben wir $g\leq h$, das heißt es existiert eine (nicht notwendigerweise effiziente) Funktion $f$ sodass $g(w)=h(f(w))$, und gleichzeitig ist $|f(w)|\leq q(|w|)$ für ein geeignetes Polynom $q$.

Betrachte folgende Multifunktion $f'$:
%\[ f'(w) = \{ y \mid y\in\Sigma^{\leq q(|w|)}, g(w)=h(y) \}. \]
\[ f'(w) \mapsto y \iff \exists y\in\Sigma^{\leq q(|w|)}, g(w)=h(y). \]
Es lässt sich leicht zeigen, dass $f'\in\NPMV$, über einen geeigneten NPTM-Transduktor. 
Es ist sogar $f'\in\NPMV_t$, denn für jedes $w$ mindestens $f(w)\in \fset{}f'(w)$.

Nach (2) gilt also $f'\in\NPMV_t\subseteq_c \FP$, also existiert eine Funktion $f''\in\FP$ welche eine Verfeinerung von $f'$ ist. Diese Funktion übersetzt $g$-Beweise $w$ für $x$ effizient in $h$-Beweise für $x$: 
Sei $g(w)=x$, dann gilt
\[ f''(w) = y \quad\text{mit } y\in\Sigma^{\leq q(|w|)}, x=g(w)=h(y) \]
also ist $h(f''(w))=x$ bzw. $f''(w)$ ein $h$-Beweis für $x$, wie gewünscht.


\item (7)$\implies$(9): klar, denn $\mathtt{rKAN}$ ist $\leqlp$-vollständig, ist Levin-paddable, und das Standardbeweissystem $\mathit{std}_\mathtt{rKAN}$ ist (wie jedes Standardbeweissystem einer NP-Relation) optimal. Zusammen mit (7) ist es also auch p-optimal.
\end{prooflist}
\end{proof}

\begin{theorem}[Formulierung Hypothese $\mathsf{Q}$ durch NP-Relationen, $\forall$-Variante]
    Folgende Aussagen sind äquivalent:
    \begin{enumerate}
        \item Hypothese $\mathsf{Q}$
        \item[(8\,$'\!$)] Für jede $\leqlp$-vollständige Levin-paddable NP-Relation $R$, und für alle NPTM $N$ mit $L(N)=\Proj(R)$ gilt: es existiert eine Funktion $h\in\mathrm{FP}$ mit
            \[ N(x) \text{ akz. mit RW $\alpha$} \implies (x,h(x,\alpha))\in R. \]
        \item[(9\,$'\!$)] Für jede $\leqlp$-vollständige Levin-paddable NP-Relationen $R$ ist das Standardbeweissystem $\mathit{std}_R$ p-optimal.
        \item[(10\,$'\!$)] Für jede NP-Relation $P$ und jede Menge $S\in \P$ mit $S\subseteq \Proj(P)$ gilt: es existiert eine Funktion $g\in\mathrm{FP}$ sodass
            \[ x\in S \implies (x, g(x))\in P. \]
    \end{enumerate}
\end{theorem}
\begin{proof}
\begin{prooflist}
\item (8$'$)$\implies$(1), (9$'$)$\implies$(1), (10$'$)$\implies$(1):
    Die NP-Relation $\mathtt{rKAN}$ ist $\leq_\mathrm{L,1,inv}^\mathrm p$-vollständig (und damit auch Levin-paddable).
    Gilt also (8$'$), dann auch die existentielle Variante (8) von vorigem Satz~\ref{thm:q}, und damit (1).
    Beweise der anderen Implikationen sind analog.

%\item (1)$\implies$(8$'$): Sei $R$ eine beliebige $\leqlp$-vollständige Levin-paddable NP-Relation. Lemma~\ref{lemma:q-generalized} gilt bezüglich dieser Relation $R$: da (1) gilt, existiert also für jede NPTM $N$ mit $L(N)=\Proj(R)$ ein $h\in \FP$ mit $(x,h(x,\alpha))\in R$ wenn immer $N(x)$ mit Rechenweg $\alpha$ akzeptiert, wie gewünscht.
\item (1)$\implies$(8$'$): Folgt aus Lemma~\ref{lemma:q-generalized}.

\item (8$'$)$\implies$(9$'$): Sei $R$ eine beliebige $\leqlp$-vollständige Levin-paddable NP-Relation. Nach Voraussetzung können wir für jede NPTM $N$, $L(N)=\Proj(R)$ eine Funktion $h\in FP$ angeben sodass 
    \[ N(x) \text{ akz. mit RW $\alpha$} \implies (x,h(x,\alpha))\in R. \]
    Das ist genau die erste der der äquivalenten Aussagen in Lemma~\ref{lemma:stdps-q}. Damit gilt auch bezüglich \emph{diesem} gewählten $R$ auch die zweite der äquivalenten Aussagen, nämlich dass $\mathit{std}_R$ p-optimal ist.

\item (1)$\implies$(10$'$): Im Beweis von Satz~\ref{thm:q} wurde (1)$\implies$(5)$\implies$(10) gezeigt. Beachte insbesondere dass im Beweis zur zweiten Implikation sogar eine stärkere Aussage bewiesn wurde, welche die Aussage (10) für alle NP-Relationen (und nicht nur vollständige) zeigt. Das entspricht genau der hier aufgestellten Aussage (10$'$), wie gewünscht.

\end{prooflist}
\end{proof}

\section{Karp-Vollständigkeit vs. Levin-Vollständigkeit}

\begin{conjecture}[Karp-vs-Levin-Vermutung; $\mathsf{KvL}$]\label{conj:kvl}
    Es existiert eine NP-Relation $R$ sodass $\Proj(R)$ $\leqmp$-vollständig für $\NP$ ist, aber $R$ ist nicht $\leqlp$-vollständig (für alle NP-Relationen).
\end{conjecture}

\begin{theorem}\label{thm:kvl-implies-q}
    $\mathsf{KvL} \implies \neg\mathsf{Q}$.
\end{theorem}
\begin{proof}
    Wir zeigen die Kontraposition, und starten mit der Voraussetzung $\mathsf{Q}$.
    Wir wollen nun $\neg\mathsf{KvL}$ zeigen. Sei hierfür $R$ eine beliebige NP-Relation sodass $\Proj(R)$ $\leqmp$-vollständig ist.
    Damit gilt also schon für alle weiteren NP-Relationen $A$, dass $\Proj(A)\leqmp\Proj(R)$.
    Nach Satz~\ref{thm:q} gilt also auch die Aussage \ref{thm:q}(6), und damit $A\leqlp R$. Also ist $R$ auch $\leqlp$-vollständig, wie gewünscht und wir haben $\neg\mathsf{KvL}$ gezeigt.
\end{proof}

Was sind natürlich notwendige Bedingungen für die Hypothese $\mathsf{KvL}$? Diese Frage erscheint tatsächlich wesentlich schwieriger als gedacht. Insbesondere scheint es unklar, ob aus irgend einer von Pudláks Hypothesen die Aussage $\mathsf{KvL}$ folgt.

Besonders interessant erscheint aber die Beziehung zur Hypothese $\neg\mathsf{Q}$, also genau die Umkehrung von Satz~\ref{thm:kvl-implies-q}.
Betrachten wir hier exemplarisch den Fall für Relationen, die $\mathtt{SAT}$ bezeugen.

Starten wir mit $\neg\mathsf{Q}$, dann haben wir über Satz~\ref{thm:q} auch die Negation von Aussage~\ref{thm:q}(8). Nachdem $\mathtt{rSAT}$ eine $\leqlp$-vollständige Levin-paddable NP-Relation ist, muss auch eine NPTM $N$ existiere, für die $L(N)=\mathtt{SAT}$, aber für alle Funktionen $h\in\FP$ gilt
\begin{equation} N(\phi) \text{ akz. mit Rechenweg $w$} \rlap{\hspace*{2pt}\raisebox{2.3pt}{$\quad\not$}}\implies  (\phi, h(\phi, w))\in\mathtt{rSAT}.\label{eq:weak-irreducibility} \end{equation}
In anderen Worten: es existiert zwar eine NPTM $N$ welche die ($\leqmp$-vollständige Menge) $\mathtt{SAT}$ entscheidet, aber aus den akzeptierenden Rechenwegen $w$ von $N(x)$ auf $x\in \mathtt{SAT}$ kann nicht effizient eine akzeptierende Belegung für $x$ abgeleitet werden.

Wir können $N$ äquivalent als NP-Relation $R_N$ repräsentieren, mit $(\phi, w) \in R_N$ genau dann wenn $N(x)$. mit Rechenweg $w$ akzeptiert.
Damit kann Gleichung~\ref{eq:weak-irreducibility} so verstanden werden, dass $\mathtt{rSAT} \not\leqlp R_N$ \emph{falls die Reduktionsfunktion $f$ die Identitätsfunktion ist}.
An dieser Stelle muss erneut hervorgehoben werden, dass im Allgemeinen $\mathtt{rSAT} \leqlp R_N$ mit Funktionen $f,g$ gelten kann, notwendig hierfür ist aber dass $f\neq\mathrm{id}$.

Gleichzeitig wäre die Existenz einer solchen Reduktion überraschend. Angenommen $\mathtt{rSAT} \leqlp R_N$ wird von $f,g$ ($f\neq\mathrm{id}$) realisiert, dann haben wir nach Definition
\[ N(f(\phi)) \text{ akz. mit Rechenweg $w$} \implies \phi \text{ wird von Belegung $g(\phi, w)$ erfüllt}. \]
Einerseits ist es also nicht möglich, aus $w$ effizient eine akzeptierende Belegung für $f(\phi)\neq\phi$ zu bestimmen.
Andererseits reicht der „Beweis“ $w$ aber aus, um (zusammen mit der Information $\phi$) effizient wieder eine erfüllende Belegung für $\phi$ zu berechnen. 

Ich vermute, dass solche Funktionen $f,g$ nicht jeweils für alle NPTM $N$ mit $L(N)=\mathtt{SAT}$ existieren können.
Tatsächlich können wir die eben formulierte Vermutung auch in der Theorie der Beweissystemen formulieren.
Wir definieren zunächst eine abgeschwächte Variante der p-Simulation.
\begin{definition}
    Seien $h,h'$ Beweissysteme für $L$. Das Beweissystem $h$ \emph{p-simuliert effektiv} $h'$ falls Funktionen $f,g\in\FP$ existieren sodass
    \begin{enumerate}
        \item $x\in L \implies f(x)\in L$,
        \item $ h'(w)=f(x) \implies h(g(x, w)) = x. $
    \end{enumerate}
    Wir schreiben in diesem Fall auch $h\leq^\mathrm p_\mathrm{eff} h'$.\marginnote{\note{Nicht vergessen: es ist offenbar unklar, wie das Symbol $\leq$ bzgl. Simulation gebraucht wird. Dose/Glaßer würden in der Def. das Ordnungszeichen spiegeln, Krajíček/Pudlák dagegen genau so wie hier.}}
\end{definition}
In anderen Worten, falls $h\leq^p_\mathrm{eff} h'$, dann kann $h$ zwar nicht \emph{jeden} $h'$-Beweis $w$ für $x\in L$ in einen $h$-Beweis für (das gleiche) $x$ effizient umrechnen, es kann aber zumindest alle \emph{relevanten} $h'$-Beweise effizient umrechnen, nämlich für jedes $x\in L$ die $h'$-Beweise für $f(x)$ in $h$-Beweise für $x$.

Klar ist: p-Simulation impliziert effektive p-Simulation impliziert Simulation unter Beweissystemen.

\begin{conjecture}[$\mathsf{KvL}$ formuliert unter Beweissystemen]\label{conj:kvl-ps}
    Das Standardbeweissystem $\mathit{sat}$ für $\mathtt{SAT}$ kann nicht alle anderen optimalen Beweissysteme für $\mathtt{SAT}$ effektiv p-simulieren.

    In anderen Worten, es existiert optimales Beweissystem $h$ sodass $\mathit{std}\not\leq^\mathrm p_\mathrm{eff} h$.
\end{conjecture}
Dass die Formulierung der Vermutungen \ref{conj:kvl} und \ref{conj:kvl-ps} äquivalent sind, zeigt folgende Beobachtung:
\begin{observation}
    Folgende Aussagen sind äquivalent:
    \begin{enumerate}
        \item Jede NP-Relation $R$ mit  $\leqmp$-vollständigem $\Proj(R)$, ist auch $\leqlp$-vollständig.
        \item Für alle optimalen Beweissysteme $h$ für $\mathtt{SAT}$ gilt $\mathit{std}\leq^\mathrm p_\mathrm{eff} h$.
    \end{enumerate}
\end{observation}
\begin{proof}
    (1)$\Rightarrow$(2): 
    Wir zeigen, dass $\mathit{sat}$ jedes andere optimale Beweissystem $h$ effektiv p-simulieren kann.
    Nachdem $h$ optimal ist, hat es auch kurze Beweise: für jedes $\phi\in\mathtt{SAT}$ einen $h$-Beweis $w$ mit $|w|\leq q(|\phi|)$.  Definiere
    \[ R_h = \{ (\phi, w) \mid |w|\leq q(|\phi|), h(w) = \phi\}. \]
    Diese Relation ist offenbar eine NP-Relation und $\Proj(R_h) = \mathtt{SAT}$ und damit ist $\Proj(R_h)$ auch $\leqmp$-vollständig. 

    Nach Voraussetzung ist also  $R_h$ auch $\leqlp$-vollständig.
    Insbesondere gilt also  auch $\mathtt{rSAT} \leqlp R_h$. Damit existieren also Funktionen $f,g\in\FP$ sodass
    $x\in \mathtt{SAT} \leftrightarrow f(x)\in\mathtt{SAT}$ und
    \[ (f(\phi), w) \in R_h \implies (\phi, g(\phi, w))\in \mathtt{rSAT}. \]
    Nach Definition gilt also 
    \[ h(w)=f(\phi) \implies \mathit{sat}(g(\phi, w) = \phi, \]
    und damit ist $\mathit{sat} \leq^\mathrm p_\mathrm{eff} h$.
    \medskip

    (2)$\Rightarrow$(1): 
    Sei $R$ eine NP-Relation wobei $\Proj(R)$ $\leqmp$-vollständig ist. Wir zeigenen nun, dass $R$ auch $\leqlp$-vollständig ist.
    Aus der $\leqmp$-Vollständigkeit folgt unmittelbar die Existenz einer Reduktionsfunktion $f$ mit 
    \[ \phi\in\mathtt{SAT} \iff f(\phi) \in \Proj(R). \]
    Definiere
    \[ h(w) = \begin{cases} \phi & \text{falls $w=(x,y,\phi)$ und $f(\phi)=x$ und $(x,y)\in R$} \\ \bot & \text{sonst.} \end{cases} \]
    Wir zeigen, dass $h$ ein Beweissystem für $\mathtt{SAT}$ ist. Es ist offenbar dass $h\in\FP$. Die Funktion $h$ ist korrekt: wenn $h(x,y,\phi)=\phi$ dann ist $f(\phi)=x\in\Proj(R)$ und nach Eigenschaft von $f$ auch $\phi\in\mathtt{SAT}$.
    Die Funktion $h$ ist vollständig: Sei $\phi\in\mathtt{SAT}$. Dann ist schon $f(\phi)\in\Proj(R)$ und es gibt ein $y$ mit $(f(\phi),y)\in R$. Also ist $(f(\phi), y,\phi)$ ein $h$-Beweis für $\phi$.

    Außerdem ist klar, dass $h$ ehrlich ist.
    Definiere
    \[ R_h = \{ (\phi, w) \mid h(w) = \phi\}, \]
    diese Relation ist damit eine NP-Relation.
    Wir wollen nun zeigen dass $\mathtt{rSAT}\leqlp R_h \leqlp Q$, womit dann $Q$ auch $\leqlp$-vollständig ist, wie gewünscht.

    Wir starten mit der zweiten Reduktion.
    Es gilt $R_h \leqlp Q$, denn es gilt
    \[ (f(\phi), y)\in Q \implies h(\underbrace{f(\phi), y, \phi}_{g(\phi, y)})=\phi \implies (\phi, g(\phi, y))\in R_h, \]
    wobei $g(\phi, y)=(f(\phi), y, \phi)$.

    Für die erste Reduktion nutzen wir die Voraussetzung.
    Es gilt nach Voraussetzung $\mathit{std}\leq^\mathrm p_\mathrm{eff} h$. Damit existieren also Funktionen $f', g'\in \FP$ mit
    \begin{gather*}
        \phi\in\mathtt{SAT} \iff f'(\phi)\in\mathtt{SAT} \\
        h(w)=f'(\phi) \implies \mathit{sat}(g'(\phi, w)) = \phi.
    \end{gather*}
    Jetzt ist aber auch klar, dass $f', g'$ die Reduktion $\mathtt{rSAT}\leqlp R_h$ realisieren, denn nun gilt
    \[ (f'(\phi), w)\in R_h \implies (\phi, g'(\phi, w))\in \mathtt{rSAT}. \]
\end{proof}

Mit der Definition der effektiven p-Simulation und der eben beweisenen äquivalenten Formulierung der KvL-Vermutung lässt sich nun zumindest die Hypothese $\mathsf{SAT}$ so verstärken, dass diese hinreichend für $\mathsf{KvL}$ ist.


\begin{conjecture}[$\mathsf{SAT^{eff}}$]
    Kein optimales Beweissystem für $\mathtt{SAT}$ kann alle anderen optimalen Beweissysteme für $\mathtt{SAT}$ effektiv p-simulieren.
    In anderen Worten, für jedes optimales Beweissystem $h$ existiert ein  optimales Beweissysteme $h'$ sodass $h\not\leq^\mathrm p_\mathrm{eff} h'$.
\end{conjecture}

\begin{theorem}
    \begin{enumerate}
        \item $\mathsf{SAT^{eff}}\implies \mathsf{SAT}$
        \item $\mathsf{SAT^{eff}}\implies \mathsf{KvL}$
    \end{enumerate}
\end{theorem}
\begin{proof}
    Zu (1): Klar aus Kontraposition. Wenn ein p-optimales Beweissystem $h$ für $\mathtt{SAT}$ existiert, dann kann dieses (optimale) $h$ auch alle anderen Beweissysteme p-simulieren, und damit insbesondere auch alle optimalen Beweissyste $h'$ effektiv p-simulieren.
    \medskip

    Zu (2): Wieder klar aus Kontraposition. Unter $\neg\mathsf{KvL}$ folgt mit der Formulierung aus Vermutung~\ref{conj:kvl-ps} dass das (optimale) Standardbeweissystem $\mathit{sat}$ alle optimalen Beweissysteme effektiv p-simulieren kann. Dann existiert also auch \emph{ein} optimales Beweissystem welches dies leistet.
\end{proof}

\section{Bekannte Implikationen, Offene Orakel}

\begin{figure}
    \hspace*{-12mm}\includegraphics[page=1]{figures.pdf}
    \caption{Bekannte (relativierenden) Implikationen zwischen den betrachteten Hypothesen und weiteren. Satz~\ref{thm:figure-implications} gibt Belegstellen für jede dieser Implikationen an.}\label{fig:figure-implications}
\end{figure}

\begin{theorem}\label{thm:figure-implications}
    Es gelten die in Abbildung \ref{fig:figure-implications} abgebildeten Implikationen und Äquivalenzen.
\end{theorem}
\begin{proof}
\begin{prooflist}
\item $\hDisjNP \Rightarrow \mathsf{CON}^N$ nach \textcite{kobler_optimal_2003}.
\item $\hUP \Rightarrow \hTAUT$ nach \textcite{kobler_optimal_2003}.
\item $\mathsf{CON}^N\Rightarrow \mathrm{NEE\neq coNEE}$ nach \textcite{kobler_optimal_2003}.
\item $\NP\cap\coNP\neq \P \Rightarrow \neq\hQ' \Rightarrow \NPMVt \not\subseteq_c \TFNP \Rightarrow \neq\hQ$ nach \textcite{fenner_inverting_2003}.
\item $\neg\hQ\Rightarrow \exists$ NP-Relation die nicht auf Entscheidung reduzierbar ist, denn unter $\neg\hQ$ gilt mit Satz~\ref{thm:q} auch die Negation von \ref{thm:q}(1), also eine NPTM $N$ mit $L(N)=\Sigma^*$ wobei keine Funktion $g\in\FP$ existiert, welche für alle $x$ durch $g(x)$ einen akzeptierenden Rechenweg von $N(x)$ bestimmt.
    Definiere die NP-Relation $R_N$ mit $(x,\alpha)\in R_N$ genau dann wenn $N(x)$ mit Rechenweg $\alpha$ existiert. Nun gilt nach Vorigem auch $R\not\in_c \FP=\FP^{\Sigma^*} = \FP^{L(R)}$.
\item $\mathrm{EE\neq NEE}\Rightarrow \exists$ NP-Relation die nicht auf Entscheidung reduzierbar ist, nach \textcite{bellare_complexity_1994}.
\item $\mathrm{EXP\neq NEXP}\Rightarrow \exists$ NP-Relation die nicht auf Entscheidung reduzierbar ist, nach Impagliazzo und Sudan (private Kommunikation berichtet von \cite{bellare_complexity_1994}, Abschn. 1.5.4).
%UPneqP DisjNPinsep
%SATeff SAT
%SatEff KvL
% KvL negQ
\item $\hNPcoNP\Rightarrow \hTAUT\lor\hSAT$ nach \textcite{beyersdorff_nondeterministic_2009}.
\item $\NPMVt$ hat keine vollständige Funktion $\Rightarrow \hSAT$ nach \textcite{beyersdorff_nondeterministic_2009}.
\item $\NPMVt$ hat keine vollständige Funktion $\Rightarrow \NP\neq\coNP$ nach Satz~\ref{}.
\item $\hNPcoNP \Rightarrow \hTFNP \Rightarrow \NPMVt$ hat keine vollständig Funktion, nach \textcite{pudlak_incompleteness_2017}.
\item $\NP\cap\coNP\neq \P \Rightarrow\exists$ P-inseparierbares $\DisjNP$-Paar, denn wenn alle $\DisjNP$-Paare P-separierbar, dann ist auch für jede Menge $L\in\NP\cap\coNP$ jeweils das $\DisjNP$-Paar $(L,\overline{L})$ P-separierbar und damit $L\in\P$.
%\item $\hDisjNP \Rightarrow \exists$ P-inseparierbares $\DisjNP$-Paar; ist klar, denn wenn alle $\DisjNP$-Paare P-separierbar wären, dann wären auch alle Paare $\leqmpp$-vollständig.
%\item $\hDisjCoNP \Rightarrow \exists$ P-inseparierbares $\DisjCoNP$-Paar; ist aus selben Gründen klar.
%\item $\mathsf{CON}^N \Rightarrow \hTAUT$ klar, weil aus p-Optimalität auch Optimalität folgt.
%\item $\mathsf \hSAT \Rightarrow \neg\hQ $ klar, denn wenn $\mathit{sat}$ p-optimal ist, dann existiert \emph{ein} p-optimales Beweissystem für $\mathtt{SAT}$.
%\item $\hUP \Rightarrow \UP\neq\P$ klar.
%\item $\hNPcoNP \Rightarrow \NP\cap\coNP\neq \P$ klar.
%\item $\exists$ P-inseparierbares $\DisjNP$-Paar $\Righarrow \P\neq \NP$ klar.
%\item $\UP\cap\coUP\Rightarrow \UP\neq \P$, $\UP\cap\coUP\Rightarrow \NP\cap\coNP\neq\P$ klar.
%\item $\mathrm{NEE\neq coNEE \Rightarrow NEXP \neq coNEXP \Rightarrow NP \neq coNP \Rightarrow P\neq NP}$ klar.
%\item $\mathrm{NEE\neq coNEE \Rightarrow EE \neq NEE}$ klar.
\end{prooflist}
\end{proof}

\printbibliography

