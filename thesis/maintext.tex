\chapter{Definitionen}

\begin{itemize}
    \item Vllt. sollten wir NP-Relationen als FNP-Probleme definieren. Andererseits wird in der aktuellen Literatur (T)FNP primär auf Funktionen verstanden.
    \item NP-Relation definieren: Zertifikate mind. so lange wie Probleme, alternativ streng monoton steigend
    \item Levin-Reduktion definieren
\end{itemize}

\begin{theorem}
    Die kanonische NP-Relation
    \[ \mathtt{rKAN} = \{ ((N,x,1^n), \alpha) \mid \text{$\alpha$ ist ein akz. Rechenweg auf $N(x)$ und $|\alpha|\leq n$} \} \]
    ist $\leq_\mathrm{L,inv}^\mathrm p$-vollständig.
\end{theorem}
\begin{proof}
    Sei $R$ eine beliebige NP-Relation mit Zertifikatsschranke $r$, i.e. $(x,y)\in R\implies |y|\leq r(|x|)$. Sei $M$ die PTM welche $R$ entscheidet, mit Laufzeitschranke $p$. Sei $N$ eine NPTM welche auf Eingabe $x$ zunächst ein Zertifikat $y, |y|\leq r(|x|)$ rät, und dann testet ob $M(\langle x,y\rangle )$ akzeptiert. Die Laufzeit von $N$ ist beschränkt auf $p(|\langle x,y\rangle|)\in O(p(r(|x|))$; sei daher $q$ ein Polynom, welches die Laufzeit von $N$ beschränkt.

    Definiere die Reduktionsfunktion $f(x)=(N, x, 1^{q(|x|)})$. Es gilt $x\in \Proj(R)\implies f(x)\in \Proj(\mathtt{rKAN})$.
    Wenn $x\in\Proj(R)$, dann existiert ein $y, |y|\leq r(|x|)$ sodass $(x,y)\in R$. Dann wird auch $N(x)$ akzeptieren, nämlich auf jenem Pfad welcher $y$ rät. Es existiert also ein Rechenweg $\alpha$ mit $|\alpha|\leq q(|x|)$ sodass $N(x)$ auf $\alpha$ akzptiert. Dann gilt aber auch $(f(x), \alpha)=((N,x,1^{q(|x|)}),\alpha)\in \mathtt{rKAN}$.
    Die Rückrichtung $x\not\in \Proj(R)\implies f(x)\not\in\Proj(R)$ folgt analog.
    Es ist klar, dass $f$ Polynomialzeit-berechenbar und -invertierbar ist. 

    Es lässt sich außerdem einfach eine Rekonstruktionsfunktion $g\in \FP$ angeben, die für $g(f(x), \alpha)=y$ aus $\alpha$ das entsprechende geratene Zertifikat $y$ aus $\alpha$ berechenen kann.
\end{proof}

\chapter{Suchprobleme und die Hypothese Q}

\begin{itemize}
    \item Will Q in den Pudlák-Baum einordnen: dafür ist es notwendig, diese ordentlich zu relativieren. Insb. will ich zeigen, dass einige bisherige Resultate natürlicherweise auf „Standardbeweissysteme“ vollständiger Mengen übertragen (nicht nur das Standardbeweissystem für SAT).
\end{itemize}

\begin{definition}[Levin-Paddability]
    Eine NP-Relation $R$ ist \emph{Levin-paddable} wenn 
    Funktionen $\mathit{pad}\in\FP$ und $\mathit{padsol}\in\FP$ existieren sodass
    \begin{enumerate}
        \item $x\in \Proj(R) \iff \mathit{pad}(x, 1^n) \in \Proj(R)$,
        \item $(\mathit{pad}(x, 1^n), y)\in R \implies (x, \mathit{padsol}(x, 1^n, y)) \in R$,
        \item $|\mathit{pad}(x, 1^n)|\geq n$.
    \end{enumerate}
\end{definition}

\begin{lemma}\label{lemma:stdps-q}
    Sei $R$ eine NP-Relation die Levin-paddable ist. Folgende Aussagen sind äquivalent:
    \begin{enumerate}
        \item Für alle NPTM $N$ mit $L(N)=\Proj(R)$ lassen sich akzepierende Rechenwege von $N$ in Zertifikate umrechnen: es existiert eine Funktion $h\in\mathrm{FP}$ sodass
            \[ N(x) \text{ akz. mit RW $\alpha$} \implies (x,h(x,\alpha))\in R. \]
        \item Das Standard-Beweissystem $\mathit{std}_R$ bzgl. $R$ ist p-optimal. Zur Erinnerung: 
            \[ \mathit{std}_R(w) = \begin{cases} x & \text{wenn $w=(x,y)$ und $(x,y)\in R$}\\
            \top & \text{sonst}\end{cases}. \] 
    \end{enumerate}
\end{lemma}

\begin{lemma}\label{lemma:q-generalized}
%Sei $A\in\mathrm{NP}$ mit folgender Eigenschaft von Vollständigkeit: es existiert eine Menge $B\in\mathrm P$ sodass $A=\{x\mid \exists y, |y|\leq p(|x|), (x,y)\in B\}$ und für alle Mengen $A'\in\mathrm{NP}$, $A'=\{x\mid \exists y, |y|\leq p'(|x|), (x,y)\in B'\}$ existieren zwei Funktionen $r,r^{-1},t\in\mathrm{FP}$ sodass
%\[ x\in A' \iff r(x) \in A, \quad (r(x),z)\in B \implies (x, t(x,z)) \in B'. \]
%($A'\leq_m^p A$ via invertierbarem $r$, Funktion $t$ bildet Zertifikate für $r(x)\in A$ auf Zertifikate für $x\in A'$ ab. Vgl. Reduktionsbegriff unter TFNP-Problemen. Vgl. Levin-Reduktionsbegriff.)
Sei $R$ eine $\leqlp$-vollständige NP-Relation, mit der zusätzlichen Eigenschaft dass für die jeweilige entsprechende Problem-Reduktionsfunktion $f\colon Q\to R$ für $Q\leqlp R$ immer gilt, dass $f$ ehrlich ist. %$\img f = \{f(x)\mid x\in\Sigma^*\}\in \P$.
Folgende Aussagen sind äquivalent:
\begin{enumerate}
    \item Für alle NPTM $N$ mit $L(N)=\Proj(R)$ lassen sich akzepierende Rechenwege von $N$ in Zertifikate umrechnen: es existiert eine Funktion $h\in\mathrm{FP}$ sodass
        \[ N(x) \text{ akz. mit RW $\alpha$} \implies (x,h(x,\alpha))\in R. \]
    \item Für alle NPTM $N$ mit $L(N)=\Sigma^*$ lassen sich aus Eingabe $x$ Rechenwege von $N(x)$ effizient bestimmen: es existiert $r\in\mathrm{FP}$ sodass $N(x)$ auf Rechenweg $r(x)$ akzeptiert. (Das ist die Aussage Q.)
\end{enumerate}
\end{lemma}


\begin{lemma}
    Die in Lemma~\ref{lemma:stdps-q} und~\ref{lemma:q-generalized} genannten Voraussetzungen an die NP-Relation $R$ werden von allen solchen $R$ erfüllt, die $\leqlp$-vollständig sind und Levin-paddable sind.
\end{lemma}
\begin{proof}
    Es ist sofort klar, dass $R$ die Voraussetzungen von Lemma~\ref{lemma:stdps-q} erfüllt.
    Es bleibt nur zu zeigen, dass für jede NP-Relation $Q$ eine $\leqlp$-Reduktion angegeben werden kann, bei dem die Problem-Reduktionsfunktion ehrlich ist.
    Wir nutzen hierbei aus, dass $R$ eine Levin-paddable Relation ist.

    Nachdem $R$ vollständig ist, gilt $Q\leqlp R$; sei $f,g\in\FP$ die Reduktions- bzw. Rekonstruktionsfunktion welche diese Reduktion realisieren. Wir werden nun Funktionen $f', g'\in\FP$ angeben, welche die gleiche Reduktion realisieren, aber $f'$ ist nichtverkürzend, und damit ehrlich, wie gewünscht.

    Sei $\mathit{pad}, \mathit{padsol}$ die zu $R$ zugehörigen Padding-Funktionen. Definiere
    \[ f'(x) = \mathit{pad}(f(x), 1^{|x|}). \]
    Es gilt
    \[ x\in\Proj(Q) \iff f(x)\in \Proj(R) \iff \mathit{pad}(f(x), 1^{|x|})=f'(x)\in\Proj(R), \]
    wobei erste Implikation die Eigenschaft der Reduktionsfunktion $f$ ist, und die zweite aus der Definition von Levin-Paddability folgt.
    Aus der Definition von  Levin-Paddability $f'$ folgt auch $|f'(x)|\geq |x|$ und damit ist $f'$ nichtverkürzend.

    Definiere
    \[ g'(x, z) = g(x, \mathit{padsol}(f(x), 1^{|x|}, z)). \]
    Sei nun $(f'(x), z)\in R$. Die Funktion $g'$ berechnet nun ein Zertifikat $y$ für $x$: Wir haben $(\mathit{pad}(f(x), 1^{|x|}), z)\in R$, also gilt nach Levin-Paddability dass \[(f(x), \mathit{padsol}(f(x), 1^{|x|}, z))\in R,\] 
    und nach Definition der Rekonstruktionsfunktion $g$ gilt dann
    \[(x, g(x, \mathit{padsol}((f(x), 1^{|x|}, z)))\in Q,\]
    und das ist genau $(x, g'(x, z))\in Q$, wie gewünscht.
\end{proof}

\section{Welche Suchprobleme sind paddable?}

\begin{observation}
    Die kanonische Levin-vollständige NP-Relation $\mathtt{rKAN}$ ist Levin-paddable.
\end{observation}

\begin{observation}\label{obs:joinable-sind-levinpaddable}
    Jede $\leq_\mathrm{L,inv}^\mathrm{p}$-vollständige NP-Relation $R$ ist auch Levin-paddable.
\end{observation}

\begin{observation}\label{obs:joinable-sind-levinpaddable}
    %Jede universelle Relation ist Levin-paddable. Dieses Resultat gilt nur im unrelativierten Fall.
    Jede NP-Relation mit einem \emph{building block} und die \emph{joinable} ist, ist auch Levin-paddable.
\end{observation}
\begin{corollary}
    Im unrelativierten Fall erfüllt jede universelle Relation $R$ die in 
    Lemma~\ref{lemma:stdps-q} und~\ref{lemma:q-generalized} genannten Voraussetzungen an die NP-Relation $R$.

    Das sind u.a. $\mathtt{rSAT}, \mathtt{rHAM}, \mathtt{rINDSET}, \mathtt{rKNAPSACK}, \mathtt{rMAXCUT}$.
\end{corollary}
\begin{proof}[Beweis zu Beobachtung~\ref{obs:joinable-sind-levinpaddable}]
    Sei $R$ eine NP-Relation, mit zugehörigem Polynom $q$, welches die Zertifikatsgröße spezifiziert. Zur Erinnerung, dieses Polynom ist streng monoton steigend, und aus $(x,y)\in R$ folgt $|y|=q(|x|)$.
    Wir zeigen zunächst, wie wir für beliebige Instanz $x$ und $n\in\mathbb N$ auf eine Instanz $x'$ der Länge $\geq n$ hochpadden.

    %Nachdem $R$ universell ist, existiert eine zertifikatserhaltende Reduktion $f\in FP$ von $\mathtt{rSAT}$ auf $R$. 
    %Es lässt sich leicht eine Familie $\phi_1, \phi_2, \ldots$ an positiven Instanzen für $\mathtt{rSAT}$ erdenken, für die $|\phi_i|\geq i$ und $\phi_i$ effizient auf Eingabe $1^i$ konstruiert werden kann.
    %Zur Erinnerung: wir haben für $\mathtt{rSAT}$ das zugehörige Polynom der Zertifikatsgröße so gewählt, dass die Zertifikate genau so lange wie die Probleminstanzen sind. Damit gilt
    %für $(\phi_i, w)\in \mathtt{rSAT}$ dann auch $|w|=|\phi_i|\geq i$.

    %Sei nun $(z,\alpha)=f(\phi_{q(n)})$, d.h. $\phi_i$ ist auf eine $R$-Instanz $z$ reduziert worden. Nach Definition von zertifikatserhaltenden Reduktionen gilt $|\alpha|=|\phi_{q(n)}|\geq {q(n)}$, und $\alpha$ besteht aus paarweise verschiedenen Indizes. Wir haben nun nach Definition
    %\[ \{ y'[\alpha] \mid y'\in \Sigma^{q(|z|)}, (z,y')\in R \} = \{ y\mid y\in \Sigma^{|\phi_{q(n)}|}, (\phi_{q(n)},y)\in\mathtt{rSAT}\} \neq \emptyset, \]
    %in anderen Worten, aus Zertifikaten $y'$ für $z$ können wir durch geeignete Projektion via $\alpha$ die Zertifikate für $w$ rekonstruieren. Davon existiert mindestens eins.
    %Alle Zertifikate $y'$ haben eine feste Länge $q(|z|)$, und da $\alpha$ aus $q(n)$ paarweise verschiedenen Indizes besteht, 
    %gilt $q(|z|) \geq q(n)$. % und aus Monotonie folgt $|z|\geq n$.

    Nach Voraussetzung hat  die Relation $R$ einen \emph{building block} $\mathit{block}$. Es lässt sich leicht aus der Definition eines \emph{building block} ableiten, dass $|\mathit{block}|>0$ und $\mathit{block}\in\Proj(R)$. Damit gilt auch dass die Zertifikate $y$ zu $\mathit{block}$ die Länge $l=q(|\mathit{block}|)\geq |\mathit{block}|\geq 1$ haben.

    Nach Voraussetzungen  ist die Relation $R$ auch \emph{joinable}, das heißt wir haben eine Funktion $\mathit{join}\in\FP$. Sei 
    \[ (x',\delta)=\mathit{join}(x, \underbrace{\mathit{block}, \mathit{block}, \ldots, \mathit{block}}_{\text{$q(n)$ mal}}).\]
    Nach Definition ?? gilt
    \[ |\delta|=q(|x|)+q(n)\cdot q(|\mathit{block}|)\geq q(n). \]
    Beob., dass alle Zertifikate $y'$ für $x'$ die feste Länge $q(|x'|)$ haben. 
    Zur Erinnerung: wir haben
    \begin{multline}\label{eq:levinpad-join} \{ y'[\delta] \mid y'\in\Sigma^{q(|x'|)}, (x', y')\in R \} = \{ yy_1y_2\cdots y_{q(n)} \mid y\in\Sigma^{q(|x|)}, y_1,y_2, \ldots \in \Sigma^{l},\\ (x,y),(\mathit{block}, y_1),(\mathit{block}, y_2), \ldots \in R\} \end{multline}
    Die Sequenz $\delta$ besteht nach Definition aus paarweise verschiedenen Indizes, daher können wir argumentieren, dass auch alle Zertifikate $y'$ (mit vorgegebener Länge $q(|x'|)$) mindestens die Länge $|\delta|$ haben.
    Damit gilt
    \[ q(|x'|) \geq |\delta| \geq q(n) \]
    und nach Monotonie haben wir $|x'|\geq n$.

    Sei nun $\mathit{pad}$ genau jene polynomialzeit-berechenbare Funktion, die aus $x$ und $1^n$ die Instanz $x'$ konstruiert:
    \[ \mathit{pad}(x, 1^n) = x' \quad\text{ wobei }
    (x',\delta)=\mathit{join}(x, \underbrace{\mathit{block}, \mathit{block}, \ldots, \mathit{block}}_{\text{$q(n)$ mal}}).\]
    Dann gilt schon sofort, dass $|\mathit{pad}(x, 1^n)|=|x'|\geq n$ wie gewünscht.

    Wir zeigen jetzt, dass die Zugehörigkeit zu $\Proj(R)$ erhalten bleibt:
    Gilt $x\not\in\Proj(R)$, dann ist die rechte Menge in (\ref{eq:levinpad-join}) leer, also auch die linke Menge und damit $x'=\mathit{pad}(x, 1^n)\not\in \Proj(R)$.
    Falls anders herum $x\in\Proj(R)$, dann ist die rechte Menge nicht leer, existiert ja ein Zertifikat $y$ für $x$ und je ein weiters $y_i$ für $\mathit{block}$. Also ist auch die linke Menge nicht leer, damit $\mathit{pad}(x, 1^n)\in \Proj(R)$.

    Die noch verbleibende Funktion $\mathit{padsol}$ ist durch die bitweise Projektion durch $\delta$ leicht möglich:
    \[
        \mathit{padsol}(x, 1^n, y') = y'[\delta[1:q(|x|)]] \quad\text{ wobei } (\cdot, \delta) = \mathit{join}(x, \underbrace{\mathit{block}, \mathit{block}, \ldots, \mathit{block}}_{\text{$q(n)$ mal}}).\]
    Wir verifizieren: Sei $(\mathit{pad}(x, 1^n), y')\in R$, dann ist nach (\ref{eq:levinpad-join}) $y'[\delta]=yy_1y_2\cdots$ wobei $y\in\Sigma^{q(|x|)}$, $(x, y)\in R$. 
    Wir haben
    \[ \mathit{padsol}(x, 1^n, y') = y'[\delta[1:q(|x|)]] = (yy_1y_2\cdots)[1:q(|x|)] = y \]
    und damit $(x, \mathit{padsol}(x, 1^n, y')) = (x, y)\in R$, wie gewünscht.
\end{proof}

