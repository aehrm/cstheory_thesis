%! TEX root = ./thesis.tex
\chapter{Definitionen}

\begin{itemize}
    \item Vllt. sollten wir NP-Relationen als FNP-Probleme definieren. Andererseits wird in der aktuellen Literatur (T)FNP primär auf Funktionen verstanden.
    \item NP-Relation definieren: Zertifikate mind. so lange wie Probleme, alternativ streng monoton steigend
    \item Levin-Reduktion definieren
\end{itemize}

\begin{theorem}
    Die kanonische NP-Relation
    \[ \mathtt{rKAN} = \{ ((N,x,1^n), \alpha) \mid \text{$\alpha$ ist ein akz. Rechenweg auf $N(x)$ und $|\alpha|\leq n$} \} \]
    ist $\leq_\mathrm{L,inv}^\mathrm p$-vollständig.
\end{theorem}
\begin{proof}
    Sei $R$ eine beliebige NP-Relation mit Zertifikatsschranke $r$, i.e. $(x,y)\in R\implies |y|\leq r(|x|)$. Sei $M$ die PTM welche $R$ entscheidet, mit Laufzeitschranke $p$. Sei $N$ eine NPTM welche auf Eingabe $x$ zunächst ein Zertifikat $y, |y|\leq r(|x|)$ rät, und dann testet ob $M(\langle x,y\rangle )$ akzeptiert. Die Laufzeit von $N$ ist beschränkt auf $p(|\langle x,y\rangle|)\in O(p(r(|x|))$; sei daher $q$ ein Polynom, welches die Laufzeit von $N$ beschränkt.

    Definiere die Reduktionsfunktion $f(x)=(N, x, 1^{q(|x|)})$. Es gilt $x\in \Proj(R)\implies f(x)\in \Proj(\mathtt{rKAN})$.
    Wenn $x\in\Proj(R)$, dann existiert ein $y, |y|\leq r(|x|)$ sodass $(x,y)\in R$. Dann wird auch $N(x)$ akzeptieren, nämlich auf jenem Pfad welcher $y$ rät. Es existiert also ein Rechenweg $\alpha$ mit $|\alpha|\leq q(|x|)$ sodass $N(x)$ auf $\alpha$ akzptiert. Dann gilt aber auch $(f(x), \alpha)=((N,x,1^{q(|x|)}),\alpha)\in \mathtt{rKAN}$.
    Die Rückrichtung $x\not\in \Proj(R)\implies f(x)\not\in\Proj(R)$ folgt analog.
    Es ist klar, dass $f$ injektiv ist, dass $f$ Polynomialzeit-berechenbar und -invertierbar ist. 

    Es lässt sich außerdem einfach eine Rekonstruktionsfunktion $g\in \FP$ angeben, die für $g(f(x), \alpha)=y$ aus $\alpha$ das entsprechende geratene Zertifikat $y$ aus $\alpha$ berechenen kann.
\end{proof}

\chapter{Suchprobleme und die Hypothese Q}

\begin{itemize}
    \item Will Q in den Pudlák-Baum einordnen: dafür ist es notwendig, diese ordentlich zu relativieren. Insb. will ich zeigen, dass einige bisherige Resultate natürlicherweise auf „Standardbeweissysteme“ vollständiger Mengen übertragen (nicht nur das Standardbeweissystem für SAT).
\end{itemize}


\begin{definition}[Levin-Paddability]
    Eine NP-Relation $R$ ist \emph{Levin-paddable} wenn 
    Funktionen $\mathit{pad}\in\FP$ und $\mathit{padsol}\in\FP$ existieren, sowie ein Polynom $r$ sodass
    \begin{enumerate}
        \item $x\in \Proj(R) \iff \mathit{pad}(x, 1^n) \in \Proj(R)$,
        \item $(\mathit{pad}(x, 1^n), y)\in R \implies (x, \mathit{padsol}(x, 1^n, y)) \in R$,
        \item $r(|\mathit{pad}(x, 1^n)|)\geq n$. (Funktion $\mathit{pad}$ ist ehrlich bzgl. der zweiten Komponente.)
    \end{enumerate}
\end{definition}

\begin{lemma}\label{lemma:stdps-q}
    Sei $R$ eine NP-Relation die Levin-paddable ist. Folgende Aussagen sind äquivalent:\marginnote{Generalisierung von Messner}
    \begin{enumerate}
        \item Für alle NPTM $N$ mit $L(N)=\Proj(R)$ lassen sich akzepierende Rechenwege von $N$ in Zertifikate umrechnen: es existiert eine Funktion $h\in\mathrm{FP}$ sodass
            \[ N(x) \text{ akz. mit RW $\alpha$} \implies (x,h(x,\alpha))\in R. \]
        \item Das Standard-Beweissystem $\mathit{std}_R$ bzgl. $R$ ist p-optimal. Zur Erinnerung: 
            \[ \mathit{std}_R(w) = \begin{cases} x & \text{wenn $w=(x,y)$ und $(x,y)\in R$}\\
            \top & \text{sonst}\end{cases}. \] 
    \end{enumerate}
\end{lemma}

\begin{lemma}\label{lemma:q-generalized}
%Sei $A\in\mathrm{NP}$ mit folgender Eigenschaft von Vollständigkeit: es existiert eine Menge $B\in\mathrm P$ sodass $A=\{x\mid \exists y, |y|\leq p(|x|), (x,y)\in B\}$ und für alle Mengen $A'\in\mathrm{NP}$, $A'=\{x\mid \exists y, |y|\leq p'(|x|), (x,y)\in B'\}$ existieren zwei Funktionen $r,r^{-1},t\in\mathrm{FP}$ sodass
%\[ x\in A' \iff r(x) \in A, \quad (r(x),z)\in B \implies (x, t(x,z)) \in B'. \]
%($A'\leq_m^p A$ via invertierbarem $r$, Funktion $t$ bildet Zertifikate für $r(x)\in A$ auf Zertifikate für $x\in A'$ ab. Vgl. Reduktionsbegriff unter TFNP-Problemen. Vgl. Levin-Reduktionsbegriff.)
Sei $R$ eine $\leqlp$-vollständige NP-Relation, mit der zusätzlichen Eigenschaft dass für die jeweilige entsprechende Problem-Reduktionsfunktion $f\colon Q\to R$ für $Q\leqlp R$ immer gilt, dass $f$ ehrlich ist.\marginnote{Generalisierung von  Fenner et al.} %$\img f = \{f(x)\mid x\in\Sigma^*\}\in \P$.
Folgende Aussagen sind äquivalent:
\begin{enumerate}
    \item Für alle NPTM $N$ mit $L(N)=\Proj(R)$ lassen sich akzepierende Rechenwege von $N$ in Zertifikate umrechnen: es existiert eine Funktion $h\in\mathrm{FP}$ sodass
        \[ N(x) \text{ akz. mit RW $\alpha$} \implies (x,h(x,\alpha))\in R. \]
    \item Für alle NPTM $N$ mit $L(N)=\Sigma^*$ lassen sich aus Eingabe $x$ Rechenwege von $N(x)$ effizient bestimmen: es existiert $r\in\mathrm{FP}$ sodass $N(x)$ auf Rechenweg $r(x)$ akzeptiert. (Das ist die Aussage Q.)
\end{enumerate}
\end{lemma}


\begin{lemma}
    Die in Lemma~\ref{lemma:stdps-q} und~\ref{lemma:q-generalized} genannten Voraussetzungen an die NP-Relation $R$ werden von allen solchen $R$ erfüllt, die $\leqlp$-vollständig sind und Levin-paddable sind.
\end{lemma}
\begin{proof}
    Es ist sofort klar, dass $R$ die Voraussetzungen von Lemma~\ref{lemma:stdps-q} erfüllt.
    Es bleibt nur zu zeigen, dass für jede NP-Relation $Q$ eine $\leqlp$-Reduktion angegeben werden kann, bei dem die Problem-Reduktionsfunktion ehrlich ist.
    Wir nutzen hierbei aus, dass $R$ eine Levin-paddable Relation ist.

    Nachdem $R$ vollständig ist, gilt $Q\leqlp R$; sei $f,g\in\FP$ die Reduktions- bzw. Rekonstruktionsfunktion welche diese Reduktion realisieren. Wir werden nun Funktionen $f', g'\in\FP$ angeben, welche die gleiche Reduktion realisieren, aber $f'$ ehrlich, wie gewünscht.

    Sei $\mathit{pad}, \mathit{padsol}$ die zu $R$ zugehörigen Padding-Funktionen. Definiere
    \[ f'(x) = \mathit{pad}(f(x), 1^{|x|}). \]
    Es gilt
    \[ x\in\Proj(Q) \iff f(x)\in \Proj(R) \iff \mathit{pad}(f(x), 1^{|x|})=f'(x)\in\Proj(R), \]
    wobei erste Implikation die Eigenschaft der Reduktionsfunktion $f$ ist, und die zweite aus der Definition von Levin-Paddability folgt.
    Aus der Definition von  Levin-Paddability folgt auch $r(|f'(x)|)\geq |x|$ für ein geeignetes Polynom $r$, und damit ist auch $f'$ ehrlich.

    Definiere
    \[ g'(x, z) = g(x, \mathit{padsol}(f(x), 1^{|x|}, z)). \]
    Sei nun $(f'(x), z)\in R$. Die Funktion $g'$ berechnet nun ein Zertifikat $y$ für $x$: Wir haben $(\mathit{pad}(f(x), 1^{|x|}), z)\in R$, also gilt nach Levin-Paddability dass \[(f(x), \mathit{padsol}(f(x), 1^{|x|}, z))\in R,\] 
    und nach Definition der Rekonstruktionsfunktion $g$ gilt dann
    \[(x, g(x, \mathit{padsol}((f(x), 1^{|x|}, z)))\in Q,\]
    und das ist genau $(x, g'(x, z))\in Q$, wie gewünscht.
\end{proof}

\section{Welche Suchprobleme sind paddable?}

\begin{observation}
    Die kanonische Levin-vollständige NP-Relation $\mathtt{rKAN}$ ist Levin-paddable.
\end{observation}

\begin{observation}\label{obs:invcomplete-sind-levinpaddable}
    %Jede $\leq_\mathrm{L,inv}^\mathrm{p}$-vollständige NP-Relation $R$ ist auch Levin-paddable.
    \begin{enumerate}
        \item Gilt $\mathtt{rKAN}\leq_\mathrm{L}^\mathrm{p} R$, und ist die zugehörige Reduktionsfuktion $f$ ehrlich, dann ist $R$ Levin-paddable
        \item Jede $\leq_\mathrm{L,inv}^\mathrm{p}$-vollständige NP-Relation $R$ ist auch Levin-paddable.
    \end{enumerate}
\end{observation}
\begin{corollary}
    Jede $\leq_\mathrm{L,inv}^\mathrm{p}$-vollständige Relation $R$ erfüllt die in 
    Lemma~\ref{lemma:stdps-q} und~\ref{lemma:q-generalized} genannten Voraussetzungen an die NP-Relation $R$.

    Das sind im unrelativierten Fall u.a. $\mathtt{rSAT}$, $\mathtt{rSETCOVER}$, $\mathtt{rVERTEXCOVER}$, $\mathtt{rCLIQUE}$, $\mathtt{r3COLORABILITY}$.\marginnote{aus den Goldreich-Textbook}
\end{corollary}
\begin{proof}[Beweis zu Beobachtung~\ref{obs:invcomplete-sind-levinpaddable}]
    Aussage (2) folgt unmittelbar aus (1): Wir haben $\mathtt{rKAN}\leq_\mathrm{L,inv}^\mathrm{p} R$ und damit ist die entsprechende Reduktionsfunktion $f$ p-invertierbar, und damit ehrlich.

    Für (1) nutzen wir die Levin-Paddability von $\mathtt{rKAN}$ aus: übersetze Instanz $x$ von $R$ nach $\mathtt{rKAN}$, padde dort hoch, und überetze zu $R$-Instanz $x'$ zurück. Ist dann $y'$ ein Zertifikat für $x'$, dann lässt sich dies auf ähnlichem Weg wieder zu einem Zertifikat für $x$ zurückrechnen.

    Seien $f, g$ die Reduktions- bzw. Reproduktionsfunktion, welche $\mathtt{rKAN}\leq_\mathrm{L}^\mathrm p R$ bezeugen, und seinen analog $f', g'$ jene Funktionen, welche $R\leq_\mathrm{L}^\mathrm p \mathtt{rKAN}$ bezeugen. Erstere existieren nach Voraussetzung, zweitere existieren weil $\mathtt{rKAN}$ $\leq_\mathrm{L}^\mathrm p$-vollständig ist.
    Nach Voraussetzung ist $f$ ehrlich. %, und da $f'$ p-invertierbar ist, ist auch $f'$ ehrlich. 
    Und nach Beobachtung ?? existieren für $\mathtt{rKAN}$ Padding-Funktionen $\mathit{pad}_\mathtt{rKAN}$, $\mathit{padsol}_\mathtt{rKAN}$.
    Sei $q$ ein entsprechendes Polynom mit $q(|\mathit{pad}_\mathtt{rKAN}(x, 1^n)|)\geq n$, $q(|f(x)|) \geq |x|$.

    Definiere nun
    \[ \mathit{pad}_R(x, 1^n) = f(\mathit{pad}_\mathtt{rKAN}(f'(x), 1^n)). \]
    Die Zugehörigkeit zu $\Proj(R)$ bleibt erhalten:
    \begin{multline*}
        x\in \Proj(R) \iff f'(x) \in \mathtt{KAN} \iff \mathit{pad}_\mathtt{rKAN}(f'(x), 1^n) \in \mathtt{KAN}\\ \iff f(\mathit{pad}_\mathtt{rKAN}(f'(x), 1^n)) \in \Proj(R) \iff \mathit{pad}_R(x, 1^n) \in\Proj(R).
    \end{multline*}
    Ferner gilt
    \begin{align*} &q(q(|\mathit{pad}_R(x, 1^n)|)) \\&= q(q(|f(\mathit{pad}_\mathtt{rKAN}(f'(x), 1^n)|))\\&\geq q(|\mathit{pad}_\mathtt{rKAN}(f'(x), 1^n)|)\\ &\geq n.
    \end{align*}
    und damit ist $\mathit{pad}_R$ wie gewünscht ehrlich bzgl. $n$ (mit Polynom $q\circ q$).

    Es verbleibt noch die Funktion $\mathit{padsol}_R$. Nehme hierfür an dass wir ein $y'$ haben mit $(\mathit{pad}_R(x, 1^n), y')\in R$.
    Wir können über $g, g'$ das Zertifikat $y'$ zu Zertifikat $y$ mit $(x, y)\in R$ zurück übersetzen:
    Sei $p=\mathit{pad}_\mathtt{rKAN}(f'(x), 1^n)$, dann gilt
    \[ (f(p), y')\in R \implies (p, \underbrace{g(p, y')}_z)\in \mathtt{rKAN}. \]
    Definere $z=g(p, y')$.
    Nun haben wir
    \begin{gather*} (p, z)=(\mathit{pad}_\mathtt{rKAN}(f'(x), 1^n), z)\in\mathtt{rKAN}  \\\quad\implies (f'(x), \underbrace{\mathit{padsol}_\mathtt{rKAN}(f'(x), 1^n, z)}_{z'})\in\mathtt{rKAN} \end{gather*}
    und mit $z'=\mathit{padsol}_\mathtt{rKAN}(f'(x), 1^n, z)$ gilt
    \[ (f'(x), z') \in \mathtt{rKAN} \implies (x, \underbrace{g'(x, z')}_{y}) \in R. \]
    %\[ \mathit{padsol}_R(x, 1^n, y') = g(\mathit{pad}_\mathtt{rKAN}(f'(x), 1^n), y')
    %\[ q(q(q(|\mathit{pad}_R(x, 1^n)|))) = q(q(q(|f(\mathit{pad}_\mathtt{rKAN}(f'(x), 1^n))|)))
    Es ist leicht zu sehen, dass sich eine Funktion $\mathit{padsol}_R\in\FP$ angeben kann, die aus $x, 1^n$ dieses entsprechende $y$ berechnen kann.
\end{proof}

\begin{observation}\label{obs:joinable-sind-levinpaddable}
    %Jede universelle Relation ist Levin-paddable. Dieses Resultat gilt nur im unrelativierten Fall.
    Jede NP-Relation mit einem \emph{building block} und die \emph{joinable} ist, ist auch Levin-paddable.
\end{observation}
\begin{corollary}
    Im unrelativierten Fall erfüllt jede universelle Relation $R$ die in 
    Lemma~\ref{lemma:stdps-q} und~\ref{lemma:q-generalized} genannten Voraussetzungen an die NP-Relation $R$.

    Das sind u.a. $\mathtt{rSAT}, \mathtt{rHAM}, \mathtt{rINDSET}, \mathtt{rKNAPSACK}, \mathtt{rMAXCUT}$.
\end{corollary}
\begin{proof}[Beweis zu Beobachtung~\ref{obs:joinable-sind-levinpaddable}]
    Sei $R$ eine NP-Relation, mit zugehörigem Polynom $q$, welches die Zertifikatsgröße spezifiziert. Zur Erinnerung, dieses Polynom ist streng monoton steigend, und aus $(x,y)\in R$ folgt $|y|=q(|x|)$.
    Wir zeigen zunächst, wie wir für beliebige Instanz $x$ und $n\in\mathbb N$ auf eine Instanz $x'$ hochpadden, in dem Sinne dass $q(|x'|) \geq n$.

    %Nachdem $R$ universell ist, existiert eine zertifikatserhaltende Reduktion $f\in FP$ von $\mathtt{rSAT}$ auf $R$. 
    %Es lässt sich leicht eine Familie $\phi_1, \phi_2, \ldots$ an positiven Instanzen für $\mathtt{rSAT}$ erdenken, für die $|\phi_i|\geq i$ und $\phi_i$ effizient auf Eingabe $1^i$ konstruiert werden kann.
    %Zur Erinnerung: wir haben für $\mathtt{rSAT}$ das zugehörige Polynom der Zertifikatsgröße so gewählt, dass die Zertifikate genau so lange wie die Probleminstanzen sind. Damit gilt
    %für $(\phi_i, w)\in \mathtt{rSAT}$ dann auch $|w|=|\phi_i|\geq i$.

    %Sei nun $(z,\alpha)=f(\phi_{q(n)})$, d.h. $\phi_i$ ist auf eine $R$-Instanz $z$ reduziert worden. Nach Definition von zertifikatserhaltenden Reduktionen gilt $|\alpha|=|\phi_{q(n)}|\geq {q(n)}$, und $\alpha$ besteht aus paarweise verschiedenen Indizes. Wir haben nun nach Definition
    %\[ \{ y'[\alpha] \mid y'\in \Sigma^{q(|z|)}, (z,y')\in R \} = \{ y\mid y\in \Sigma^{|\phi_{q(n)}|}, (\phi_{q(n)},y)\in\mathtt{rSAT}\} \neq \emptyset, \]
    %in anderen Worten, aus Zertifikaten $y'$ für $z$ können wir durch geeignete Projektion via $\alpha$ die Zertifikate für $w$ rekonstruieren. Davon existiert mindestens eins.
    %Alle Zertifikate $y'$ haben eine feste Länge $q(|z|)$, und da $\alpha$ aus $q(n)$ paarweise verschiedenen Indizes besteht, 
    %gilt $q(|z|) \geq q(n)$. % und aus Monotonie folgt $|z|\geq n$.
    \strut\marginnote{Es sieht nicht danach aus, dass wir über die Anzahl an Zertifikaten für $x'$ die Länge abschätzen können: ist $x\not\in\Proj(R)$ dann hätte auch $x'$ keine Zertifikate, und könnte sich erlauben entsprechend kurz zu sein.}
    Nach Voraussetzung hat  die Relation $R$ einen \emph{building block} $\mathit{block}$. Es lässt sich leicht aus der Definition eines \emph{building block} ableiten, dass $|\mathit{block}|>0$ und $\mathit{block}\in\Proj(R)$. Damit gilt auch dass die Zertifikate $y$ zu $\mathit{block}$ die Länge $l=q(|\mathit{block}|)\geq |\mathit{block}|\geq 1$ haben.

    Nach Voraussetzungen  ist die Relation $R$ auch \emph{joinable}, das heißt wir haben eine Funktion $\mathit{join}\in\FP$. Sei 
    \[ (x',\delta)=\mathit{join}(x, \underbrace{\mathit{block}, \mathit{block}, \ldots, \mathit{block}}_{\text{$n$ mal}}).\]
    Wir werden nun über die Länge $|\delta|$ auf die Länge von Zertifikaten zu $x'$ schließen, und damit $|x'|$ beschränken.
    Nach Definition ?? gilt
    \[ |\delta|=q(|x|)+q(n)\cdot q(|\mathit{block}|)=q(|x|)+q(n)\cdot l\geq n. \]
    Beob. dass unter Definition ?? alle Zertifikate $y'$ für $x'$ die feste Länge $q(|x'|)$ haben. 
    Zur Erinnerung: wir haben
    \begin{multline}\label{eq:levinpad-join} \{ y'[\delta] \mid y'\in\Sigma^{q(|x'|)}, (x', y')\in R \} = \{ yy_1y_2\cdots y_{n} \mid y\in\Sigma^{q(|x|)}, y_1,y_2, \ldots \in \Sigma^{l},\\ (x,y),(\mathit{block}, y_1),(\mathit{block}, y_2), \ldots \in R\} \end{multline}
    Die Sequenz $\delta$ besteht nach Definition aus paarweise verschiedenen Indizes, daher können wir argumentieren, dass auch alle Zertifikate $y'$ (mit vorgegebener Länge $q(|x'|)$) mindestens die Länge $|\delta|$ haben.
    Damit gilt
    \[ q(|x'|) \geq |\delta| \geq n \]
    wie gewünscht.

    Sei nun $\mathit{pad}$ genau jene polynomialzeit-berechenbare Funktion, die aus $x$ und $1^n$ die Instanz $x'$ konstruiert:
    \[ \mathit{pad}(x, 1^n) = x' \quad\text{ wobei }
    (x',\delta)=\mathit{join}(x, \underbrace{\mathit{block}, \mathit{block}, \ldots, \mathit{block}}_{\text{$q(n)$ mal}}).\]
    Dann gilt schon sofort, dass $q(|\mathit{pad}(x, 1^n)|)=q(|x'|)\geq n$ wie gewünscht.

    Wir zeigen jetzt, dass die Zugehörigkeit zu $\Proj(R)$ erhalten bleibt:
    Gilt $x\not\in\Proj(R)$, dann ist die rechte Menge in (\ref{eq:levinpad-join}) leer, also auch die linke Menge und damit $x'=\mathit{pad}(x, 1^n)\not\in \Proj(R)$.
    Falls anders herum $x\in\Proj(R)$, dann ist die rechte Menge nicht leer, existiert ja ein Zertifikat $y$ für $x$ und je ein weiters $y_i$ für $\mathit{block}$. Also ist auch die linke Menge nicht leer, damit $\mathit{pad}(x, 1^n)\in \Proj(R)$.

    Die noch verbleibende Funktion $\mathit{padsol}$ ist durch die bitweise Projektion durch $\delta$ leicht möglich:
    \[
        \mathit{padsol}(x, 1^n, y') = y'[\delta[1:q(|x|)]] \quad\text{ wobei } (\cdot, \delta) = \mathit{join}(x, \underbrace{\mathit{block}, \mathit{block}, \ldots, \mathit{block}}_{\text{$n$ mal}}).\]
    Wir verifizieren: Sei $(\mathit{pad}(x, 1^n), y')\in R$, dann ist nach (\ref{eq:levinpad-join}) $y'[\delta]=yy_1y_2\cdots$ wobei $y\in\Sigma^{q(|x|)}$, $(x, y)\in R$. 
    Wir haben
    \[ \mathit{padsol}(x, 1^n, y') = y'[\delta[1:q(|x|)]] = (yy_1y_2\cdots)[1:q(|x|)] = y \]
    und damit $(x, \mathit{padsol}(x, 1^n, y')) = (x, y)\in R$, wie gewünscht.
\end{proof}

\section{Alles zusammensetzen\ldots}

\begin{theorem}[Äquivalente Formulierungen der Hypothese $Q$; Fenner et al., 2003]
    Folgende Aussagen sind äquivalent:
    \begin{enumerate}
        \item Jede surjektive ehrliche Funktion $f\in\FP$ ist p-invertierbar. %Zur Erinnerung: p-Invertierbarkeit bedeutet dass eine Funktion $g\in\FP$ existiert mit $f(g(y))=y$.
        \item $\mathrm{NPMV}_t\subseteq_c \mathrm{FP}$
        \item $\P=\NP\cap\coNP$ und $\mathrm{NPMV}_t\subseteq_c \mathrm{NPSV}_t$
        \item Für jede NPTM $N$ mit $L(N)=\Sigma^*$ existiert eine Funktion $g\in\FP$ sodass für alle $x$ das Bild $g(x)$ eine akzeptierende Berechnung von $N(x)$ ist.
        \item Für jede Menge $L\in \P$  und jede NPTM $N$ mit $L(N)=L$ existiert eine Funktion $h\in \FP$ mit 
            \[ x\in L \implies N(x) \text{ akz. mit RW $h(x)$}. \]
        \item Für jedes Paar von NP-Relationen $A, B$ gilt:
            \[ \Proj(A) \leqmp \Proj(B) \iff A \leqlp B, \]
            wobei die zwei entsprechenden Reduktionsfuktionen sogar identisch sind.
        \item Für jedes Beweissystem $h$ gilt: $h$ ist optimal $\iff$ $h$ ist p-optimal. 
        \item Es existiert eine $\leqlp$-vollständige Levin-paddable NP-Relation $R$ sodass für alle NPTM $N$ mit $L(N)=\Proj(R)$ gilt: es existiert eine Funktion $h\in\mathrm{FP}$ mit
            \[ N(x) \text{ akz. mit RW $\alpha$} \implies (x,h(x,\alpha))\in R. \]
        \item[(\arabic{enumi}\,$'\!$)] Für jede $\leqlp$-vollständige Levin-paddable NP-Relation $R$, und für alle NPTM $N$ mit $L(N)=\Proj(R)$ gilt: es existiert eine Funktion $h\in\mathrm{FP}$ mit
            \[ N(x) \text{ akz. mit RW $\alpha$} \implies (x,h(x,\alpha))\in R. \]
        \item Es existiert eine $\leqlp$-vollständige Levin-paddable NP-Relation $R$ für welche das Standardbeweissystem $\mathit{std}_R$ p-optimal ist.
        \item[(\arabic{enumi}\,$'\!$)] Für jede $\leqlp$-vollständige Levin-paddable NP-Relationen $R$ ist das Standardbeweissystem $\mathit{std}_R$ p-optimal.
        \item Es existiert eine $\leq_\mathrm{L,1-1,inv}^\mathrm p$-vollständige NP-Relation $R$ sodass für jede Menge $S\in \P$ mit $S\subseteq \Proj(R)$ gilt: es existiert eine Funktion $g\in\mathrm{FP}$ sodass
            \[ x\in S \implies (x, g(x))\in R. \]
        \item[(\arabic{enumi}\,$'\!$)] Für jede $\leq_\mathrm{L,1-1,inv}^\mathrm p$-vollständige NP-Relation $R$ und jede Menge $S\in \P$ mit $S\subseteq \Proj(R)$ gilt: es existiert eine Funktion $g\in\mathrm{FP}$ sodass
            \[ x\in S \implies (x, g(x))\in R. \]
    \end{enumerate}
\end{theorem}
\begin{proof}
\begin{prooflist}
\item (1)$\iff$(2)$\iff$(3)$\iff$(4)$\iff$(5): nach Fenner et al. (2003, Thm. 2).

\item (4)$\iff$(8)$\iff$(9): nach Lemma ?? und ??

\item (5)$\implies$(10): Wähle für $R$ eine beliebige $\leq_\mathrm{L,1-1,inv}^\mathrm p$-vollständige NP-Relation (diese existiert immer, z.B. $\mathtt{rKAN}$), wobei Polynom $q$ die Zertifikatsgröße beschränkt. Sei nun $S\subseteq \Proj(R)$ mit $S\in \P$. Definiere die NPTM $N$, welche auf Eingabe $x$ folgendes leistet: teste zuerst ob $x\in S$; falls nicht, lehne sofort ab. Rate dann ein $y\in\Sigma^{\leq q(|x|)}$ und akzeptiere genau dann wenn $(x,y)\in R$. 

    Klar ist, dass $L(N)=S$. Nach (5) existiert nun eine Funktion $h\in \FP$, die für $x\in S$ einen akzeptierenden Rechenweg $h(x)$ von $N(x)$ ausgibt. Wir können sogar aus $h(x)$ das geratene Zertifikat $y$ extrahieren. Es ist daher leicht eine Funktion $g\in \FP$ anzugeben für die $(x,g(x))\in R$ für alle $x\in S$.

\item (10)$\implies$(5): Sei $L\in \P$ und sei $N$ eine NPTM mit $L(N)=L$, wobei das Polynom $q$ die Laufzeit beschränkt. Wir wollen eine Funktion $h\in\FP$ definieren sodass $h(x)$ ein akzeptierender Rechenweg von $N(x)$ für $x\in L$ ist. Definiere die NP-Relation
    \[ Q = \{ (x, y) \mid \text{$N(x)$ akzeptiert mit Rechenweg $y\in\Sigma^{\leq q(|x|)}$} \}. \]
    Nachdem (10) gilt, haben wir eine $\leq_\mathrm{L,1-1,inv}^\mathrm p$-vollständige NP-Relation $R$. Damit gilt $Q\leqlp R$ mittels Reduktions- bwz. Rekonstruktionsfunktion $f, k\in FP$. Insbesondere existiert eine Inverse $f^{-1}\in\FP$ zu $f$.

    Sei $S=f(L)$ die Bildmenge der Elemente aus $L$, also 
    \[ S= \{f(x) \mid x\in L\}. \]
    Es ist leicht zu sehen dass $S\subseteq \Proj(R)$. Außerdem ist $S\in \P$: teste $z\in S$ indem getestet wird ob $f^{-1}(z)=x\neq\bot$ und ob $x\in L$.

    Damit sind die Voraussetzungen von (10) erfüllt, und es existiert eine Funktion $g\in \FP$ sodass $(z,g(z))\in R$ für alle $z\in S$. Damit gilt
    \begin{gather*}x\in L \implies f(x) \in S \implies (f(N,x), g(f(x)))\in R\\ \implies ((x), k(g(f(x))))\in Q\\
    \implies N(x) \text{ akz. mit Rechenweg $\underbrace{k(g(f(x)))}_{h(x)}$. }
\end{gather*}
    Definiere nun die gesuchte Funktion $h\in \FP$ mit $h(x) = k(g(f(x)))$. Damit gilt für alle $x\in L$ dass $N(x)$ mit Rechenweg $h(x)$ akzeptiert, wie gewünscht.

\item (4)$\implies$(6): Die Richtunng von rechts nach links ist klar. Für die andere Richtung sei $\Proj(A) \leqmp \Proj(B)$ mit $A,B$ NP-Relationen. Sei $q$ hierbei das Polynom was die Zertifikatslänge in $A$ begrenzt.
    Wir wollen nun eine Levin-Reduktion von $A$ auf $B$ angeben. Sei $f\in \FP$ die Funktion, welche die Reduktion $\Proj(A) \leqmp \Proj(B)$ realisiert.

    Definiere folgende NPTM $N$, die wie folgt auf Eingabe $w$ arbeitet: \\
    \begin{algorithm}[H]
        \lIf{$w$ nicht von der Form $(x, y')$}{akzeptiere}
        $(x, y')\gets w$\;
        \lIf{$(f(x), y')\not\in B$}{akzeptiere}
        \Else{
        \tcc{Ab hier gilt $f(x)\in \Pr(B)$ und $x\in \Pr(A)$}
        Rate nichtdeterministisch $y\in \Sigma^{q(|x|)}$\;
        Akzeptiere genau dann wenn $(x,y)\in A$.
        }
    \end{algorithm}
    Es ist leicht zu sehen dass $L(N)=\Sigma^*$. Nach (4) existiert nun also eine Funktion $g$ sodass, für alle $w$, $g(w)$ eine akzeptierender Rechwenweg von $N(w)$ ist.

    Damit lässt die Levin-Reduktion von $A$ auf $B$ angeben: wähle $f$ als Reduktionsfunktion, und definiere $g'$ als Rekonstruktionsfunktion, welche aus dem akzeptierenden Rechenweg $g(x, y')$ das geratene Zertifikat $y$ von Zeile 6 ausliest. Dann gilt
    \begin{gather*}
        (f(x), y') \in B \implies N(x, y') \text{ akz. auf einem RW in Z. 6, rät dabei $y$}\\
        \implies (x,y)=(x, g'(x, y'))\in A
    \end{gather*}
    wie gewünscht. Wir haben $A\leqlp$ via $f, g'$.
\item (6)$\implies$(8): Sei $R$ die $\leqlp$-vollständige und Levin-paddable NP-Relation $R$ und sei $N$ eine NPTM $N$ mit $L(N)=\Proj(R)$, wobei das Polynom $q$ die Laufzeit von $N$ beschränkt.  Definiere
    \[ Q = \{ (x, y) \mid \text{$N(x)$ akzeptiert mit Rechenweg $y\in\Sigma^{\leq q(|x|)}$} \}. \]

    Es ist klar, dass $\Proj(R)\leqmp \Proj(Q)=L(N)=\Proj(R)$ über die Identiätsfunktion.
    Nach (6) gilt nun auch $R\leqlp Q$. mit Identitätsfunktion als Reduktionsfunktion und $h\in \FP$ als Rekonstruktionsfunktion.
    Damit gilt
    \[ \text{$N(x)$ akz. mit RW $\alpha$} \implies (x, \alpha)\in Q \implies (x, h(x, \alpha))\in R \]
    wie gewünscht.


\item (2)$\implies$(7): Die Richtung von rechts nach links ist klar. Sei für die andere Richtung $h$ ein optimales Beweissystem für eine Menge $L$. Wir wollen zeigen, dass $h$ auch p-optimal ist. Sei dafür $g$ ein weiteres Beweissystem für $L$. Nach Voraussetzung haben wir $g\leq h$, das heißt es existiert eine (nicht notwendigerweise effiziente) Funktion $f$ sodass $g(w)=h(f(w))$, und gleichzeitig ist $|f(w)|\leq q(|w|)$ für ein geeignetes Polynom $q$.

Betrachte folgende Multifunktion $f'$:
%\[ f'(w) = \{ y \mid y\in\Sigma^{\leq q(|w|)}, g(w)=h(y) \}. \]
\[ f'(w) \mapsto y \iff \exists y\in\Sigma^{\leq q(|w|)}, g(w)=h(y). \]
Es lässt sich leicht zeigen, dass $f'\in\NPMV$, über einen geeigneten NPTM-Transduktor. 
Es ist sogar $f'\in\NPMV_t$, denn für jedes $w$ mindestens $f(w)\in \fset{}f'(w)$.

Nach (2) gilt also $f'\in\NPMV_t\subseteq_c \FP$, also existiert eine Funktion $f''\in\FP$ welche eine Verfeinerung von $f'$ ist. Diese Funktion übersetzt $g$-Beweise $w$ für $x$ effizient in $h$-Beweise für $x$: 
Sei $g(w)=x$, dann gilt
\[ f''(w) = y \quad\text{mit } y\in\Sigma^{\leq q(|w|)}, x=g(w)=h(y) \]
also ist $h(f''(w))=x$ bzw. $f''(w)$ ein $h$-Beweis für $x$, wie gewünscht.


\item (7)$\implies$(9): klar, denn $\mathtt{rKAN}$ ist $\leqlp$-vollständig, ist Levin-paddable, und das Standardbeweissystem $\mathit{std}_\mathtt{rKAN}$ ist (wie jedes Standardbeweissystem einer NP-Relation) optimal. Zusammen mit (7) ist es also auch p-optimal.
\end{prooflist}
\end{proof}

