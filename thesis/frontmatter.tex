\newgeometry{margin=3cm}
\pagenumbering{roman}
\pagestyle{empty}

\begin{center}{\setstretch{1.7}

\vfill\sffamily

{\Large\textbf{Masterarbeit}}\par
    \vspace*{1cm}
{\addfontfeature{LetterSpace=15.0}\huge{KOMPLEXITÄT VON SUCHPROBLEMEN UND BEWEISSYSTEMEN}}\par
    \vspace*{.3cm}
{\addfontfeature{LetterSpace=15.0}\Large{ANTON EHRMANNTRAUT}}\par}
\vspace*{6cm}

\includegraphics[width=4cm]{siegel.pdf}

\vspace*{4cm}


%\setstretch{1.1}\par
\bgroup\sffamily\addfontfeature{LetterSpace=10.0}
{\large\MakeUppercase{\today}}\vspace*{.7cm}

{\large BETREUER: PROF. DR. CHRISTIAN GLASSER}\vspace*{.5cm}


{\large
JULIUS\,-\,MAXIMILIANS\,-\,UNIVERSITÄT WÜRZBURG\\
LEHRSTUHL FÜR INFORMATIK I\\
ALGORITHMEN UND KOMPLEXITÄT
}
\egroup


\end{center}

\cleardoublepage
\restoregeometry

\section*{Zusammenfassung}
In der vorliegenden Masterarbeit werden NP-Suchprobleme betrachtet. Während NP-Entscheidungsprobleme in der Komplexitätstheorie traditionell im Vordergrund stehen, sind NP-Suchprobleme weniger erforscht und viele Fragen sind noch offen.
So zum Beispiel die Hypothese $\hQ$ (\citeauthor{fenner_inverting_1996}, CCC 1996), nach jene NP-Suchprobleme effizient lösbar sind, welche garantiert immer eine Lösung haben.
Selbst der Zusammenhang zwischen der Karp-Vollständigkeit von NP-Entscheidungsproblemen und der Levin-Vollständigkeit von NP-Suchproblemen ist ungeklärt, und ist in der Forschung bisher noch nicht detailliert untersucht worden.

Diese Arbeit trägt zur Vertiefung des Verständnisses dieser Fragen und Hypothesen bei, indem sie den Zusammenhang zwischen NP-Suchproblemen und der Optimalität von Beweissystemen erarbeitet. Damit schließt sich diese Arbeit an das von \citeauthor{pudlak_incompleteness_2017} (\emph{Bull. of Symb. Logic}, 2017) initiierte Forschungsprogramm an, welches die Vollständigkeit von Komplexitätsklassen mit der Optimalität von Beweissystemen in Beziehung setzt. Die Hypothese $\hQ$ wird als Aussage über die P-Optimalität von Standardbeweissystemen für NP-vollständige Suchprobleme charakterisiert, sowie $\neg\hQ$ als Abschwächung der Pudlákschen Hypothesen $\hSAT$ und $\hNPcoNP$ erkannt.
Über eine Orakelkonstruktion ergibt sich ferner eine paarweise Unabhängigkeit der Hypothese $\hQ$ mit den Pudlákschen Hypothesen, außer in den Fällen, in denen wir relativierende Implikationen kennen.

\begin{otherlanguage}{english}
\section*{Abstract}
This master's thesis examines NP search problems. While NP decision problems have traditionally been the focus in complexity theory, NP search problems are less explored, and many questions remain open.
For example the hypothesis $\hQ$ (\citeauthor{fenner_inverting_1996}, CCC 1996), which posits that NP search problems having a guaranteed solution can be solved efficiently.
Even the relationship between the Karp-completeness of NP decision problems and the Levin-completeness of NP search problems is unclear and has not yet been thoroughly investigated in in the literature.

This work contributes to a deeper understanding of these questions and hypotheses by developing the relationship between NP search problems and the optimality of proof systems. Thus, this thesis connects to a research program initiated by \citeauthor{pudlak_incompleteness_2017} (\emph{Bull. of Symb. Logic}, 2017), which relates the completeness of complexity classes to the optimality of proof systems. The hypothesis $\hQ$ is characterized as a statement about the P-optimality of the standard proof systems for NP-complete search problems, and $\neg\hQ$ is recognized as a weakening of Pudlák's hypotheses $\hSAT$ and $\hNPcoNP$.
Furthermore, an oracle construction witnesses the pairwise independence of the hypothesis $\hQ$ from Pudlák's hypotheses, except in cases where we know relativizing implications.
\end{otherlanguage}
\cleardoublepage

\tableofcontents
\thispagestyle{empty}
\cleardoublepage
\pagenumbering{arabic}
\pagestyle{main}

