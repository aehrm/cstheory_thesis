\newgeometry{margin=3cm}
\pagenumbering{roman}
\pagestyle{empty}

\begin{center}{\setstretch{1.7}

\vfill\sffamily

{\Large\textbf{Masterarbeit}}\par
    \vspace*{1cm}
{\addfontfeature{LetterSpace=15.0}\huge{KOMPLEXITÄT VON SUCHPROBLEMEN UND BEWEISSYSTEMEN}}\par
    \vspace*{.3cm}
{\addfontfeature{LetterSpace=15.0}\Large{ANTON EHRMANNTRAUT}}\par}
\vspace*{6cm}

\includegraphics[width=4cm]{siegel.pdf}

\vspace*{4cm}


%\setstretch{1.1}\par
\bgroup\sffamily\addfontfeature{LetterSpace=10.0}
{\large\MakeUppercase{\today}}\vspace*{.7cm}

{\large BETREUER: PROF. DR. CHRISTIAN GLASSER}\vspace*{.5cm}


{\large
JULIUS\,-\,MAXIMILIANS\,-\,UNIVERSITÄT WÜRZBURG\\
LEHRSTUHL FÜR INFORMATIK I\\
ALGORITHMEN UND KOMPLEXITÄT
}
\egroup


\end{center}

\cleardoublepage
\restoregeometry

\section*{Zusammenfassung}
In der vorliegenden Masterarbeit werden NP-Suchprobleme betrachtet. Während NP-Entscheidungsprobleme in der Komplexitätstheorie traditionell im Vordergrund stehen, sind NP-Suchprobleme weniger erforscht und viele Fragen sind noch offen.
So zum Beispiel die Hypothese $\hQ$ (\citeauthor{fenner_inverting_1996}, CCC 1996), nach jene NP-Suchprobleme effizient lösbar sind, welche garantiert immer eine Lösung haben.
Selbst der Zusammenhang zwischen der Karp-Vollständigkeit von NP-Entscheidungsproblemen und der Levin-Vollständigkeit von NP-Suchproblemen ist ungeklärt, und ist in der Forschung bisher noch nicht detailliert untersucht worden.

Diese Arbeit trägt zur Vertiefung des Verständnisses dieser Fragen und Hypothesen bei, indem sie den Zusammenhang zwischen NP-Suchproblemen und der Optimalität von Beweissystemen erarbeitet. Damit schließt sich diese Arbeit an das von Pudlák (\emph{Bull. of Symb. Logic}, 2017) initiierte Forschungsprogramm an, welches die Vollständigkeit von Komplexitätsklassen mit der Optimalität von Beweissystemen in Beziehung setzt. Die Hypothese $\hQ$ wird als Aussage über die P-Optimalität von Standardbeweissystemen für NP-vollständige Suchprobleme charakterisiert, sowie $\neg\hQ$ als Abschwächung der Pudlákschen Hypothesen $\hSAT$ und $\hNPcoNP$ erkannt.
Über eine Orakelkonstruktion ergibt sich ferner eine paarweise Unabhängigkeit der Hypothese $\hQ$ mit den Pudlákschen Hypothesen, außer in den Fällen, in denen wir relativierende Implikationen kennen.
\cleardoublepage

\tableofcontents
\thispagestyle{empty}
\cleardoublepage
\pagenumbering{arabic}
\pagestyle{main}

