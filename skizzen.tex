\documentclass[nofonts]{uebung}
%\setmainfont{NewCM10-Book}[ItalicFont={NewCM10-BookItalic}, BoldFont={NewCM10-Bold}]
%\setmathfont{NewCMMath-Book}
\usepackage[left=2cm,top=2cm, bottom=2cm,right=6cm,marginparwidth=4cm]{geometry}
\setmainfont{Latin Modern Roman}
\usepackage{unicode-math}
\setmathfont{Latin Modern Math}
\setmathfont[version=lm,range={\setminus}, Scale=MatchUppercase]{STIX Two Math}

\setmainlanguage{german}


\usepackage{amsthm}
\newtheorem{theorem}{Theorem}
\newtheorem{proposition}[theorem]{Proposition}
\newtheorem{lemma}[theorem]{Lemma}
\newtheorem{claim}[theorem]{Behauptung}
\newtheorem{corollary}[theorem]{Corollary}
\newtheorem{observation}[theorem]{Observation}
\newtheorem{definition}[theorem]{Definition}


\setlist[enumerate]{label=(\arabic*), itemsep=0pt}
\setlist[itemize]{itemsep=0pt}
\setlist{beginpenalty=10000, midpenalty=10000}

\newlist{statements}{enumerate}{1}
\setlist[statements]{resume,label=(A\arabic*)}

\usepackage{turnstile}

\def\P{\ensuremath{\mathrm{P}}}
\def\NP{\ensuremath{\mathrm{NP}}}
\def\NE{\ensuremath{\mathrm{NE}}}
\def\NEE{\ensuremath{\mathrm{NEE}}}
\def\FP{\ensuremath{\mathrm{FP}}}
\def\UP{\ensuremath{\mathrm{UP}}}
\def\DisjNP{\ensuremath{\mathrm{DisjNP}}}
\def\DisjCoNP{\ensuremath{\mathrm{DisjCoNP}}}
\def\DisjUP{\ensuremath{\mathrm{DisjUP}}}
\def\DisjCoUP{\ensuremath{\mathrm{DisjCoUP}}}
\def\coNP{\ensuremath{\mathrm{coNP}}}
\def\coNE{\ensuremath{\mathrm{coNE}}}
\def\coNEE{\ensuremath{\mathrm{coNEE}}}
\def\coUP{\ensuremath{\mathrm{coUP}}}
\def\NPcoNP{\ensuremath{\mathrm{NP}\cap\mathrm{coNP}}}
\def\TFNP{\ensuremath{\mathrm{TFNP}}}
\def\TALLY{\ensuremath{\mathrm{TALLY}}}
\def\NPMV{\ensuremath{\mathrm{NPMV}}}
\def\NPMVt{\ensuremath{\mathrm{NPMV_t}}}
\def\NPSV{\ensuremath{\mathrm{NPSV}}}
\def\NPSVt{\ensuremath{\mathrm{NPSV_t}}}
\def\NPbV{\ensuremath{\mathrm{NPbV}}}
\def\NPbVt{\ensuremath{\mathrm{NPbV_t}}}
\def\NPkV{\ensuremath{\mathrm{NP}k\mathrm{V}}}
\def\NPkVt{\ensuremath{\mathrm{NP}k\mathrm{V_t}}}
\def\TAUT{\ensuremath{\mathrm{TAUT}}}
\def\SAT{\ensuremath{\mathrm{SAT}}}
\def\PF{\ensuremath{\mathrm{PF}}}
\DeclareMathOperator{\dom}{dom}
\DeclareMathOperator{\img}{img}
\DeclareMathOperator{\supp}{supp}
\def\hUP{\ensuremath{\mathsf{UP}}}
\def\hDisjNP{\ensuremath{\mathsf{DisjNP}}}
\def\hDisjCoNP{\ensuremath{\mathsf{DisjCoNP}}}
\def\hNPcoNP{\ensuremath{\mathsf{NP}{}\cap{}\mathsf{coNP}}}
\def\hCON{\ensuremath{\mathsf{CON}}}
\def\hSAT{\ensuremath{\mathsf{SAT}}}
\def\hTFNP{\ensuremath{\mathsf{TFNP}}}
\def\leqmpp{\ensuremath{\leq_\mathrm{m}^\mathrm{pp}}}
\def\leqmp{\ensuremath{\leq_\mathrm{m}^\mathrm{p}}}


\begin{document}

\renewcommand{\thesubsection}{\arabic{subsection}}
\numberwithin{theorem}{subsection}

\subsection{Orakelkonstruktion $\hDisjNP$, $\hUP$, und $Q$ ($O_1$)}

U.a. Verbesserung der Konstruktion von Khaniki ($\hCON^N\land \TFNP=\FP$, i.e $Q$).

Konstruktion wie bei DG.

Sei $e(0)=2, e(i+1)=2^{e(i)}$. Sei hier $\{H_m\}_{m\in\mathbb N}$ eine Familie von paarweise disjunkten, unendlichen Teilmenge von $e(\mathbb N)$. (Ebenen $H_m$ gehören zur Zeugensprache bzgl. DisjNP-Maschinenpaar $M_a, M_b$.)
Starte mit $\mathrm{PSPACE}$-vollständiger Menge $C$ welche keine Wörter der Länge $e(\cdot)$ enthält.
Definiere folgende Zeugensprachen:
        \begin{gather*}
            A_m^O \coloneqq \{ 0^n \mid n\in H_m, \text{existiert $x\in \Sigma^{n}$ mit } x\in O \text{ und $x$ endet mit $0$} \}\\
            B_m^O \coloneqq \{ 0^n \mid n\in H_m, \text{existiert $x\in \Sigma^{n}$ mit } x\in O \text{ und $x$ endet mit $1$} \}\\
            C_m^O \coloneqq \{ 0^n \mid n\in H_m, \text{existiert $x\in \Sigma^{n}$ mit } x\in O  \}
        \end{gather*}
        Fakt: wenn $|O\cap \Sigma^{n}|\leq 1$ für alle $n\in H_m$, dann $(A_m^O, B_m^O)\in\DisjUP^O$.\\
        Wenn $|O\cap \Sigma^n|\leq 1$ für alle $n\in H_m$, dann $C_m^O \in \UP^O$.
\medskip

Idee: erreiche entweder dass $M_a$, $M_b$ nicht disjunkt akzeptieren (Task $\tau^1_{a,b}$), oder dass das Zeugenpaar $(A_m,B_m)$ nicht auf $(L(M_a),L(M_b))$ reduzierbar ist (Task $\tau^1_{a,b,r}$ für Transduktor $F_r$).

Symmetrisch: erreiche das $M_a$ nicht kategorisch akzeptiert (Task $\tau^3_{a}$), oder dass die Zeugensprache $C_m$ nicht auf $L(M_a)$ reduzierbar ist (Task $\tau^2_{a,r}$ für Transduktor $T_r$).

Gleichzeitig versuchen wir für möglichst viele $M_j$ erreichen, dass diese nicht total sind. (Task $\tau^2_j$)
%Dies hat in gewisser Weise Priorität über die o.g. Diagonalisierung.
Am Ende sind die verbleibenden totalen Maschinen $M^O_j$ sehr speziell, denn sie sind auch für gewisse Teilmengen von $O$ total.
In Kombination mit dem Fakt dass $\P^C=\mathrm{PSPACE}^C$ können wir relevante Wörter in $O-C$ errechnen und so einen akzeptierenden Weg von $M^O_j(x)$ ausgeben -- damit erzielen wir $Q$.
\medskip

Sei wie üblich $t\in \mathcal T$ wenn der Definitionsbereich endlich ist, nur die Tasks der Form $\tau^1_j, \tau^2_{a,b}, \tau^3_a$ enthält, $t$ diese Tasks auf $\mathbb N$ abbildet, und injektiv auf dem Support ist.

Ein Orakel $w\in\Sigma^*$ ist $t$-valide wenn $t\in\mathcal T$ und folgendes gilt:
\begin{enumerate}[label={V\arabic*}]
    \item Wenn $x<|w|$ und $|x|\not\in e(\mathbb N)$, dann gilt $x\in w\iff x\in C$.\\
        (Orakel $w$ und $C$ stimmen auf Wörtern mit Länge $\neq e(\cdot)$ überein.)
    \item Für alle $n=e(i)$ gilt $|w\cap \Sigma^n|\leq 2$.\\
        (Orakel $w$ ist dünn auf den Ebenen der Länge $e(\cdot)$.)
    \item Wenn $t(\tau^1_j)=0$, dann existiert ein $z$ sodass $M_j^w(z)$ definitiv ablehnt.\\
        ($L(M_j)\neq \Sigma^*$ relativ zum finalen Orakel.)
    \item Wenn $t(\tau^2_{a,b})=0$, dann existiert ein $z$ sodass $M_a^w(z)$ und $M_b^w(z)$ definitiv akzeptieren.\\
        (Wenn $t(\tau^2_{a,b})=0$, dann $L(M_a)\cap L(M_b)\neq \emptyset$ relativ zum finalen Orakel.)
    \item Wenn $0<t(\tau^2_{a,b})=m$, dann gilt für alle $n\in H_m$ dass $|\Sigma^{n}\cap w|\leq 1$.\\
        (Wenn $0<t(\tau^2_{a,b})=m$, dann $(A_m,B_m)\in\DisjNP$.)
    \item Wenn $t(\tau^3_{a})=0$, dann existiert ein $z$ sodass $M_a^w(z)$ definitiv auf zwei Rechenwegen akzeptiert.\\
        (Wenn $t(\tau^3_{a})=0$, dann $L(M_a)\not\in \UP$ relativ zum finalen Orakel.)
    \item Wenn $0<t(\tau^3_{a})=m$, dann gilt für alle $n\in H_m$ dass $|\Sigma^n\cap w|\leq 1$.\\
        (Wenn $0<t(\tau^3_{a})=m$, dann $C_m\in\UP$.)
\end{enumerate}

Sei $T$ eine abzählbare Aufzählung der o.g. Tasks sodass $\tau^2_{a,b,r}$ immer nach $\tau^2_{a,b}$ kommt, sowie $\tau^3_{a,r}$ immer nach $\tau^3_a$ kommt.

[ . . . Üblicher Text zur stufenweisen Erweiterung von $w_s$ und $t_s$ . . . ]

Wir definieren nun Stufe $s>0$, diese startet mit einem $t_{s-1}\in\mathcal T$ und eine $t_{s-1}$-validen Orakel $w_{s-1}$ welche nun den kleinsten Task bearbeitet, welcher noch in $T$ ist. Dieser wird unmittelbar nach der Bearbeitung aus $T$ entfernt. In der Bearbeitung wird das Orakel strikt verlängert.
\begin{itemize}

    \item $\tau^1_j$: Setze $t'=t_{s-1}\cup\{\tau^1_j\mapsto 0\}$. Existiert ein $t'$-valides Orakel $v\sqsupsetneq w_{s-1}$, dann setze $t_s\coloneqq t'$ und $w_s\coloneqq v$.

        Ansonsten setze $t_s\coloneqq t_{s-1}$ und setze $w_s\coloneqq w_{s-1}y$ für geeignetes $y\in\{0,1\}$ sodass $w_s$ auch $t_s$-valide ist. (Das ist möglich nach Behauptung 1.1.)

    \item $\tau^2_{a,b}$: Setze $t'=t_{s-1}\cup\{\tau^2_{a,b}\mapsto 0\}$. Existiert ein $t'$-valides Orakel $v\sqsupsetneq w_{s-1}$, dann setze $t_s\coloneqq t'$ und $w_s\coloneqq v$. Entferne außerdem alle Tasks der Form $\tau^2_{a,b,r}$ von $T$.

        Ansonsten wähle ein hinreichend großes $m\not\in \img(t_s)$ sodass $w_s$ kein Wort der Länge $\min H_m$ definiert. Setze $t_s\coloneqq t_{s-1}\cup \{ \tau^2_{a,b}\mapsto m \}$; damit ist $w_{s-1}$ auch $t_s$-valide. Setze $w_s\coloneqq w_{s-1}y$ für geeignetes $y\in\{0,1\}$ sodass $w_s$ auch $t_s$-valide ist. (Das ist möglich nach Behauptung 1.1.)

    \item $\tau^2_{a,b,r}$: Wir wissen dass $t_{s-1}(\tau^2_{a,b})=m>0$. Setze $t_s=t_{s-1}$ und wähle ein $t_s$-valides Orakel $w_s\sqsupsetneq w_{s-1}$ sodass bezüglich einem $n\in\mathbb N$ eine der folgenden Aussagen gilt:
        \begin{itemize}[nosep,endpenalty=10000]
            \item $0^n\in A_m^v$ für alle $v\sqsupseteq w_s$ und $M_a(F_r(0^n))$ lehnt relativ zu $w_s$ definitiv ab.
            \item $0^n\in B_m^v$ für alle $v\sqsupseteq w_s$ und $M_b(F_r(0^n))$ lehnt relativ zu $w_s$ definitiv ab.
        \end{itemize} (Das ist möglich nach Behauptung 1.2.)

    \item $\tau^3_{a}$: Setze $t'=t_{s-1}\cup\{\tau^3_{a,b}\mapsto 0\}$. Existiert ein $t'$-valides Orakel $v\sqsupsetneq w_{s-1}$, dann setze $t_s\coloneqq t'$ und $w_s\coloneqq v$. Entferne außerdem alle Tasks der Form $\tau^3_{a,b,r}$ von $T$.

        Ansonsten wähle ein hinreichend großes $m\not\in \img(t_s)$ sodass $w_s$ kein Wort der Länge $\min H_m$ definiert. Setze $t_s\coloneqq t_{s-1}\cup \{ \tau^3_{a,b}\mapsto m \}$; damit ist $w_{s-1}$ auch $t_s$-valide. Setze $w_s\coloneqq w_{s-1}y$ für geeignetes $y\in\{0,1\}$ sodass $w_s$ auch $t_s$-valide ist. (Das ist möglich nach Behauptung 1.1.)

    \item $\tau^3_{a,r}$: Wir wissen dass $t_{s-1}(\tau^3_{a})=m>0$. Setze $t_s=t_{s-1}$ und wähle ein $t_s$-valides Orakel $w_s\sqsupsetneq w_{s-1}$ sodass bezüglich einem $n\in\mathbb N$ eine der folgenden Aussagen gilt:
        \begin{itemize}[nosep,endpenalty=10000]
            \item $0^n\in C_m^v$ für alle $v\sqsupseteq w_s$ und $M_a(F_r(0^n))$ lehnt relativ zu $w_s$ definitiv ab.
            \item $0^n\not\in C_m^v$ für alle $v\sqsupseteq w_s$ und $M_b(F_r(0^n))$ akzeptiert relativ zu $w_s$ definitiv.
        \end{itemize} (Das ist möglich nach Behauptung 1.3.)
\end{itemize}

\begin{claim}
    Für jedes $t\in\mathcal T$ und jedes $t$-valide $w$ existiert ein $b\in\{0,1\}$ sodass $wb$ auch $t$-valide ist.
\end{claim}

\begin{claim}
    Die Bearbeitung eines Tasks $\tau^2_{a,b,r}$ ist möglich: gilt $t_{s-1}(\tau^2_{a,b})=m>0$, dann lässt sich $w_{s-1}$ so zu $t_{s-1}$-validem $u\sqsupsetneq w_{s-1}$ erweitern, dass eine der o.g. Fälle eintritt.
\end{claim}
\begin{proof}[Skizze.]
    Widerspruchsbeweis. Erweitere $w_{s-1}$ mit Behauptung 1.1 so weit zu $u$, dass genau alle Wörter der Länge $<n=e(i)\in H_m$ definiert sind, wobei das $i$ hinreichend groß gewählt wird. Sei $u(X), X\subseteq \Sigma^n$ das Orakel was entsteht, wenn die Ebene $e(i)$ mit genau den Wörtern aus $X$ gefüllt wird, bzw. $u(X)=u\cup X \cup C$. Beob. dass $u(X), |X|\leq 1$ auch $t_{s-1}$-valide ist.

    Nach Annahme gilt
    \begin{itemize}
        \item für gerades $\alpha\in \Sigma^n$ gilt $0^n\in A_m^{u(\{\alpha\})}$ und daher akzeptiert $M_a(F_r(0^n))$ definitiv relativ zu $u(\{\alpha\})$.
        \item für ungerades $\beta\in \Sigma^n$ gilt $0^n\in B_m^{u(\{\alpha\})}$ und daher akzeptiert $M_b(F_r(0^n))$ definitiv relativ zu $u(\{\beta\})$.
    \end{itemize}
    Kombinatorische Standardmethoden zeigen dann, dass relativ zu $u(\{\alpha,\beta\})$ mit geeignetem geraden $\alpha$, ungeradem $\beta$ sowohl $M_a(F_r(0^n))$ also auch $M_b(F_r(0^n))$ relativ zu $u(\{\alpha,\beta\})$ akzeptieren.
    Damit wäre aber auch $u(\{\alpha,\beta\})$ ein geeignetes Orakel in der Bearbeitung von Task $\tau^2_{a,b}$ und wir hätten $t_{s-1}(\tau^2_{a,b})=0$.
\end{proof}

\begin{claim}
    Die Bearbeitung eines Tasks $\tau^3_{a,r}$ ist möglich: gilt $t_{s-1}(\tau^3_{a,b})=m>0$, dann lässt sich $w_{s-1}$ so zu $t_{s-1}$-validem $u\sqsupsetneq w_{s-1}$ erweitern, dass eine der o.g. Fälle eintritt.
\end{claim}
\begin{proof}[Skizze.]
    Widerspruchsbeweis symmetrisch zu Behauptunng 2. Betrachte wieder die identisch definierten Orakel $u(X)$.
    Nach Annahme gilt
    \begin{itemize}
        \item es gilt $0^n\in C_m^{u(\emptyset)}$ und daher lehnt $M_a(F_r(0^n))$ definitiv relativ zu $u(\emptyset)$ ab. [Das ist wichtig um zu zeigen dass es zwei \emph{unterschiedliche} Berechnungen gibt.]
        \item für gerades $\alpha\in \Sigma^n$ gilt $0^n\in C_m^{u(\{\alpha\})}$ und daher akzeptiert $M_a(F_r(0^n))$ definitiv relativ zu $u(\{\alpha\})$.
        \item für ungerades $\beta\in \Sigma^n$ gilt $0^n\in C_m^{u(\{\beta\})}$ und daher akzeptiert $M_a(F_r(0^n))$ definitiv relativ zu $u(\{\beta\})$.
    \end{itemize}
    Sei für $\xi\in\Sigma^n$ die Menge $Q_\xi$ die Menge an Orakelfragen auf dem akzeptierenden Rechenweg von $M_a(F_r(0^n))$ relativ zu $u(\{\xi\})$.
    Es gilt $\xi\in Q_\xi$, denn andernfalls würde $u(\emptyset)$ und $u(\{\xi\})$ auf $Q_\xi$ übereinstimmen und wir hätten dass auch $M_a(F_r(0^n))$ relativ zu $u(\emptyset)$ akzeptiert. Das widerspricht der Annahme.

    Kombinatorische Standarmethoden zeigen dann, dass es ein gerades $\alpha$, ungerades $\beta$ gibt mit $\alpha\not\in Q_\beta$, $\beta\not\in Q_\alpha$ und so $M_a(F_r(0^n))$ relativ zu $u(\{\alpha,\beta\})$ auf zwei Rechenwegen akzeptiert, je mit Orakelfragen $Q_\alpha$ und $Q_\beta$. Diese Rechenwege sind nicht gleich, da $\alpha\in Q_\alpha$ nach obiger Behauptung, aber $\beta\not\in Q_\beta$.

    Damit wäre aber auch $u(\{\alpha,\beta\})$ ein geeignetes Orakel in der Bearbeitung von Task $\tau^3_{a}$ und wir hätten $t_{s-1}(\tau^3_{a})=0$.
\end{proof}

Damit ist die Konstruktion möglich. Sei $O=\lim_{s\to\infty} w_s$.

\begin{claim}
    Kein Paar aus $\DisjNP^O$ ist $\leqmpp$-hart für $\DisjUP$.
\end{claim}

\begin{claim}
    Keine Menge aus $\UP^O$ ist $\leqmp$-vollständig.
\end{claim}


\begin{claim}
    Sei $M_j$ eine totale Maschine, d.h. $L(M^O)=\Sigma^*$.
    Es existiert eine Länge $n$ mit folgender Eigenschaft: 
    falls $T\subseteq O$ mit $O$ auf Wörtern der Länge $\neq e(\cdot)$ und Wörtern $\leq n$ übereinstimmt, dann $L(M_j^T)=\Sigma^*$.
\end{claim}
\begin{proof}[Skizze.]
    Sei $s$ die Stufe bei der $\tau^1_j$ bearbeitet wurde. Setze $n=|w_{s-1}|$.
    Wir zeigen nun, dass dieses $n$ die behauptete Eigenschaft erfüllt.
    Angenommen, dies gilt nicht, dann existiert ein $T\subseteq O$ dass mit $O$ auf Wörtern der Länge $\neq e(\cdot)$ und Wörtern $\leq n$ übereinstimmt, aber für ein Wort $z$ lehnt $M_j^T(z)$ ab.
    Sei $t'=t_{s-1}\cup \{\tau^1_j\mapsto 0\}$ und sei $v=T\cap\Sigma^{m}$ wobei $m\geq p_j(|x|), n+1$. 
    Beob. dass $M_j^v(z)$ definitiv ablehnt.

    Wir zeigen, das $v$ auch $t'$-valide ist;  damit wäre $v$ eine geeignete Erweiterung in Stufe $s$ und wir hätten $t_s=t'$. 
    Das bedeutet nach V3, dass $M_j^{w_s}(z)$ definitiv ablehnt, damit auch $M_j^O(z)$ ablehnt, was der Voraussetzung widerspricht.

    Wir zeigen dass $v$ auch $t'$-valide ist:
    V1, V2, V5, V7 sind sofort erfüllt nach Definition von $T$ als Teilmenge von $O$ bzw. übereinstimmend mit $O$ auf Wörtern der Länge $\neq e(\cdot)$.
    Da $v\sqsupsetneq w_{s-1}$, sind auch V4 und V6 erfüllt, außer der neue V3-Fall von $t'(\tau_j^2)=0$.

    Aber auch hier erfüllen wir entsprechende Instanz von V3, da ja $M_j^v(z)$ definitiv ablehnt.
\end{proof}

\begin{claim}
    Sei $M_j$ eine totale Maschine, d.h. $L(M_j^O)=\Sigma^*$. Dann existiert eine Funktion $g\in \FP^O$ sodass $g(x)$ einen akzeptierenden Rechenweg von $M^O_j(x)$ ausgibt. Damit gilt nach Definition die Hypothese $Q$ relativ zu $O$.
\end{claim}
\begin{proof}[Skizze.]
    Es reicht aus, dass $g\in\FP^O$ nur Wörter hinreichender Länge verarbeiten muss.
    Sei $n$ hinreichend groß, sodass diese vorige Behauptung 1.6 erfüllt ist.
    Damit gilt: wenn $T\subseteq O$ mit $O$ auf Wörtern der Länge $\neq e(\cdot)$ und Wörtern der Länge $\leq n$ übereinstimmt, dann $L(M_j^T)=\Sigma^*$.

    Sei also nun ein solches $x$ gegeben. 
    Wir werden obige Eigenschaft ausnutzen und iterativ eine Menge $D\subseteq O$ an Orakelwörtern der Länge $e(\cdot)$ aufbauen, welche für die Berechnung $M_j^O(z)$ relevant ist, bis wir alle solchen relevanten Wörter gefunden haben.
    Wir starten hierbei mit der Menge aller Orakelwörter in $O$, welche Länge $\leq n$  und Länge $=e(\cdot)$ haben.
    Beob. dass damit $C\cup D\subseteq O$ und mit $O$ auf Wörtern der Länge $\neq e(\cdot)$ und Wörtern der Länge $\leq n$ übereinstimmt, also auch $L(M_j^{C\cup D})=\Sigma^*$.
    Da uns $D$ vorliegt, können wir sogar diese Orakelwerte in $M$ hineincodieren, sodass $\smash{M'^C_j(z)}$ äquivalent arbeitet. Und da $\smash{\P^C=\mathrm{PSPACE}^C}$, können wir in $\FP^O$ auch einen akzeptierenden Rechenweg von $M'^C_j(x)$ bestimmen.

    \noindent
    \SetKwFor{Loop}{repeat}{}{end}
    \begin{algorithm}[H]
        $D\gets \{ w\mid w\in O, |w|\leq n, \exists i.|w|=e(i)\}$\;
        \Loop{}{
            Sei $\alpha$ ein akzeptierender Rechenweg auf $M_j^{C\cup D}(x)$ und $Q$ die Menge an Orakelfragen\;
            \eIf{existiert eine Frage $q\in Q$ für die $q\in O-C$ aber $q\not\in D$}
            {
                $D\gets D \cup \{q\}$\;
            }
            {
                \Return{$\alpha$}
            }
        }
    \end{algorithm}

    Korrektheit: Beobachte zunächst die Invariante dass $D\subseteq O\cap \{ w \mid \exists i.|w|=e(i)\}$.
    Wie oben argumentiert gilt außerdem, dass $C\cup D\subseteq O$ und mit $O$ auf Wörtern der Länge $\neq e(\cdot)$ und Wörtern der Länge $\leq n$ übereinstimmt. Damit $L(M_j^{C\cup D})=\Sigma^*$ und insbesondere existiert dann auch ein akzeptierender Rechenweg auf $M_j^{C\cup D}(x)$, und Zeile 3 wohldefiniert.

    Terminiert nun der Algorithmus mit einem Rechenweg $\alpha$, wissen wir auch dass für alle Orakelfragen $q\in Q$ entweder $q\in C$ gilt oder $q\not\in O$ oder $q\in O, D$ gilt.
    Damit stimmt $C\cup D$ mit $O$ auf $Q$ überein, und auch $M_j^O(x)$ akzeptiert mit Rechenweg $\alpha$.

    Laufzeit: Wir zeigen dass der Algorithmus in polynomiell beschränkter deterministischer Zeit (abhängig von $|x|$) relativ zu $O$ arbeitet. 
    Wir wissen, dass für jede Orakelfrage $q\in Q$ gilt, dass $|q|\leq p_j(|x|)$.
    Zusammen mit o.g. Invariante gilt $D\subseteq O\cap \{w \mid \exists i.|w|=e(i)\leq p_j(|x|)\}$.
    Nach V2 gilt damit $\ell(D)\leq p_j(|x|)\cdot p_j(|x|)\cdot 2$, denn in den je $\leq p_j(|x|)$ Ebenen der Länge $e(i)\leq p_j(|x|)$ existieren höchstens zwei Wörter der Länge $e(i)\leq p_j(|x|)$.
    Damit folgt auch unmittelbar, dass der Algorithmus nach höchstens polynomiell vielen Iterationen terminiert.

    Zeile 3 kann damit auch in polynomiell beschränkter deterministischer Zeit berechnet werden. Wie oben skizziert kann der Rechenweg in deterministisch polynomieller Zeit abh. von $|x|$ und $\ell(D)$ relativ zu $C$ bestimmt werden.
    Nach Konstruktion ist leicht zu sehen, dass dieser Rechenweg dann auch relativ zu $O$ bestimmt werden kann.
\end{proof}

\clearpage

\subsection{Orakel mit $\hDisjCoNP$ und $\hDisjNP$ und $\P=\UP$ ($O_2$)}

Sei $e(0)=2, e(i+1)=2^{2^{e(i)}}$. (Doppelt exponentiell!) Sei hier $\{H_m\}_{m\in\mathbb N}$ eine Familie von paarweise disjunkten, unendlichen Teilmenge von $e(\mathbb N)$. (Ebenen $H_m$ gehören zur Zeugensprache bzgl. Disj(Co)NP-Maschinenpaar $M_a, M_b$.)
Starte mit $\mathrm{PSPACE}$-vollständiger Menge $C$ welche keine Wörter der Länge $e(\cdot)$ enthält.
Definiere folgende Zeugensprachen:
\begin{gather*}
    A_m^O \coloneqq \{ 0^n \mid n\in H_m, \text{für alle $x\in \Sigma^{n}$ gilt } x\in O \rightarrow \text{ $x$ endet mit $0$} \}\\
    B_m^O \coloneqq \{ 0^n \mid n\in H_m, \text{für alle $x\in \Sigma^{n}$ gilt } x\in O \rightarrow \text{ $x$ endet mit $1$} \}\\
    D_m^O \coloneqq \{ 0^n \mid n\in H_m, \text{es existiert ein $x\in \Sigma^{n}$ mit $x\in O$ und $x$ endet mit $0$} \}\\
    \begin{split} E_m^O \coloneqq &\{ 0^n \mid n\in H_m, \text{es existiert ein $x\in \Sigma^{n}$ mit $x\in O$ und $x$ endet mit $1$} \} \\ &\cup \overline{\{ z\in\Sigma^* \mid |z|=e(i) \text{ für ein $i$} \}} \end{split}
\end{gather*}
Fakt: $|O\cap \Sigma^n|\geq 1$ für alle $n\in H_m \implies (A_m^O, B_m^O)\in\DisjCoNP$.\\
Fakt: $O\cap \Sigma^{n-1}0\neq\emptyset$ genau dann wenn  $O\cap \Sigma^{n-1}1=\emptyset$ für alle $n\in H_m \implies \overline{D_m^O}=E_m^O$ und $(D_m^O, E_m^O)\in\DisjNP$ und $D_m^O\in\NP\cap\coNP$.
\medskip

Idee: erreiche entweder dass $M_a$, $M_b$ nicht disjunkt ablehnen (Task $\tau^2_{a,b}$), oder dass das Zeugenpaar $(A_m,B_m)$ nicht auf $(L(M_a),L(M_b))$ reduzierbar ist (Task $\tau^2_{a,b,r}$ für Transduktor $F_r$).
Ebenso, für $\hDisjNP$, erreiche entweder dass $M_a$, $M_b$ nicht disjunkt akzeptieren (Task $\tau^3_{a,b}$), oder dass das Zeugenpaar $(A_m,B_m)$ nicht auf $(L(M_a),L(M_b))$ reduzierbar ist (Task $\tau^3_{a,b,r}$ für Transduktor $F_r$).

Gleichzeitig versuchen wir für möglichst viele $M_j$ erreichen, dass diese nicht kategorisch sind. (Task $\tau^1_j$)
%Dies hat in gewisser Weise Priorität über die o.g. Diagonalisierung.
Am Ende sind die verbleibenden totalen Maschinen $M^O_j$ sehr speziell, denn sie sind auch für gewisse Teilmengen von $O$ kategorisch.
In Kombination mit dem Fakt dass $\P^C=\mathrm{PSPACE}^C$ können wir relevante Wörter in $O-C$ errechnen und so einen akzeptierenden Weg von $M^O_j(x)$ ausgeben -- damit entscheiden wir $L(M^O_j)$ und haben also auch $\P=\UP$.
\medskip

Sei wie üblich $t\in \mathcal T$ wenn der Definitionsbereich endlich ist, nur die Tasks der Form $\tau^2_{a,b}, \tau^3_{a,b}, \tau^1_j$ enthält, $t$ diese Tasks auf $\mathbb N$ abbildet, und injektiv auf dem Support ist.

Ein Orakel $w\in\Sigma^*$ ist $t$-valide wenn $t\in\mathcal T$ und folgendes gilt:
\begin{enumerate}[label={V\arabic*}]
    \item Wenn $x<|w|$ und $|x|\not\in e(\mathbb N)$, dann gilt $x\in w\iff x\in C$.\\
        (Orakel $w$ und $C$ stimmen auf Wörtern mit Länge $\neq e(\cdot)$ überein.)
    %\item Für alle $n=e(i)$ gilt $|w\cap \Sigma^n|\leq 2$.\\
    %    (Orakel $w$ ist dünn auf den Stufen der Länge $e(\cdot)$.)
    \item Wenn $t(\tau^1_j)=0$, dann existiert ein $z$ sodass $M_j^w(z)$ auf zwei Rechenwegen akzeptiert.\\
        ($M_j$ nicht kategorisch relativ zum finalen Orakel.)
    \item Wenn $t(\tau^2_{a,b})=0$, dann existiert ein $z$ sodass $M_a^w(z)$ und $M_b^w(z)$ definitiv ablehnen.\\
        (Wenn $t(\tau^2_{a,b})=0$, dann $(\overline{L(M_a)}, \overline{ L(M_b)})\not\in \DisjCoNP$ relativ zum finalen Orakel.)
    \item Wenn $0<t(\tau^2_{a,b})=m$, dann gilt für alle $n\in H_m$: wenn $w$ für alle Wörter der Länge $n$ definiert ist, dann $|\Sigma^n\cap w|\geq 1$.\\
        (Wenn $0<t(\tau^2_{a,b})=m$, dann $(A_m,B_m)\in\DisjCoNP$.)
    \item Wenn $t(\tau^3_{a,b})=0$, dann existiert ein $z$ sodass $M_a^w(z)$ und $M_b^w(z)$ definitiv akzeptieren.\\
        (Wenn $t(\tau^3_{a,b})=0$, dann $({L(M_a)}, { L(M_b)})\not\in \DisjNP$ relativ zum finalen Orakel.)
    \item Wenn $0<t(\tau^3_{a,b})=m$, dann gilt für alle $n\in H_m$: wenn $w$ für alle Wörter der Länge $n$ definiert ist, dann entweder $|\Sigma^{n-1}0\cap w|=0$ oder $|\Sigma^{n-1}1\cap w|=0$ aber nicht beides.\\
        (Wenn $0<t(\tau^3_{a,b})=m$, dann $(D_m,E_m)\in\DisjNP, D_m\in\NP\cap\coNP$.)
\end{enumerate}

Sei $T$ eine abzählbare Aufzählung der o.g. Tasks sodass $\tau^2_{a,b,r}$ immer nach $\tau^2_{a,b}$, sowie $\tau^3_{a,b,r}$ immer nach $\tau^3_{a,b}$ kommt.
%Wir sagen dass ein Task $\tau$ \emph{priorisierter} als Task $\tau'$ ist, wenn $\tau$ in $T$ vor $\tau'$ kommt.

[ . . . Üblicher Text zur stufenweisen Erweiterung von $w_s$ und $t_s$ . . . ]

%Wir definieren nun Stufe $s>0$, diese startet mit einem $t_{s-1}\in\mathcal T$ und eine $t_{s-1}$-validen Orakel $w_{s-1}$ welche nun die Tasks $\{\tau_{(1)}, \tau_{(2)}, \dots, \tau_{(s)}\}$ behandelt -- $\tau^{(s)}$ kommt dazu. Je nach Typ von Task $\tau^{(s)}$ führen wir nun Folgendes durch:
Wir definieren nun Stufe $s>0$, diese startet mit einem $t_{s-1}\in\mathcal T$ und eine $t_{s-1}$-validen Orakel $w_{s-1}$ welche nun den kleinsten Task bearbeitet, welcher noch in $T$ ist. Dieser wird unmittelbar nach der Bearbeitung aus $T$ entfernt. In der Bearbeitung wird das Orakel strikt verlängert.
\begin{itemize}

    \item $\tau^1_j$: Setze $t'=t_{s-1}\cup\{\tau^1_j\mapsto 0\}$. Existiert ein $t'$-valides Orakel $v\sqsupsetneq w_{s-1}$, dann setze $t_s\coloneqq t'$ und $w_s\coloneqq v$.

        Ansonsten setze $t_s\coloneqq t_{s-1}$ und setze $w_s\coloneqq w_{s-1}y$ für geeignetes $y\in\{0,1\}$ sodass $w_s$ auch $t_s$-valide ist. (Das ist möglich nach Behauptung 2.1.)

    \item $\tau^2_{a,b}$: Setze $t'=t_{s-1}\cup\{\tau^2_{a,b}\mapsto 0\}$. Existiert ein $t'$-valides Orakel $v\sqsupsetneq w_{s-1}$, dann setze $t_s\coloneqq t'$ und $w_s\coloneqq v$. Entferne außerdem alle Tasks der Form $\tau^2_{a,b,r}$ von $T$.

        Ansonsten wähle ein hinreichend großes $m\not\in \img(t_s)$ sodass $w_s$ kein Wort der Länge $\min H_m$ definiert. Setze $t_s\coloneqq t_{s-1}\cup \{ \tau^2_{a,b}\mapsto m \}$; damit ist $w_{s-1}$ auch $t_s$-valide. Setze $w_s\coloneqq w_{s-1}y$ für geeignetes $y\in\{0,1\}$ sodass $w_s$ auch $t_s$-valide ist. (Das ist möglich nach Behauptung 2.1.)

    \item $\tau^2_{a,b,r}$: Wir wissen dass $t_{s-1}(\tau^2_{a,b})=m>0$. Setze $t_s=t_{s-1}$ und wähle ein $t_s$-valides Orakel $w_s\sqsupsetneq w_{s-1}$ sodass bezüglich einem $n\in\mathbb N$ eine der folgenden Aussagen gilt:
        \begin{itemize}[nosep,endpenalty=10000]
            \item $0^n\in A_m^v$ für alle $v\sqsupseteq w_s$ und $M_a(F_r(0^n))$ akzeptiert definitiv relativ zu $w_s$.
            \item $0^n\in B_m^v$ für alle $v\sqsupseteq w_s$ und $M_b(F_r(0^n))$ akzeptiert definitiv relativ zu $w_s$.
        \end{itemize} (Das ist möglich nach Behauptung 2.2.)

    \item $\tau^3_{a,b}$: Setze $t'=t_{s-1}\cup\{\tau^3_{a,b}\mapsto 0\}$. Existiert ein $t'$-valides Orakel $v\sqsupsetneq w_{s-1}$, dann setze $t_s\coloneqq t'$ und $w_s\coloneqq v$. Entferne außerdem alle Tasks der Form $\tau^3_{a,b,r}$ von $T$.

        Ansonsten wähle ein hinreichend großes $m\not\in \img(t_s)$ sodass $w_s$ kein Wort der Länge $\min H_m$ definiert. Setze $t_s\coloneqq t_{s-1}\cup \{ \tau^3_{a,b}\mapsto m \}$; damit ist $w_{s-1}$ auch $t_s$-valide. Setze $w_s\coloneqq w_{s-1}y$ für geeignetes $y\in\{0,1\}$ sodass $w_s$ auch $t_s$-valide ist. (Das ist möglich nach Behauptung 2.1.)

    \item $\tau^3_{a,b,r}$: Wir wissen dass $t_{s-1}(\tau^3_{a,b})=m>0$. Setze $t_s=t_{s-1}$ und wähle ein $t_s$-valides Orakel $w_s\sqsupsetneq w_{s-1}$ sodass bezüglich einem $n\in\mathbb N$ eine der folgenden Aussagen gilt:
        \begin{itemize}[nosep,endpenalty=10000]
            \item $0^n\in D_m^v$ für alle $v\sqsupseteq w_s$ und $M_a(F_r(0^n))$ lehnt definitiv relativ zu $w_s$ ab.
            \item $0^n\in E_m^v$ für alle $v\sqsupseteq w_s$ und $M_b(F_r(0^n))$ lehnt definitiv relativ zu $w_s$ ab.
        \end{itemize} (Das ist möglich nach Behauptung 2.2.)
\end{itemize}

\begin{claim}
    Für jedes $t\in\mathcal T$ und jedes $t$-valide $w$ existiert ein $b\in\{0,1\}$ sodass $wb$ auch $t$-valide ist.
\end{claim}

\begin{claim}
    Die Bearbeitung eines Tasks $\tau^2_{a,b,r}$ ist möglich: gilt $t_{s-1}(\tau^2_{a,b})=m>0$, dann lässt sich $w_{s-1}$ so zu $t_{s-1}$-validem $u\sqsupsetneq w_{s-1}$ erweitern, dass eine der o.g. Fälle eintritt.
\end{claim}
\begin{proof}[Skizze.]
    Direkter Beweis. %Sei $\hat{s}$ die Stufe, in der $\tau^1_{a,b}$ bearbeitet wurde. Beobachte dass $t_{s-1}$ auch $\hat{s}-1$-valide ist.
    Erweitere durch Beauptung 1 das Orakel $w_{s-1}$ so weit zu $u$, dass genau alle Wörter der Länge $<n=e(i)\in H_m$ definiert sind, wobei das $i$ hinreichend groß gewählt wird. Sei $u(X), X\subseteq \Sigma^n$ das Orakel was entsteht, wenn die Stufe $e(i)$ mit genau den Wörtern aus $X$ gefüllt wird, bzw. $u(X)=u\cup X\cup C$.

    Sei $\hat{s}$ die Stufe, in der $\tau^2_{a,b}$ bearbeitet wurde.
    Da nach Behauptung 2.1 $u$ auch $t_{s-1}$ valide ist, ist es auch $t_{\hat{s}-1}$-valide.
    Es ist sogar $u(\emptyset)$ auch $t_{\hat{s}-1}$-valide, denn $t_{\hat{s}-1}(\tau^2_{a,b})$ ist undefiniert.

    Sei $y=T_r(0^n)$.
    Wir wissen, dass eine der Maschinen $M_a(y)$ oder $M_b(y)$ relativ zu $u(\emptyset)$ akzeptieren muss.
    Andernfalls wäre $u(\emptyset)$ ein geeignetes Orakel zur Zerstörung des Paars $M_a$, $M_b$ in der Bearbeitung von Task $\tau^2_{a,b}$ und wir hätten $t_{s-1}(\tau^2_{a,b})=0$.

    Ohne Beschränkung akzeptiert also $M_a(y)^{u(\emptyset)}$ auf einem Rechenweg mit polynomiell vielen Orakelfragen $Q$.
    Wähle ein $\alpha\in \Sigma^n-Q$ was mit $0$ endet. Dann akzeptiert auch $M_a(y)^{u(\{\alpha\})}$ auf dem gleichen Rechenweg und $0^n\in A_m^{u(\{\alpha\})}$. Es ist leicht zu sehen, dass $u(\{\alpha\})$ auch $t_{s}$-valide ist.
\end{proof}

\begin{claim}
    Die Bearbeitung eines Tasks $\tau^3_{a,b,r}$ ist möglich: gilt $t_{s-1}(\tau^3_{a,b})=m>0$, dann lässt sich $w_{s-1}$ so zu $t_{s-1}$-validem $u\sqsupsetneq w_{s-1}$ erweitern, dass eine der o.g. Fälle eintritt.
\end{claim}

Damit ist die Konstruktion möglich. Sei $O=\lim_{s\to\infty} w_s$.

\begin{claim}
    Kein Paar aus $\DisjCoNP^O$ ist $\leqmpp$-vollständig.
\end{claim}
\begin{claim}
    Kein Paar aus $\DisjNP^O$ ist $\leqmpp$-hart für $\NP^O\cap\coNP^O$.
    Damit gilt im Übrigen dass keine Sprache aus $\NP^O\cap\coNP^O$ auch $\leqmp$-vollständig ist.
\end{claim}

Wir wollen nun zeigen, dass wir UP-Sprachen in P entscheiden können. Sei im Folgenden $M_j$ eine kategorische Maschine relativ zu $O$. Um $x\in L(M_j)$ zu entscheiden nutzen wir aus, dass $\mathrm{PSPACE}^C=\P^C$, um so iterativ eine Menge $D\subseteq O$ an Orakelwörtern der Länge $e(\cdot)$ aufbauen, welche für die Berechnung $M_j(x)$ relevant ist, bis wir nach einigen Iterationen alle solchen relevanten Wörter gefunden haben. 
Wir beschränken uns im Folgenden auf Eingaben hinreichender Länge, sodass es für $x$ ein eindeutiges $i$ gibt, sodass
\[ e(i-1) \leq \log(|x|) < e(i), \quad 2p_j(|x|)< 2^{e(i)-1} < e(i+1).\tag{\ast} \]

Wir definieren Folgendes: %Seien $W,W'\subseteq \{q\in \Sigma^*\mid \exists i.|q|=e(i)\}$ zwei disjunkte Mengen mit $W\subseteq O,$ $W'\subseteq \overline{O}$.
Eine \emph{$(U, W, W')$ respektierende akzeptierende Berechnung $P$ von $M_j$ auf Eingabe $x$} ist ein akzeptierender Rechenweg $P$ von $M_j(x)$ relativ zu einem Orakel $v\subseteq\Sigma^*$, wobei 
\[ U, W,W'\subseteq \Sigma^{e(i)}, \quad W\subseteq O, \quad W'\subseteq\overline{O} \]
und für $v$ gilt:
\begin{enumerate}[noitemsep,label=\arabic*.]
    \item $v$ ist definiert für genau die Wörter der Länge $\leq p_j(|x|)$.
    \item $v(q)=O(q)$ für alle $q$ mit $|q|\neq  e(i)$, wobei hier das $i$ diejenige eindeutige Zahl ist für die obigen Ungleichungen ($\ast$) bzgl. $|x|$ gelten.
    \item $v(q)=1$ für alle $q$ mit $q\in W$.
    \item $v(q)=0$ für alle $q$ mit $q\in W'$.
    \item $v(q)=1 \implies q\in U$ für alle $q\in\Sigma^{e(i)}$. [$v$ enthält auf Ebene $e(i)$ nur Wörter, die auch in $U$ vorkommen.]
\end{enumerate}

Sei $P^\mathrm{all}$ die Menge der Orakelfragen auf $P$, und sei $P^\mathrm{yes}=P^\mathrm{all}\cap v$, $P^\mathrm{no}=P^\mathrm{all}\cap \overline{v}$.
Beob. dass für festes $U=\Sigma^{e(i)}$ (bzw. $U=\Sigma^{e(i)-1}0$, $U=\Sigma^{e(i)-1}1$) wegen $\mathrm{PSPACE}^C=\P^C$ das Ermitteln einer $(U, W, W')$ respektierenden akzeptierenden Berechnung einfach in Polynomialzeit (abh. von $|x|$ und $\ell(W),\ell(W')$) relativ zu $O$ möglich ist: insbesondere stimmt $O$ mit $C$ auf Wörtern der Länge $\neq e(\cdot)$ überein, und alle anderen Wörter der Länge $e(0), e(1), \dots, e(i-1)$ können vorab mit Queries an $O$ in Polynomialzeit erfragt werden.
Entsprechende Belegungen von $v$ für Wörter der Länge $e(i)$ können z.B. in PSPACE enumeriert werden.

Sei $s$ die Stufe bei der $\tau^1_j$ betrachtet wurde.
Zusätzlich zur Einschränkung $(\ast)$ diskutieren wir ab jetzt nur noch Eingaben, für welche das Orakel $w_{s-1}$ keine Wörter der Länge $e(i)$ definiert.
Sei $M=\bigcup \{ H_m \mid t_s(\tau^3_{a,b})=m>0 \}$ eine Menge an Ebenen, welche einer DisjNP-Zeugensprache \emph{in Stufe $s$} zugewiesen ist.
Beob. dass $M\in \P$, denn es sind höchstens endlich viele $H_m$ in der Vereinigung, welche je in $\P$ entscheidbar sind.
\marginnote{Wie Nutzen der Menge $U$ erklären?}


\begin{claim}
    %Sei $M_j$ eine kategorische Maschine relativ zu $O$.
    %Ab hinreichend großer Eingabelänge $n_0$ gilt für alle Eingaben $x$, $|x|\geq n_0$, alle $W, W'$:
    Seien $P_1, P_2$ zwei $(U, W, W')$ respektierenden akzeptierenden Berechnungen von $M_j$ auf Eingabe $x$.
    %Folgende Aussage gilt uneingeschränkt falls $e(i)\not\in M$, und sonst falls $e(i)\in M$ dann mit der Einschränkung dass $U=\Sigma^{e(i)-1}0$ oder $U=\Sigma^{e(i)-1}1$ gelten muss.
    Folgende Aussage gilt jeweils bezüglich dieser beiden Einschränkungen auf $U$:
    \begin{itemize}[nosep]
        \item $e(i)\in M$, sowie $U=\Sigma^{e(i)-1}0$ oder $U=\Sigma^{e(i)-1}1$,
        \item $e(i)\not\in M$ und $U=\Sigma^{e(i)}$.
    \end{itemize}

    Wenn $P^\mathrm{all}_1$ und $P^\mathrm{all}_2$ beide je eine (nicht notwendigerweise identische) Orakelfrage $q_1, q_2$ der Länge $e(i)$ enthalten, welche in $U-(W\cup W')$ liegt, dann haben diese zwei Berechnungen eine (identische) Orakelfrage $q$ der Länge $e(i)$ gemeinsam, welche in $U-(W\cup W')$ liegt.
\end{claim}
\begin{proof}[Skizze.]
    Wir beweisen zunächst den ersten Fall.
    Angenommen, dies gilt nicht, also sei $x$ sowie $U,W,W'\subseteq \Sigma^{e(i)}$, $W\subseteq O$, $W'\subseteq\overline{O}$ gegeben.
    Seien außerdem $P_1$ und $P_2$ zwei $(U, W, W')$ respektierenden akzeptierenden Berechnungen von $M_j$ auf Eingabe $x$,
    welche je eine Orakelfrage der Länge $e(i)$ enthalten, welche nicht in $W\cup W'$ liegt,
    aber keine Orakelfrage aus $U-(W\cup W')$ gemeinsam haben.
    Dann sind schon $P_1$ und $P_2$ verschieden.
    Seien ferner $v_1, v_2$ die zugehörigen Orakel, also für welche $M_j(x)$ akzeptiert und Eigenschaften~1--4 erfüllen.

    Wir werden nun ein $t_{s-1}$-valides Orakel $u\sqsupsetneq w_{s-1}$ konstruieren welches mit $v_1$ auf $P^\mathrm{all}_1$ übereinstimmt, und welches mit $v_2$ auf $P^\mathrm{all}_2$ übereinstimmt.
    Außerdem wird es auf Ebene $\Sigma^{e(i)}$ nur Wörter aus $U$ enthalten, woraus wir zeigen können dass $u$ sogar ein geeignetes Orakel zur Zerstörung der UP-Machine in der Bearbeitung von Task $\tau^2_{j}$ in Stufe $s$ ist, ohne Beschränkungen bzgl. DisjNP-Zeugensprachen zu verletzen. Insbesondere akzeptiert dann auch $M_j^O(x)$ auf den zwei verschiedenen Rechenwegen $P_1$, $P_2$, was der Voraussetzung widerspricht.

    %Setze $D=(\Sigma^{e(i)}\cap P^\mathrm{yes}_1) \cup (\Sigma^{e(i)}\cap P^\mathrm{yes}_2) \cup \{\alpha\}$ wobei $\alpha\in \Sigma^{e(i)}-(P^\mathrm{all}_1\cup P^\mathrm{all}_2)$.
    Sei $Y=(P^\mathrm{yes}_1\cup P^\mathrm{yes}_2)\cap\Sigma^{e(i)}$, und $N=(P^\mathrm{no}_1\cup P^\mathrm{no}_2)\cap\Sigma^{e(i)}$.
    Wir zeigen $Y\cap N = \emptyset$. (Das soll uns helfen nachzuweisen, dass ein geeignetes $u$ existieren kann.)
    Nehme an es gibt ein $q\in Y\cap N$ der Länge $e(i)$.
    \begin{itemize}[noitemsep]
        \item Ist $q\not\in U$ dann schon sofort dass $q\not\in P^\mathrm{yes}, P^\mathrm{yes}_2$ was $q\in N$ widerspricht.
        \item Ist $q\in W$ dann gilt schon sofort dass $q\not\in P^\mathrm{no}, P^\mathrm{no}_2$ was $q\in N$ widerspricht.
        \item Ist $q\in W'$ dann gilt schon sofort dass $q\not\in P^\mathrm{yes}, P^\mathrm{yes}_2$ was $q\in Y$ widerspricht.
        \item Andernfalls ist $q\in U-( W\cup W')$, dann gilt $q\in P^\mathrm{yes}_1\cap P^\mathrm{no}_2$ oder $q\in P^\mathrm{yes}_2\cap P^\mathrm{no}_1$.
            In beiden Fällen hätten wir aber, dass $P_1$ und $P_2$ eine Orakelfrage der Länge $e(i)$ teilen, welche in $U-(W\cup W')$ liegt. Das widerspricht der urpsrünglichen Annahme.
    \end{itemize}

    Es gilt also $Y\cap N =\emptyset$. Wir beobachten außerdem dass $Y\subseteq U$ nach Punkt~5 der Definition gilt. Wähle ein $\alpha\in U-N$. Dieses existiert da $|N|\leq 2p_j(|x|)<2^{e(i)-1} = |U|$ nach ($\ast$).
    Sei nun $u$ das Orakel was genau alle Wörter der Länge $\leq p_j(|x|)$ definiert sind, und
    \[ u(z)= \begin{cases} O(z) & \text{falls $|z|\neq e(i)$}\\ 1 & \text{falls $z=\alpha$} \\1 & \text{falls $z\in Y$} \\ 0&\text{sonst,} \end{cases}
    \]
    also wie $O^{\leq p_j(|x|)}$ aufgebaut ist, außer dass die Ebene $e(i)$ mit genau den Wörtern aus $Y$ gefüllt wird. bzw. $u\cap\Sigma^{e(i)} = Y\cup \{\alpha\}$.
    Es ist leicht zu sehen dass $u\sqsupsetneq w_{s-1}$.
    Beob. dass 
    \[ u\cap N = \Sigma^{e(i)}\cap u \cap N = (Y\cup \{\alpha\}) \cap N= Y\cap N=\emptyset.\]

    Das Orakel $u$ stimmt mit $v_1$ auf $P^\mathrm{all}_1$ überein. Sei hierfür $q\in P^\mathrm{all}_1$.
    Ist $|q|\neq e(i)$, dann gilt schon nach Definition $v_1(q)=O(q)=u(q)$. Sei daher im Folgenden $|q|=e(i)$.
    Ist $q\in P^\mathrm{yes}_1$, dann auch $q\in v_1$. Außerdem dann auch $q\in Y$, daher $q\in u$.
    Ansonsten ist $q\in P^\mathrm{no}_1$, dann auch $q\not\in v_1$. Außerdem dann auch $q\in N$, daher $q\not\in u$ nach obiger Beobachtung.
    
    Auf symmetrische Weise stimmt $u$ mit $v_2$ auf $P^\mathrm{all}_2$ überein.
    Wir zeigen nun dass $u$ auch $t_{s-1}$-valide ist.
    Nach obiger Argumentation wäre dann $u$ eine geeignete Erweiterung von $w_{s-1}$ für welche $M_j^O(x)$ nicht mehr kategorisch ist, was der Wahl von $M_j^O(x)$ widerspricht.

    Nach Konstruktion ist V1 erfüllt; V2, V3, V5 sind wegen $u\sqsupsetneq w_{s-1}$ erfüllt. Angenommen V3 ist verletzt. Dies kann nur an der Ebene $e(i)$ liegen. Aber dann wäre $e(i)\in H_{m}$ mit $m=t_{s-1}(\tau^2_{a,b})$ und $e(i) \not\in M$; Widerspruch zur Einschränkung.

    Angenommen V5 ist verletzt.
    Wieder kann das nur an der Ebene $e(i)$ liegen.
    Aber hier gilt $u\cap\Sigma^{e(i)}=Y\cap\{\alpha\}\subseteq U$ und nach Wahl von $U$ ist damit $|\Sigma^{n-1}0\cap w|=0$ oder $|\Sigma^{n-1}1\cap w|=0$ aber nicht beides, ist ja $\alpha\in u\cap\Sigma^{e(i)}$.
    \medskip

    Wir beweisen jetzt den zweiten Fall. Hier läuft die Konstruktion von $u$ identisch,
    und wieder gilt dass $u$ mit $v_1$ auf $P^\mathrm{all}_1$ übereinstimmt, sowie $u$ mit $v_2$ auf $P^\mathrm{all}_2$ übereinstimmt.
    Nach obiger Argumentation wäre dann $u$ eine geeignete Erweiterung von $w_{s-1}$ für welche $M_j^O(x)$ nicht mehr kategorisch ist, was der Wahl von $M_j^O(x)$ widerspricht.

    Nach Konstruktion ist V1 erfüllt; V2, V3, V5 sind wegen $u\sqsupsetneq w_{s-1}$ erfüllt. Angenommen V5 ist verletzt. Dies kann nur an der Ebene $e(i)$ liegen. Aber dann wäre $e(i)\in H_{m}$ mit $m=t_{s-1}(\tau^3_{a,b})$ und $e(i)\in M$; Widerspruch zur Einschränkung.

    Angenommen V3 ist verletzt.
    Wieder kann das nur an der Ebene $e(i)$ liegen.
    Aber hier gilt $u\cap\Sigma^{e(i)}=Y\cap\{\alpha\}$ und nach Wahl von $U$ ist damit $|\Sigma^{n}\cap w|>0$, ist ja $\alpha\in u\cap\Sigma^{e(i)}$.
\end{proof}

\begin{claim}
    $\P=\UP$ relativ zu $O$.
\end{claim}
\begin{proof}[Skizze.]
    Sei $S\in \UP^O$. Es existiert nach Definition eine Maschine $M_j$ mit $L(M_j)^O=S$.
    Wir zeigen für hinreichend lange $x$ wie man $x\in S$ in Polyonmialzeit relativ zu $O$ entscheiden kann.
    
    Sei im Folgenden $x$ hinreichend lange wie oben diskutiert, also für dieses $(\ast)$ mit eindeutigem $i$ gilt, sowie $w_{s-1}$ keine Wörter der Länge $e(i)$ definiert, wobei $s$ die Stufe ist, bei der $\tau^2_j$ betrachtet wurde.
    Sei wieder $M=\bigcup \{ H_m \mid t_s(\tau^3_{a,b})=m>0 \}$.
    Für diese Maschine $M_j$ und eine solche Eingabe $x$ gilt dann Behauptung~2.6.

    Wir werden diese Eigenschaft ausnutzen und iterativ Mengen $W, W'$ aufbauen, welche für die Berechnung $M_j^O(x)$ relevant ist, bis wir alle solchen relevanten Wörter gefunden haben.
    Betrachte dafür zunächst folgende Subroutine:

    \noindent%
    \SetKwProg{Fn}{Function}{:}{}%
    \SetKw{Assert}{assert}%
    \begin{algorithm}[H]
        \Fn{Search($U$)}{
            \Assert{$e(i)\in M\implies (U=\Sigma^{e(i)-1}0 \lor U=\Sigma^{e(i)-1}1)$}\;
        $W\gets\emptyset,\, W'\gets\emptyset$\;
        \For{$k$ von $0$ bis $p_j(|x|)+1$}{
            $P\gets$ eine $(U, W, W')$ respektierende akzeptierende Berechnung $P$ von $M_j$ auf $x$ mit $|(P^\mathrm{all}\cap U)-(W\cup W')|$ minimal, oder $\bot$ falls keine existiert\;
            \If{$P=\bot$}
            {
                \Return{„$x\not\in S$“}
            }
            \If{alle $q\in P^\mathrm{all}$ mit $|q|=e(i)$ sind in $W\cup W'$}{
                \Return{„$x\in S$“}
            }
            \ForEach{$q\in P^\mathrm{all}$ mit $|q|=e(i)$}
            {
                %\lIf{$q\in O-U$}{\Return{„$x\not\in S$“}
                \lIf{$q\in O$}{$W\gets W\cup\{q\}$}
                \lIf{$q\not\in O$}{$W'\gets W'\cup\{q\}$}
            }
            %\If{$P$ ist immer noch eine $(W, W')$ respektierende akzeptierende Berechnung}
            %{
                %\Return{„$x\in S$“}
            %}
        }
        \Return{„$x\not\in S$“}
        }
    \end{algorithm}
    Es ist leicht zu sehen dass der Algorithmus eine polynomielle Laufzeitschranke einhält.
    Wir zeigen nun folgende Aussage: Wenn die Assertion in Z.~2 zutrifft dann macht der Algorithmus keinen falsch-positiv-Fehler. Falls zusätzlich $O\cap \Sigma^{e(i)}\subseteq U$, dann macht der Algorithmus keinen falsch-negativ-Fehler.

    Wir beobachten die Invariante dass $W\subseteq O\cap \Sigma^{e(i)}$ und $W'\subseteq \overline{O}\cap\Sigma^{e(i)}$. 
    Terminiert also der Algorithmus in Z.~10 mit „$x\in S$“ dann ist auch $x\in S$: Sei $v$ das Orakel der $(U, W, W')$ respektierenden akzeptierenden Berechnung $P$ von $M_j(x)$. Es ist nun leicht zu sehen dass $v$ und $O$ auf $P^\mathrm{all}$ übereinstimmen. Damit gilt auch dass $M_j^O(x)$ akzeptiert und damit $x\in S$.
    Der Algorithmus macht also schon mal keinen falsch-positiv-Fehler.

    Es verbleibt zu zeigen dass wenn $x\in S$ und  $O\cap \Sigma^{e(i)}\subseteq U$ dann der Algorithmus auch in Z.~10 mit „$x\in S$“ terminiert, also keinen falsch-negativ-Fehler macht.
    Sei hierfür $P^*$ der längste akzeptierende Rechenweg von $M_j^O(x)$.
    Beob. dass mit der o.\,g. Invariante sowie der Bedingung $O\cap\Sigma^{e(i)}\subseteq U$ gilt, dass $P^*$ auch immer ein $(U,W,W')$ respektierender akzeptierender Rechenweg ist.
    Damit ist die Bedingung in Z.~6 nie erfüllt.
    %Damit ist auch die Bedingugn in Z.~2 immer erfüllt und der Algorithmus terminiert sicher nicht ablehnend in Z.~12.
    Wir zeigen nun noch, dass nach $\leq p_j(|x|)+1$ vielen Iterationen auch die Bedingung in Z.~9 erfüllt ist.
    Hierfür zeigen wir, dass $|(P^{*\mathrm{all}}\cap U)-(W\cup W')|$ in jeder Iteration um $\geq 1$ abnimmt. Da $|P^{*\mathrm{all}}|\leq p_j(|x|)$ ist nach $\leq p_j(|x|)+1$ vielen Iterationen $|(P^{*\mathrm{all}}\cap U)-(W\cup W')|=0$.
    Nach $\leq p_j(|x|)+1$ vielen Iterationen wird also in Z.~5 eine Berechnung $P$ ausgewählt, bei der alle $q\in P^{\mathrm{all}}$ mit $|q|=e(i)$ in $W\cup W'$ liegen. Dann ist die Bedingung in Z.~10 erfüllt und der Algorithmus terminiert akzeptierend.

    Steht der Algorithmus in Z.~12, dann gilt sowohl für das ausgewählte $P$, als auch für $P^*$ dass beide je eine (nicht notwendigerweise identische) Orakelfrage der Länge $e(i)$ enthalten, welche nicht in $W\cup W'$ liegt. (Andernfalls  wäre $P$ in Z.~5 anders ausgewählt worden.)

    Sowohl $P$ als auch $P^*$ sind $(U, W, W')$ respektierende akzeptierende Rechenwege. Nachdem die Assertion gilt, ist Behauptung~2.6 anwendbar. Also haben diese zwei Berechnungen eine identische Orakelfrage $q\in P^\mathrm{all}\cap P^{*\mathrm{all}}\cap \Sigma^{e(i)}$ gemeinsam, welche in $U-(W\cup W')$ liegt.
    Diese wird in den Zz.~12--15 irgendwann der Menge $W\cup W'$ hinzugefügt.
    Damit nimmt auch $|(P^{*\mathrm{all}}\cap U)-(W\cup W')|$ um $\geq 1$ ab, wie behauptet.
    \medskip

    Betrachte nun folgenden Entscheidungsalgorithmus für $S=L(M_j^O)$.

    \noindent
    \begin{algorithm}[H]
        \setcounter{AlgoLine}{17}
        \eIf{$e(i)\in M$}
        {
            \uIf{{Search}($\,\Sigma^{e(i-1)0}$) = „$x\in S$“}
            {
                \Return{„$x\in S$“}
            }
            \uElseIf{{Search}($\,\Sigma^{e(i-1)1}$) = „$x\in S$“}
            {
                \Return{„$x\in S$“}
            }
            \Else{
                \Return{„$x\not\in S$“}
            }
        }
        {
            \Return \emph{Search}($\Sigma^{e(i)}$)\;
        }
        \medskip
    \end{algorithm}

    Es ist leicht zu überprüfen dass in allen Fällen die Subroutine so ausgeführt wird dass die Assertion immer erfüllt ist.
    Ebenso ist leicht zu sehen, dass dieser Algorithmus in Polynomialzeit läuft.

    Wir überprüfen nun Korrektheit in zwei Fällen.
    Im Fall $e(i)\not\in M$ rufen wir die Subroutine mit $U=\Sigma^{e(i)}$ auf, und nach obiger Argumentation macht \emph{Search} sowohl keinen falsch-positiv- also auch keinen falsch-negativ-Fehler.
    Der zurückgegebene Wert ist also korrekt.

    Im Fall $e(i)\in M$ bekommen wir in beiden Aufrufen von \emph{Search} zumindest keinen falsch-positiven Fehler. Wenn also in Zz.~20 oder 22 mit „$x\in S$“ terminiert wird, dann war auch $x\in S$.
    Fener wissen wir wegen $e(i)\in M$ und V6 dass entweder $O\cap\Sigma^{e(i)}\subseteq \Sigma^{e(i)-1}0$ oder $O\cap\Sigma^{e(i)}\subseteq \Sigma^{e(i)-1}1$.
    Wenn also $x\in S$, dann macht einer der Aufrufe von \emph{Search} in Zz.~19 oder 21 keinen falsch-negativ-Fehler und die zugehörige If-Bedingung wird positiv ausfallen, und der Algorithmus terminiert mit „$x\in S$“.
\end{proof}

\clearpage
\subsection{Orakelkonstruktion $\hDisjNP$, $\P=\UP$, und $Q$}

Sei $e(0)=2, e(i+1)=2^{2^{e(i)}}$. Sei hier $\{H_m\}_{m\in\mathbb N}$ eine Familie von paarweise disjunkten, unendlichen Teilmenge von $e(\mathbb N)$. (Ebenen $H_m$ gehören zur Zeugensprache bzgl. DisjNP-Maschinenpaar $M_a, M_b$.)
Starte mit $\mathrm{PSPACE}$-vollständiger Menge $C$ welche keine Wörter der Länge $e(\cdot)$ enthält.
Definiere folgende Zeugensprachen:
        \begin{gather*}
            A_m^O \coloneqq \{ 0^n \mid n\in H_m, \text{existiert $x\in \Sigma^{n}$ mit } x\in O \text{ und $x$ endet mit $0$} \}\\
            B_m^O \coloneqq \{ 0^n \mid n\in H_m, \text{existiert $x\in \Sigma^{n}$ mit } x\in O \text{ und $x$ endet mit $1$} \}
        \end{gather*}
        Fakt: wenn $|O\cap \Sigma^{n-1}0|=0$ oder $|O\cap \Sigma^{n-1}1|=0$ für alle $n\in H_m$, dann $(A_m^O, B_m^O)\in\DisjNP^O$.
\medskip

Idee: erreiche entweder dass $M_a$, $M_b$ nicht disjunkt akzeptieren (Task $\tau^1_{a,b}$), oder dass das Zeugenpaar $(A_m,B_m)$ nicht auf $(L(M_a),L(M_b))$ reduzierbar ist (Task $\tau^1_{a,b,r}$ für Transduktor $F_r$).

Gleichzeitig versuchen wir für möglichst viele $M_j$ erreichen, dass diese (a) nicht total sind (Task $\tau^2_j$), und diese (b) nicht kategorisch sind (Task $\tau^3_j$).
%Dies hat in gewisser Weise Priorität über die o.g. Diagonalisierung.
Am Ende sind die verbleibenden totalen Maschinen $M^O_j$ sehr speziell, denn sie sind auch für gewisse Teilmengen von $O$ total.
In Kombination mit dem Fakt dass $\P^C=\mathrm{PSPACE}^C$ können wir relevante Wörter in $O-C$ errechnen und so einen akzeptierenden Weg von $M^O_j(x)$ ausgeben -- damit erzielen wir $Q$ bzw. $\P=\UP$.
Besonders aufgepasst muss hier auf den $\UP$-Entscheidungsalgorithmus: im Korrektheitsbeweis müssen wir sicherstellen, dass höchstens polynomiell viele Wörter in eine Ebene gesetzt werden. Das ist etwas schwieriger, weil üblicherweise im Beweis bis zu $2\cdot p_j(|x|)$ Wörter eingesetzt werden (Menge $Y$); und dieses Polynom ist kann nicht von der Eingabelänge allein beschränkt werden da $j$ beliebig.

Sei wie üblich $t\in \mathcal T$ wenn der Definitionsbereich endlich ist, nur die Tasks der Form $\tau^1_{a,b}, \tau^2_j$ enthält, $t$ diese Tasks auf $\mathbb N$ abbildet, und injektiv auf dem Support ist.

\begin{enumerate}[label={V\arabic*}]
    \item Wenn $x<|w|$ und $|x|\not\in e(\mathbb N)$, dann gilt $x\in w\iff x\in C$.\\
        (Orakel $w$ und $C$ stimmen auf Wörtern mit Länge $\neq e(\cdot)$ überein.)
    \item Für alle $n=e(i)$ gilt $|w\cap \Sigma^n|\leq 2^{c}$ mit $c=|\{j\mid t(\tau^2_j)=0 \}|$.
        Ferner definiert $w$ alle Wörter der Länge $e(c)$.\\
        (Auf den Ebenen der Länge $e(\cdot)$ sind exponeniell so viele Wörter wie Tasks $\tau^2_j$ „negativ“ behandelt werden. Wir werden sehen dass mit dieser Eigenschaft das Orakel dünn auf Ebenen der Länge $e(\cdot)$ ist.)
    \item Wenn $t(\tau^1_j)=0$, dann existiert ein $z$ sodass $M_j^w(z)$ definitiv ablehnt.\\
        ($L(M_j)\neq \Sigma^*$ relativ zum finalen Orakel.)
    \item Wenn $t(\tau^2_j)=0$, dann existiert ein $z$ sodass $M_j^w(z)$ definitiv auf zwei Rechenwegen akzeptiert.\\
        ($M_j$ nicht kategorisch relativ zum finalen Orakel.)
    \item Wenn $t(\tau^3_{a,b})=0$, dann existiert ein $z$ sodass $M_a^w(z)$ und $M_b^w(z)$ definitiv akzeptieren.\\
        (Wenn $t(\tau^3_{a,b})=0$, dann $L(M_a)\cap L(M_b)\neq \emptyset$ relativ zum finalen Orakel.)
    \item Wenn $0<t(\tau^3_{a,b})=m$, dann gilt für alle $n\in H_m$ dass $|\Sigma^{n-1}0\cap w|=0$ oder $|\Sigma^{n-1}1\cap w|=0$.\\
        (Wenn $0<t(\tau^3_{a,b})=m$, dann $(A_m,B_m)\in\DisjNP$.)
\end{enumerate}

Sei $T$ eine abzählbare Aufzählung der o.g. Tasks sodass $\tau^3_{a,b,r}$ immer nach $\tau^3_{a,b}$ kommt.

[ . . . Üblicher Text zur stufenweisen Erweiterung von $w_s$ und $t_s$ . . . ]

Wir definieren nun Stufe $s>0$, diese startet mit einem $t_{s-1}\in\mathcal T$ und eine $t_{s-1}$-validen Orakel $w_{s-1}$ welche nun den kleinsten Task bearbeitet, welcher noch in $T$ ist. Dieser wird unmittelbar nach der Bearbeitung aus $T$ entfernt. In der Bearbeitung wird das Orakel strikt verlängert.
\begin{itemize}

    \item $\tau^1_j$: Setze $t'=t_{s-1}\cup\{\tau^1_j\mapsto 0\}$. Existiert ein $t'$-valides Orakel $v\sqsupsetneq w_{s-1}$, dann setze $t_s\coloneqq t'$ und $w_s\coloneqq v$.

        Ansonsten setze $t_s\coloneqq t_{s-1}$ und setze $w_s\coloneqq w_{s-1}y$ für geeignetes $y\in\{0,1\}$ sodass $w_s$ auch $t_s$-valide ist. (Das ist möglich nach Behauptung 3.1.)

    \item $\tau^2_j$: Setze $t'=t_{s-1}\cup\{\tau^2_j\mapsto 0\}$. Existiert ein $t'$-valides Orakel $v\sqsupsetneq w_{s-1}$, dann setze $t_s\coloneqq t'$ und $w_s\coloneqq v$.

        Ansonsten setze $t_s\coloneqq t_{s-1}$ und setze $w_s\coloneqq w_{s-1}y$ für geeignetes $y\in\{0,1\}$ sodass $w_s$ auch $t_s$-valide ist. (Das ist möglich nach Behauptung 3.1.)

    \item $\tau^3_{a,b}$: Setze $t'=t_{s-1}\cup\{\tau^3_{a,b}\mapsto 0\}$. Existiert ein $t'$-valides Orakel $v\sqsupsetneq w_{s-1}$, dann setze $t_s\coloneqq t'$ und $w_s\coloneqq v$. Entferne außerdem alle Tasks der Form $\tau^2_{a,b,r}$ von $T$.

        Ansonsten wähle ein hinreichend großes $m\not\in \img(t_s)$ sodass $w_s$ kein Wort der Länge $\min H_m$ definiert. Setze $t_s\coloneqq t_{s-1}\cup \{ \tau^3_{a,b}\mapsto m \}$; damit ist $w_{s-1}$ auch $t_s$-valide. Setze $w_s\coloneqq w_{s-1}y$ für geeignetes $y\in\{0,1\}$ sodass $w_s$ auch $t_s$-valide ist. (Das ist möglich nach Behauptung 3.1.)

    \item $\tau^3_{a,b,r}$: Wir wissen dass $t_{s-1}(\tau^2_{a,b})=m>0$. Setze $t_s=t_{s-1}$ und wähle ein $t_s$-valides Orakel $w_s\sqsupsetneq w_{s-1}$ sodass bezüglich einem $n\in\mathbb N$ eine der folgenden Aussagen gilt:
        \begin{itemize}[nosep,endpenalty=10000]
            \item $0^n\in A_m^v$ für alle $v\sqsupseteq w_s$ und $M_a(F_r(0^n))$ lehnt relativ zu $w_s$ definitiv ab.
            \item $0^n\in B_m^v$ für alle $v\sqsupseteq w_s$ und $M_b(F_r(0^n))$ lehnt relativ zu $w_s$ definitiv ab.
        \end{itemize} (Das ist möglich nach Behauptung 3.2.)
\end{itemize}

\begin{claim}
    Für jedes $t\in\mathcal T$ und jedes $t$-valide $w$ existiert ein $b\in\{0,1\}$ sodass $wb$ auch $t$-valide ist.
\end{claim}

\begin{claim}
    Die Bearbeitung eines Tasks $\tau^3_{a,b,r}$ ist möglich: gilt $t_{s-1}(\tau^3_{a,b})=m>0$, dann lässt sich $w_{s-1}$ so zu $t_{s-1}$-validem $u\sqsupsetneq w_{s-1}$ erweitern, dass eine der o.g. Fälle eintritt.
\end{claim}
\begin{proof}[Hinweis.]
    Unterschied ist V2. Trotzdem dürfen wir zwei Wörter in die Ebene einer Stufe $e(i)$ einsetzen, und verletzen dabei V2 nicht.
\end{proof}

Damit ist die Konstruktion möglich. Sei $O=\lim_{s\to\infty} w_s$.

\begin{claim}
    Kein Paar aus $\DisjNP^O$ ist $\leqmpp$-hart für $\DisjUP$.
\end{claim}


\begin{claim}
    Sei $M_j$ eine totale Maschine, d.h. $L(M^O)=\Sigma^*$.
    Es existiert eine Länge $n$ mit folgender Eigenschaft: 
    falls $T\subseteq O$ mit $O$ auf Wörtern der Länge $\neq e(\cdot)$ und Wörtern $\leq n$ übereinstimmt, dann $L(M_j^T)=\Sigma^*$.
\end{claim}

\begin{claim}
    Das Orakel $O$ ist dünn auf den Ebenen der Länge $e(i)$. Insbesondere gilt $|O\cap \Sigma^{e(i)}|\leq e(i)$ für alle $i$.
\end{claim}
\begin{proof}[Skizze.]
    Sein eine Ebene $e(i)$ beliebig.
    Sei $c_k=|\{j\mid t_{k}(\tau^2_j)=0 \}|$.
    Nachdem in der Folge $c_0, c_1, c_2, \dots$ die Terme immer um $\leq 1$ ansteigen, existiert ein kleinstes $s$ sodass $c_s=i$.
    Damit hat nach V2 das Orakel $w_s\sqsubsetneq O$ alle Wörter der Länge $c(i)$ definiert.
    Ferner gilt $|w_s\cap\Sigma^{e(i)}|\leq 2^{c_s}$.
    Also haben wir $|O\cap\Sigma^{e(i)}|=|w_s\cap\Sigma^{e(i)}|\leq 2^{i}\leq e(i)$.
\end{proof}

\begin{claim}
    Sei $M_j$ eine totale Maschine, d.h. $L(M_j^O)=\Sigma^*$. Dann existiert eine Funktion $g\in \FP^O$ sodass $g(x)$ einen akzeptierenden Rechenweg von $M^O_j(x)$ ausgibt. Damit gilt nach Definition die Hypothese $Q$ relativ zu $O$.
\end{claim}
\begin{proof}[Hinweis.]
    Wie im vorigen Abschnitt. Analog gilt für die Menge an erfassten Orakelwörtern $D$ nach V2: $\ell(D) \leq p_j(|x|)\cdot p_j(|x|)\cdot p_j(|x|)$ denn in den je $\leq p_j(|x|)$ Ebenen der Länge $e(i)\leq p_j(|x|)$ existieren nach Behauptung 3.5 höchstens $e(i)\leq p_j(|x|)$ Wörter der Länge $e(i)\leq p_j(|x|)$.
    Damit folgt auch, dass der Algorithmus nach höchstens polynomiell vielen Iterationen terminiert.
\end{proof}

Um $\UP$-Sprache $S=L(M_j^O)$ in $\P$ zu entscheiden, gehen wir wie in vorigem Abschnitt vor.
Sei $s$ die Stufe bei der $\tau^1_j$ betrachtet wurde.
Sei $c=|\{j\mid t_{s-1}(\tau^2_j)=0|$.
Wir beschränken uns im Folgenden wieder auf Eingaben hinreichender Länge, sodass es für $x$ ein eindeutiges $i$ gibt, sodass
\[ e(c+1)\leq e(i), \quad \leq e(i-1) \leq \log(|x|), \quad 2p_j(|x|)< 2^{e(i)-1} < e(i+1).\tag{\ast} \]

Zusätzlich zur Einschränkung $(\ast)$ diskutieren wir ab jetzt nur noch Eingaben, für welche das Orakel $w_{s-1}$ keine Wörter der Länge $e(i)$ definiert.
Sei $M=\bigcup \{ H_m \mid t_s(\tau^3_{a,b})=m>0 \}$ eine Menge an Ebenen, welche einer DisjNP-Zeugensprache \emph{in Stufe $s$} zugewiesen ist.

Wir verfeinern die Definition einer \emph{$(U, W, W')$ respektierende akzeptierende Berechnung $P$ von $M_j$ auf Eingabe $x$}, und verlangen zusätzlich
\begin{enumerate}[label={\arabic*.},nosep]
    \setcounter{enumi}{5}
    \item $|v\cap \Sigma^{e(i)}|\leq 2^{c}$.
\end{enumerate}

\begin{claim}
    %Sei $M_j$ eine kategorische Maschine relativ zu $O$.
    %Ab hinreichend großer Eingabelänge $n_0$ gilt für alle Eingaben $x$, $|x|\geq n_0$, alle $W, W'$:
    Seien $P_1, P_2$ zwei $(U, W, W')$ respektierenden akzeptierenden Berechnungen von $M_j$ auf Eingabe $x$.
    %Folgende Aussage gilt uneingeschränkt falls $e(i)\not\in M$, und sonst falls $e(i)\in M$ dann mit der Einschränkung dass $U=\Sigma^{e(i)-1}0$ oder $U=\Sigma^{e(i)-1}1$ gelten muss.
    Folgende Aussage gilt jeweils bezüglich dieser beiden Einschränkungen auf $U$:
    \begin{itemize}[nosep]
        \item $e(i)\in M$ und $U=\Sigma^{e(i)-1}0$ oder $U=\Sigma^{e(i)-1}1$,
        \item $e(i)\not\in M$ und $U=\Sigma^{e(i)}$.
    \end{itemize}

    Wenn $P^\mathrm{all}_1$ und $P^\mathrm{all}_2$ beide je eine (nicht notwendigerweise identische) Orakelfrage $q_1, q_2$ der Länge $e(i)$ enthalten, welche in $U-(W\cup W')$ liegt, dann haben diese zwei Berechnungen eine (identische) Orakelfrage $q$ der Länge $e(i)$ gemeinsam, welche nicht in $W\cup W'$ liegt.
\end{claim}
\begin{proof}[Skizze.]
    Wieder beweisen wir zunächst den ersten Fall, der zweite Fall ist noch leichter.
    Wir konstruieren ein $u$ ähnlich wie im originalen Beweis, verzichten aber auf das zusätzliche Wort $\alpha$, also sodass $u\cap\Sigma^{e(i)}=Y\subseteq U$.

    Es gilt dass $u$ mit $v_1$ auf $P^\mathrm{all}_1$ übereinstimmt, sowie $u$ mit $v_2$ auf $P^\mathrm{all}_2$ übereinstimmt.
    Sei $t'=t_{s-1}\cup \{\tau^2_j\mapsto 0\}$.
    Wir zeigen dass $u$ auch $t'$-valide ist
    Damit wäre $u$ dann eine geeignete Erweiterung von $w_{s-1}$ in Bearbeitung von Task $\tau^2_j$, für welche $M_j^O(x)$ nicht mehr kategorisch ist, was der Wahl von $M_j^O(x)$ widerspricht.

    Beob. dass $|P^\mathrm{yes}_1|,|P^\mathrm{yes}_2|\leq 2^{c}$,
    damit gilt $|u\cap\Sigma^{e(i)}|=|Y|\leq 2^{c+1}$.
    Aber nun gilt $c'=|\{j\mid t'(\tau^2_j)=0\}|=c+1$, damit $|u\cap\Sigma^{e(i)}|\leq 2^{c'}$.
    Außerdem definiert $u$ nach Konstruktion alle Wörter der Länge $e(i)\geq e(c+1)=e(c')$ und $u$ erfüllt damit auf jeden Fall V2.
    Es ist nun auch leicht zu sehen, dass V1, V3, V4, V5, V6 erfüllt sind, daher ist $u$ wie behauptet $t'$-valide.
    \medskip
\end{proof}

Damit gilt mit gleichem Verfahren

\begin{claim}
    $\P=\UP$ relativ zu $O$.
\end{claim}

\clearpage
\subsection{Orakel mit $\hDisjCoNP$ und alle Paare aus $\DisjNP$ sind $\P$-separierbar ($O_3$)}

Sei $e(0)=2, e(i+1)=2^{2^{e(i)}}$. (Doppelt exponentiell!) Sei hier $\{H_m\}_{m\in\mathbb N}$ eine Familie von paarweise disjunkten, unendlichen Teilmenge von $e(\mathbb N)$. (Ebenen $H_m$ gehören zur Zeugensprache bzgl. DisjCoNP-Maschinenpaar $M_a, M_b$.)
Starte mit $\mathrm{PSPACE}$-vollständiger Menge $C$ welche keine Wörter der Länge $e(\cdot)$ enthält.
Definiere folgende Zeugensprachen:
\begin{gather*}
    A_m^O \coloneqq \{ 0^n \mid n\in H_m, \text{für alle $x\in \Sigma^{n}$ gilt } x\in O \rightarrow \text{ $x$ endet mit $0$} \}\\
    B_m^O \coloneqq \{ 0^n \mid n\in H_m, \text{für alle $x\in \Sigma^{n}$ gilt } x\in O \rightarrow \text{ $x$ endet mit $1$} \}
\end{gather*}
Fakt: $|O\cap \Sigma^n|\geq 1$ für alle $n\in H_m \implies (A_m^O, B_m^O)\in\DisjCoNP$.
\medskip

Idee: erreiche entweder dass $M_a$, $M_b$ nicht disjunkt ablehnen (Task $\tau^2_{a,b}$), oder dass das Zeugenpaar $(A_m,B_m)$ nicht auf $(L(M_a),L(M_b))$ reduzierbar ist (Task $\tau^2_{a,b,r}$ für Transduktor $F_r$).

Gleichzeitig versuchen wir für möglichst viele Paare $M_a, M_b$ zu erreichen, dass diese nicht disjunkt akzeptieren. (Task $\tau^1_{a,b}$)
%Dies hat in gewisser Weise Priorität über die o.g. Diagonalisierung.
Am Ende sind die verbleibenden Maschinenpaare $M_a^O, M_b^O$ sehr speziell, denn sie sind auch für gewisse Teilmengen von $O$ kategorisch.
In Kombination mit dem Fakt dass $\P^C=\mathrm{PSPACE}^C$ können wir relevante Wörter in $O-C$ errechnen und so $L(M_a^O)$ von $L(M_b^O)$ trennen.
\medskip

Sei wie üblich $t\in \mathcal T$ wenn der Definitionsbereich endlich ist, nur die Tasks der Form $\tau^1_{a,b}, \tau^2_{a,b}$ enthält, $t$ diese Tasks auf $\mathbb N$ abbildet, und injektiv auf dem Support ist.

Ein Orakel $w\in\Sigma^*$ ist $t$-valide wenn $t\in\mathcal T$ und folgendes gilt:
\begin{enumerate}[label={V\arabic*}]
    \item Wenn $x<|w|$ und $|x|\not\in e(\mathbb N)$, dann gilt $x\in w\iff x\in C$.\\
        (Orakel $w$ und $C$ stimmen auf Wörtern mit Länge $\neq e(\cdot)$ überein.)
    %\item Für alle $n=e(i)$ gilt $|w\cap \Sigma^n|\leq 2$.\\
    %    (Orakel $w$ ist dünn auf den Stufen der Länge $e(\cdot)$.)
    \item Wenn $t(\tau^1_{a,b})=0$, dann existiert ein $z$ sodass $M_a^w(z)$ und $M_b^w(z)$ definitiv akzeptieren.\\
        ($M_a, M_b$ nicht disjunkt relativ zum finalen Orakel.)
    \item Wenn $t(\tau^2_{a,b})=0$, dann existiert ein $z$ sodass $M_a^w(z)$ und $M_b^w(z)$ definitiv ablehnen.\\
        (Wenn $t(\tau^2_{a,b})=0$, dann $(\overline{L(M_a)}, \overline{ L(M_b)})\not\in \DisjCoNP$ relativ zum finalen Orakel.)
    \item Wenn $0<t(\tau^2_{a,b})=m$, dann gilt für alle $n\in H_m$: wenn $w$ für alle Wörter der Länge $n$ definiert ist, dann $|\Sigma^n\cap w|\geq 1$.\\
        (Wenn $0<t(\tau^2_{a,b})=m$, dann $(A_m,B_m)\in\DisjCoNP$.)
\end{enumerate}

Sei $T$ eine abzählbare Aufzählung der o.g. Tasks sodass $\tau^2_{a,b,r}$ immer nach $\tau^2_{a,b}$ kommt.
%Wir sagen dass ein Task $\tau$ \emph{priorisierter} als Task $\tau'$ ist, wenn $\tau$ in $T$ vor $\tau'$ kommt.

[ . . . Üblicher Text zur stufenweisen Erweiterung von $w_s$ und $t_s$ . . . ]

%Wir definieren nun Stufe $s>0$, diese startet mit einem $t_{s-1}\in\mathcal T$ und eine $t_{s-1}$-validen Orakel $w_{s-1}$ welche nun die Tasks $\{\tau_{(1)}, \tau_{(2)}, \dots, \tau_{(s)}\}$ behandelt -- $\tau^{(s)}$ kommt dazu. Je nach Typ von Task $\tau^{(s)}$ führen wir nun Folgendes durch:
Wir definieren nun Stufe $s>0$, diese startet mit einem $t_{s-1}\in\mathcal T$ und eine $t_{s-1}$-validen Orakel $w_{s-1}$ welche nun den kleinsten Task bearbeitet, welcher noch in $T$ ist. Dieser wird unmittelbar nach der Bearbeitung aus $T$ entfernt. In der Bearbeitung wird das Orakel strikt verlängert.
\begin{itemize}

    \item $\tau^1_{a,b}$: Setze $t'=t_{s-1}\cup\{\tau^1_{a,b}\mapsto 0\}$. Existiert ein $t'$-valides Orakel $v\sqsupsetneq w_{s-1}$, dann setze $t_s\coloneqq t'$ und $w_s\coloneqq v$.

        Ansonsten setze $t_s\coloneqq t_{s-1}$ und setze $w_s\coloneqq w_{s-1}y$ für geeignetes $y\in\{0,1\}$ sodass $w_s$ auch $t_s$-valide ist. (Das ist möglich nach Behauptung 4.1.)

    \item $\tau^2_{a,b}$: Setze $t'=t_{s-1}\cup\{\tau^2_{a,b}\mapsto 0\}$. Existiert ein $t'$-valides Orakel $v\sqsupsetneq w_{s-1}$, dann setze $t_s\coloneqq t'$ und $w_s\coloneqq v$. Entferne außerdem alle Tasks der Form $\tau^2_{a,b,r}$ von $T$.

        Ansonsten wähle ein hinreichend großes $m\not\in \img(t_s)$ sodass $w_s$ kein Wort der Länge $\min H_m$ definiert. Setze $t_s\coloneqq t_{s-1}\cup \{ \tau^2_{a,b}\mapsto m \}$; damit ist $w_{s-1}$ auch $t_s$-valide. Setze $w_s\coloneqq w_{s-1}y$ für geeignetes $y\in\{0,1\}$ sodass $w_s$ auch $t_s$-valide ist. (Das ist möglich nach Behauptung 4.1.)

    \item $\tau^2_{a,b,r}$: Wir wissen dass $t_{s-1}(\tau^2_{a,b})=m>0$. Setze $t_s=t_{s-1}$ und wähle ein $t_s$-valides Orakel $w_s\sqsupsetneq w_{s-1}$ sodass bezüglich einem $n\in\mathbb N$ eine der folgenden Aussagen gilt:
        \begin{itemize}[nosep,endpenalty=10000]
            \item $0^n\in A_m^v$ für alle $v\sqsupseteq w_s$ und $M_a(F_r(0^n))$ akzeptiert definitiv relativ zu $w_s$.
            \item $0^n\in B_m^v$ für alle $v\sqsupseteq w_s$ und $M_b(F_r(0^n))$ akzeptiert definitiv relativ zu $w_s$.
        \end{itemize} (Das ist möglich nach Behauptung 4.2.)
\end{itemize}

\begin{claim}
    Für jedes $t\in\mathcal T$ und jedes $t$-valide $w$ existiert ein $b\in\{0,1\}$ sodass $wb$ auch $t$-valide ist.
\end{claim}

\begin{claim}
    Die Bearbeitung eines Tasks $\tau^2_{a,b,r}$ ist möglich: gilt $t_{s-1}(\tau^2_{a,b})=m>0$, dann lässt sich $w_{s-1}$ so zu $t_{s-1}$-validem $u\sqsupsetneq w_{s-1}$ erweitern, dass eine der o.g. Fälle eintritt.
\end{claim}

Damit ist die Konstruktion möglich. Sei $O=\lim_{s\to\infty} w_s$.

\begin{claim}
    Kein Paar aus $\DisjCoNP^O$ ist $\leqmpp$-vollständig.
\end{claim}

Wir wollen nun zeigen, dass wir disjunkte Paare von NP-Sprachen in P separieren können. Sei im Folgenden $M_a, M_b$ ein komplementär akzeptierendes Paar an Maschine relativ zu $O$. Um die Sprachen zu trennen nutzen wir aus, dass $\mathrm{PSPACE}^C=\P^C$, um so iterativ eine Menge $D\subseteq O$ an Orakelwörtern der Länge $e(\cdot)$ aufbauen, welche für die Berechnungen $M_a(x), M_b(x)$ relevant ist, bis wir nach einigen Iterationen alle solchen relevanten Wörter gefunden haben. 
Wir beschränken uns im Folgenden auf Eingaben hinreichender Länge, sodass es für $x$ ein eindeutiges $i$ gibt, sodass
\[ e(i-1) \leq \log(|x|) < e(i), \quad p_a(|x|)+p_b(|x|)< 2^{e(i)} < e(i+1).\tag{\ast} \]

Wir definieren Folgendes: %Seien $W,W'\subseteq \{q\in \Sigma^*\mid \exists i.|q|=e(i)\}$ zwei disjunkte Mengen mit $W\subseteq O,$ $W'\subseteq \overline{O}$.
Sei $j\in \{a,b\}$
Eine \emph{$(W, W')$ respektierende akzeptierende Berechnung $P$ von $M_j$ auf Eingabe $x$} ist ein akzeptierender Rechenweg $P$ von $M_j(x)$ relativ zu einem Orakel $v\subseteq\Sigma^*$, wobei 
\[ W,W'\subseteq \Sigma^{e(i)}, \quad W\subseteq O, \quad W'\subseteq\overline{O} \]
und für $v$ gilt:
\begin{enumerate}[noitemsep,label=\arabic*.]
    \item $v$ ist definiert für genau die Wörter der Länge $\leq p_j(|x|)$.
    \item $v(q)=O(q)$ für alle $q$ mit $|q|\neq  e(i)$, wobei hier das $i$ diejenige eindeutige Zahl ist für die obigen Ungleichungen ($\ast$) bzgl. $|x|$ gelten.
    \item $v(q)=1$ für alle $q$ mit $q\in W$.
    \item $v(q)=0$ für alle $q$ mit $q\in W'$.
    %\item $v(q)=1 \implies q\in U$ für alle $q\in\Sigma^{e(i)}$. [$v$ enthält auf Ebene $e(i)$ nur Wörter, die auch in $U$ vorkommen.]
\end{enumerate}

Sei $P^\mathrm{all}$ die Menge der Orakelfragen auf $P$, und sei $P^\mathrm{yes}=P^\mathrm{all}\cap v$, $P^\mathrm{no}=P^\mathrm{all}\cap \overline{v}$.
Beob. dass  das Ermitteln einer $(W, W')$ respektierenden akzeptierenden Berechnung einfach in Polynomialzeit (abh. von $|x|$ und $\ell(W),\ell(W')$) relativ zu $O$ möglich ist: insbesondere stimmt $O$ mit $C$ auf Wörtern der Länge $\neq e(\cdot)$ überein, und alle anderen Wörter der Länge $e(0), e(1), \dots, e(i-1)$ können vorab mit Queries an $O$ in Polynomialzeit erfragt werden.
Entsprechende Belegungen von $v$ für Wörter der Länge $e(i)$ können z.B. in PSPACE enumeriert werden.

Sei $s$ die Stufe bei der $\tau^1_{a,b}$ betrachtet wurde.
Zusätzlich zur Einschränkung $(\ast)$ diskutieren wir ab jetzt nur noch Eingaben, für welche das Orakel $w_{s-1}$ keine Wörter der Länge $e(i)$ definiert.


\begin{claim}
    %Sei $M_j$ eine kategorische Maschine relativ zu $O$.
    %Ab hinreichend großer Eingabelänge $n_0$ gilt für alle Eingaben $x$, $|x|\geq n_0$, alle $W, W'$:
    Seien $P_a, P_b$ je $(W, W')$ respektierenden akzeptierenden Berechnungen von $M_a$ bzw. $M_b$ auf Eingabe $x$.
    %Folgende Aussage gilt uneingeschränkt falls $e(i)\not\in M$, und sonst falls $e(i)\in M$ dann mit der Einschränkung dass $U=\Sigma^{e(i)-1}0$ oder $U=\Sigma^{e(i)-1}1$ gelten muss.
    Wenn $P^\mathrm{all}_a$ und $P^\mathrm{all}_b$ beide je eine (nicht notwendigerweise identische) Orakelfrage $q_a, q_b$ der Länge $e(i)$ enthalten, welche in $\Sigma^{e(i)}-(W\cup W')$ liegt, dann haben diese zwei Berechnungen eine (identische) Orakelfrage $q$ der Länge $e(i)$ gemeinsam, welche nicht in $W\cup W'$ liegt.
\end{claim}
\begin{proof}[Skizze.]
    Angenommen, dies gilt nicht, also sei $x$ sowie $W,W'\subseteq \Sigma^{e(i)}$, $W\subseteq O$, $W'\subseteq\overline{O}$ gegeben.
    Seien außerdem $P_a$ und $P_b$ je zwei $(W, W')$ respektierenden akzeptierenden Berechnungen von $M_a$, $M_b$ auf Eingabe $x$,
    welche je eine Orakelfrage der Länge $e(i)$ enthalten, welche nicht in $W\cup W'$ liegt,
    aber keine Orakelfrage aus $\Sigma^{e(i)}-(W\cup W')$ gemeinsam haben.
    Dann sind schon $P_a$ und $P_b$ verschieden.
    Seien ferner $v_a, v_b$ die zugehörigen Orakel, also für welche $M_j(x)$ akzeptiert und Eigenschaften~1--4 erfüllen.

    Wir werden nun ein $t_{s-1}$-valides Orakel $u\sqsupsetneq w_{s-1}$ konstruieren welches mit $v_a$ auf $P^\mathrm{all}_b$ übereinstimmt, und welches mit $v_b$ auf $P^\mathrm{all}_b$ übereinstimmt.
    Außerdem wird es auf Ebene $\Sigma^{e(i)}$ mindestens ein Wort enthalten, woraus wir zeigen können dass $u$ sogar ein geeignetes Orakel zur Zerstörung dieses DisjNP-Machinenpaars in der Bearbeitung von Task $\tau^1_{a,b}$ in Stufe $s$ ist, ohne Beschränkungen bzgl. DisjCoNP-Zeugensprachen zu verletzen. Insbesondere akzeptiert dann sowohl $M_a^O(x)$ als auch $M_b^O(x)$ was der Voraussetzung widerspricht.

    Sei $Y=(P^\mathrm{yes}_a\cup P^\mathrm{yes}_b)\cap\Sigma^{e(i)}$, und $N=(P^\mathrm{no}_a\cup P^\mathrm{no}_b)\cap\Sigma^{e(i)}$.
    Wir zeigen $Y\cap N = \emptyset$. (Das soll uns helfen nachzuweisen, dass ein geeignetes $u$ existieren kann.)
    Nehme an es gibt ein $q\in Y\cap N$ der Länge $e(i)$.
    \begin{itemize}[noitemsep]
        \item Ist $q\in W$ dann gilt schon sofort dass $q\not\in P^\mathrm{no}_a, P^\mathrm{no}_b$ was $q\in N$ widerspricht.
        \item Ist $q\in W'$ dann gilt schon sofort dass $q\not\in P^\mathrm{yes}_a, P^\mathrm{yes}_b$ was $q\in Y$ widerspricht.
        \item Andernfalls ist $q\in  W\cup W'$, dann gilt $q\in P^\mathrm{yes}_a\cap P^\mathrm{no}_b$ oder $q\in P^\mathrm{yes}_a\cap P^\mathrm{no}_b$.
            In beiden Fällen hätten wir aber, dass $P_a$ und $P_b$ eine Orakelfrage der Länge $e(i)$ teilen, welche in $W\cup W'$ liegt. Das widerspricht der urpsrünglichen Annahme.
    \end{itemize}

    Es gilt also $Y\cap N =\emptyset$. Wähle ein $\alpha\in \Sigma^{e(i)}-N$. Dieses existiert da $|N|\leq p_a(|x|)+p_b(|x|)<2^{e(i)}$ nach ($\ast$).
    Sei nun $u$ das Orakel was genau alle Wörter der Länge $\leq p_j(|x|)$ definiert sind, und
    \[ u(z)= \begin{cases} O(z) & \text{falls $|z|\neq e(i)$}\\ 1 & \text{falls $z=\alpha$} \\1 & \text{falls $z\in Y$} \\ 0&\text{sonst,} \end{cases}
    \]
    also wie $O^{\leq p_j(|x|)}$ aufgebaut ist, außer dass die Ebene $e(i)$ mit genau den Wörtern aus $Y$ gefüllt wird. bzw. $u\cap\Sigma^{e(i)} = Y\cup \{\alpha\}$.
    Es ist leicht zu sehen dass $u\sqsupsetneq w_{s-1}$.
    Beob. dass 
    \[ u\cap N = \Sigma^{e(i)}\cap u \cap N = (Y\cup \{\alpha\}) \cap N= Y\cap N=\emptyset.\]

    Das Orakel $u$ stimmt mit $v_a$ auf $P^\mathrm{all}_a$ überein. Sei hierfür $q\in P^\mathrm{all}_a$.
    Ist $|q|\neq e(i)$, dann gilt schon nach Definition $v_a(q)=O(q)=u(q)$. Sei daher im Folgenden $|q|=e(i)$.
    Ist $q\in P^\mathrm{yes}_a$, dann auch $q\in v_a$. Außerdem dann auch $q\in Y$, daher $q\in u$.
    Ansonsten ist $q\in P^\mathrm{no}_a$, dann auch $q\not\in v_a$. Außerdem dann auch $q\in N$, daher $q\not\in u$ nach obiger Beobachtung.
    
    Auf symmetrische Weise stimmt $u$ mit $v_b$ auf $P^\mathrm{all}_b$ überein.
    Wir zeigen nun dass $u$ auch $t_{s-1}$-valide ist.
    Nach obiger Argumentation wäre dann $u$ eine geeignete Erweiterung von $w_{s-1}$ für welche $M_a^O(x)$ und $M_b^O(x)$ akzeptieren, also nicht mehr disjunkt, was der Wahl von $M_a, M_b$ widerspricht.

    Nach Konstruktion ist V1 und V2 erfüllt; V3 ist wegen $u\sqsupsetneq w_{s-1}$ erfüllt.
    Angenommen V4 ist verletzt.
    Wieder kann das nur an der Ebene $e(i)$ liegen.
    Aber hier gilt $|u\cap\Sigma^{e(i)}|=|Y\cap\{\alpha\}|\geq 1$.
\end{proof}

\begin{claim}
    Zu jedem disjunkten \NP-Paar $(L_a, L_b)$ existiert ein Separator aus $\P$.
\end{claim}
\begin{proof}
    Sei $L_a, L_b\in\NP^O, L_a\cap L_b=\emptyset$. Es existiert nach Definition ein Paar an Maschinen $M_a, M_b$ mit $L(M_a^O)=L_a$, $L(M_b^O)=L_b$.
    Wir zeigen für hinreichend lange $x$ wie man $L_1$ von $L_2$ in Polyonmialzeit relativ zu $O$ trennen kann.
    
    Sei im Folgenden $x$ hinreichend lange wie oben diskutiert, also für dieses $(\ast)$ mit eindeutigem $i$ gilt, sowie $w_{s-1}$ keine Wörter der Länge $e(i)$ definiert, wobei $s$ die Stufe ist, bei der $\tau^1_{a,b}$ betrachtet wurde.

    Wir werden diese Eigenschaft ausnutzen und iterativ Mengen $W, W'$ aufbauen, welche für die Berechnungen $M_a^O(x), M_b^O(x)$ relevant sind, bis wir alle solchen relevanten Wörter gefunden haben.
    Das machen wir je abwechselnd für $M_a^O(x)$ und $M_b^O(x)$.
    Betrachte dafür folgende Subroutine:

    \noindent%
    \SetKwProg{Fn}{Function}{:}{}%
    \SetKw{Assert}{assert}%
    \begin{algorithm}[H]
        $W\gets\emptyset,\, W'\gets\emptyset$\;
        \For{$k$ von $0$ bis $\max\{p_a(|x|), p_b(|x|)\}+1$}{
            $P_a\gets$ eine $(W, W')$ respektierende akzeptierende Berechnung $P$ von $M_a$ auf $x$ mit $|(P^\mathrm{all}\cap\Sigma^{e(i)})-(W\cup W')|$ minimal, oder $\bot$ falls keine existiert\;
            $P_b\gets$ eine $(W, W')$ respektierende akzeptierende Berechnung $P$ von $M_b$ auf $x$ mit $|(P^\mathrm{all}\cap\Sigma^{e(i)})-(W\cup W')|$ minimal, oder $\bot$ falls keine existiert\;
            \lIf{$P_a= \bot$}{\Return{„$x\in L_b$“}}
            \lIf{$P_b= \bot$}{\Return{„$x\in L_a$“}}
            \tcc{ab hier sind $P_a, P_b$ je zwei $(W, W')$ respektierende akzeptierende Berechnungen}
            \If{alle $q\in P_a^\mathrm{all}$ mit $|q|=e(i)$ sind in $W\cup W'$}{
                \Return{„$x\in L_a$“}
            }
            \If{alle $q\in P_b^\mathrm{all}$ mit $|q|=e(i)$ sind in $W\cup W'$}{
                \Return{„$x\in L_b$“}
            }
            \ForEach{$q\in P_a^\mathrm{all}\cup P_b^\mathrm{all}$ mit $|q|=e(i)$}
            {
                \lIf{$q\in O$}{$W\gets W\cup\{q\}$}
                \lIf{$q\not\in O$}{$W'\gets W'\cup\{q\}$}
            }
        }
        \Return{„$x\not\in L_a\cup L_b$“}
    \end{algorithm}
Es ist leicht zu sehen dass der Algorithmus eine polynomielle Laufzeitschranke einhält.
Wir beobachten die Invariante dass $W\subseteq O\cap\Sigma^{e(i)}$ und $W'\subseteq \overline{O}\cap\Sigma^{e(i)}$. 

Wir zeigen zunächst dass der Algorithmus keine falsch-positiven Fehler macht.
Für den ersten Fall nehme an dass der Algorithmus mit „$x\in L_a$“ terminiert aber es gilt $x\not\in L_a$ und $x\in L_b$.
Terminiert der Algorithmus in Z.~6, dann war $P_b=\bot$, was nach obiger Invariante bedeutet dass $M_b^O(x)$ ablehnt (denn sonst existiert immer eine $(W, W')$ respektierende akzeptierende Berechnung); Widerspruch zur Annahme.

Terminiert der Algorithmus in Z.~8, können wir den Widerspruch $x\in L_a$ zeigen: Sei $v$ das Orakel der $(W, W')$ respektierenden akzeptierenden Berechnung $P$ von $M_a(x)$. Es ist nun leicht zu sehen dass $v$ und $O$ auf $P_a^\mathrm{all}$ übereinstimmen. Damit gilt auch dass $M_a^O(x)$ akzeptiert und damit $x\in L_a$.

Der symmetrische Fall mit $L_b$ läuft analog. Damit macht der Algorithmus also zumindest schon keine falsch-positiven Fehler.
\medskip

Es verbleibt zu zeigen dass der Algorithus keine falsch-negativen Fehler macht.
Wir zeigen dies für den Fall dass $x\in L_a$, der andere Fall $x\in L_b$ läuft analog.
Sei hierfür $P_a^*$ der längste akzeptierende Rechenweg von $M_a^O(x)$.
Beob. mit obiger Invariante dass $P_a^*$ auch immer ein $(W,W')$ respektierender akzeptierender Rechenweg ist.
Nachdem der Algorithmus keine falsch-positven Fehler macht, sind die Bedingungen in Zz.~5 und 10 nie erfüllt.

Wir zeigen nun, dass nach $\leq p_j(|x|)+1$ vielen Iterationen auch die Bedingung in Z.~7 erfüllt ist.
Hierfür zeigen wir, dass $|P_a^{*\mathrm{all}}\cap \Sigma^{e(i)}-(W\cup W')|$ in jeder Iteration um $\geq 1$ abnimmt. Da $|P_a^{*\mathrm{all}}|\leq p_j(|x|)$ ist nach $\leq p_j(|x|)+1$ vielen Iterationen $|P_a^{*\mathrm{all}}\cap \Sigma^{e(i)}-(W\cup W')|=0$.
Nach $\leq p_j(|x|)+1$ vielen Iterationen wird also in Z.~3 eine Berechnung $P_a$ ausgewählt, bei der alle $q\in P_a^{\mathrm{all}}$ mit $|q|=e(i)$ in $W\cup W'$ liegen. Dann ist die Bedingung in Z.~7 erfüllt und der Algorithmus terminiert akzeptierend.

Steht der Algorithmus in Z.~13, dann gilt sowohl für das ausgewählte $P_b$, als auch für $P_a^*$ dass beide je eine (nicht notwendigerweise identische) Orakelfrage der Länge $e(i)$ enthalten, welche nicht in $W\cup W'$ liegt. (Andernfalls  wäre $P_a$ in Z.~3 anders ausgewählt worden.)
Damit ist Behauptung~4.4 anwendbar. Also haben diese zwei Berechnungen eine identische Orakelfrage $q\in P_b^\mathrm{all}\cap P_a^{*\mathrm{all}}\cap \Sigma^{e(i)}$ gemeinsam, welche nicht in $W\cup W'$ liegt.
Diese wird in den Zz.~13--16 dann auch irgendwann der Menge $W\cup W'$ hinzugefügt.
Damit nimmt auch $|P_a^{*\mathrm{all}}\cap \Sigma^{e(i)}-(W\cup W')|$ um $\geq 1$ ab, wie behauptet.
\end{proof}

\end{document}
